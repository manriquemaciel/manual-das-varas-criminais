%% Generated by Sphinx.
\def\sphinxdocclass{report}
\documentclass[letterpaper,10pt,brazil]{sphinxmanual}
\ifdefined\pdfpxdimen
   \let\sphinxpxdimen\pdfpxdimen\else\newdimen\sphinxpxdimen
\fi \sphinxpxdimen=.75bp\relax
\ifdefined\pdfimageresolution
    \pdfimageresolution= \numexpr \dimexpr1in\relax/\sphinxpxdimen\relax
\fi
%% let collapsible pdf bookmarks panel have high depth per default
\PassOptionsToPackage{bookmarksdepth=5}{hyperref}

\PassOptionsToPackage{booktabs}{sphinx}
\PassOptionsToPackage{colorrows}{sphinx}

\PassOptionsToPackage{warn}{textcomp}
\usepackage[utf8]{inputenc}
\ifdefined\DeclareUnicodeCharacter
% support both utf8 and utf8x syntaxes
  \ifdefined\DeclareUnicodeCharacterAsOptional
    \def\sphinxDUC#1{\DeclareUnicodeCharacter{"#1}}
  \else
    \let\sphinxDUC\DeclareUnicodeCharacter
  \fi
  \sphinxDUC{00A0}{\nobreakspace}
  \sphinxDUC{2500}{\sphinxunichar{2500}}
  \sphinxDUC{2502}{\sphinxunichar{2502}}
  \sphinxDUC{2514}{\sphinxunichar{2514}}
  \sphinxDUC{251C}{\sphinxunichar{251C}}
  \sphinxDUC{2572}{\textbackslash}
\fi
\usepackage{cmap}
\usepackage[T1]{fontenc}
\usepackage{amsmath,amssymb,amstext}
\usepackage{babel}



\usepackage{tgtermes}
\usepackage{tgheros}
\renewcommand{\ttdefault}{txtt}



\usepackage[Sonny]{fncychap}
\ChNameVar{\Large\normalfont\sffamily}
\ChTitleVar{\Large\normalfont\sffamily}
\usepackage{sphinx}

\fvset{fontsize=auto}
\usepackage{geometry}


% Include hyperref last.
\usepackage{hyperref}
% Fix anchor placement for figures with captions.
\usepackage{hypcap}% it must be loaded after hyperref.
% Set up styles of URL: it should be placed after hyperref.
\urlstyle{same}


\usepackage{sphinxmessages}
\setcounter{tocdepth}{1}



\title{manual das varas criminais}
\date{29 ago. 2025}
\release{1.0}
\author{TJAM}
\newcommand{\sphinxlogo}{\vbox{}}
\renewcommand{\releasename}{Release}
\makeindex
\begin{document}

\ifdefined\shorthandoff
  \ifnum\catcode`\=\string=\active\shorthandoff{=}\fi
  \ifnum\catcode`\"=\active\shorthandoff{"}\fi
\fi

\pagestyle{empty}
\sphinxmaketitle
\pagestyle{plain}
\sphinxtableofcontents
\pagestyle{normal}
\phantomsection\label{\detokenize{index::doc}}


\begin{sphinxadmonition}{note}{Nota:}
\sphinxAtStartPar
Manual em construção
\end{sphinxadmonition}

\sphinxAtStartPar
Prezados colaboradores e servidores,

\sphinxAtStartPar
É com imenso prazer que apresentamos o Manual 00, o primeiro passo em direção à melhoria e otimização dos procedimentos judiciais nas Varas Criminais do Tribunal de Justiça do Amazonas (TJAM). Este manual é resultado do comprometimento e dedicação de uma equipe multidisciplinar, que trabalhou em conjunto para desenvolver soluções inovadoras e práticas para nossos desafios cotidianos.


\chapter{Apresentação da ideia}
\label{\detokenize{index:apresentacao-da-ideia}}
\sphinxAtStartPar
A dinâmica do Poder Judiciário exige constante aprimoramento e inovação. O TJAM, ao longo dos anos, tem enfrentado um expressivo volume processual, que, somado a procedimentos pouco uniformes, gargalos em audiências e deficiências em sistemas, apontou a urgente necessidade de padronização, racionalização e uniformização dos procedimentos e métodos de trabalho de nossas Unidades Judiciais. Diante deste cenário, foi criada uma comissão para propor soluções eficazes e eficientes, com foco em procedimentos que auxiliem as atividades desempenhadas por estas Unidades.


\chapter{Objetivos do Manual}
\label{\detokenize{index:objetivos-do-manual}}
\sphinxAtStartPar
Este manual tem por objetivo:
\begin{itemize}
\item {} 
\sphinxAtStartPar
\sphinxstylestrong{Padronizar e otimizar} os procedimentos judiciais nas Varas Criminais, garantindo assim uma maior celeridade processual.

\item {} 
\sphinxAtStartPar
\sphinxstylestrong{Aumentar a eficiência} no fluxo de trabalho interno, reduzindo gargalos e agilizando processos.

\item {} 
\sphinxAtStartPar
\sphinxstylestrong{Estabelecer claramente} as competências institucionais de cada unidade, de forma a garantir a correta distribuição de tarefas e responsabilidades.

\end{itemize}


\chapter{Metodologia utilizada}
\label{\detokenize{index:metodologia-utilizada}}
\sphinxAtStartPar
A metodologia empregada na construção deste manual envolveu:
\begin{itemize}
\item {} 
\sphinxAtStartPar
Fluxo de padronização, racionalização e uniformização dos procedimentos.

\item {} 
\sphinxAtStartPar
Oficinas participativas, permitindo que as partes envolvidas trouxessem suas contribuições.

\item {} 
\sphinxAtStartPar
Reuniões (presenciais e remotas) com a equipe do projeto.

\item {} 
\sphinxAtStartPar
Análise dos documentos normativos das Varas para maior aporte jurídico.

\end{itemize}


\chapter{Conteúdo do Manual}
\label{\detokenize{index:conteudo-do-manual}}
\sphinxAtStartPar
Este manual contém:
\begin{itemize}
\item {} 
\sphinxAtStartPar
\sphinxstylestrong{Textos explicativos} sobre cada procedimento, claramente detalhados e estruturados.

\item {} 
\sphinxAtStartPar
\sphinxstylestrong{Vídeos demonstrativos} acessados por QRCode, que ilustram de forma prática a execução dos procedimentos.

\item {} 
\sphinxAtStartPar
\sphinxstylestrong{Exemplos} práticos que ajudam na compreensão dos procedimentos.

\item {} 
\sphinxAtStartPar
\sphinxstylestrong{Modelos de documentos} a serem utilizados, padronizados conforme as diretrizes estabelecidas.

\end{itemize}

\begin{sphinxadmonition}{attention}{Atenção:}
\sphinxAtStartPar
Informações que necessitem de atenção aparecerão assim.
\end{sphinxadmonition}

\begin{sphinxadmonition}{tip}{Dica:}
\sphinxAtStartPar
Dicas aparecerão assim.
\end{sphinxadmonition}

\begin{sphinxadmonition}{note}{Nota:}
\sphinxAtStartPar
Notas adicionais sobre o tema aparecerão assim
\end{sphinxadmonition}

\sphinxAtStartPar
Agradecemos a todos os envolvidos neste projeto e esperamos que este manual seja uma ferramenta valiosa para a melhoria contínua de nossos procedimentos e práticas. Juntos, caminharemos em direção a um Judiciário mais eficiente e ágil!

\sphinxstepscope


\section{Processos Entrados (Análise do cartório)}
\label{\detokenize{01processos_entrados:processos-entrados-analise-do-cartorio}}\label{\detokenize{01processos_entrados::doc}}

\subsection{Introdução}
\label{\detokenize{01processos_entrados:introducao}}
\sphinxAtStartPar
Nas Varas Criminais os autos processuais entram através das seguintes filas:
\begin{itemize}
\item {} 
\sphinxAtStartPar
Processos Redistribuídos/Transferidos

\item {} 
\sphinxAtStartPar
Processos Entrados (Inquéritos peticionados pela delegacia)

\item {} 
\sphinxAtStartPar
Processos a serem Recebidos de Outros Foros

\end{itemize}

\sphinxAtStartPar
Em sua maioria, os processos das Varas Criminais são oriundos da Central de Inquérito, quando do oferecimento da denúncia pelo Ministério Público, logo, a principal porta de entrada de processos é a fila “Processos Redistribuídos/Transferidos”.

\begin{sphinxadmonition}{note}{Nota:}
\sphinxAtStartPar
A fundamentação está na Resolução n° 06/2019 \sphinxhyphen{} TJAM, em seu art. 1º, \S{}3º, nestes termos: “A competência da Vara de Inquéritos Policiais se exaure após o oferecimento da denúncia pelo Ministério Público, ocasião em que as medidas cautelares, requeridas no curso da ação penal, serão de competência do Juízo de Conhecimento.”
\end{sphinxadmonition}

\begin{sphinxadmonition}{tip}{Dica:}
\sphinxAtStartPar
Processos entrados, geralmente, são os Inquéritos peticionados pela delegacia equivocadamente, pois deveriam ser direcionados à Vara de Inquéritos Policiais.
\end{sphinxadmonition}


\subsection{Checklist}
\label{\detokenize{01processos_entrados:checklist}}
\sphinxAtStartPar
Nesta fase pré\sphinxhyphen{}processual a Vara deverá realizar as seguintes diligências, antes de encaminhar concluso para o Juiz.
\begin{itemize}
\item {} 
\sphinxAtStartPar
Cadastro de Partes/Representantes

\item {} 
\sphinxAtStartPar
Inclusão do Histórico de Partes

\item {} 
\sphinxAtStartPar
Inclusão de Tarjas processuais

\item {} 
\sphinxAtStartPar
Verificação do Assunto Principal

\item {} 
\sphinxAtStartPar
Conferência do Banco Nacional de Mandados de Prisão

\item {} 
\sphinxAtStartPar
Expedição de Certidão de Antecedentes Criminais

\end{itemize}

\begin{sphinxadmonition}{note}{Nota:}
\sphinxAtStartPar
A atualização do cadastro de partes e representantes obedece à Resolução CNJ nº 331/2020, que determinou o saneamento do DataJud por Unidade Judiciária, bem como é critério para pontuação do Selo CNJ de qualidade, nos termos do art 8º, inciso III do Anexo IV da Portaria nº 82, de 31 de março de 2023. Serão considerados apenas CPF e CNPJ. Em relação à parte ativa em Ações Penais Pública Incondicionadas, cadastrar a vítima com o código de terceiro “50\sphinxhyphen{}Vítima” e o MP como “Autor” “Código 4441052”. Em relação à parte passiva, sempre que possível buscar nos próprios autos CPF do réu e inserir no cadastro. Quando não disponível, solicitar autorização judicial para busca nos sistemas e posterior inclusão.
\end{sphinxadmonition}

\begin{sphinxadmonition}{warning}{Aviso:}
\sphinxAtStartPar
A correta indicação de assuntos também obedece à Resolução CNJ nº 331/2020, que determinou o saneamento do DataJud por Unidade Judiciária, bem como é critério para pontuação do Selo CNJ de qualidade, nos termos do art 8º, inciso III do Anexo IV da Portaria nº 82, de 31 de março de 2023. Nesse caso, sempre verificar se o assunto corresponde ao fato tipificado na denúncia e se o cadastro está correto, devendo sempre constar o último código disponível da árvore. Para alteração acessar no SAJ “Retificação de Processo” / “Assuntos”.
\end{sphinxadmonition}


\begin{sphinxseealso}{Ver também:}

\sphinxAtStartPar
Saiba mais sobre como realizar:

\sphinxAtStartPar
Cadastro de Partes/Representantes: {\hyperref[\detokenize{projud_23_cadastroparte::doc}]{\sphinxcrossref{\DUrole{doc}{Cadastro de Parte e CPF/CNPJ}}}}

\sphinxAtStartPar
Inclusão de Histórico de Partes: {\hyperref[\detokenize{projud_23_cadastroparte::doc}]{\sphinxcrossref{\DUrole{doc}{Cadastro de Parte e CPF/CNPJ}}}}

\sphinxAtStartPar
Inclusão de Tarjas processuais: {\hyperref[\detokenize{projud_11_telainicialprocesso::doc}]{\sphinxcrossref{\DUrole{doc}{Tela Inicial do Processo}}}}

\sphinxAtStartPar
Verificação do Assunto Principal: {\hyperref[\detokenize{projud_23_cadastroparte::doc}]{\sphinxcrossref{\DUrole{doc}{Cadastro de Parte e CPF/CNPJ}}}}

\sphinxAtStartPar
Expedição de Certidão de Antecedentes Criminais: {\hyperref[\detokenize{projud_35_enviarconcluso::doc}]{\sphinxcrossref{\DUrole{doc}{Enviar Concluso}}}}


\end{sphinxseealso}



\subsection{Inclusão de Tarjas Processuais}
\label{\detokenize{01processos_entrados:inclusao-de-tarjas-processuais}}
\sphinxAtStartPar
Tarjas são mecanismos que auxiliam na visualização de determinadas características do processo.

\sphinxAtStartPar
Determinadas informações são fundamentais no momento de classificar o processo, tais como: segredo de justiça, réu preso, réu preso por outro processo, meta 02, entre outros. A Vara, a partir de suas necessidades, deverá estabelecer um padrão de inclusão de tarjas de forma a facilitar os trabalhos no Fluxo SAJ/PG5.


\begin{sphinxseealso}{Ver também:}

\sphinxAtStartPar
Saiba mais sobre como adicionar tarjas no sistema:

\sphinxAtStartPar
Tarjas e sigilo: {\hyperref[\detokenize{projud_13_alterarsigilo::doc}]{\sphinxcrossref{\DUrole{doc}{Alterar Sigilo de Processos, Documentos e Movimentações}}}}


\end{sphinxseealso}



\subsection{Conferência do Banco Nacional de Mandados de Prisão}
\label{\detokenize{01processos_entrados:conferencia-do-banco-nacional-de-mandados-de-prisao}}

\begin{sphinxseealso}{Ver também:}

\sphinxAtStartPar
Saiba mais sobre cadastro de mandado de prisão:

\sphinxAtStartPar
Cadastro de mandado de prisão: {\hyperref[\detokenize{projud_32_expedicaomandado::doc}]{\sphinxcrossref{\DUrole{doc}{Expedição de Mandado para Oficial de Justiça}}}}


\end{sphinxseealso}



\subsection{Expedição de Certidão de Antecedentes Criminais}
\label{\detokenize{01processos_entrados:expedicao-de-certidao-de-antecedentes-criminais}}
\sphinxAtStartPar
A expedição de nova Certidão de Antecedentes Criminais é importante para averiguar a possibilidade de aplicação de medidas diversas da prisão ou outros benefícios como a Suspensão Condicional do Processo, sendo importante a juntada do referido documento atualizado.

\sphinxAtStartPar
Para iniciar a criação de uma nova certidão de antecedentes criminais dentro do SAJ/PG5 acesse ”Certidão/Cadastro de Pedido…”.

\sphinxAtStartPar
Será apresentada a tela abaixo, para iniciar a consulta clique no botão “Novo”

\sphinxAtStartPar
Os únicos campos obrigatórios são o Nome e Pessoa (Física ou Jurídica), entretanto, quanto mais informações, melhor será a busca dentro da Base de dados do SAJ/PG5.
A pesquisa incluirá todos os processos com o mesmo nome da parte e mostrará quais dados estão coincidindo ou não para que seja feita a análise de que se trata da mesma pessoa, ou caso de homônimo.

\sphinxAtStartPar
Após preencher os campos clique em “Salvar”. Aguarde alguns instantes e clique após em “Restaurar”. Caso nenhum dos botões “Analisar” ou “Emitir” fique ativo, aguarde mais alguns instantes e clique novamente em “Restaurar”.


\begin{sphinxseealso}{Ver também:}

\sphinxAtStartPar
Saiba mais sobre como gerar certidões e realizar remessas:

\sphinxAtStartPar
Expedição de certidões e remessas: {\hyperref[\detokenize{projud_35_enviarconcluso::doc}]{\sphinxcrossref{\DUrole{doc}{Enviar Concluso}}}}


\end{sphinxseealso}



\subsection{Autos conclusos}
\label{\detokenize{01processos_entrados:autos-conclusos}}
\sphinxAtStartPar
Superada todas as pendências necessárias para subsidiar a decisão do magistrado, o servidor deverá fazer conclusão dos autos, mediante certidão de conclusão, informando ainda quaisquer pendências que verificou estarem pendentes de saneamento. Nessa hora, após o documento ser lançado nos autos, deve\sphinxhyphen{}se mover o processo para a fila “Conclusos \sphinxhyphen{} Decisão Interlocutória”.


\begin{sphinxseealso}{Ver também:}

\sphinxAtStartPar
Saiba mais sobre como concluir um processo no sistema:

\sphinxAtStartPar
Enviar concluso ao magistrado: {\hyperref[\detokenize{projud_35_enviarconcluso::doc}]{\sphinxcrossref{\DUrole{doc}{Enviar Concluso}}}}

\sphinxAtStartPar
Utilização de agrupadores para organização: {\hyperref[\detokenize{projud_51_agrupador::doc}]{\sphinxcrossref{\DUrole{doc}{Agrupador}}}}


\end{sphinxseealso}


\sphinxstepscope


\section{Análise da denúncia (Jurídico)}
\label{\detokenize{02analisedenuncia:analise-da-denuncia-juridico}}\label{\detokenize{02analisedenuncia::doc}}

\subsection{Introdução}
\label{\detokenize{02analisedenuncia:introducao}}
\sphinxAtStartPar
O recebimento da denúncia consiste numa etapa processual que congrega inúmeras determinações e atividades imprescindíveis para o regular desenvolvimento judicial e cartorário da ação penal proposta.
Por isso, trata\sphinxhyphen{}se de momento que exige a atenção do magistrado e da magistrada, presidentes do feito, bem como da secretaria da Vara.


\subsection{Denúncia rejeitada}
\label{\detokenize{02analisedenuncia:denuncia-rejeitada}}
\sphinxAtStartPar
Oferecida a denúncia, deve a secretaria da Vara lançar esse evento no histórico de partes.
Em seguida, os autos devem ser encaminhados conclusos ao Juiz.
Caso a denúncia seja rejeitada, a decisão é publicada e a secretaria deve lançar no histórico de partes o evento: rejeição da denúncia.
Dever\sphinxhyphen{}se\sphinxhyphen{}á aguardar o decurso do prazo recursal.
Caso seja interposto recurso em sentido estrito, cujo processamento poderá ocorrer nos próprios autos ou em autos apartados, dever\sphinxhyphen{}se\sphinxhyphen{}á tornar os autos conclusos para decisão judicial.


\begin{sphinxseealso}{Ver também:}

\sphinxAtStartPar
Saiba mais sobre:

\sphinxAtStartPar
Cadastro e atualização do histórico de partes: {\hyperref[\detokenize{projud_23_cadastroparte::doc}]{\sphinxcrossref{\DUrole{doc}{Cadastro de Parte e CPF/CNPJ}}}}

\sphinxAtStartPar
Como realizar conclusões: {\hyperref[\detokenize{projud_35_enviarconcluso::doc}]{\sphinxcrossref{\DUrole{doc}{Enviar Concluso}}}}

\sphinxAtStartPar
Cadastro de movimentações: {\hyperref[\detokenize{projud_11_telainicialprocesso::doc}]{\sphinxcrossref{\DUrole{doc}{Tela Inicial do Processo}}}}


\end{sphinxseealso}



\subsection{Denúncia recebida}
\label{\detokenize{02analisedenuncia:denuncia-recebida}}
\sphinxAtStartPar
Oferecida a denúncia, deve a secretaria da Vara lançar esse evento no histórico de partes.
Em seguida, os autos devem ser encaminhados conclusos ao Juiz.
Caso a denúncia seja recebida, a decisão é publicada e a secretaria deve lançar no histórico de partes o evento: recebimento da denúncia.
Sugere\sphinxhyphen{}se ao magistrado ou magistrada que, nesse momento em que tenha recebido a denúncia, possa concentrar o máximo de deliberações possíveis na decisão de recebimento justamente para impedir que os autos fiquem em constantes e infinitas tramitações entre a secretaria e o gabinete do Juiz, o que, certamente, atrasa o andamento processual.
Portanto, é relevante que se possa concentrar o máximo de deliberações a serem cumpridas pela secretaria da Vara.
Dentre as deliberações judiciais e as atividades cartorárias a serem cumpridas, elencamos abaixo as seguintes:


\begin{sphinxseealso}{Ver também:}

\sphinxAtStartPar
Saiba mais sobre:

\sphinxAtStartPar
Conclusão com agrupador para recebimento da denúncia: {\hyperref[\detokenize{projud_51_agrupador::doc}]{\sphinxcrossref{\DUrole{doc}{Agrupador}}}}

\sphinxAtStartPar
Cadastro de denúncia e cálculo da prescrição: {\hyperref[\detokenize{projud_53_cadastrodenuncia::doc}]{\sphinxcrossref{\DUrole{doc}{Cadastro da Denúncia ou Queixa para fins da Prescrição}}}}

\sphinxAtStartPar
Evolução de classe processual: {\hyperref[\detokenize{projud_19_evolucao_retificacao::doc}]{\sphinxcrossref{\DUrole{doc}{Evolução e Retificação de Classe Processual e Prioridade}}}}

\sphinxAtStartPar
Cadastro de infrações e penas: {\hyperref[\detokenize{projud_52_cadastroinfracoes::doc}]{\sphinxcrossref{\DUrole{doc}{Cadastro de Infrações e Penas para fins de Prescrição}}}}


\end{sphinxseealso}



\subsubsection{Citação}
\label{\detokenize{02analisedenuncia:citacao}}
\sphinxAtStartPar
Com o recebimento da denúncia e a ordem de citação, a secretaria deverá empreender os seguintes esforços abaixo:


\begin{sphinxseealso}{Ver também:}

\sphinxAtStartPar
Saiba mais sobre:

\sphinxAtStartPar
Citação de réu preso ou solto via mandado: {\hyperref[\detokenize{projud_32_expedicaomandado::doc}]{\sphinxcrossref{\DUrole{doc}{Expedição de Mandado para Oficial de Justiça}}}}

\sphinxAtStartPar
Controle de cumprimento de mandado: {\hyperref[\detokenize{projud_59_juntadahabeascorpus::doc}]{\sphinxcrossref{\DUrole{doc}{Juntada de Habeas Corpus}}}}

\sphinxAtStartPar
Cadastro de citação por edital: {\hyperref[\detokenize{projud_44_cartaprecatoriaeletronica::doc}]{\sphinxcrossref{\DUrole{doc}{Carta Precatória Eletrônica}}}} e {\hyperref[\detokenize{projud_33_ordenarcumprimento::doc}]{\sphinxcrossref{\DUrole{doc}{Ordenar Cumprimentos}}}}


\end{sphinxseealso}

\begin{itemize}
\item {} 
\sphinxAtStartPar
Mandado de citação de réu ou ré presos
\begin{quote}

\sphinxAtStartPar
A secretaria deve, num primeiro momento, identificar se o réu ou ré encontra\sphinxhyphen{}se preso em alguma unidade prisional do Estado do Amazonas.
Caso esteja preso no sistema prisional do Estado, deve confeccionar o mandado de citação e, juntamente com a denúncia, encaminhar à Unidade Prisional, via malote digital.
É importante que o próprio servidor possa controlar o prazo de retorno do cumprimento da diligência. 05 (cinco) dias é tempo razoável para o cumprimento.
\end{quote}

\item {} 
\sphinxAtStartPar
Mandado de citação de réu ou ré soltos
\begin{quote}

\sphinxAtStartPar
A secretaria, ao se deparar com réu ou ré soltos, deve expedir mandado de citação para o endereço registrado na denúncia.
Necessário controlar o prazo de cumprimento do mandado. Caso exceda os 30 (trinta) dias, deve expedir ofício à Central de Mandados para cobrar o retorno da diligência pelo oficial de justiça.
Ao retornar o mandado de citação, deve o servidor verificar o teor da certidão do oficial de justiça, caso em que deve observar as seguintes diretrizes:
\begin{itemize}
\item {} 
\sphinxAtStartPar
\sphinxstylestrong{certidão positiva}: caso o réu ou ré tenha sido citado, pessoalmente ou mesmo por hora certa, deve a secretaria da Vara aguardar o decurso do prazo de 10 (dez) dias para a constituição de advogado e oferecimento da resposta escrita. Decorrido o prazo, sem constituição de advogado e apresentação de resposta escrita, deve a secretaria cadastrar a Defensoria Pública no cadastro de partes e representantes e, mediante ato ordinatório, intimar a Defensoria Pública para representar o réu ou ré e apresentar a resposta escrita, desde que tenha ordem judicial previamente proferida.

\item {} 
\sphinxAtStartPar
\sphinxstylestrong{certidão negativa}: caso o réu ou ré NÃO tenha sido citado pessoalmente, ou seja, a diligência tenha sido frustrada, deve a secretaria emitir ato ordinatório para notificar o Ministério Público para tomar ciência da impossibilidade de citação pessoal ou por hora certa, no endereço que havia fornecido na denúncia.

\end{itemize}
\end{quote}

\item {} 
\sphinxAtStartPar
Carta precatória
\begin{quote}

\sphinxAtStartPar
Caso o réu ou ré esteja preso em outra comarca, deve a secretaria da Vara confeccionar carta precatória a ser assinada pelo Juiz e enviar em seguida ao juízo deprecado. Deve, em mais esta oportunidade, a secretaria da Vara controlar o prazo de retorno da carta precatória, outrora fixado pelo Juiz.
\end{quote}

\item {} 
\sphinxAtStartPar
Citação por edital
\begin{quote}

\sphinxAtStartPar
A citação por edital surge como hipótese aplicável no caso de o réu ou ré não ter sido citado pessoalmente ou por hora certa, nos termos do art.

\sphinxAtStartPar
Com o comando judicial, deve a secretaria da Vara confeccionar o edital contendo as exigências constantes do art. XXX
Esse edital lançado nos autos, será publicado no Diário de Justiça Eletrônico \sphinxhyphen{} DJE com o fim de ser levado ao conhecimento do réu ou ré que não fora localizado para ser citado.
Esse edital possui o prazo de 15 (quinze) dias. Com isso, uma vez emitido nos autos e publicado, deve a secretaria da Vara controlar o decurso do prazo de 15 (quinze) dias. Uma vez decorrido o prazo, iniciar\sphinxhyphen{}se\sphinxhyphen{}á o prazo de 10 (dez) dias para o réu ou ré constituir advogado e apresentar sua resposta escrita. Caso decorra o prazo in albis, a secretaria da Vara deve tornar os autos conclusos ao Juiz ou já proceder a iniciativa de suspender o processo e o curso do prazo prescricional, nos termos do art. 366 do CPP, desde que tenha ordem judicial nesse sentido.
\end{quote}

\end{itemize}


\begin{sphinxseealso}{Ver também:}

\sphinxAtStartPar
Saiba mais sobre:

\sphinxAtStartPar
Cadastro de edital de citação no Projudi: {\hyperref[\detokenize{projud_33_ordenarcumprimento::doc}]{\sphinxcrossref{\DUrole{doc}{Ordenar Cumprimentos}}}}


\end{sphinxseealso}

\begin{itemize}
\item {} 
\sphinxAtStartPar
Evolução de Classe
\begin{quote}

\sphinxAtStartPar
Com o recebimento da denúncia, deve a secretaria da Vara proceder à evolução de classe e o devido preenchimento do histórico de partes. Essa atividade é estritamente fundamental para o controle estatístico de processos e definição de estratégias futuras pelo Tribunal de Justiça.
A evolução de classe consiste na atividade cartorária de lançar no sistema SAJ a informação de que o outrora inquérito policial ou peças de informações passaram a ser tratados como ação penal, haja vista o juízo de admissibilidade proferido pelo Juiz com a decisão de recebimento da denúncia.
\end{quote}

\end{itemize}


\begin{sphinxseealso}{Ver também:}

\sphinxAtStartPar
Saiba mais sobre:

\sphinxAtStartPar
Como evoluir a classe processual: {\hyperref[\detokenize{projud_19_evolucao_retificacao::doc}]{\sphinxcrossref{\DUrole{doc}{Evolução e Retificação de Classe Processual e Prioridade}}}}


\end{sphinxseealso}


\sphinxstepscope


\section{Após recebimento da denúncia}
\label{\detokenize{03recebimento_denuncia:apos-recebimento-da-denuncia}}\label{\detokenize{03recebimento_denuncia::doc}}

\subsection{Introdução}
\label{\detokenize{03recebimento_denuncia:introducao}}
\sphinxAtStartPar
Após recebida a denúncia dá\sphinxhyphen{}se início ao processo penal, a partir desse momento a Vara Criminal deve proceder a evolução da classe processual, atualização do histórico de partes e efetuar os atos de citação da parte.


\subsection{Checklist}
\label{\detokenize{03recebimento_denuncia:checklist}}
\sphinxAtStartPar
Nesta fase processual a Vara deverá realizar as seguintes diligências.
\begin{itemize}
\item {} 
\sphinxAtStartPar
Realizar a Evolução de Classe

\item {} 
\sphinxAtStartPar
Atualizar o Histórico de Partes

\item {} 
\sphinxAtStartPar
Atualizar o Cadastro de Partes

\item {} 
\sphinxAtStartPar
Emitir Mandados de Citação

\item {} \begin{description}
\sphinxlineitem{Analisar o cumprimento do Mandado de Citação}\begin{itemize}
\item {} 
\sphinxAtStartPar
Consultar sistemas (mandado negativo)

\item {} 
\sphinxAtStartPar
Emitir Edital de Citação (caso frustradas todas as tentativas de citação)

\end{itemize}

\end{description}

\item {} 
\sphinxAtStartPar
Analisar prazo de apresentação de resposta à acusação

\item {} 
\sphinxAtStartPar
Vista a Defensoria (caso não apresentada resposta à acusação)

\end{itemize}


\begin{sphinxseealso}{Ver também:}

\sphinxAtStartPar
Saiba mais sobre:

\sphinxAtStartPar
Como realizar a evolução de classe: {\hyperref[\detokenize{projud_19_evolucao_retificacao::doc}]{\sphinxcrossref{\DUrole{doc}{Evolução e Retificação de Classe Processual e Prioridade}}}}

\sphinxAtStartPar
Atualização de cadastro e histórico de partes: {\hyperref[\detokenize{projud_23_cadastroparte::doc}]{\sphinxcrossref{\DUrole{doc}{Cadastro de Parte e CPF/CNPJ}}}}

\sphinxAtStartPar
Cadastro de partes com citação online: {\hyperref[\detokenize{projud_24_cadastropartecitacaoonline::doc}]{\sphinxcrossref{\DUrole{doc}{Cadastro de Parte com Citação Online}}}}

\sphinxAtStartPar
Emissão e expedição de mandado de citação: {\hyperref[\detokenize{projud_32_expedicaomandado::doc}]{\sphinxcrossref{\DUrole{doc}{Expedição de Mandado para Oficial de Justiça}}}}

\sphinxAtStartPar
Como analisar o cumprimento do mandado e emitir edital: {\hyperref[\detokenize{projud_33_ordenarcumprimento::doc}]{\sphinxcrossref{\DUrole{doc}{Ordenar Cumprimentos}}}}

\sphinxAtStartPar
Citação por edital: {\hyperref[\detokenize{projud_44_cartaprecatoriaeletronica::doc}]{\sphinxcrossref{\DUrole{doc}{Carta Precatória Eletrônica}}}}

\sphinxAtStartPar
Contagem e análise de prazo: {\hyperref[\detokenize{projud_31_contagemprazo::doc}]{\sphinxcrossref{\DUrole{doc}{Contagem do Prazo nas Intimações e Citações}}}}

\sphinxAtStartPar
Vista à Defensoria e intimações: {\hyperref[\detokenize{projud_34_intimarperitooj::doc}]{\sphinxcrossref{\DUrole{doc}{Intimar Perito e Oficial de Justiça}}}}


\end{sphinxseealso}



\subsection{Evolução de Classe}
\label{\detokenize{03recebimento_denuncia:evolucao-de-classe}}
\sphinxAtStartPar
Após o recebimento da denúncia os autos deixam de ser um mero procedimento (inquérito policial, TCO e etc) e passa a ser uma ação penal, devendo contar para o acervo da Vara e em todos os parâmetros de produtividade e metas estabelecidas.
As varas devem se atentar para a obrigatoriedade deste procedimento de evolução de classe, conforme estabelecido no provimento XXXX e recomendação XXXX da Coordenação das Varas Criminais.


\begin{sphinxseealso}{Ver também:}

\sphinxAtStartPar
Saiba mais sobre:

\sphinxAtStartPar
Evolução de classe após o recebimento da denúncia: {\hyperref[\detokenize{projud_19_evolucao_retificacao::doc}]{\sphinxcrossref{\DUrole{doc}{Evolução e Retificação de Classe Processual e Prioridade}}}}

\sphinxAtStartPar
Cadastro da denúncia para fins de cálculo de prescrição: {\hyperref[\detokenize{projud_53_cadastrodenuncia::doc}]{\sphinxcrossref{\DUrole{doc}{Cadastro da Denúncia ou Queixa para fins da Prescrição}}}}


\end{sphinxseealso}


\sphinxstepscope


\section{Análise da Defesa Escrita}
\label{\detokenize{04analiseDefesa_pautaraij:analise-da-defesa-escrita}}\label{\detokenize{04analiseDefesa_pautaraij:analise-defesa}}\label{\detokenize{04analiseDefesa_pautaraij::doc}}
\sphinxAtStartPar
Se recebida a denúncia, os autos seguem para a Defesa apresentar a \sphinxstylestrong{Resposta à Acusação} ou \sphinxstylestrong{Defesa Preliminar}.


\subsection{Prazo para Resposta à Acusação}
\label{\detokenize{04analiseDefesa_pautaraij:prazo-para-resposta-a-acusacao}}\label{\detokenize{04analiseDefesa_pautaraij:prazo-resposta}}
\sphinxAtStartPar
Conforme o art. 396 do CPP%
\begin{footnote}[1]\sphinxAtStartFootnote
\sphinxstylestrong{Art. 396 do CPP} \textendash{} “Nos procedimentos ordinário e sumário, oferecida a denúncia ou queixa, o juiz, se não a rejeitar liminarmente, recebê\sphinxhyphen{}la\sphinxhyphen{}á e ordenará a citação do acusado para responder à acusação, por escrito, no prazo de 10 (dez) dias.”
%
\end{footnote}, o réu citado terá o prazo de \sphinxstylestrong{10 (dez) dias} para responder à acusação. A contagem deverá observar a \sphinxstylestrong{Súmula 710 do STF}%
\begin{footnote}[2]\sphinxAtStartFootnote
\sphinxstylestrong{Súmula 710\sphinxhyphen{}STF} \textendash{} “No processo penal, contam\sphinxhyphen{}se os prazos da data da intimação, e não da juntada aos autos do mandado ou da carta precatória ou de ordem.”
%
\end{footnote}, iniciando\sphinxhyphen{}se a partir da data da citação.

\sphinxAtStartPar
Na prática forense, recomenda\sphinxhyphen{}se que nos mandados de citação (liberdade ou presos), o agente certifique se o réu possui condições de constituir advogado particular:
\begin{itemize}
\item {} 
\sphinxAtStartPar
Se \sphinxstylestrong{afirmar que possui}, a Secretaria aguarda o prazo legal e, se não houver defesa, \sphinxstylestrong{nomeia a Defensoria Pública} (art. 396\sphinxhyphen{}A, \S{}2º do CPP)%
\begin{footnote}[3]\sphinxAtStartFootnote
\sphinxstylestrong{Art. 396\sphinxhyphen{}A, \S{}2º do CPP} \textendash{} “Não apresentada a resposta no prazo legal, ou se o acusado, citado, não constituir defensor, o juiz nomeará defensor para oferecê\sphinxhyphen{}la, concedendo\sphinxhyphen{}lhe vista dos autos por 10 (dez) dias.”
%
\end{footnote}.

\item {} 
\sphinxAtStartPar
Se \sphinxstylestrong{afirmar que não possui}, a Secretaria \sphinxstylestrong{já nomeia a Defensoria Pública}, otimizando o trâmite.

\item {} 
\sphinxAtStartPar
A Defensoria tem \sphinxstylestrong{prazo em dobro} para manifestações (art. 128, I, LC nº 80/94)%
\begin{footnote}[4]\sphinxAtStartFootnote
\sphinxstylestrong{Art. 128, I, da LC nº 80/94} \textendash{} “São prerrogativas dos membros da Defensoria Pública do Estado: I \textendash{} receber, inclusive quando necessário, mediante entrega dos autos com vista, intimação pessoal em qualquer processo e grau de jurisdição ou instância administrativa, contando\sphinxhyphen{}se\sphinxhyphen{}lhes em dobro todos os prazos.”
%
\end{footnote}.

\end{itemize}


\begin{savenotes}\sphinxattablestart
\sphinxthistablewithglobalstyle
\centering
\begin{tabulary}{\linewidth}[t]{TTT}
\sphinxtoprule
\sphinxstyletheadfamily 
\sphinxAtStartPar
Situação
&\sphinxstyletheadfamily 
\sphinxAtStartPar
Conduta Recomendada
&\sphinxstyletheadfamily 
\sphinxAtStartPar
Consequência para o Procedimento
\\
\sphinxmidrule
\sphinxtableatstartofbodyhook
\sphinxAtStartPar
Réu com advogado ou silêncio
&
\sphinxAtStartPar
Secretaria aguarda o prazo do art. 396\sphinxhyphen{}A do CPP
&
\sphinxAtStartPar
Após o prazo, nomeia\sphinxhyphen{}se a DPE se não houver defesa
\\
\sphinxhline
\sphinxAtStartPar
Réu afirma não ter condições financeiras
&
\sphinxAtStartPar
Certificação nos autos e nomeação imediata da DPE
&
\sphinxAtStartPar
Evita controle de prazo e agiliza tramitação
\\
\sphinxhline
\sphinxAtStartPar
Nomeação da Defensoria Pública
&
\sphinxAtStartPar
Aplica\sphinxhyphen{}se prazo em dobro automaticamente
&
\sphinxAtStartPar
Sem necessidade de pedido específico
\\
\sphinxbottomrule
\end{tabulary}
\sphinxtableafterendhook\par
\sphinxattableend\end{savenotes}


\subsection{Análise da Resposta à Acusação}
\label{\detokenize{04analiseDefesa_pautaraij:analise-da-resposta-a-acusacao}}\label{\detokenize{04analiseDefesa_pautaraij:analise-resposta}}
\sphinxAtStartPar
A Resposta à Acusação tem natureza \sphinxstylestrong{mista}, permitindo à Defesa:
\begin{itemize}
\item {} 
\sphinxAtStartPar
Alegar nulidades

\item {} 
\sphinxAtStartPar
Prejudiciais de mérito

\item {} 
\sphinxAtStartPar
Ausência de pressupostos processuais ou condições da ação

\item {} 
\sphinxAtStartPar
Ou ainda, antecipar teses de mérito

\end{itemize}

\sphinxAtStartPar
Duas situações práticas surgem com frequência:

\sphinxAtStartPar
\sphinxstylestrong{1. Defesa apresenta preliminares ou mérito:}
\begin{itemize}
\item {} 
\sphinxAtStartPar
Deve\sphinxhyphen{}se abrir \sphinxstylestrong{vista ao Ministério Público} por 10 dias, assegurando contraditório e paridade de armas. Tal prática, embora não prevista expressamente na legislação, tem respaldo na jurisprudência do STJ%
\begin{footnote}[5]\sphinxAtStartFootnote
\sphinxstylestrong{STJ \sphinxhyphen{} AgRg no RHC 124304/SP} \textendash{} “A manifestação acusatória após a defesa inicial, embora não prevista em lei, vem justamente a atender ao princípio do contraditório…”. (Rel. Min. Nefi Cordeiro, DJe 15/05/2020)
%
\end{footnote}, como forma de evitar cerceamento de defesa.

\item {} 
\sphinxAtStartPar
Após manifestação (ou silêncio), o Juiz analisa a hipótese de \sphinxstylestrong{absolvição sumária} conforme art. 397 do CPP.

\item {} 
\sphinxAtStartPar
Também pode avaliar outras questões preliminares: \sphinxstylestrong{conexão, continência, litispendência, prescrição, coisa julgada ou nulidades}.

\end{itemize}

\sphinxAtStartPar
\sphinxstylestrong{2. Defesa silencia sobre questões preliminares:}
\begin{itemize}
\item {} 
\sphinxAtStartPar
Deve\sphinxhyphen{}se \sphinxstylestrong{certificar} nos autos que não houve arguição relevante.

\item {} 
\sphinxAtStartPar
Em seguida, o feito é incluído em pauta para a \sphinxstylestrong{audiência de instrução e julgamento}, respeitando as prioridades legais e normativas do CNJ.

\end{itemize}


\begin{savenotes}\sphinxattablestart
\sphinxthistablewithglobalstyle
\centering
\begin{tabulary}{\linewidth}[t]{TTT}
\sphinxtoprule
\sphinxstyletheadfamily 
\sphinxAtStartPar
Situação
&\sphinxstyletheadfamily 
\sphinxAtStartPar
Conduta Recomendada
&\sphinxstyletheadfamily 
\sphinxAtStartPar
Consequência para o Procedimento
\\
\sphinxmidrule
\sphinxtableatstartofbodyhook
\sphinxAtStartPar
Defesa com preliminares, prejudiciais ou mérito
&
\sphinxAtStartPar
Vista ao MP por 10 dias (contraditório)
&
\sphinxAtStartPar
Após manifestação ou silêncio, análise judicial (art. 397)
\\
\sphinxhline
\sphinxAtStartPar
Defesa sem preliminares ou alegações
&
\sphinxAtStartPar
Certificar e pautar audiência de instrução e julgamento
&
\sphinxAtStartPar
Processo segue sem necessidade de decisão intermediária
\\
\sphinxbottomrule
\end{tabulary}
\sphinxtableafterendhook\par
\sphinxattableend\end{savenotes}


\begin{sphinxseealso}{Ver também:}
\begin{itemize}
\item {} 
\sphinxAtStartPar
\DUrole{xref,std,std-ref}{recebimento\_denuncia}

\item {} 
\sphinxAtStartPar
\DUrole{xref,std,std-ref}{pautar\_aij}

\item {} 
\sphinxAtStartPar
\DUrole{xref,std,std-ref}{sentenca\_final}

\item {} 
\sphinxAtStartPar
\DUrole{xref,std,std-ref}{certificar\_transito}

\end{itemize}


\end{sphinxseealso}


\sphinxstepscope


\section{Pautar AIJ}
\label{\detokenize{05pautaraij:pautar-aij}}\label{\detokenize{05pautaraij::doc}}
\sphinxAtStartPar
Apesar de parecer algo rotineiro, organizar uma pauta eficiente é algo que demanda lógica e estratégia.


\subsection{05.1 \textendash{} Do Prazo}
\label{\detokenize{05pautaraij:do-prazo}}
\sphinxAtStartPar
Segundo o art. 400 do CPP, a audiência de instrução e julgamento deverá ocorrer em até \sphinxstylestrong{60 (sessenta) dias} após a decisão que confirma o recebimento da denúncia e afasta as hipóteses de absolvição sumária ou reconhecimento de causas prejudiciais (ex: prescrição, decadência).

\begin{sphinxadmonition}{note}{Nota:}
\sphinxAtStartPar
Apesar do teor do art. 399 indicar que “recebida a denúncia ou queixa, o juiz designará dia e hora para a audiência (…)”, esse comando deve ser interpretado de forma sistemática com os demais dispositivos do CPP.
Isso porque o rito prevê antes a resposta à acusação (art. 396), a análise das preliminares e prejudiciais (art. 397) e, só então, a confirmação do recebimento da denúncia.
Assim, não faz sentido lógico ou prático pautar AIJ desde o recebimento da denúncia inicial.
\end{sphinxadmonition}


\subsection{05.2 \textendash{} Da organização da Pauta}
\label{\detokenize{05pautaraij:da-organizacao-da-pauta}}
\sphinxAtStartPar
A organização da pauta de audiências deve observar sempre as prioridades legais e normativas do CNJ. As mais comuns são:
\begin{enumerate}
\sphinxsetlistlabels{\arabic}{enumi}{enumii}{}{.}%
\item {} 
\sphinxAtStartPar
Réus presos provisoriamente pela unidade jurisdicional.

\item {} 
\sphinxAtStartPar
Crimes hediondos (art. 394\sphinxhyphen{}A do CPP).

\item {} 
\sphinxAtStartPar
Violência contra a mulher (art. 394\sphinxhyphen{}A do CPP e art. 1.048, II do CPC).

\item {} 
\sphinxAtStartPar
Crimes contra idosos ou quando o réu for idoso (art. 71 do Estatuto do Idoso \textendash{} Lei nº 10.741/2003; Res. nº 14/2007 CNJ; art. 1.048, I do CPC).

\item {} 
\sphinxAtStartPar
Portadores de doenças graves (art. 1.048, I do CPC).

\item {} 
\sphinxAtStartPar
Pessoas com deficiência (art. 9º, VII da Lei nº 13.146/2015).

\item {} 
\sphinxAtStartPar
Crianças ou adolescentes como vítimas ou testemunhas (art. 4º do ECA; art. 227 da CF/88; Lei nº 13.431/2017; art. 1.048, II do CPC).

\item {} 
\sphinxAtStartPar
Vítimas ou testemunhas ameaçadas protegidas pelo PROVITA/PPCAM (art. 19\sphinxhyphen{}A da Lei nº 9.807/1999; Recomendação nº 07 do CNJ).

\item {} 
\sphinxAtStartPar
Processos incluídos em metas do CNJ (Meta 02, Meta 06, Meta 07) e processos antigos.

\end{enumerate}

\begin{sphinxadmonition}{warning}{Aviso:}
\sphinxAtStartPar
Sabe\sphinxhyphen{}se que atualmente é quase impossível realizar AIJ dentro do prazo legal do art. 400 do CPP.
\sphinxstylestrong{Recomenda\sphinxhyphen{}se pautar imediatamente a audiência}, ainda que em prazo longo, para evitar responsabilização da Secretaria/Direção.
\end{sphinxadmonition}

\sphinxAtStartPar
\sphinxstylestrong{Recomendações práticas:}
\begin{itemize}
\item {} 
\sphinxAtStartPar
Deixar espaços vazios na pauta para encaixe de processos posteriores que possuam prioridade legal ou normativa.

\item {} 
\sphinxAtStartPar
Observar a quantidade de partes a serem ouvidas (ofendidos, testemunhas, réus), evitando fracionamentos desnecessários.

\item {} 
\sphinxAtStartPar
Como o Projudi emite pautas em formato corrido, recomenda\sphinxhyphen{}se usar planilhas (Google Sheets) para gerir melhor a pauta, reservando espaços para casos prioritários.

\item {} 
\sphinxAtStartPar
Evitar pautar audiências em datas de feriados, pontos facultativos ou recessos definidos pela Corte Estadual.

\end{itemize}


\subsection{Referências e Observações}
\label{\detokenize{05pautaraij:referencias-e-observacoes}}\begin{enumerate}
\sphinxsetlistlabels{\arabic}{enumi}{enumii}{}{.}%
\item {} 
\sphinxAtStartPar
\sphinxstylestrong{Meta 08/CNJ (Justiça Estadual):} Julgar 75\% dos casos de feminicídio distribuídos até 31/12/2023 e 90\% dos casos de violência doméstica e familiar até 31/12/2023.

\item {} 
\sphinxAtStartPar
\sphinxstylestrong{Idosos:} O Estatuto da Pessoa Idosa considera idoso quem tem 60+ anos. Aos 80+, aplica\sphinxhyphen{}se prioridade especial (super prioridade).

\item {} 
\sphinxAtStartPar
\sphinxstylestrong{Meta 02/CNJ:} Julgar pelo menos 80\% dos processos distribuídos até 31/12/2021 no 1º grau e 100\% dos processos com mais de 15 anos (2010 ou anteriores).

\item {} 
\sphinxAtStartPar
\sphinxstylestrong{Meta 06/CNJ:} Julgar 50\% das ações ambientais distribuídas até 31/12/2024.

\item {} 
\sphinxAtStartPar
\sphinxstylestrong{Meta 07/CNJ:} Julgar 50\% dos processos envolvendo direitos das comunidades indígenas e quilombolas até 31/12/2024.

\item {} 
\sphinxAtStartPar
\sphinxstylestrong{Prêmio CNJ de Qualidade (Portaria nº 411/2024):} Processos antigos são os distribuídos até 2022; a unidade deve manter menos de 20\% do acervo nessa condição.

\end{enumerate}

\sphinxstepscope


\section{Atos Intimatórios}
\label{\detokenize{06atosintimatorios:atos-intimatorios}}\label{\detokenize{06atosintimatorios::doc}}

\subsection{Objetivos}
\label{\detokenize{06atosintimatorios:objetivos}}\begin{itemize}
\item {} 
\sphinxAtStartPar
Garantir a uniformidade e eficiência na execução dos atos intimatórios.

\item {} 
\sphinxAtStartPar
Assegurar o cumprimento dos prazos processuais.

\item {} 
\sphinxAtStartPar
Facilitar a compreensão e execução das atividades por parte dos servidores do Judiciário.

\end{itemize}


\subsection{Procedimentos Gerais}
\label{\detokenize{06atosintimatorios:procedimentos-gerais}}

\subsubsection{Definição de Atos Intimatórios}
\label{\detokenize{06atosintimatorios:definicao-de-atos-intimatorios}}
\sphinxAtStartPar
Atos intimatórios são aqueles destinados a comunicar às partes ou a outros interessados os atos e termos do processo, bem como a convocação para a prática de atos processuais.


\subsubsection{Expedição de Cartas Precatórias}
\label{\detokenize{06atosintimatorios:expedicao-de-cartas-precatorias}}
\sphinxAtStartPar
Para a realização de citações e intimações fora da jurisdição do juízo de origem, expede\sphinxhyphen{}se uma carta precatória, que deve conter:
\begin{itemize}
\item {} 
\sphinxAtStartPar
Identificação do juízo deprecante e do juízo deprecado.

\item {} 
\sphinxAtStartPar
Finalidade específica da citação ou intimação.

\item {} 
\sphinxAtStartPar
Data, hora e local onde o destinatário deve comparecer, se aplicável.

\item {} 
\sphinxAtStartPar
Anexos pertinentes, como cópia da denúncia ou da decisão a ser intimada.

\end{itemize}


\begin{sphinxseealso}{Ver também:}

\sphinxAtStartPar
Saiba mais sobre:

\sphinxAtStartPar
Como confeccionar cartas precatórias eletrônicas no Projudi: {\hyperref[\detokenize{projud_44_cartaprecatoriaeletronica::doc}]{\sphinxcrossref{\DUrole{doc}{Carta Precatória Eletrônica}}}}

\sphinxAtStartPar
Como ordenar cumprimento de carta precatória: {\hyperref[\detokenize{projud_33_ordenarcumprimento::doc}]{\sphinxcrossref{\DUrole{doc}{Ordenar Cumprimentos}}}}


\end{sphinxseealso}



\subsubsection{Atos Ordinatórios}
\label{\detokenize{06atosintimatorios:atos-ordinatorios}}
\sphinxAtStartPar
Os servidores podem praticar atos ordinatórios, que não têm caráter decisório e visam a garantir a celeridade processual. Entre eles, estão:
\begin{itemize}
\item {} 
\sphinxAtStartPar
Expedição de documentos e juntada aos autos.

\item {} 
\sphinxAtStartPar
Notificações, citações e intimações eletrônicas.

\item {} 
\sphinxAtStartPar
Encaminhamento de cartas precatórias à Central de Mandados para cumprimento.

\end{itemize}


\begin{sphinxseealso}{Ver também:}

\sphinxAtStartPar
Saiba mais sobre:

\sphinxAtStartPar
Como movimentar o processo com ato ordinatório: {\hyperref[\detokenize{projud_06_analisejuntada_multiplaeunitaria::doc}]{\sphinxcrossref{\DUrole{doc}{Análise Múltipla e Unitária de Juntadas}}}}

\sphinxAtStartPar
Expedição de intimações e notificações eletrônicas: {\hyperref[\detokenize{projud_32_expedicaomandado::doc}]{\sphinxcrossref{\DUrole{doc}{Expedição de Mandado para Oficial de Justiça}}}}

\sphinxAtStartPar
Intimação de perito e oficial de justiça: {\hyperref[\detokenize{projud_34_intimarperitooj::doc}]{\sphinxcrossref{\DUrole{doc}{Intimar Perito e Oficial de Justiça}}}}


\end{sphinxseealso}



\subsection{Procedimento de Intimação}
\label{\detokenize{06atosintimatorios:procedimento-de-intimacao}}

\subsubsection{Citação para Audiência de Conciliação e Transação Penal}
\label{\detokenize{06atosintimatorios:citacao-para-audiencia-de-conciliacao-e-transacao-penal}}
\sphinxAtStartPar
\sphinxstylestrong{Finalidade:}

\sphinxAtStartPar
A citação para audiência de conciliação e transação penal visa a informar ao acusado sobre a audiência marcada para tentativa de conciliação entre as partes e/ou proposta de transação penal pelo Ministério Público.

\sphinxAtStartPar
\sphinxstylestrong{Procedimento:}
\begin{itemize}
\item {} 
\sphinxAtStartPar
A citação deve ser pessoal, realizada por Oficial de Justiça ou por via postal com aviso de recebimento.

\item {} 
\sphinxAtStartPar
O mandado de citação deve conter a data, hora e local da audiência.

\item {} 
\sphinxAtStartPar
A ausência do acusado, devidamente citado, não impede a realização da audiência, podendo ser julgada à revelia.

\end{itemize}


\begin{sphinxseealso}{Ver também:}

\sphinxAtStartPar
Saiba mais sobre:

\sphinxAtStartPar
Como pautar e agendar audiências: {\hyperref[\detokenize{projud_45_listasaudiencias::doc}]{\sphinxcrossref{\DUrole{doc}{Listas de Audiências}}}}, {\hyperref[\detokenize{projud_46_comopautaraudiencia::doc}]{\sphinxcrossref{\DUrole{doc}{Como Pautar Audiência}}}}

\sphinxAtStartPar
Cadastro de medidas alternativas \sphinxhyphen{} Transação Penal: {\hyperref[\detokenize{projud_54_cadastromedidasalternativas::doc}]{\sphinxcrossref{\DUrole{doc}{Cadastro de Medidas Alternativas: Transação Penal}}}}


\end{sphinxseealso}



\subsubsection{Citação para Suspensão Condicional do Processo (SCP)}
\label{\detokenize{06atosintimatorios:citacao-para-suspensao-condicional-do-processo-scp}}
\sphinxAtStartPar
\sphinxstylestrong{Finalidade:}

\sphinxAtStartPar
Informar ao acusado sobre a proposta de suspensão condicional do processo, nos termos do artigo 89 da Lei 9.099/95.

\sphinxAtStartPar
\sphinxstylestrong{Procedimento:}
\begin{itemize}
\item {} 
\sphinxAtStartPar
A citação deve ser pessoal e preferencialmente realizada pelo Oficial de Justiça.

\item {} 
\sphinxAtStartPar
Deve ser informado o teor da proposta e os benefícios de aceitar a suspensão condicional.

\item {} 
\sphinxAtStartPar
Em caso de recusa, o processo seguirá o seu curso normal.

\end{itemize}


\begin{sphinxseealso}{Ver também:}

\sphinxAtStartPar
Saiba mais sobre:

\sphinxAtStartPar
Cadastro da suspensão condicional do processo: {\hyperref[\detokenize{projud_55_cadastrosuspensao::doc}]{\sphinxcrossref{\DUrole{doc}{Cadastro de Suspensões (Parte)}}}}


\end{sphinxseealso}



\subsubsection{Citação para Audiência de Instrução e Julgamento (AIJ)}
\label{\detokenize{06atosintimatorios:citacao-para-audiencia-de-instrucao-e-julgamento-aij}}
\sphinxAtStartPar
\sphinxstylestrong{Finalidade:}

\sphinxAtStartPar
Convocar o acusado para a audiência de instrução e julgamento, onde serão ouvidas as testemunhas, peritos e realizado o interrogatório do réu.

\sphinxAtStartPar
\sphinxstylestrong{Procedimento:}
\begin{itemize}
\item {} 
\sphinxAtStartPar
A citação deve ser pessoal, realizada por Oficial de Justiça, via postal com aviso de recebimento, ou por meio eletrônico, quando permitido.

\item {} 
\sphinxAtStartPar
O mandado deve conter data, hora e local da audiência, além de alertar o acusado sobre a necessidade de comparecimento sob pena de prosseguimento à revelia.

\item {} 
\sphinxAtStartPar
Em caso de não localização do réu, deverá ser feita a citação por edital, conforme o artigo 363, \S{}1º, do CPP.

\end{itemize}


\begin{sphinxseealso}{Ver também:}

\sphinxAtStartPar
Saiba mais sobre:

\sphinxAtStartPar
Agendamento de audiências: {\hyperref[\detokenize{projud_46_comopautaraudiencia::doc}]{\sphinxcrossref{\DUrole{doc}{Como Pautar Audiência}}}}

\sphinxAtStartPar
Movimentação de audiência e inserção de termo: {\hyperref[\detokenize{projud_47_movimentacaoaudiencia::doc}]{\sphinxcrossref{\DUrole{doc}{Movimentação de Audiência}}}}

\sphinxAtStartPar
Citação por edital e cumprimento de mandado: {\hyperref[\detokenize{projud_33_ordenarcumprimento::doc}]{\sphinxcrossref{\DUrole{doc}{Ordenar Cumprimentos}}}}


\end{sphinxseealso}



\subsubsection{Intimação de Partes e Testemunhas}
\label{\detokenize{06atosintimatorios:intimacao-de-partes-e-testemunhas}}
\sphinxAtStartPar
\sphinxstylestrong{Finalidade:}

\sphinxAtStartPar
Informar as partes e testemunhas sobre a data e hora das audiências, decisões, despachos, e demais atos processuais.

\sphinxAtStartPar
\sphinxstylestrong{Procedimento:}
\begin{itemize}
\item {} 
\sphinxAtStartPar
A intimação pode ser realizada pessoalmente, via postal com aviso de recebimento, por meio eletrônico, ou, em casos excepcionais, por edital.

\item {} 
\sphinxAtStartPar
Para testemunhas, pode ser requisitada a presença através de ofício ao órgão ou instituição onde se encontram vinculadas (ex.: Polícia Militar para testemunhas policiais).

\end{itemize}


\begin{sphinxseealso}{Ver também:}

\sphinxAtStartPar
Saiba mais sobre:

\sphinxAtStartPar
Como realizar intimações a partes e oficiais de justiça: {\hyperref[\detokenize{projud_34_intimarperitooj::doc}]{\sphinxcrossref{\DUrole{doc}{Intimar Perito e Oficial de Justiça}}}}

\sphinxAtStartPar
Expedição de mandado para oficial de justiça: {\hyperref[\detokenize{projud_32_expedicaomandado::doc}]{\sphinxcrossref{\DUrole{doc}{Expedição de Mandado para Oficial de Justiça}}}}


\end{sphinxseealso}



\subsection{Controle de Prazos}
\label{\detokenize{06atosintimatorios:controle-de-prazos}}
\sphinxAtStartPar
A secretaria deve manter um controle rigoroso dos prazos para a realização dos atos intimatórios, especialmente em relação a presos provisórios, conforme o artigo 316, parágrafo único, do Código de Processo Penal.

\sphinxAtStartPar
É fundamental certificar nos autos o término dos prazos e proceder ao registro no sistema eletrônico SAJ, para garantir a efetiva suspensão ou continuidade do andamento processual, conforme o caso.


\begin{sphinxseealso}{Ver também:}

\sphinxAtStartPar
Saiba mais sobre:

\sphinxAtStartPar
Contagem dos prazos processuais no Projudi: {\hyperref[\detokenize{projud_31_contagemprazo::doc}]{\sphinxcrossref{\DUrole{doc}{Contagem do Prazo nas Intimações e Citações}}}}, {\hyperref[\detokenize{projud_60_contagemprazosprocessuais::doc}]{\sphinxcrossref{\DUrole{doc}{Contagem dos Prazos Processuais}}}}

\sphinxAtStartPar
Suspensão e interrupção de prazos: {\hyperref[\detokenize{projud_38_interrupcaoprazo::doc}]{\sphinxcrossref{\DUrole{doc}{Interrupção do Prazo}}}}, {\hyperref[\detokenize{projud_39_suspensaosobrestamento::doc}]{\sphinxcrossref{\DUrole{doc}{Suspensão e Sobrestamento do Processo}}}}


\end{sphinxseealso}


\sphinxstepscope


\section{Audiência de Instrução e Julgamento}
\label{\detokenize{07termoAIJ:audiencia-de-instrucao-e-julgamento}}\label{\detokenize{07termoAIJ::doc}}
\sphinxAtStartPar
A audiência de instrução constitui ato processual relevante no curso da ação penal.
Trata\sphinxhyphen{}se da oportunidade conferida às partes para produzir suas provas, sobretudo as testemunhais.
O que é produzido na audiência deve ser registrado nos autos. Para isso, o Juiz, com o auxílio da secretaria da Vara, emite o termo de audiência, no qual são consignados os principais e relevantes eventos ocorridos.

\sphinxAtStartPar
Na data agendada, a secretaria da Vara deve realizar o pregão no horário designado para a audiência.
Uma vez realizado o pregão, a secretaria deve registrar os presentes e informar ao Juiz para a devida anotação no termo de audiência e para deliberações seguintes.

\sphinxAtStartPar
Registrada a presença da acusação, geralmente representada pelo Ministério Público, e da defesa, seja por advogado ou defensor público, o Juiz deve iniciar o ato processual com a oitiva da vítima, se possível, e das demais testemunhas.
Via de regra, são inquiridas primeiramente as testemunhas de acusação e, em seguida, as de defesa.
Após a inquirição das testemunhas, deve ser interrogado o réu ou a ré.


\subsection{Situações a serem observadas:}
\label{\detokenize{07termoAIJ:situacoes-a-serem-observadas}}

\subsubsection{Indicação de Endereços pelas Partes}
\label{\detokenize{07termoAIJ:indicacao-de-enderecos-pelas-partes}}
\sphinxAtStartPar
Designada a data para a audiência de instrução criminal, as partes (acusação e defesa) devem ser intimadas para indicarem, no prazo de 05 (cinco) dias, todos os endereços e telefones disponíveis das vítimas e testemunhas arroladas.
O não fornecimento dessas informações pode levar à preclusão e expedição dos atos intimatórios nos endereços cadastrados.
Caso a testemunha não compareça, recomenda\sphinxhyphen{}se ao magistrado(a) a declaração de perecimento do direito de produção daquela prova, conforme a Recomendação Conjunta nº 01, de 21 de março de 2024, art. 3º, ressalvadas as exceções do art. 2º da referida Recomendação.


\begin{sphinxseealso}{Ver também:}

\sphinxAtStartPar
Saiba mais sobre:

\sphinxAtStartPar
Intimação de partes e testemunhas: {\hyperref[\detokenize{projud_34_intimarperitooj::doc}]{\sphinxcrossref{\DUrole{doc}{Intimar Perito e Oficial de Justiça}}}}

\sphinxAtStartPar
Atos intimatórios: {\hyperref[\detokenize{projud_58_cadastrocumprimentomedida::doc}]{\sphinxcrossref{\DUrole{doc}{Cumprimentos de Medidas Alternativas}}}}


\end{sphinxseealso}



\subsubsection{Realização Parcial da Instrução Criminal e Redesignação}
\label{\detokenize{07termoAIJ:realizacao-parcial-da-instrucao-criminal-e-redesignacao}}
\sphinxAtStartPar
Caso apenas algumas testemunhas de acusação compareçam e a acusação insista na inquirição das ausentes, deverá pugnar pela redesignação da audiência.
Nesse caso, deve apresentar informações atualizadas sobre o endereço das testemunhas ou vítimas, acompanhadas da fonte das novas informações. (Recomendação Conjunta nº 01, de 21 de março de 2024, art. 2º).

\sphinxAtStartPar
Essas informações devem ser juntadas aos autos até 15 (quinze) dias antes da data pautada para a audiência, sob pena de preclusão da produção da referida prova (Recomendação Conjunta nº 01, de 21 de março de 2024, art. 2º, \S{}\S{}1º e 2º).


\begin{sphinxseealso}{Ver também:}

\sphinxAtStartPar
Saiba mais sobre:

\sphinxAtStartPar
Movimentação de audiência, cancelamento e redesignação: {\hyperref[\detokenize{projud_47_movimentacaoaudiencia::doc}]{\sphinxcrossref{\DUrole{doc}{Movimentação de Audiência}}}}


\end{sphinxseealso}



\subsubsection{Encerramento da Instrução Criminal}
\label{\detokenize{07termoAIJ:encerramento-da-instrucao-criminal}}
\sphinxAtStartPar
Concluídas a oitiva da vítima (se existente nos autos), a inquirição das testemunhas de acusação e defesa, e o interrogatório do réu ou ré, as partes têm a oportunidade de apresentar requerimentos finais.
Podem solicitar a realização de diligências, desde que sua necessidade decorra de circunstâncias ou fatos apurados na instrução. (CPP, art. 402).


\begin{sphinxseealso}{Ver também:}

\sphinxAtStartPar
Saiba mais sobre:

\sphinxAtStartPar
Inserção de termo de audiência e registro da instrução: {\hyperref[\detokenize{projud_47_movimentacaoaudiencia::doc}]{\sphinxcrossref{\DUrole{doc}{Movimentação de Audiência}}}}


\end{sphinxseealso}



\subsubsection{Fase de Alegações Finais}
\label{\detokenize{07termoAIJ:fase-de-alegacoes-finais}}
\sphinxAtStartPar
Encerrada a instrução criminal, as partes devem apresentar suas derradeiras alegações antes da sentença do Juiz.
Caso qualquer das partes deixe de oferecer as alegações finais, recomenda\sphinxhyphen{}se à secretaria proceder a nova intimação via ato ordinatório, sem prejuízo de tornar os autos conclusos ao Juiz para deliberação.

\sphinxAtStartPar
Depositados os memoriais, tanto da acusação quanto da defesa, os autos devem ser remetidos ao Juiz para prolação da sentença.


\begin{sphinxseealso}{Ver também:}

\sphinxAtStartPar
Saiba mais sobre:

\sphinxAtStartPar
Como movimentar os autos após a audiência e encaminhar conclusos ao juiz: {\hyperref[\detokenize{projud_35_enviarconcluso::doc}]{\sphinxcrossref{\DUrole{doc}{Enviar Concluso}}}}


\end{sphinxseealso}


\sphinxstepscope


\section{Disposições Finais da Sentença}
\label{\detokenize{08sentenca_disposicoesfinais:disposicoes-finais-da-sentenca}}\label{\detokenize{08sentenca_disposicoesfinais::doc}}
\sphinxAtStartPar
É importante que o Juízo de 1º Grau elabore de forma prática, sucessiva e de simples compreensão pela Secretaria as disposições finais da Sentença.


\subsection{08.1 \textendash{} Finalidade das Disposições Finais}
\label{\detokenize{08sentenca_disposicoesfinais:finalidade-das-disposicoes-finais}}
\sphinxAtStartPar
As disposições finais da sentença representam etapa essencial e estratégica no processo penal. Sua correta elaboração:
\begin{enumerate}
\sphinxsetlistlabels{\arabic}{enumi}{enumii}{}{.}%
\item {} 
\sphinxAtStartPar
Garante a imediata execução dos atos administrativos e jurisdicionais após o trânsito em julgado.

\item {} 
\sphinxAtStartPar
Evita conclusões desnecessárias ao juízo.

\item {} 
\sphinxAtStartPar
Minimiza o risco de prescrição ou inércia processual.

\item {} 
\sphinxAtStartPar
Assegura o cumprimento das penas e comunicações institucionais obrigatórias.

\item {} 
\sphinxAtStartPar
Promove a celeridade e eficiência da tramitação na fase de execução penal, seja no âmbito da própria Vara ou na Vara de Execuções Penais.

\end{enumerate}


\subsection{08.2 \textendash{} Da reavaliação de prisão e da situação processual do Sentenciado}
\label{\detokenize{08sentenca_disposicoesfinais:da-reavaliacao-de-prisao-e-da-situacao-processual-do-sentenciado}}
\sphinxAtStartPar
A sentença também é o momento de realizar a reavaliação do prazo \sphinxstylestrong{nonagesimal} para manutenção da prisão cautelar, consoante previsão do art. 316, parágrafo único, do CPP, bem como a decretação de medidas protetivas ou cautelares solicitadas pelo Ministério Público.

\sphinxAtStartPar
Sugere\sphinxhyphen{}se que as disposições finais sejam redigidas com base nas seguintes diretrizes:

\sphinxAtStartPar
\sphinxstylestrong{Exemplos:}
\begin{itemize}
\item {} 
\sphinxAtStartPar
Se permanece preso pelo presente processo:

\begin{sphinxVerbatim}[commandchars=\\\{\}]
“O sentenciado se encontra preso pelo presente processo. Mantenho tal status e nego o direito de recorrer em liberdade, considerando que permaneceu em custódia durante toda a instrução criminal e não há alteração nos fundamentos da prisão preventiva.”
\end{sphinxVerbatim}

\item {} 
\sphinxAtStartPar
Se responde em liberdade:

\begin{sphinxVerbatim}[commandchars=\\\{\}]
“O sentenciado não se encontra preso pelo presente processo. Mantenho tal status, permitindo\PYGZhy{}lhe recorrer em liberdade, salvo se preso por outro motivo.”
\end{sphinxVerbatim}

\end{itemize}

\sphinxAtStartPar
\sphinxstylestrong{Avaliar a necessidade de:}
\begin{itemize}
\item {} 
\sphinxAtStartPar
Manutenção de medidas anteriormente decretadas:

\begin{sphinxVerbatim}[commandchars=\\\{\}]
“O sentenciado permanece sujeito às medidas protetivas constantes às fls. XX.”
\end{sphinxVerbatim}

\item {} 
\sphinxAtStartPar
Decretação de novas medidas, com base no art. 20 da Lei nº 14.344/2022:

\begin{sphinxVerbatim}[commandchars=\\\{\}]
“Decreto as medidas protetivas previstas no art. 20, incisos II, III, IV, V, VI e IX da Lei nº 14.344/22, por serem indispensáveis à segurança da vítima.”
\end{sphinxVerbatim}

\item {} 
\sphinxAtStartPar
Determinar a expedição imediata de mandado de medida protetiva, em regime \sphinxstylestrong{urgente}, para distribuição à Central de Mandados.

\end{itemize}


\subsection{08.3 \textendash{} Providências pós trânsito em julgado}
\label{\detokenize{08sentenca_disposicoesfinais:providencias-pos-transito-em-julgado}}

\subsubsection{08.3.1 \textendash{} Guia de Execução Penal}
\label{\detokenize{08sentenca_disposicoesfinais:guia-de-execucao-penal}}\begin{itemize}
\item {} 
\sphinxAtStartPar
Réu preso:

\begin{sphinxVerbatim}[commandchars=\\\{\}]
“Expeça\PYGZhy{}se guia de execução penal vinculada ao mandado de prisão preventiva já em vigor.”
“Encaminhe\PYGZhy{}se à Vara de Execução Penal conforme o regime fixado.”
\end{sphinxVerbatim}

\item {} 
\sphinxAtStartPar
Regime fechado (réu solto):

\begin{sphinxVerbatim}[commandchars=\\\{\}]
“Expeça\PYGZhy{}se mandado de prisão junto ao BNMP.”
“Oficie\PYGZhy{}se às autoridades policiais (EX: DEPCA, DIP, POLINTER, PF).”
\end{sphinxVerbatim}

\item {} 
\sphinxAtStartPar
Regime semiaberto ou aberto:

\begin{sphinxVerbatim}[commandchars=\\\{\}]
“Expeça\PYGZhy{}se guia de execução penal e encaminhem\PYGZhy{}se os autos à VEP competente.”
\end{sphinxVerbatim}

\item {} 
\sphinxAtStartPar
Penas alternativas (restritivas de direito ou multa) ou casos de suspensão condicional da pena:

\begin{sphinxVerbatim}[commandchars=\\\{\}]
“Encaminhe\PYGZhy{}se à VEMEPA, conforme art. 78 da Lei Complementar nº 261/2023.”
\end{sphinxVerbatim}

\end{itemize}


\subsubsection{08.3.2 \textendash{} Rol dos Culpados}
\label{\detokenize{08sentenca_disposicoesfinais:rol-dos-culpados}}
\sphinxAtStartPar
“Lance\sphinxhyphen{}se o nome do réu no rol dos culpados.”


\subsubsection{08.3.3 \textendash{} Comunicações Institucionais}
\label{\detokenize{08sentenca_disposicoesfinais:comunicacoes-institucionais}}\begin{itemize}
\item {} 
\sphinxAtStartPar
Polícia Militar / SSP (em caso de perda de cargo público):

\begin{sphinxVerbatim}[commandchars=\\\{\}]
“Oficie\PYGZhy{}se ao Comando Geral da PM/AM para ciência da perda do cargo.”
\end{sphinxVerbatim}

\item {} 
\sphinxAtStartPar
Tribunal Regional Eleitoral / INFODIP:

\begin{sphinxVerbatim}[commandchars=\\\{\}]
“Comunique\PYGZhy{}se ao TRE, nos termos do art. 71, \S{}2º, do Código Eleitoral, c/c art. 15, III, da CF/88.”
“Anote\PYGZhy{}se a suspensão dos direitos políticos junto ao INFODIP, durante a execução da pena.”
\end{sphinxVerbatim}

\item {} 
\sphinxAtStartPar
Vítima:

\begin{sphinxVerbatim}[commandchars=\\\{\}]
“Comunique\PYGZhy{}se à vítima, por seu representante legal, o desfecho da causa, nos termos do art. 201, \S{}2º, do CPP.”
\end{sphinxVerbatim}

\item {} 
\sphinxAtStartPar
Cartório de Registro Civil (perda do poder familiar):

\begin{sphinxVerbatim}[commandchars=\\\{\}]
“Oficie\PYGZhy{}se ao Cartório XXº RCPN para averbação da perda do poder familiar, nos termos do art. 163, parágrafo único, do ECA.”
\end{sphinxVerbatim}

\item {} 
\sphinxAtStartPar
Casos de crimes graves (art. 9º\sphinxhyphen{}A da LEP):

\begin{sphinxVerbatim}[commandchars=\\\{\}]
“Determine\PYGZhy{}se a submissão do réu ao procedimento de coleta de DNA, se ainda não realizado, no ingresso ao estabelecimento prisional.”
\end{sphinxVerbatim}

\end{itemize}


\subsubsection{08.3.4 \textendash{} Conclusão e Segredo de Justiça}
\label{\detokenize{08sentenca_disposicoesfinais:conclusao-e-segredo-de-justica}}
\sphinxAtStartPar
Quando se tratar de processo que tramita em segredo de justiça:

\sphinxAtStartPar
“Publique\sphinxhyphen{}se. Mantenham\sphinxhyphen{}se os autos em segredo de justiça, conforme art. 234\sphinxhyphen{}B do Código Penal.”
“Sentença registrada eletronicamente.”


\subsubsection{08.3.5 \textendash{} Recomendações finais}
\label{\detokenize{08sentenca_disposicoesfinais:recomendacoes-finais}}\begin{itemize}
\item {} 
\sphinxAtStartPar
Evitar encerramentos genéricos ou abertos (“expeça\sphinxhyphen{}se o necessário”).

\item {} 
\sphinxAtStartPar
Adaptar o modelo às peculiaridades do caso, especialmente quanto à situação prisional e natureza da pena.

\item {} 
\sphinxAtStartPar
Utilizar linguagem clara, precisa e imperativa, seguindo o Pacto Nacional pela Linguagem Simples (CNJ).

\end{itemize}


\subsection{Referências e Precedentes}
\label{\detokenize{08sentenca_disposicoesfinais:referencias-e-precedentes}}\begin{enumerate}
\sphinxsetlistlabels{\arabic}{enumi}{enumii}{}{.}%
\item {} 
\sphinxAtStartPar
STF \sphinxhyphen{} HC 167083/SP, Rel. Min. Marco Aurélio, julgamento em 18/08/2020, Primeira Turma, DJe 18/11/2020.
TJ\sphinxhyphen{}DF 0707868\sphinxhyphen{}21.2020.8.07.0000, Rel. Roberval Casemiro Belinati, julgamento em 02/04/2020.

\item {} 
\sphinxAtStartPar
Portaria nº 2897/2023, art. 2º \textendash{} expedição de mandado de prisão e arquivamento provisório do processo até a captura do réu.
Após o envio dos documentos à Vara de Execução Penal via SEEU, o feito deve ser baixado em definitivo.

\item {} 
\sphinxAtStartPar
STF \textendash{} RE 601182 (Tese de Repercussão Geral):
“A suspensão de direitos políticos prevista no art. 15, III, da Constituição Federal aplica\sphinxhyphen{}se no caso de substituição da pena privativa de liberdade pela restritiva de direitos.”
Tribunal Pleno, Rel. Min. Marco Aurélio, Red. acórdão Min. Alexandre de Moraes, julgamento em 08/05/2019, publicação em 02/10/2019.

\end{enumerate}

\sphinxstepscope


\section{Ciência da Sentença}
\label{\detokenize{08_5cienciasentenca:ciencia-da-sentenca}}\label{\detokenize{08_5cienciasentenca::doc}}
\sphinxAtStartPar
A Sentença é o ato judicial formal que põe fim à fase de conhecimento, podendo, no processo penal, ser:
\begin{itemize}
\item {} 
\sphinxAtStartPar
\sphinxstylestrong{Condenatória}

\item {} 
\sphinxAtStartPar
\sphinxstylestrong{Absolutória}

\item {} 
\sphinxAtStartPar
\sphinxstylestrong{Extintiva da punibilidade}

\end{itemize}

\sphinxAtStartPar
Subsidiariamente, também existem as chamadas \sphinxstylestrong{sentenças terminativas}, que encerram o procedimento sem julgamento de mérito (ex: perda de objeto).


\subsection{Quadro comparativo dos tipos de sentença}
\label{\detokenize{08_5cienciasentenca:quadro-comparativo-dos-tipos-de-sentenca}}

\begin{savenotes}\sphinxattablestart
\sphinxthistablewithglobalstyle
\centering
\begin{tabular}[t]{\X{20}{100}\X{20}{100}\X{20}{100}\X{20}{100}\X{20}{100}}
\sphinxtoprule
\sphinxstyletheadfamily 
\sphinxAtStartPar
Tipo de Sentença
&\sphinxstyletheadfamily 
\sphinxAtStartPar
Fundamento Legal
&\sphinxstyletheadfamily 
\sphinxAtStartPar
Efeito Principal
&\sphinxstyletheadfamily 
\sphinxAtStartPar
Exemplos
&\sphinxstyletheadfamily 
\sphinxAtStartPar
Observações
\\
\sphinxmidrule
\sphinxtableatstartofbodyhook
\sphinxAtStartPar
Condenatória
&
\sphinxAtStartPar
Art. 387 CPP; Art. 91 CP
&
\sphinxAtStartPar
Aplica pena ao réu
&
\sphinxAtStartPar
Homicídio com provas de autoria e materialidade
&
\sphinxAtStartPar
Pode implicar prisão, multa, perda de bens
\\
\sphinxhline
\sphinxAtStartPar
Absolutória
&
\sphinxAtStartPar
Art. 386 CPP (I a VIII)
&
\sphinxAtStartPar
Reconhece que o réu não deve ser punido
&
\sphinxAtStartPar
Fato atípico ou ausência de provas
&
\sphinxAtStartPar
Pode haver responsabilidade civil
\\
\sphinxhline
\sphinxAtStartPar
Extinção da Punibilidade
&
\sphinxAtStartPar
Art. 107 CP
&
\sphinxAtStartPar
Estado perde o direito de punir
&
\sphinxAtStartPar
Prescrição, morte do agente, ANPP
&
\sphinxAtStartPar
Não há julgamento de mérito da culpa
\\
\sphinxbottomrule
\end{tabular}
\sphinxtableafterendhook\par
\sphinxattableend\end{savenotes}


\subsection{09.1 \textendash{} Da intimação da sentença}
\label{\detokenize{08_5cienciasentenca:da-intimacao-da-sentenca}}
\sphinxAtStartPar
Para validade da sentença é necessário que haja \sphinxstylestrong{intimação das partes e publicação}.


\subsubsection{09.1.1 \textendash{} Ministério Público e Defensoria Pública}
\label{\detokenize{08_5cienciasentenca:ministerio-publico-e-defensoria-publica}}
\sphinxAtStartPar
No Projudi, a intimação do MP e da DPE não é automática.
É necessário remeter manualmente os autos a tais órgãos, que possuem prerrogativa de \sphinxstylestrong{intimação pessoal}, conforme:
\begin{itemize}
\item {} 
\sphinxAtStartPar
Arts. 180, 183 \S{}1º, 186 \S{}1º, CPC;

\item {} 
\sphinxAtStartPar
Art. 270, parágrafo único, CPC;

\item {} 
\sphinxAtStartPar
Art. 3º CPP (aplicação subsidiária do CPC).

\end{itemize}


\subsubsection{09.1.2 \textendash{} Advocacia privada e DJEN}
\label{\detokenize{08_5cienciasentenca:advocacia-privada-e-djen}}
\sphinxAtStartPar
O CNJ regulamentou a intimação pelo \sphinxstylestrong{DJEN (Diário de Justiça Eletrônico Nacional)}:
\begin{itemize}
\item {} 
\sphinxAtStartPar
Resolução CNJ nº 455/2022 (instituição).

\item {} 
\sphinxAtStartPar
Resolução CNJ nº 569/2024 (alterações).

\end{itemize}

\sphinxAtStartPar
\sphinxstylestrong{Publicações obrigatórias no DJEN (art. 13, Res. 455/2022):}
\begin{itemize}
\item {} 
\sphinxAtStartPar
Despachos, decisões interlocutórias e dispositivos de sentença.

\item {} 
\sphinxAtStartPar
Intimações eletrônicas que não exigem vista pessoal.

\item {} 
\sphinxAtStartPar
Listas de distribuição.

\item {} 
\sphinxAtStartPar
Atos destinados à plataforma de editais do CNJ.

\item {} 
\sphinxAtStartPar
Outros previstos em lei.

\end{itemize}

\sphinxAtStartPar
\sphinxstylestrong{Contagem de prazos:}
Res. CNJ 569/2024 alterou o art. 11, \S{}3º da Res. 455/2022 \(\rightarrow\) prazos passam a ser contados \sphinxstylestrong{a partir da publicação no DJEN} (art. 224 CPC).

\sphinxAtStartPar
\sphinxstylestrong{Obrigatoriedade:}
A partir de \sphinxstylestrong{16/05/2025}, o DJEN passou a ser o \sphinxstylestrong{meio oficial único} para publicação e contagem de prazos em todo o Judiciário.

\begin{sphinxadmonition}{note}{Nota:}
\sphinxAtStartPar
No TJAM, a SETIC integrou no Projudi a aba
“Pré\sphinxhyphen{}Análise \textgreater{} Dados da Conclusão \textgreater{} Dados para Publicação”.
A assessoria deve inserir o conteúdo da decisão para publicação.
O sistema então emite uma certidão de publicação.
\end{sphinxadmonition}


\subsection{09.2 \textendash{} Retorno de conclusão}
\label{\detokenize{08_5cienciasentenca:retorno-de-conclusao}}
\sphinxAtStartPar
No Projudi, após manifestação judicial, os autos seguem para:

\sphinxAtStartPar
\sphinxstylestrong{Mesa do Analista \textgreater{} Análise de Juntadas \textgreater{} Retorno de Conclusão \textgreater{} Dt. Retorno \textgreater{} “finalizar conclusão pendente”}

\begin{sphinxadmonition}{warning}{Aviso:}
\sphinxAtStartPar
Antes de finalizar, deve\sphinxhyphen{}se incluir o \sphinxstylestrong{localizador/fila processual} pertinente.
\end{sphinxadmonition}


\subsection{09.3 \textendash{} Dos prazos}
\label{\detokenize{08_5cienciasentenca:dos-prazos}}
\sphinxAtStartPar
Prazo para recorrer da sentença penal:
\begin{itemize}
\item {} 
\sphinxAtStartPar
\sphinxstylestrong{Apelação:} 5 dias

\item {} 
\sphinxAtStartPar
\sphinxstylestrong{Embargos de Declaração:} 2 dias

\end{itemize}

\sphinxAtStartPar
\sphinxstylestrong{Regras de contagem (CPP, art. 798, \S{}\S{}1º e 3º):}
\begin{itemize}
\item {} 
\sphinxAtStartPar
Excluir o primeiro dia.

\item {} 
\sphinxAtStartPar
Não contar finais de semana e feriados.

\end{itemize}


\subsubsection{Quadro prazos de Apelação}
\label{\detokenize{08_5cienciasentenca:quadro-prazos-de-apelacao}}

\begin{savenotes}\sphinxattablestart
\sphinxthistablewithglobalstyle
\centering
\begin{tabular}[t]{\X{20}{100}\X{20}{100}\X{20}{100}\X{20}{100}\X{20}{100}}
\sphinxtoprule
\sphinxstyletheadfamily 
\sphinxAtStartPar
Parte
&\sphinxstyletheadfamily 
\sphinxAtStartPar
Prazo Interposição
&\sphinxstyletheadfamily 
\sphinxAtStartPar
Prazos Razões/Contrarrazões
&\sphinxstyletheadfamily 
\sphinxAtStartPar
Fundamento
&\sphinxstyletheadfamily 
\sphinxAtStartPar
Observações
\\
\sphinxmidrule
\sphinxtableatstartofbodyhook
\sphinxAtStartPar
Advogado Particular
&
\sphinxAtStartPar
5 dias (DJEN)
&
\sphinxAtStartPar
8/8 dias
&
\sphinxAtStartPar
Art. 593 CPP; Res. 455/22; 569/24
&
\sphinxAtStartPar
Conta\sphinxhyphen{}se do dia útil seguinte à publicação
\\
\sphinxhline
\sphinxAtStartPar
Defensoria Pública
&
\sphinxAtStartPar
10 dias (prazo em dobro \textendash{} LC 80/94, art. 5º \S{}5º)
&
\sphinxAtStartPar
16/16 dias
&
\sphinxAtStartPar
LC 80/94; STF/STJ
&
\sphinxAtStartPar
Prazo em dobro mesmo em eletrônico (Lei 11.419/06, art. 5º \S{}3º)
\\
\sphinxhline
\sphinxAtStartPar
Ministério Público
&
\sphinxAtStartPar
5 dias
&
\sphinxAtStartPar
8/8 dias
&
\sphinxAtStartPar
Lei 11.419/06, art. 5º; Art. 593 CPP
&
\sphinxAtStartPar
Conta\sphinxhyphen{}se da ciência (expressa ou tácita \textendash{} 10 dias)
\\
\sphinxhline
\sphinxAtStartPar
Assistente de Acusação
&
\sphinxAtStartPar
5 dias
&
\sphinxAtStartPar
8/8 dias
&
\sphinxAtStartPar
Art. 598 CPP
&
\sphinxAtStartPar
Pode apelar mesmo sem o MP
\\
\sphinxhline
\sphinxAtStartPar
Apelação Adesiva
&
\sphinxAtStartPar
Mesmo prazo das contrarrazões
&
\sphinxAtStartPar
—
&
\sphinxAtStartPar
Art. 500 CPC c/c art. 3º CPP
&
\sphinxAtStartPar
Só vale se houver apelação principal
\\
\sphinxbottomrule
\end{tabular}
\sphinxtableafterendhook\par
\sphinxattableend\end{savenotes}


\subsubsection{Quadro prazos de Embargos}
\label{\detokenize{08_5cienciasentenca:quadro-prazos-de-embargos}}

\begin{savenotes}\sphinxattablestart
\sphinxthistablewithglobalstyle
\centering
\begin{tabular}[t]{\X{20}{120}\X{20}{120}\X{20}{120}\X{20}{120}\X{20}{120}\X{20}{120}}
\sphinxtoprule
\sphinxstyletheadfamily 
\sphinxAtStartPar
Parte
&\sphinxstyletheadfamily 
\sphinxAtStartPar
Prazo
&\sphinxstyletheadfamily 
\sphinxAtStartPar
Fundamento
&\sphinxstyletheadfamily 
\sphinxAtStartPar
Cabimento
&\sphinxstyletheadfamily 
\sphinxAtStartPar
Efeitos
&\sphinxstyletheadfamily 
\sphinxAtStartPar
Forma
\\
\sphinxmidrule
\sphinxtableatstartofbodyhook
\sphinxAtStartPar
Qualquer parte
&
\sphinxAtStartPar
2 dias
&
\sphinxAtStartPar
Art. 382 CPP
&
\sphinxAtStartPar
Obscuridade, contradição, omissão ou ambiguidade
&
\sphinxAtStartPar
Interrompe prazos recursais (art. 613 \S{}1º CPP)
&
\sphinxAtStartPar
Petição simples ou termo nos autos
\\
\sphinxbottomrule
\end{tabular}
\sphinxtableafterendhook\par
\sphinxattableend\end{savenotes}


\subsection{09.4 \textendash{} Da intimação do réu}
\label{\detokenize{08_5cienciasentenca:da-intimacao-do-reu}}\begin{itemize}
\item {} 
\sphinxAtStartPar
\sphinxstylestrong{Réu preso preventivamente:} intimação no presídio (art. 392, I CPP).
\sphinxhyphen{} Verificação de local via sistema SEAP (link).

\item {} 
\sphinxAtStartPar
\sphinxstylestrong{Réu absolvido preso} ou \sphinxstylestrong{prisão preventiva revogada:} expedir Alvará de Soltura (art. 596 CPP).

\item {} 
\sphinxAtStartPar
\sphinxstylestrong{Réu solto com advogado ou defensor público:} basta intimação do defensor (art. 392, II e III CPP).
\sphinxhyphen{} Súmula 19/TJAM: intimação do defensor supre intimação pessoal do réu solto (DJE 23/08/2024).

\item {} 
\sphinxAtStartPar
\sphinxstylestrong{Réu ausente/revel ou foragido:} intimação por edital (art. 392 \S{}1º CPP):
\sphinxhyphen{} 90 dias \(\rightarrow\) pena ≥ 1 ano.
\sphinxhyphen{} 60 dias \(\rightarrow\) demais casos.

\end{itemize}

\sphinxstepscope


\section{Certidão de Trânsito em Julgado}
\label{\detokenize{09certificartransito:certidao-de-transito-em-julgado}}\label{\detokenize{09certificartransito::doc}}

\subsection{Definição de Trânsito em Julgado}
\label{\detokenize{09certificartransito:definicao-de-transito-em-julgado}}
\sphinxAtStartPar
O trânsito em julgado ocorre quando não há mais possibilidade de recurso no processo, seja porque o prazo para interposição de recurso expirou, ou porque todas as possibilidades de recurso foram esgotadas. A partir desse momento, a decisão torna\sphinxhyphen{}se definitiva.


\subsection{Procedimentos Preliminares}
\label{\detokenize{09certificartransito:procedimentos-preliminares}}
\sphinxAtStartPar
Antes de realizar a certificação de trânsito em julgado no SAJPG5, certifique\sphinxhyphen{}se de:
\begin{itemize}
\item {} 
\sphinxAtStartPar
Verificar se o prazo recursal para todas as partes expirou.

\item {} 
\sphinxAtStartPar
Verificar se há eventuais recursos pendentes ou interpostos no sistema.

\item {} 
\sphinxAtStartPar
Conferir se a intimação das partes sobre a decisão foi devidamente realizada e registrada no sistema.

\end{itemize}


\subsection{Acesso ao Processo no SAJPG5}
\label{\detokenize{09certificartransito:acesso-ao-processo-no-sajpg5}}\begin{enumerate}
\sphinxsetlistlabels{\arabic}{enumi}{enumii}{}{.}%
\item {} 
\sphinxAtStartPar
Acesse o sistema SAJ/PG5 utilizando seu login e senha.

\item {} 
\sphinxAtStartPar
\sphinxstylestrong{Localize o processo} utilizando o número único ou buscando pelo nome das partes. Utilize o campo de consulta de processos para encontrar as informações necessárias.

\end{enumerate}


\subsection{Verificação dos Prazos e Recursos}
\label{\detokenize{09certificartransito:verificacao-dos-prazos-e-recursos}}\begin{enumerate}
\sphinxsetlistlabels{\arabic}{enumi}{enumii}{}{.}%
\item {} 
\sphinxAtStartPar
Acesse a aba de movimentações processuais para verificar se o prazo de recurso já se encerrou. Acompanhe as movimentações referentes à publicação da decisão e ao início do prazo recursal.

\item {} 
\sphinxAtStartPar
Certifique\sphinxhyphen{}se de que não há recursos interpostos ou pendências. No sistema, verifique se houve a interposição de apelações ou outros recursos, e se esses recursos já foram julgados.

\end{enumerate}


\subsection{Certificação de Trânsito em Julgado}
\label{\detokenize{09certificartransito:certificacao-de-transito-em-julgado}}
\sphinxAtStartPar
Após a verificação de que não há mais possibilidade de recurso, siga os seguintes passos para certificar o trânsito em julgado:
\begin{enumerate}
\sphinxsetlistlabels{\arabic}{enumi}{enumii}{}{.}%
\item {} 
\sphinxAtStartPar
\sphinxstylestrong{Acesse a aba de certificações}: Dentro do processo, vá até o campo destinado à emissão de certidões e documentos.

\item {} 
\sphinxAtStartPar
\sphinxstylestrong{Escolha a opção de “Certidão de Trânsito em Julgado”}: No sistema SAJ/PG5, selecione o modelo de certidão adequado. Normalmente, os modelos já estão pré\sphinxhyphen{}configurados no sistema.

\item {} 
\sphinxAtStartPar
\sphinxstylestrong{Preenchimento dos campos necessários}: Informe a data da decisão, partes envolvidas e a data final do prazo recursal. Certifique\sphinxhyphen{}se de que os dados inseridos estão corretos, evitando erros na emissão da certidão.

\item {} 
\sphinxAtStartPar
\sphinxstylestrong{Descrição da Certidão}: A certidão deve mencionar claramente que o prazo para interposição de recursos transcorreu sem manifestação das partes ou que todos os recursos foram julgados, e que, portanto, a decisão transitou em julgado.
\begin{itemize}
\item {} 
\sphinxAtStartPar
\sphinxstylestrong{Exemplo de Certidão:}
\textgreater{} “Certifico que, em \_\_/\_\_/\_\_\_\_, transcorreu in albis o prazo legal, sem interposição de recurso pelas partes, tornando\sphinxhyphen{}se assim transitada em julgado a sentença proferida nos autos nº \_\_\_\_\_\_\_\_, conforme o artigo 502 do Código de Processo Civil.”

\end{itemize}

\item {} 
\sphinxAtStartPar
\sphinxstylestrong{Salvar e Concluir}: Após preencher os campos, clique em \sphinxstylestrong{“Salvar”} e \sphinxstylestrong{“Concluir”} a emissão da certidão no sistema. A certidão ficará disponível nos autos e poderá ser visualizada por todas as partes.

\end{enumerate}


\begin{sphinxseealso}{Ver também:}

\sphinxAtStartPar
Saiba mais sobre como certificar o trânsito em julgado no Projudi:

\sphinxAtStartPar
Procedimento para trânsito em julgado por parte ou processo: {\hyperref[\detokenize{projud_40_transitaremjulgado::doc}]{\sphinxcrossref{\DUrole{doc}{Transitar em Julgado}}}}


\end{sphinxseealso}



\subsection{Movimentação e Registro do Trânsito em Julgado}
\label{\detokenize{09certificartransito:movimentacao-e-registro-do-transito-em-julgado}}\begin{enumerate}
\sphinxsetlistlabels{\arabic}{enumi}{enumii}{}{.}%
\item {} 
\sphinxAtStartPar
\sphinxstylestrong{Movimente o processo no SAJPG5}, indicando que o trânsito em julgado foi certificado. Para isso, utilize a opção de “Movimentações Processuais” e selecione a movimentação correspondente ao trânsito em julgado.

\item {} 
\sphinxAtStartPar
\sphinxstylestrong{Encaminhe para o Arquivo ou Execução}: Dependendo do tipo de processo (cível, criminal, etc.), após o trânsito em julgado, o processo pode ser encaminhado para:
\sphinxhyphen{} Arquivamento;
\sphinxhyphen{} Cumprimento de sentença (no caso de processos cíveis);
\sphinxhyphen{} Execução penal (no caso de processos criminais).

\item {} 
\sphinxAtStartPar
\sphinxstylestrong{Notificação das Partes}: Caso necessário, utilize o SAJPG5 para notificar as partes interessadas sobre o trânsito em julgado, especialmente em processos criminais ou de execução.

\end{enumerate}


\begin{sphinxseealso}{Ver também:}

\sphinxAtStartPar
Saiba mais sobre:

\sphinxAtStartPar
Como movimentar o processo após trânsito em julgado: {\hyperref[\detokenize{projud_40_transitaremjulgado::doc}]{\sphinxcrossref{\DUrole{doc}{Transitar em Julgado}}}}

\sphinxAtStartPar
Arquivamento dos autos após o trânsito: {\hyperref[\detokenize{projud_41_arquivamento::doc}]{\sphinxcrossref{\DUrole{doc}{Arquivamento dos Autos}}}}

\sphinxAtStartPar
Execução penal e cumprimento de sentença (medidas posteriores): {\hyperref[\detokenize{projud_53_cadastrodenuncia::doc}]{\sphinxcrossref{\DUrole{doc}{Cadastro da Denúncia ou Queixa para fins da Prescrição}}}} e {\hyperref[\detokenize{projud_54_cadastromedidasalternativas::doc}]{\sphinxcrossref{\DUrole{doc}{Cadastro de Medidas Alternativas: Transação Penal}}}}


\end{sphinxseealso}



\subsection{Finalização}
\label{\detokenize{09certificartransito:finalizacao}}
\sphinxAtStartPar
Após a certificação do trânsito em julgado e o registro da movimentação no sistema, o processo pode seguir para as próximas etapas, conforme a natureza da decisão (execução, arquivamento, etc.).


\subsection{Modelos de Certidões}
\label{\detokenize{09certificartransito:modelos-de-certidoes}}

\begin{sphinxseealso}{Ver também:}

\sphinxAtStartPar
Saiba mais sobre:

\sphinxAtStartPar
Criação e personalização de modelos de documentos no sistema: {\hyperref[\detokenize{projud_48_criandomodelo::doc}]{\sphinxcrossref{\DUrole{doc}{Criando Modelos de Documento}}}}

\sphinxAtStartPar
Uso de variáveis em modelos (como nome da parte, data do trânsito etc.): {\hyperref[\detokenize{projud_49_variaveis::doc}]{\sphinxcrossref{\DUrole{doc}{Variáveis}}}}


\end{sphinxseealso}


\sphinxstepscope


\section{Da Expedição de Guias de Execução}
\label{\detokenize{10expedir_guia_execucao:da-expedicao-de-guias-de-execucao}}\label{\detokenize{10expedir_guia_execucao::doc}}
\sphinxAtStartPar
A expedição das Guias de Execução constitui etapa essencial após a prolação de sentença penal condenatória, assegurando a correta remessa das informações ao Banco Nacional de Monitoramento de Prisões (BNMP) e ao Sistema Eletrônico de Execução Unificado (SEEU), bem como a comunicação aos órgãos competentes.


\subsection{13.1 \textendash{} Guias Provisórias}
\label{\detokenize{10expedir_guia_execucao:guias-provisorias}}
\sphinxAtStartPar
As Guias Provisórias são expedidas quando o réu responde preso durante todo o processo (prisão preventiva), sendo determinadas na sentença ou na decisão que recebe a apelação, conforme Resolução CNJ nº 113/2010.

\sphinxAtStartPar
\sphinxstylestrong{Procedimentos:}
\begin{itemize}
\item {} 
\sphinxAtStartPar
Elaborar a Guia junto ao BNMP, atentando às informações obrigatórias.

\item {} 
\sphinxAtStartPar
Encaminhar à assinatura do(a) magistrado(a).

\item {} 
\sphinxAtStartPar
Juntar a Guia nos autos e expedir ofício ao estabelecimento prisional comunicando a expedição, com envio pelo Malote Digital.

\item {} 
\sphinxAtStartPar
Proceder ao cadastro da guia no SEEU / Encaminhar via Malote Digital para a Vara de Execuções.

\end{itemize}

\begin{sphinxadmonition}{note}{Nota:}
\sphinxAtStartPar
Nos casos de regime fechado, a Guia Provisória somente pode ser expedida se houver mandado de prisão devidamente cumprido com a assinatura do réu.
\end{sphinxadmonition}


\subsection{13.2 \textendash{} Guias Definitivas}
\label{\detokenize{10expedir_guia_execucao:guias-definitivas}}
\sphinxAtStartPar
A Guia Definitiva somente pode ser expedida após o trânsito em julgado da sentença penal condenatória.

\begin{sphinxadmonition}{warning}{Aviso:}
\sphinxAtStartPar
A guia relacionada ao cumprimento de pena privativa de liberdade em regime fechado só pode ser feita se houver, no processo, o mandado devolvido cumprido com a assinatura do réu.

\sphinxAtStartPar
Nos casos de réus condenados em regime semiaberto, aberto ou penas restritivas de direito e/ou multa, expedir a respectiva Guia independente de expedição de mandado de prisão ou intimação prévia do condenado, conforme Portaria da Presidência TJAM 2897/2023.
\end{sphinxadmonition}

\sphinxAtStartPar
As Guias Definitivas podem ser em três regimes: fechado, semiaberto e aberto.

\sphinxAtStartPar
As Guias Provisórias geralmente serão em regime fechado, pois pressupõem que o réu se encontrou preso durante todo o processo.

\sphinxAtStartPar
Se a pena do condenado comportar substituição de pena privativa de liberdade em restritiva de direitos, deverá ser expedida a Guia de Execução em regime aberto e incluída a observação da conversão na própria Guia.

\sphinxAtStartPar
\sphinxstylestrong{Passos principais:}
\begin{itemize}
\item {} 
\sphinxAtStartPar
Verificar se consta nos autos a certidão de trânsito em julgado (1º ou 2º grau), confirmando o decurso do prazo recursal.

\item {} 
\sphinxAtStartPar
Elaborar a Guia no BNMP, preenchendo integralmente os dados requeridos.

\item {} 
\sphinxAtStartPar
Submeter à assinatura do(a) magistrado(a).

\item {} 
\sphinxAtStartPar
Exportar a peça do BNMP e juntar nos autos.

\item {} 
\sphinxAtStartPar
Realizar a comunicação eleitoral (INFODIP \textendash{} art. 71, \S{}2º do Código Eleitoral).

\item {} 
\sphinxAtStartPar
Expedir comunicação à vítima, nos termos do art. 201, \S{}2º, do CPP.

\end{itemize}

\sphinxAtStartPar
\sphinxstylestrong{Sugestão:}

\sphinxAtStartPar
Organizar, em pasta digital, as peças obrigatórias a serem encaminhadas à Vara de Execuções (Denúncia, Decisão que Recebeu a Denúncia, Decisão que aditou a Denúncia (caso existente), Sentença Condenatória, Acórdão Integral de Apelação (se houver), Resp/STJ ou RE/STF (se houver), Certidão de Trânsito, Alvará se Soltura (se houver), Mandado de Prisão, Guia Definitiva, Comprovante Infodip, Comprovante transferência de Guia BNMP) no Google Drive institucional para organizar melhor e cadastrar no SEEU.

\sphinxAtStartPar
\sphinxstylestrong{Selecionar corretamente a Vara de Execução competente:}
\begin{itemize}
\item {} 
\sphinxAtStartPar
VEP Manaus \textendash{} Fechado (penas \textgreater{} 8 anos);

\item {} 
\sphinxAtStartPar
VEP Manaus \textendash{} Semiaberto (penas \textgreater{} 4 e ≤ 8 anos);

\item {} 
\sphinxAtStartPar
VEP Manaus \textendash{} Aberto (penas ≤ 4 anos, sem substituição);

\item {} 
\sphinxAtStartPar
VEMEPA Manaus (penas restritivas de direitos e/ou multa).

\end{itemize}

\sphinxAtStartPar
\sphinxstylestrong{Etapas finais:}
\begin{itemize}
\item {} 
\sphinxAtStartPar
Gerar comprovante de protocolo da distribuição no SEEU e importar aos autos.

\item {} 
\sphinxAtStartPar
Realizar a transferência de competência no BNMP, vinculando todas as peças (mandado, guia, certidões).

\item {} 
\sphinxAtStartPar
Atualizar dados no Processo e proceder ao arquivamento definitivo.

\end{itemize}

\sphinxstepscope


\section{Recebimento da Apelação}
\label{\detokenize{11decisao_recebimento_apelacao:recebimento-da-apelacao}}\label{\detokenize{11decisao_recebimento_apelacao::doc}}
\sphinxAtStartPar
A Apelação Criminal é um recurso interposto em face de decisões de mérito criminais, sendo a mais comum a Sentença.
Diferentemente da sistemática do Processo Civil, no Processo Penal quem exerce o \sphinxstylestrong{primeiro juízo de admissibilidade recursal} é o Juízo de 1º Grau.

\sphinxAtStartPar
Além disso, a interposição da Apelação no Processo Penal possui múltiplas formas de apresentação.


\subsection{10.1 \textendash{} Interposição com fulcro no art. 600, \S{}4º do CPP}
\label{\detokenize{11decisao_recebimento_apelacao:interposicao-com-fulcro-no-art-600-4o-do-cpp}}
\sphinxAtStartPar
Nessa hipótese, a parte manifesta apenas a intenção de apelar, por meio de petição ou termo (nos casos de sentença proferida em AIJ ou Sessão de Julgamento), com base no art. 600, \S{}4º, do CPP:
\begin{quote}

\sphinxAtStartPar
“Se o apelante declarar, na petição ou no termo, ao interpor a apelação, que deseja arrazoar na superior instância serão os autos remetidos ao tribunal ad quem onde será aberta vista às partes, observados os prazos legais, notificadas as partes pela publicação oficial.”
\end{quote}

\sphinxAtStartPar
\sphinxstylestrong{Providências da Secretaria:}
\begin{itemize}
\item {} 
\sphinxAtStartPar
Verificar a \sphinxstylestrong{tempestividade} da interposição (5 dias após intimação da sentença).

\item {} 
\sphinxAtStartPar
Receber a apelação (se presentes os requisitos de admissibilidade).

\item {} 
\sphinxAtStartPar
Declarar expressamente o recebimento nos autos e determinar a \sphinxstylestrong{remessa imediata ao Juízo ad quem}.

\end{itemize}


\subsection{10.2 \textendash{} Apelação já interposta com as respectivas razões}
\label{\detokenize{11decisao_recebimento_apelacao:apelacao-ja-interposta-com-as-respectivas-razoes}}
\sphinxAtStartPar
A parte apresenta a petição de apelação \sphinxstylestrong{já instruída com as razões} no mesmo momento.

\sphinxAtStartPar
\sphinxstylestrong{Procedimentos:}
\begin{enumerate}
\sphinxsetlistlabels{\arabic}{enumi}{enumii}{}{.}%
\item {} 
\sphinxAtStartPar
A Secretaria deve intimar a parte contrária para apresentar contrarrazões:
\sphinxhyphen{} 08 dias para MP, Assistente de Acusação ou Defesa com advogado particular;
\sphinxhyphen{} 16 dias para réu assistido pela Defensoria Pública.

\item {} 
\sphinxAtStartPar
Após o decurso do prazo (com ou sem contrarrazões), os autos são conclusos ao magistrado para análise de admissibilidade.

\item {} 
\sphinxAtStartPar
\sphinxstylestrong{Recebido o recurso:} remessa ao Juízo ad quem.

\item {} 
\sphinxAtStartPar
\sphinxstylestrong{Não recebido o recurso:} intimar as partes para ciência. Possibilidade de interposição de \sphinxstylestrong{Recurso em Sentido Estrito} (art. 581, XV, do CPP).
\sphinxhyphen{} Se decorrido o prazo sem interposição: certificar o trânsito em julgado da sentença e adotar as providências posteriores.

\end{enumerate}


\subsection{10.3 \textendash{} Apelação sem razões e sem invocação do \S{}4º}
\label{\detokenize{11decisao_recebimento_apelacao:apelacao-sem-razoes-e-sem-invocacao-do-4o}}
\sphinxAtStartPar
Nessa hipótese, a parte apenas assina o termo de apelação, com fulcro no art. 600, caput, do CPP:
\begin{quote}

\sphinxAtStartPar
“Assinado o termo de apelação, o apelante e, depois dele, o apelado terão o prazo de oito dias cada um para oferecer razões, salvo nos processos de contravenção, em que o prazo será de três dias.”
\end{quote}

\sphinxAtStartPar
\sphinxstylestrong{Providências:}
\begin{itemize}
\item {} 
\sphinxAtStartPar
O Juízo de 1º Grau recebe a apelação e \sphinxstylestrong{intima o Apelante} para apresentar as razões no prazo legal.

\item {} 
\sphinxAtStartPar
\sphinxstylestrong{Se o Apelante apresentar razões:} intimar o Apelado para contrarrazões; após o prazo, remeter os autos ao Tribunal.

\item {} 
\sphinxAtStartPar
\sphinxstylestrong{Se o Apelante não apresentar razões:} certificar a inércia e remeter os autos ao Tribunal (art. 601 do CPP).

\item {} 
\sphinxAtStartPar
\sphinxstylestrong{Se o Apelado não apresentar contrarrazões:} certificar o decurso do prazo e remeter ao Tribunal.

\end{itemize}


\subsection{Quadro comparativo das hipóteses}
\label{\detokenize{11decisao_recebimento_apelacao:quadro-comparativo-das-hipoteses}}

\begin{savenotes}\sphinxattablestart
\sphinxthistablewithglobalstyle
\centering
\begin{tabular}[t]{\X{25}{110}\X{35}{110}\X{20}{110}\X{30}{110}}
\sphinxtoprule
\sphinxstyletheadfamily 
\sphinxAtStartPar
Hipótese
&\sphinxstyletheadfamily 
\sphinxAtStartPar
Descrição
&\sphinxstyletheadfamily 
\sphinxAtStartPar
Base Legal
&\sphinxstyletheadfamily 
\sphinxAtStartPar
Providências da Secretaria
\\
\sphinxmidrule
\sphinxtableatstartofbodyhook
\sphinxAtStartPar
Interposição com fulcro no art. 600, \S{}4º
&
\sphinxAtStartPar
Parte manifesta apenas a intenção de apelar, declarando que apresentará razões na instância superior.
&
\sphinxAtStartPar
Art. 600, \S{}4º do CPP
&
\sphinxAtStartPar
Verifica a tempestividade, certifica e remete imediatamente ao Tribunal.
\\
\sphinxhline
\sphinxAtStartPar
Apelação já interposta com razões
&
\sphinxAtStartPar
Apelação já instruída com razões; Secretaria intima a parte contrária para contrarrazões; após prazo, autos ao juiz para admissibilidade e remessa.
&
\sphinxAtStartPar
Art. 600, caput do CPP; CF/88, art. 93, XIV
&
\sphinxAtStartPar
Intima parte contrária para contrarrazões (8 ou 16 dias); após, conclusão ao juiz e remessa ao Tribunal.
\\
\sphinxhline
\sphinxAtStartPar
Apelação sem razões e sem invocação do \S{}4º
&
\sphinxAtStartPar
Apelação apenas interposta; apelante intimado para razões; com ou sem razões, após contrarrazões ou decurso, remessa ao Tribunal.
&
\sphinxAtStartPar
Art. 600, caput e art. 601 do CPP
&
\sphinxAtStartPar
Intima o Apelante para razões; após razões ou inércia, certifica e remete ao Tribunal.
\\
\sphinxbottomrule
\end{tabular}
\sphinxtableafterendhook\par
\sphinxattableend\end{savenotes}

\begin{sphinxadmonition}{warning}{Aviso:}
\sphinxAtStartPar
Nos casos de réus presos provisórios, em obediência à \sphinxstylestrong{Resolução CNJ nº 113/2010}, é necessário expedir a \sphinxstylestrong{Guia de Execução Provisória} antes da remessa da Apelação ao Tribunal.
O Juiz de 1º Grau deve consignar expressamente esse comando na decisão de admissibilidade.
\end{sphinxadmonition}


\subsection{Requisitos de admissibilidade recursal}
\label{\detokenize{11decisao_recebimento_apelacao:requisitos-de-admissibilidade-recursal}}
\sphinxAtStartPar
Para que a apelação seja admitida, devem estar presentes:
\begin{itemize}
\item {} 
\sphinxAtStartPar
\sphinxstylestrong{Tempestividade};

\item {} 
\sphinxAtStartPar
\sphinxstylestrong{Legitimidade} da parte;

\item {} 
\sphinxAtStartPar
\sphinxstylestrong{Interesse recursal};

\item {} 
\sphinxAtStartPar
\sphinxstylestrong{Cabimento} do recurso;

\item {} 
\sphinxAtStartPar
\sphinxstylestrong{Regularidade formal};

\item {} 
\sphinxAtStartPar
\sphinxstylestrong{Inexistência de fato impeditivo ou extintivo}.

\end{itemize}


\subsection{Referências}
\label{\detokenize{11decisao_recebimento_apelacao:referencias}}\begin{enumerate}
\sphinxsetlistlabels{\arabic}{enumi}{enumii}{}{.}%
\item {} 
\sphinxAtStartPar
CPP, arts. 591, 600 e 601.

\item {} 
\sphinxAtStartPar
CF/88, art. 93, XIV (delegação a servidores para atos de mero expediente).

\item {} 
\sphinxAtStartPar
Art. 603 do CPP \textendash{} findos os prazos, autos remetidos ao Tribunal com ou sem razões (prazo de 5 dias; 30 dias em caso específico).

\end{enumerate}

\sphinxstepscope


\section{Encaminhar os autos ao Segundo Grau}
\label{\detokenize{12encaminhar_segundo_grau:encaminhar-os-autos-ao-segundo-grau}}\label{\detokenize{12encaminhar_segundo_grau::doc}}
\sphinxAtStartPar
Sendo recebida a Apelação ou o Recurso em Sentido Estrito, os autos deverão ser encaminhados imediatamente à segunda instância.


\subsection{11.1 \textendash{} Dos prazos}
\label{\detokenize{12encaminhar_segundo_grau:dos-prazos}}
\sphinxAtStartPar
A Apelação e o Recurso em Sentido Estrito deverão ser encaminhados ao Segundo Grau no prazo de \sphinxstylestrong{05 (cinco) dias}, conforme:
\begin{itemize}
\item {} 
\sphinxAtStartPar
Art. 601 do CPP (Apelação);

\item {} 
\sphinxAtStartPar
Art. 591 do CPP (Recurso em Sentido Estrito).

\end{itemize}


\subsection{11.2 \textendash{} Do procedimento}
\label{\detokenize{12encaminhar_segundo_grau:do-procedimento}}
\sphinxAtStartPar
Após finalizar o procedimento de interposição recursal, a Secretaria deverá encaminhar os autos ao Segundo Grau.

\sphinxAtStartPar
\sphinxstylestrong{Passos práticos:}
\begin{enumerate}
\sphinxsetlistlabels{\arabic}{enumi}{enumii}{}{.}%
\item {} 
\sphinxAtStartPar
Verificar e retirar eventuais pendências, tais como:
\sphinxhyphen{} “Retorno de conclusão”;
\sphinxhyphen{} “Parte sem CPF cadastrado”;
\sphinxhyphen{} “Ação Penal sem Denúncia (clique para cadastrar)”, entre outras.

\item {} 
\sphinxAtStartPar
Acessar a aba \sphinxstylestrong{“Movimentações”}, localizar a decisão de admissibilidade recursal e clicar em \sphinxstylestrong{“Movimentar a Partir Desta Movimentação”}.

\item {} 
\sphinxAtStartPar
Ir na aba \sphinxstylestrong{“Ações”}, no canto esquerdo da tela, e selecionar \sphinxstylestrong{“Autos ao Tribunal de Justiça”}.

\end{enumerate}


\subsection{11.3 \textendash{} Localizador do processo após o envio ao Segundo Grau}
\label{\detokenize{12encaminhar_segundo_grau:localizador-do-processo-apos-o-envio-ao-segundo-grau}}
\sphinxAtStartPar
Após a remessa dos autos à segunda instância, o processo ficará com o \sphinxstylestrong{Status: “EM INSTÂNCIA SUPERIOR”}.

\begin{sphinxadmonition}{note}{Nota:}
\sphinxAtStartPar
É importante que, além do status, a Secretaria mantenha um \sphinxstylestrong{localizador específico} para tais processos, a fim de proporcionar maior controle.
A Corregedoria\sphinxhyphen{}Geral de Justiça do Amazonas (CGJ/AM) determina que, nas inspeções judiciais, o Juízo de 1º Grau \sphinxstylestrong{diligencie sobre o julgamento de recursos remetidos há mais de 01 (um) ano}.
\end{sphinxadmonition}


\subsection{Referência normativa}
\label{\detokenize{12encaminhar_segundo_grau:referencia-normativa}}\begin{itemize}
\item {} 
\sphinxAtStartPar
\sphinxstylestrong{Provimento nº 481/2024 \textendash{} CGJ/AM} (art. 14, IV, do Manual de Correições):
“IV \textendash{} Nos processos remetidos à segunda instância há mais de 01 (um) ano, o provimento deve determinar ao Diretor/Escrivão para verificar o julgamento do recurso.”

\end{itemize}

\sphinxstepscope


\section{Arquivamento Definitivo (Baixa)}
\label{\detokenize{13baixa definitiva:arquivamento-definitivo-baixa}}\label{\detokenize{13baixa definitiva::doc}}

\subsection{14.1 \textendash{} Do Arquivamento Definitivo dos Processos Criminais}
\label{\detokenize{13baixa definitiva:do-arquivamento-definitivo-dos-processos-criminais}}
\sphinxAtStartPar
O arquivamento definitivo é a fase final do processo penal em 1º grau, na qual, após o cumprimento de todas as obrigações legais e saneamento das pendências, o feito é transferido para a condição de encerrado, com baixa definitiva no sistema.


\subsubsection{14.1.1 \textendash{} Requisitos prévios}
\label{\detokenize{13baixa definitiva:requisitos-previos}}
\sphinxAtStartPar
Antes de realizar o arquivamento definitivo, a Secretaria deve:
\begin{itemize}
\item {} 
\sphinxAtStartPar
\sphinxstylestrong{Verificar a existência de sentença penal transitada em julgado}, seja ela:
\begin{itemize}
\item {} 
\sphinxAtStartPar
\sphinxstylestrong{Condenatória:} confirmar se a Guia de Execução foi devidamente expedida e encaminhada à Vara de Execuções Penais (BNMP/SEEU), conforme Portaria PTJ nº 2897/2023.

\item {} 
\sphinxAtStartPar
\sphinxstylestrong{Absolutória:} verificar se não há pendências de bens apreendidos, medidas cautelares, medidas protetivas ou mandados de prisão em aberto.

\item {} 
\sphinxAtStartPar
\sphinxstylestrong{Extintiva da punibilidade:} conferir se houve saneamento de todas as obrigações correlatas (ex: comunicação eleitoral, baixa de cautelares, liberação de bens, certidão de extinção de punibilidade por morte no BNMP).

\end{itemize}

\item {} 
\sphinxAtStartPar
\sphinxstylestrong{Verificar a aba “Informações Adicionais \textgreater{} Apreensões” no PROJUDI:}
\begin{itemize}
\item {} 
\sphinxAtStartPar
Se constarem bens apreendidos, verificar nos autos se já houve destinação judicial (perdimento, restituição, leilão, doação, etc.).

\item {} 
\sphinxAtStartPar
Se não houver registro na aba, compulsar os autos para certificar se não restam bens pendentes de destinação.

\end{itemize}

\end{itemize}

\begin{sphinxadmonition}{note}{Nota:}
\sphinxAtStartPar
Somente após o saneamento de todas as pendências (bens, cautelares, guias, medidas), poderá ser realizado o arquivamento definitivo.
\end{sphinxadmonition}


\subsection{14.2 \textendash{} Procedimento no PROJUDI}
\label{\detokenize{13baixa definitiva:procedimento-no-projudi}}
\sphinxAtStartPar
Com as verificações concluídas, proceder da seguinte forma:
\begin{enumerate}
\sphinxsetlistlabels{\arabic}{enumi}{enumii}{}{.}%
\item {} 
\sphinxAtStartPar
Acessar a aba “Movimentações”.

\item {} 
\sphinxAtStartPar
Selecionar “Movimentar a partir desta”.

\item {} 
\sphinxAtStartPar
Clicar em “Arquivar processo”.

\item {} 
\sphinxAtStartPar
Selecionar a opção “Arquivamento Definitivo”.

\item {} 
\sphinxAtStartPar
Conferir se o sistema lançou corretamente a movimentação final e se o processo foi transferido para a pasta “Processos Encerrados”.

\end{enumerate}


\subsection{14.3 \textendash{} Observações importantes}
\label{\detokenize{13baixa definitiva:observacoes-importantes}}\begin{itemize}
\item {} 
\sphinxAtStartPar
Sempre conferir se todas as movimentações obrigatórias foram devidamente lançadas (ex: expedição de guias, comunicação eleitoral, comunicação à vítima, baixa de cautelares).

\item {} 
\sphinxAtStartPar
No caso de sentença absolutória com réu preso, verificar se o Alvará de Soltura foi expedido e cumprido.

\item {} 
\sphinxAtStartPar
Nos casos de penas restritivas de direito e multa, verificar se a Guia de Execução foi remetida à VEMEPA antes da baixa.

\item {} 
\sphinxAtStartPar
Se houver recurso pendente de julgamento, \sphinxstylestrong{não realizar o arquivamento definitivo}, mantendo o processo em tramitação até o trânsito em julgado.

\end{itemize}

\sphinxstepscope


\section{Manual PROJUDI}
\label{\detokenize{manualprojudi:manual-projudi}}\label{\detokenize{manualprojudi::doc}}
\sphinxstepscope


\subsection{Sistema Projudi: Visão Geral}
\label{\detokenize{projud_01_visaogeral:sistema-projudi-visao-geral}}\label{\detokenize{projud_01_visaogeral::doc}}

\subsubsection{Vantagens e Funcionalidades da Tela Externa}
\label{\detokenize{projud_01_visaogeral:vantagens-e-funcionalidades-da-tela-externa}}
\sphinxAtStartPar
O sistema \sphinxstylestrong{ProJUDI} é um programa de computador acessado via internet, com funcionamento simples e seguro. Seu principal objetivo é \sphinxstylestrong{agilizar a justiça}, \sphinxstylestrong{reduzir custos}, \sphinxstylestrong{aumentar a capacidade de processamento} e \sphinxstylestrong{facilitar o trabalho} de advogados, servidores e usuários em geral.

\sphinxAtStartPar
Vantagens do sistema:
\begin{itemize}
\item {} 
\sphinxAtStartPar
Sistema sem barreiras ou fronteiras

\item {} 
\sphinxAtStartPar
Acesso aos processos de qualquer lugar com internet

\item {} 
\sphinxAtStartPar
Compatível com computadores, tablets e celulares

\item {} 
\sphinxAtStartPar
Certificação digital nativa (sem necessidade de token)

\item {} 
\sphinxAtStartPar
Automação de rotinas processuais

\item {} 
\sphinxAtStartPar
Rapidez na tramitação

\item {} 
\sphinxAtStartPar
Segurança de dados com controle de acesso e assinatura digital

\item {} 
\sphinxAtStartPar
Redundância de dados e controle antivírus

\end{itemize}

\sphinxAtStartPar
Funcionalidades da tela inicial:
\begin{itemize}
\item {} 
\sphinxAtStartPar
Link para cadastro de advogado com geração de certificado digital

\item {} 
\sphinxAtStartPar
Consulta pública por foro, número do processo, partes ou OAB

\item {} 
\sphinxAtStartPar
Solicitação de certidão negativa

\item {} 
\sphinxAtStartPar
Calendário com histórico de disponibilidade do sistema

\item {} 
\sphinxAtStartPar
Listagem de grandes demandantes (para citação/intimação online)

\item {} 
\sphinxAtStartPar
Contatos do suporte técnico

\item {} 
\sphinxAtStartPar
Notícias importantes do sistema

\end{itemize}

\sphinxAtStartPar
Demonstração de acesso:
\begin{enumerate}
\sphinxsetlistlabels{\arabic}{enumi}{enumii}{}{.}%
\item {} 
\sphinxAtStartPar
Acesse o site do TJAM

\item {} 
\sphinxAtStartPar
Clique no ícone ou link “Projudi”

\item {} 
\sphinxAtStartPar
Navegue pelas funcionalidades da tela externa antes do login

\end{enumerate}


\bigskip\hrule\bigskip



\subsubsection{Perfis do Sistema Projudi}
\label{\detokenize{projud_01_visaogeral:perfis-do-sistema-projudi}}
\sphinxAtStartPar
Diferenças entre as áreas:

\sphinxAtStartPar
Área Cível:
\begin{itemize}
\item {} 
\sphinxAtStartPar
Mesa do Analista / Técnico

\item {} 
\sphinxAtStartPar
Início

\item {} 
\sphinxAtStartPar
Citações e Intimações

\item {} 
\sphinxAtStartPar
Análise de Juntada

\item {} 
\sphinxAtStartPar
Outros Cumprimentos

\item {} 
\sphinxAtStartPar
Lembretes

\item {} 
\sphinxAtStartPar
Filas Processuais

\end{itemize}

\sphinxAtStartPar
Área Criminal:

\sphinxAtStartPar
Inclui todos os itens da área cível, com acréscimos:
\begin{itemize}
\item {} 
\sphinxAtStartPar
\sphinxstylestrong{Cumprimentos de Medidas}

\item {} 
\sphinxAtStartPar
\sphinxstylestrong{Mesa do Escrivão Criminal}

\end{itemize}

\sphinxAtStartPar
Aplicações:
\begin{itemize}
\item {} 
\sphinxAtStartPar
Varas Criminais

\item {} 
\sphinxAtStartPar
Juizados Especiais Criminais (Violência Doméstica, Tráfico, Trânsito)

\item {} 
\sphinxAtStartPar
Vara da Infância e Juventude Criminal

\item {} 
\sphinxAtStartPar
Vara de Meio Ambiente Criminal

\end{itemize}

\sphinxAtStartPar
Área Cível abrange:
\begin{itemize}
\item {} 
\sphinxAtStartPar
Varas Cíveis

\item {} 
\sphinxAtStartPar
Juizados Especiais Cíveis e da Fazenda Pública

\item {} 
\sphinxAtStartPar
Registros Públicos

\item {} 
\sphinxAtStartPar
Infância e Juventude Cível

\item {} 
\sphinxAtStartPar
Meio Ambiente Cível

\end{itemize}

\sphinxAtStartPar
O curso é modular, considerando:
\begin{itemize}
\item {} 
\sphinxAtStartPar
Perfil do juiz, assessor, técnico

\item {} 
\sphinxAtStartPar
Funcionalidades específicas de 1º e 2º grau

\end{itemize}


\bigskip\hrule\bigskip



\subsubsection{Principais Funcionalidades}
\label{\detokenize{projud_01_visaogeral:principais-funcionalidades}}
\sphinxAtStartPar
O sistema ProJUDI é \sphinxstylestrong{100\% web}:
\begin{itemize}
\item {} 
\sphinxAtStartPar
Acesso de qualquer local com internet

\item {} 
\sphinxAtStartPar
Não exige download, aplicativo, nem VPN

\item {} 
\sphinxAtStartPar
Certificado digital gerado no próprio sistema

\end{itemize}

\sphinxAtStartPar
Funcionalidades destacadas:

\sphinxAtStartPar
\sphinxstylestrong{1. Análise Múltipla de Processos}

\sphinxAtStartPar
Disponível em:
\begin{itemize}
\item {} 
\sphinxAtStartPar
Citações e Intimações

\item {} 
\sphinxAtStartPar
Decurso de Prazo

\item {} 
\sphinxAtStartPar
Análise de Juntadas

\item {} 
\sphinxAtStartPar
Retorno de Conclusão

\end{itemize}

\sphinxAtStartPar
Exemplo:
\begin{itemize}
\item {} 
\sphinxAtStartPar
Selecionar processos

\item {} 
\sphinxAtStartPar
Clicar em \sphinxstyleemphasis{“Analisar em lote”}

\item {} 
\sphinxAtStartPar
Executar ações em massa: citar, intimar, suspender, arquivar

\end{itemize}

\sphinxAtStartPar
\sphinxstylestrong{2. Contagem de Prazos}

\sphinxAtStartPar
Disponível na aba \sphinxstyleemphasis{Prazos} do processo:
\begin{itemize}
\item {} 
\sphinxAtStartPar
Detalhamento automático da contagem

\item {} 
\sphinxAtStartPar
Início e término do prazo

\item {} 
\sphinxAtStartPar
Desconsideração de feriados

\item {} 
\sphinxAtStartPar
Visualização por clique na data de leitura

\end{itemize}

\sphinxAtStartPar
\sphinxstylestrong{3. Mesa do Escrivão Criminal}

\sphinxAtStartPar
Recursos exclusivos da área criminal:
\begin{itemize}
\item {} 
\sphinxAtStartPar
Listas de processos por categoria

\item {} 
\sphinxAtStartPar
Prescrições vencidas e a vencer

\item {} 
\sphinxAtStartPar
Monitoramento eletrônico

\item {} 
\sphinxAtStartPar
Prisões provisórias sem sentença (90+ dias)

\end{itemize}

\sphinxAtStartPar
\sphinxstylestrong{4. Prescrição e Decadência}
\begin{itemize}
\item {} 
\sphinxAtStartPar
Acesso pelo menu \sphinxstyleemphasis{Informações Adicionais} \textgreater{} \sphinxstyleemphasis{Infrações Penais}

\item {} 
\sphinxAtStartPar
Tela de cadastro exibe: data do fato, denúncia, sentença

\item {} 
\sphinxAtStartPar
Detalhamento do cálculo da prescrição da pena

\end{itemize}

\sphinxstepscope


\subsection{Certificado e Senha}
\label{\detokenize{projud_02_senhacertificado:certificado-e-senha}}\label{\detokenize{projud_02_senhacertificado::doc}}
\sphinxAtStartPar
Bem\sphinxhyphen{}vindo ao curso do sistema \sphinxstylestrong{ProJUDI}. Nesta primeira aula, vamos tratar sobre os \sphinxstylestrong{certificados digitais} e a \sphinxstylestrong{senha de acesso ao sistema}.


\subsubsection{Certificado Digital Interno}
\label{\detokenize{projud_02_senhacertificado:certificado-digital-interno}}
\sphinxAtStartPar
Diferentemente de outros sistemas, o ProJUDI \sphinxstylestrong{não exige token físico}. O certificado digital é gerado \sphinxstylestrong{dentro do próprio sistema}, facilitando o uso e reduzindo custos.

\sphinxAtStartPar
\sphinxstylestrong{Procedimento para gerar o certificado digital:}
\begin{enumerate}
\sphinxsetlistlabels{\arabic}{enumi}{enumii}{}{.}%
\item {} 
\sphinxAtStartPar
Acesse o sistema com seu login e senha.

\item {} 
\sphinxAtStartPar
Na tela inicial (mesa do analista), vá até a aba superior \sphinxstylestrong{“Outros”}.

\item {} 
\sphinxAtStartPar
Clique em \sphinxstylestrong{“Listar Certificados”}.

\item {} 
\sphinxAtStartPar
Caso não exista nenhum certificado, clique em \sphinxstylestrong{“Novo”}.

\item {} 
\sphinxAtStartPar
Na tela \sphinxstyleemphasis{Emitir Identidade Digital}, informe a senha desejada.

\item {} 
\sphinxAtStartPar
Clique em \sphinxstylestrong{“Criar”}.

\item {} 
\sphinxAtStartPar
Certificado criado com sucesso.

\end{enumerate}

\sphinxAtStartPar
\sphinxstylestrong{Importante:} Este certificado será utilizado para a assinatura dos documentos processuais no sistema.


\subsubsection{Revogação e Regeneração do Certificado}
\label{\detokenize{projud_02_senhacertificado:revogacao-e-regeneracao-do-certificado}}
\sphinxAtStartPar
Caso você esqueça a senha do certificado:
\begin{enumerate}
\sphinxsetlistlabels{\arabic}{enumi}{enumii}{}{.}%
\item {} 
\sphinxAtStartPar
Acesse novamente a aba \sphinxstylestrong{“Outros” \textgreater{} “Listar Certificados”}.

\item {} 
\sphinxAtStartPar
Clique sobre o certificado existente.

\item {} 
\sphinxAtStartPar
Clique em \sphinxstylestrong{“Revogar”} e confirme a operação.

\item {} 
\sphinxAtStartPar
Após revogar, clique novamente em \sphinxstylestrong{“Novo”}, crie uma nova senha e o certificado será recriado com sucesso.

\end{enumerate}


\subsubsection{Alteração de Senha de Acesso ao Sistema}
\label{\detokenize{projud_02_senhacertificado:alteracao-de-senha-de-acesso-ao-sistema}}
\sphinxAtStartPar
Se desejar alterar sua senha de login:
\begin{enumerate}
\sphinxsetlistlabels{\arabic}{enumi}{enumii}{}{.}%
\item {} 
\sphinxAtStartPar
Vá até \sphinxstylestrong{“Outros” \textgreater{} “Meus Dados”}.

\item {} 
\sphinxAtStartPar
Clique em \sphinxstylestrong{“Alterar Senha”}.

\item {} 
\sphinxAtStartPar
Digite a nova senha desejada.

\item {} 
\sphinxAtStartPar
Marque a opção \sphinxstyleemphasis{“Replicar senha para todos os perfis”}, se aplicável.

\item {} 
\sphinxAtStartPar
Clique em \sphinxstylestrong{“Salvar”}.

\item {} 
\sphinxAtStartPar
Mensagem de sucesso será exibida: \sphinxstyleemphasis{“Dados registrados com sucesso.”}

\end{enumerate}

\sphinxAtStartPar
Pronto! Sua senha foi alterada e você poderá acessar o sistema com a nova credencial.

\sphinxstepscope


\subsection{Aba Início}
\label{\detokenize{projud_03_abainicio:aba-inicio}}\label{\detokenize{projud_03_abainicio::doc}}
\sphinxAtStartPar
Nesta aula, vamos explorar a \sphinxstylestrong{Aba Início} do sistema ProJUDI e suas funcionalidades. Essa aba fornece um panorama geral da unidade judicial e facilita o acesso a informações estratégicas.


\subsubsection{Informações Apresentadas}
\label{\detokenize{projud_03_abainicio:informacoes-apresentadas}}
\sphinxAtStartPar
Logo ao acessar a aba \sphinxstylestrong{Início}, o sistema apresenta:
\begin{itemize}
\item {} 
\sphinxAtStartPar
\sphinxstylestrong{Nome da Vara ou Juizado}

\item {} 
\sphinxAtStartPar
\sphinxstylestrong{Quantidade de Processos Ativos}

\item {} 
\sphinxAtStartPar
\sphinxstylestrong{Quantidade de Processos Físicos (caso existam)}

\item {} 
\sphinxAtStartPar
\sphinxstylestrong{Processos Paralisados há mais de 30 dias}

\item {} 
\sphinxAtStartPar
\sphinxstylestrong{Processos em Remessa}

\item {} 
\sphinxAtStartPar
\sphinxstylestrong{Últimas Mensagens do Administrador do Sistema}

\end{itemize}


\subsubsection{Processos Ativos}
\label{\detokenize{projud_03_abainicio:processos-ativos}}
\sphinxAtStartPar
Ao clicar sobre o número de processos ativos, o sistema redireciona para uma página com a lista de processos da unidade. Essa página funciona como ferramenta de \sphinxstylestrong{busca e filtragem}:
\begin{itemize}
\item {} 
\sphinxAtStartPar
Filtros por situação: \sphinxstyleemphasis{ativos} ou \sphinxstyleemphasis{arquivados}

\item {} 
\sphinxAtStartPar
Legenda com ícones indicando status dos processos

\item {} 
\sphinxAtStartPar
Colunas com: número do processo, partes, datas, classe e assunto

\item {} 
\sphinxAtStartPar
Possibilidade de ordenação (ascendente ou descendente) ao clicar no triângulo nas colunas

\end{itemize}


\subsubsection{Processos Paralisados (30+ dias)}
\label{\detokenize{projud_03_abainicio:processos-paralisados-30-dias}}
\sphinxAtStartPar
Na tela inicial, há uma seção indicando a quantidade de \sphinxstylestrong{processos paralisados há mais de 30 dias}, tanto:
\begin{itemize}
\item {} 
\sphinxAtStartPar
\sphinxstylestrong{Na secretaria}

\item {} 
\sphinxAtStartPar
\sphinxstylestrong{Em remessa} (exceto os concluídos)

\end{itemize}

\sphinxAtStartPar
Ao clicar no número exibido, o sistema apresenta uma \sphinxstylestrong{lista filtrada} com os processos nessa condição, podendo também ser ordenada pela data de paralisação:
\begin{itemize}
\item {} 
\sphinxAtStartPar
Clique uma vez: mostra o processo mais antigo

\item {} 
\sphinxAtStartPar
Clique novamente: mostra o mais recente dentro do critério de paralisação (mínimo de 30 dias)

\end{itemize}


\subsubsection{Processos em Remessa}
\label{\detokenize{projud_03_abainicio:processos-em-remessa}}
\sphinxAtStartPar
Funciona de maneira semelhante aos paralisados:
\begin{itemize}
\item {} 
\sphinxAtStartPar
Clique no número para visualizar os processos em remessa

\item {} 
\sphinxAtStartPar
Ordene por data para verificar o tempo de paralisação

\end{itemize}


\subsubsection{Mensagens do Administrador}
\label{\detokenize{projud_03_abainicio:mensagens-do-administrador}}
\sphinxAtStartPar
Ao final da aba, o sistema exibe:
\begin{itemize}
\item {} 
\sphinxAtStartPar
\sphinxstylestrong{Últimas mensagens enviadas pelo administrador do sistema}, como alertas, atualizações e comunicados gerais.

\end{itemize}

\sphinxAtStartPar
Observação:

\sphinxAtStartPar
\textgreater{} A estrutura e funcionalidades da aba \sphinxstylestrong{Início} são as mesmas tanto na \sphinxstylestrong{área cível} quanto na \sphinxstylestrong{área criminal}.

\sphinxstepscope


\subsection{Análise Múltipla de Decurso de Prazo na Aba Citações e Intimações}
\label{\detokenize{projud_04_analisemultipla:analise-multipla-de-decurso-de-prazo-na-aba-citacoes-e-intimacoes}}\label{\detokenize{projud_04_analisemultipla::doc}}
\sphinxAtStartPar
Nesta aula, vamos explorar a \sphinxstylestrong{Aba Citações e Intimações} e aprender como realizar a \sphinxstylestrong{análise múltipla de decurso de prazo}, uma funcionalidade muito útil para otimizar a rotina cartorária.


\subsubsection{Objetivo da Aba}
\label{\detokenize{projud_04_analisemultipla:objetivo-da-aba}}
\sphinxAtStartPar
A aba agrupa todos os processos com \sphinxstylestrong{citações e intimações expedidas}, sejam:
\begin{itemize}
\item {} 
\sphinxAtStartPar
\sphinxstylestrong{Online} (eletrônicas)

\item {} 
\sphinxAtStartPar
\sphinxstylestrong{Offline} (pessoais ou por AR)

\end{itemize}

\sphinxAtStartPar
Colunas da Aba:
\begin{enumerate}
\sphinxsetlistlabels{\arabic}{enumi}{enumii}{}{.}%
\item {} 
\sphinxAtStartPar
\sphinxstylestrong{Urgência} \textendash{} processos com citações/intimações urgentes

\item {} 
\sphinxAtStartPar
\sphinxstylestrong{Para conferir}

\item {} 
\sphinxAtStartPar
\sphinxstylestrong{Para expedir}

\item {} 
\sphinxAtStartPar
\sphinxstylestrong{Para informar retorno de AR digital}

\item {} 
\sphinxAtStartPar
\sphinxstylestrong{Para corrigir AR}

\item {} 
\sphinxAtStartPar
\sphinxstylestrong{Decurso de prazo}

\end{enumerate}

\sphinxAtStartPar
\sphinxstylestrong{Observação:} A funcionalidade é a mesma em varas cíveis e criminais.


\subsubsection{Decurso de Prazo}
\label{\detokenize{projud_04_analisemultipla:decurso-de-prazo}}
\sphinxAtStartPar
Na última coluna (\sphinxstyleemphasis{Decurso de Prazo}), estão os processos cujo prazo já decorreu para uma ou ambas as partes. O sistema indica esses casos automaticamente.

\sphinxAtStartPar
Como acessar:
\begin{enumerate}
\sphinxsetlistlabels{\arabic}{enumi}{enumii}{}{.}%
\item {} 
\sphinxAtStartPar
Clique no número correspondente em \sphinxstyleemphasis{Decurso de Prazo}.

\item {} 
\sphinxAtStartPar
Será exibida a tela de listagem com diversas opções de filtro:
\sphinxhyphen{} Número do processo
\sphinxhyphen{} CPF / Nome da parte
\sphinxhyphen{} Data da postagem
\sphinxhyphen{} Tipo: citação, notificação, intimação
\sphinxhyphen{} Status do prazo: aguardando, cumprido, analisado, interrompido
\sphinxhyphen{} Urgência
\sphinxhyphen{} Apenas processos do seu usuário
\sphinxhyphen{} Tipos de AR: digital, devolvido, pendente

\end{enumerate}


\subsubsection{Análise Múltipla}
\label{\detokenize{projud_04_analisemultipla:analise-multipla}}
\sphinxAtStartPar
Para realizar uma análise múltipla:
\begin{enumerate}
\sphinxsetlistlabels{\arabic}{enumi}{enumii}{}{.}%
\item {} 
\sphinxAtStartPar
\sphinxstylestrong{Filtre} e \sphinxstylestrong{selecione} os processos com o mesmo tipo de procedimento.

\item {} 
\sphinxAtStartPar
Clique em \sphinxstylestrong{“Próximo passo”}.

\item {} 
\sphinxAtStartPar
Clique em \sphinxstylestrong{“Adicionar documento”}.

\item {} 
\sphinxAtStartPar
Insira um documento PDF ou selecione um modelo previamente cadastrado:
\sphinxhyphen{} Exemplo: \sphinxstyleemphasis{Certidão de Decurso de Prazo}

\item {} 
\sphinxAtStartPar
Digite sua senha e confirme a inclusão.

\end{enumerate}

\sphinxAtStartPar
Após isso:
\begin{enumerate}
\sphinxsetlistlabels{\arabic}{enumi}{enumii}{}{.}%
\item {} 
\sphinxAtStartPar
Clique em \sphinxstylestrong{“Continuar movimentando”}.

\item {} 
\sphinxAtStartPar
Escolha a ação desejada:
\sphinxhyphen{} Remeter ao Ministério Público
\sphinxhyphen{} Concluir para despacho, sentença, decisão
\sphinxhyphen{} Realizar nova intimação
\sphinxhyphen{} Arquivar, suspender, etc.

\item {} 
\sphinxAtStartPar
Preencha o formulário com as informações necessárias (ex.: tipo de decisão, magistrado).

\item {} 
\sphinxAtStartPar
Clique em \sphinxstylestrong{“Próximo passo”} e depois em \sphinxstylestrong{“Salvar”}.

\end{enumerate}

\sphinxAtStartPar
Mensagem final: \sphinxstyleemphasis{“Processos movimentados com sucesso.”}


\subsubsection{Resumo}
\label{\detokenize{projud_04_analisemultipla:resumo}}
\sphinxAtStartPar
Com a \sphinxstylestrong{análise múltipla}, é possível:
\begin{itemize}
\item {} 
\sphinxAtStartPar
Analisar diversos processos de uma só vez

\item {} 
\sphinxAtStartPar
Inserir documentos (certidões, atos ordinatórios)

\item {} 
\sphinxAtStartPar
Movimentar em lote para várias finalidades

\end{itemize}

\sphinxAtStartPar
Essa funcionalidade reduz o tempo de trabalho, aumenta a eficiência e organiza o fluxo de tramitação de forma automatizada.

\sphinxstepscope


\subsection{Análise de Juntadas}
\label{\detokenize{projud_05_analisejuntada:analise-de-juntadas}}\label{\detokenize{projud_05_analisejuntada::doc}}
\sphinxAtStartPar
Nesta aula, vamos explorar a aba \sphinxstylestrong{Análise de Juntadas}, localizada no centro da mesa do analista ou técnico. Ela possui pequenas variações entre a \sphinxstylestrong{área cível} e a \sphinxstylestrong{área criminal}, que também serão abordadas.


\subsubsection{Objetivo da Aba}
\label{\detokenize{projud_05_analisejuntada:objetivo-da-aba}}
\sphinxAtStartPar
A aba \sphinxstylestrong{Análise de Juntadas} recebe a maioria das movimentações processuais realizadas pelas partes e órgãos externos, com \sphinxstylestrong{exceção de}:
\begin{itemize}
\item {} 
\sphinxAtStartPar
Movimentações de \sphinxstylestrong{citações e intimações} (que são tratadas em aba própria)

\item {} 
\sphinxAtStartPar
Movimentações de \sphinxstylestrong{outros cumprimentos}

\end{itemize}

\sphinxAtStartPar
É, portanto, uma das abas com \sphinxstylestrong{maior fluxo de trabalho} na secretaria.


\subsubsection{Filas Disponíveis}
\label{\detokenize{projud_05_analisejuntada:filas-disponiveis}}
\sphinxAtStartPar
Na área cível, a aba apresenta as seguintes filas:
\begin{itemize}
\item {} 
\sphinxAtStartPar
\sphinxstylestrong{Urgência} \textendash{} processos com prioridade

\item {} 
\sphinxAtStartPar
\sphinxstylestrong{Juntadas e Retorno de Conclusão}

\item {} 
\sphinxAtStartPar
\sphinxstylestrong{Para realizar} \textendash{} local de trabalho principal

\end{itemize}

\sphinxAtStartPar
Essencialmente, as duas filas centrais são:
\begin{enumerate}
\sphinxsetlistlabels{\arabic}{enumi}{enumii}{}{.}%
\item {} 
\sphinxAtStartPar
\sphinxstylestrong{Juntadas}

\item {} 
\sphinxAtStartPar
\sphinxstylestrong{Retorno de Conclusão}

\end{enumerate}


\subsubsection{Diferenças na Área Criminal}
\label{\detokenize{projud_05_analisejuntada:diferencas-na-area-criminal}}
\sphinxAtStartPar
Na área criminal, além das filas anteriores, podem aparecer outras:
\begin{itemize}
\item {} 
\sphinxAtStartPar
\sphinxstylestrong{Retorno de Cartas Eletrônicas}

\item {} 
\sphinxAtStartPar
\sphinxstylestrong{Remessas ao Ministério Público}

\item {} 
\sphinxAtStartPar
\sphinxstylestrong{Retorno de Assessoria}

\item {} 
\sphinxAtStartPar
\sphinxstylestrong{Autuação de Guias de Execução Penal}

\item {} 
\sphinxAtStartPar
\sphinxstylestrong{Multas do FPM}

\item {} 
\sphinxAtStartPar
\sphinxstylestrong{Prestações Pecuniárias}, entre outros

\end{itemize}


\subsubsection{Conceitos\sphinxhyphen{}Chave}
\label{\detokenize{projud_05_analisejuntada:conceitos-chave}}
\sphinxAtStartPar
\sphinxstylestrong{Juntadas}:

\sphinxAtStartPar
É a fila onde ficam hospedados todos os processos que tiveram \sphinxstylestrong{movimentações das partes}, como:
\begin{itemize}
\item {} 
\sphinxAtStartPar
Petição inicial (do advogado ou oriunda da termação/distribuição)

\item {} 
\sphinxAtStartPar
Petições intermediárias (contestação, réplica, recursos)

\item {} 
\sphinxAtStartPar
Manifestações do Ministério Público (denúncias, pareceres, promoções)

\item {} 
\sphinxAtStartPar
Documentos da Defensoria, PGE, INSS, entre outros

\end{itemize}

\sphinxAtStartPar
Toda vez que uma parte movimenta o processo e ele retorna à secretaria, ele aparece na fila \sphinxstylestrong{Juntadas}.

\sphinxAtStartPar
\sphinxstylestrong{Retorno de Conclusão}:

\sphinxAtStartPar
É a fila onde ficam os processos que foram \sphinxstylestrong{remetidos ao magistrado} e \sphinxstylestrong{retornaram após despacho, decisão ou sentença}.

\sphinxAtStartPar
No sistema ProJUDI, o envio ao magistrado é feito com a ação \sphinxstylestrong{“Concluo”}. Após o magistrado assinar e devolver, o processo aparece no \sphinxstylestrong{Retorno de Conclusão} para nova tramitação pela secretaria.


\subsubsection{Resumo}
\label{\detokenize{projud_05_analisejuntada:resumo}}\begin{itemize}
\item {} 
\sphinxAtStartPar
Utilize \sphinxstylestrong{Juntadas} para analisar as movimentações originadas das partes e órgãos externos.

\item {} 
\sphinxAtStartPar
Utilize \sphinxstylestrong{Retorno de Conclusão} para tratar os processos que retornaram do magistrado.

\end{itemize}

\sphinxAtStartPar
Essas duas filas são essenciais para o funcionamento do cartório e fazem parte da rotina diária de trabalho nas varas cíveis e criminais.

\sphinxstepscope


\subsection{Análise Múltipla e Unitária de Juntadas}
\label{\detokenize{projud_06_analisejuntada_multiplaeunitaria:analise-multipla-e-unitaria-de-juntadas}}\label{\detokenize{projud_06_analisejuntada_multiplaeunitaria::doc}}
\sphinxAtStartPar
Nesta aula, vamos aprender como realizar a \sphinxstylestrong{análise unitária} e a \sphinxstylestrong{análise múltipla} das juntadas dentro da aba \sphinxstylestrong{Análise de Juntadas}, no sistema ProJUDI.


\subsubsection{Identificação de Processos}
\label{\detokenize{projud_06_analisejuntada_multiplaeunitaria:identificacao-de-processos}}
\sphinxAtStartPar
Os processos podem ser localizados de diversas formas:
\begin{itemize}
\item {} 
\sphinxAtStartPar
\sphinxstylestrong{Número do processo}: digite e filtre diretamente

\item {} 
\sphinxAtStartPar
\sphinxstylestrong{Por grupos}:
\sphinxhyphen{} Analista
\sphinxhyphen{} Magistrado
\sphinxhyphen{} Advogado
\sphinxhyphen{} Membro do Ministério Público

\item {} 
\sphinxAtStartPar
\sphinxstylestrong{Tipo de movimento}:
\sphinxhyphen{} Petição inicial
\sphinxhyphen{} Ciência
\sphinxhyphen{} Requerimento de habilitação
\sphinxhyphen{} Procuração
\sphinxhyphen{} Processo incidental
\sphinxhyphen{} Oferecimento de denúncia

\item {} 
\sphinxAtStartPar
\sphinxstylestrong{Situação}:
\sphinxhyphen{} Urgente
\sphinxhyphen{} Concluso

\item {} 
\sphinxAtStartPar
\sphinxstylestrong{Localizador}: permite refinar ainda mais os filtros

\end{itemize}


\subsubsection{Análise Unitária}
\label{\detokenize{projud_06_analisejuntada_multiplaeunitaria:analise-unitaria}}\begin{enumerate}
\sphinxsetlistlabels{\arabic}{enumi}{enumii}{}{.}%
\item {} 
\sphinxAtStartPar
Localize o processo desejado.

\item {} 
\sphinxAtStartPar
Clique com o botão direito sobre o número do processo e selecione \sphinxstylestrong{“Abrir em nova guia”}.

\item {} 
\sphinxAtStartPar
Analise o processo em tela separada (sugestão: monitor secundário).

\item {} 
\sphinxAtStartPar
Acesse a aba \sphinxstylestrong{“Movimentações”} do processo.

\end{enumerate}

\sphinxAtStartPar
\sphinxstylestrong{Importante:} Quando uma juntada é realizada, o sistema gera uma \sphinxstylestrong{pendência}.
\begin{itemize}
\item {} 
\sphinxAtStartPar
Para remover a pendência:
\sphinxhyphen{} Vá até a aba \sphinxstylestrong{“Pendência”}
\sphinxhyphen{} Clique sobre a pendência
\sphinxhyphen{} Selecione a opção \sphinxstylestrong{“Dispensar”}
\sphinxhyphen{} A pendência será removida

\end{itemize}

\sphinxAtStartPar
Sem a remoção da pendência, não é possível movimentar o processo. O sistema apresentará a mensagem:

\sphinxAtStartPar
Movimentação não permitida. O processo possui juntadas pendentes a serem analisadas.

\sphinxAtStartPar
Após dispensar a pendência, é possível:
\begin{itemize}
\item {} 
\sphinxAtStartPar
Enviar concluso

\item {} 
\sphinxAtStartPar
Intimar

\item {} 
\sphinxAtStartPar
Notificar

\item {} 
\sphinxAtStartPar
Ordenar cumprimento

\end{itemize}

\sphinxAtStartPar
\#\#\# Exemplo:
\begin{enumerate}
\sphinxsetlistlabels{\arabic}{enumi}{enumii}{}{.}%
\item {} 
\sphinxAtStartPar
Dispensar pendência

\item {} 
\sphinxAtStartPar
Clicar em \sphinxstylestrong{“Movimentar a partir desta movimentação”}

\item {} 
\sphinxAtStartPar
Selecionar \sphinxstylestrong{“Enviar Concluso”}

\item {} 
\sphinxAtStartPar
Escolher:
\sphinxhyphen{} Tipo de conclusão (ex.: Decisão Inicial)
\sphinxhyphen{} Nome do magistrado
\sphinxhyphen{} Agrupador (opcional)

\item {} 
\sphinxAtStartPar
Finalizar movimentação

\end{enumerate}


\subsubsection{Análise Múltipla}
\label{\detokenize{projud_06_analisejuntada_multiplaeunitaria:analise-multipla}}\begin{enumerate}
\sphinxsetlistlabels{\arabic}{enumi}{enumii}{}{.}%
\item {} 
\sphinxAtStartPar
Selecione vários processos com a mesma natureza de juntada

\item {} 
\sphinxAtStartPar
Clique em \sphinxstylestrong{“Análise Múltipla”}

\item {} 
\sphinxAtStartPar
Na tela de análise múltipla:
\sphinxhyphen{} Clique em \sphinxstylestrong{“Adicionar”}
\sphinxhyphen{} Insira um documento (ato ordinatório, certidão, etc.)
\sphinxhyphen{} Ou selecione diretamente a movimentação (ex.: enviar concluso)

\item {} 
\sphinxAtStartPar
Defina:
\sphinxhyphen{} Tipo de movimentação (ex.: decisão inicial)
\sphinxhyphen{} Magistrado responsável
\sphinxhyphen{} Agrupador (opcional)
\sphinxhyphen{} Localizador (opcional)

\item {} 
\sphinxAtStartPar
Clique em \sphinxstylestrong{“Próximo passo”}

\item {} 
\sphinxAtStartPar
Revise os processos individualmente (opcional)

\item {} 
\sphinxAtStartPar
Clique em \sphinxstylestrong{“Salvar”}

\end{enumerate}

\sphinxAtStartPar
Mensagem de sucesso:

\sphinxAtStartPar
Conclusões analisadas com sucesso.


\subsubsection{Resumo}
\label{\detokenize{projud_06_analisejuntada_multiplaeunitaria:resumo}}\begin{itemize}
\item {} 
\sphinxAtStartPar
A \sphinxstylestrong{análise unitária} permite examinar e movimentar processos individualmente.

\item {} 
\sphinxAtStartPar
A \sphinxstylestrong{análise múltipla} agiliza a tramitação de processos semelhantes.

\item {} 
\sphinxAtStartPar
O uso de pendências e filtros ajuda a organizar e priorizar o fluxo de trabalho.

\end{itemize}

\sphinxstepscope


\subsection{Análise Múltipla e Unitária de Retorno de Conclusão}
\label{\detokenize{projud_07_analisejuntada_multiplaeunitariaconclusos:analise-multipla-e-unitaria-de-retorno-de-conclusao}}\label{\detokenize{projud_07_analisejuntada_multiplaeunitariaconclusos::doc}}
\sphinxAtStartPar
Nesta aula, vamos tratar do \sphinxstylestrong{Retorno de Conclusão}, ou seja, dos processos que foram \sphinxstylestrong{despachados, decididos ou sentenciados pelo magistrado} e que retornam à secretaria para providências.


\subsubsection{Localização da Fila}
\label{\detokenize{projud_07_analisejuntada_multiplaeunitariaconclusos:localizacao-da-fila}}
\sphinxAtStartPar
A fila de \sphinxstylestrong{Retorno de Conclusão} está localizada dentro da aba \sphinxstylestrong{Análise de Juntadas}, tanto na:
\begin{itemize}
\item {} 
\sphinxAtStartPar
Área cível quanto na área criminal

\item {} 
\sphinxAtStartPar
Mesa de analista e técnico

\end{itemize}

\sphinxAtStartPar
Para acessar:
\begin{enumerate}
\sphinxsetlistlabels{\arabic}{enumi}{enumii}{}{.}%
\item {} 
\sphinxAtStartPar
Clique na aba \sphinxstyleemphasis{Análise de Juntadas}

\item {} 
\sphinxAtStartPar
Clique sobre o número ao lado de \sphinxstyleemphasis{Retorno de Conclusão}

\end{enumerate}


\subsubsection{Tela de Conclusões}
\label{\detokenize{projud_07_analisejuntada_multiplaeunitariaconclusos:tela-de-conclusoes}}
\sphinxAtStartPar
Na tela aberta (\sphinxstyleemphasis{Conclusões}), é possível filtrar:
\begin{itemize}
\item {} 
\sphinxAtStartPar
\sphinxstylestrong{Número do processo}

\item {} 
\sphinxAtStartPar
\sphinxstylestrong{Tipo de conclusão}: despacho, decisão, sentença, julgamento do mérito

\item {} 
\sphinxAtStartPar
\sphinxstylestrong{Situação}: com magistrado, aguardando cartório

\item {} 
\sphinxAtStartPar
\sphinxstylestrong{Privatividade}: pessoalidade privativa ou não

\item {} 
\sphinxAtStartPar
\sphinxstylestrong{Responsável pela conclusão}

\item {} 
\sphinxAtStartPar
\sphinxstylestrong{Agrupador} (tema abordado em aula específica)

\item {} 
\sphinxAtStartPar
\sphinxstylestrong{Data de retorno} (ordenar ascendente/descendente)

\item {} 
\sphinxAtStartPar
\sphinxstylestrong{Classe e tipo de conclusão} (ordem alfabética)

\item {} 
\sphinxAtStartPar
\sphinxstylestrong{Preanálise}: identifica o assessor que minutou a decisão

\end{itemize}


\subsubsection{Análise Unitária}
\label{\detokenize{projud_07_analisejuntada_multiplaeunitariaconclusos:analise-unitaria}}\begin{enumerate}
\sphinxsetlistlabels{\arabic}{enumi}{enumii}{}{.}%
\item {} 
\sphinxAtStartPar
Clique com o botão direito na data de retorno do processo

\item {} 
\sphinxAtStartPar
Selecione \sphinxstylestrong{“Abrir em nova guia”} (sugestão: usar monitor secundário)

\item {} 
\sphinxAtStartPar
Leia a decisão e verifique a providência a ser tomada

\item {} 
\sphinxAtStartPar
Retorne à tela de conclusões e clique em \sphinxstylestrong{“Analisar”}

\item {} 
\sphinxAtStartPar
Clique em \sphinxstylestrong{“Finalizar conclusão pendente”}

\end{enumerate}

\sphinxAtStartPar
Após finalizar, você pode:
\begin{itemize}
\item {} 
\sphinxAtStartPar
Acessar o processo

\item {} 
\sphinxAtStartPar
Ir à aba \sphinxstyleemphasis{Movimentações}

\item {} 
\sphinxAtStartPar
Clicar em \sphinxstyleemphasis{Movimentar a partir desta movimentação}

\item {} 
\sphinxAtStartPar
Realizar ações como:
\sphinxhyphen{} Intimar partes
\sphinxhyphen{} Enviar concluso
\sphinxhyphen{} Notificar
\sphinxhyphen{} Arquivar
\sphinxhyphen{} Suspender

\end{itemize}

\sphinxAtStartPar
\sphinxstylestrong{Exemplo:} Decisão determina arquivamento \(\rightarrow\) intima as duas partes \(\rightarrow\) define prazo \(\rightarrow\) salva.


\subsubsection{Análise Múltipla}
\label{\detokenize{projud_07_analisejuntada_multiplaeunitariaconclusos:analise-multipla}}\begin{enumerate}
\sphinxsetlistlabels{\arabic}{enumi}{enumii}{}{.}%
\item {} 
\sphinxAtStartPar
Filtre processos com o \sphinxstylestrong{mesmo tipo de conclusão} (ex.: sentença procedente)

\item {} 
\sphinxAtStartPar
Clique em \sphinxstylestrong{“Análise Múltipla”}

\item {} 
\sphinxAtStartPar
Na tela de análise múltipla:
\sphinxhyphen{} Escolha a movimentação desejada:
\begin{itemize}
\item {} 
\sphinxAtStartPar
Intimar partes

\item {} 
\sphinxAtStartPar
Citar partes

\item {} 
\sphinxAtStartPar
Suspender processos

\item {} 
\sphinxAtStartPar
Arquivar processos

\item {} 
\sphinxAtStartPar
Realizar remessa (MP, Defensoria, Distribuidor, etc.)

\end{itemize}

\item {} 
\sphinxAtStartPar
Informe os advogados:
\sphinxhyphen{} Promovente (autor)
\sphinxhyphen{} Promovido (réu)

\item {} 
\sphinxAtStartPar
Estipule o prazo (ex.: 15 dias úteis)

\item {} 
\sphinxAtStartPar
Clique em \sphinxstylestrong{“Próximo passo”}

\item {} 
\sphinxAtStartPar
Revise e \sphinxstylestrong{confira} os processos

\item {} 
\sphinxAtStartPar
Clique em \sphinxstylestrong{“Salvar”}

\end{enumerate}

\sphinxAtStartPar
Mensagem de sucesso:

\sphinxAtStartPar
Conclusões analisadas com sucesso.


\subsubsection{Resumo}
\label{\detokenize{projud_07_analisejuntada_multiplaeunitariaconclusos:resumo}}\begin{itemize}
\item {} 
\sphinxAtStartPar
O \sphinxstylestrong{retorno de conclusão} reúne todos os processos devolvidos pelo juiz

\item {} 
\sphinxAtStartPar
Pode ser analisado individualmente ou em lote

\item {} 
\sphinxAtStartPar
Permite movimentações posteriores, como intimações, remessas e arquivamentos

\item {} 
\sphinxAtStartPar
O uso de filtros e agrupadores torna o trabalho mais organizado e produtivo

\end{itemize}

\sphinxstepscope


\subsection{Análise Múltipla com Movimentação Múltipla}
\label{\detokenize{projud_08_multiplaeunitariamov:analise-multipla-com-movimentacao-multipla}}\label{\detokenize{projud_08_multiplaeunitariamov::doc}}
\sphinxAtStartPar
Nesta aula, vamos aprender a realizar a \sphinxstylestrong{análise múltipla com movimentações consecutivas} no sistema ProJUDI. Trata\sphinxhyphen{}se de um procedimento em que, após a análise múltipla, o servidor continua com outras movimentações sobre os mesmos processos — como intimações seguidas de suspensão, por exemplo.


\subsubsection{Cenário Prático}
\label{\detokenize{projud_08_multiplaeunitariamov:cenario-pratico}}
\sphinxAtStartPar
Exemplo: processos cujo \sphinxstylestrong{agrupador} é \sphinxstyleemphasis{Decisão IRDR e Tarifas}.

\sphinxAtStartPar
Objetivo:
\begin{enumerate}
\sphinxsetlistlabels{\arabic}{enumi}{enumii}{}{.}%
\item {} 
\sphinxAtStartPar
\sphinxstylestrong{Intimar as partes} sobre a decisão

\item {} 
\sphinxAtStartPar
Em seguida, \sphinxstylestrong{suspender os processos}

\end{enumerate}


\subsubsection{Procedimentos}
\label{\detokenize{projud_08_multiplaeunitariamov:procedimentos}}
\sphinxAtStartPar
\sphinxstylestrong{Passo 1: Acessar os processos}
\begin{itemize}
\item {} 
\sphinxAtStartPar
Vá em \sphinxstylestrong{Retorno de Conclusão}

\item {} 
\sphinxAtStartPar
Utilize o \sphinxstylestrong{filtro por agrupador}: ex.: “Decisão IRDR e Tarifas”

\item {} 
\sphinxAtStartPar
Clique em \sphinxstylestrong{Análise Múltipla}

\item {} 
\sphinxAtStartPar
Selecione os processos

\item {} 
\sphinxAtStartPar
Clique em \sphinxstylestrong{Próximo passo}

\end{itemize}

\sphinxAtStartPar
\sphinxstylestrong{Passo 2: Primeira movimentação \sphinxhyphen{} Intimação}
\begin{itemize}
\item {} 
\sphinxAtStartPar
Selecione a ação: \sphinxstylestrong{Intimar}

\item {} 
\sphinxAtStartPar
Escolha os destinatários:
\sphinxhyphen{} Advogado do promovente (autor)
\sphinxhyphen{} Advogado do promovido (réu)

\item {} 
\sphinxAtStartPar
Defina:
\sphinxhyphen{} Prazo (em dias úteis ou corridos)
\sphinxhyphen{} Localizador (opcional): ex.: \sphinxstyleemphasis{aguardando decisão IRDR}

\item {} 
\sphinxAtStartPar
Clique em \sphinxstylestrong{Próximo passo}

\item {} 
\sphinxAtStartPar
Revise os processos

\item {} 
\sphinxAtStartPar
Clique em \sphinxstylestrong{Continuar movimentando}

\end{itemize}

\sphinxAtStartPar
⚠️ \sphinxstylestrong{Atenção:} Se algum processo tiver \sphinxstylestrong{duas pendências (ex: juntada + retorno de conclusão)}, o sistema não permitirá a movimentação.
\begin{itemize}
\item {} 
\sphinxAtStartPar
Solução:
\sphinxhyphen{} Acesse o processo
\sphinxhyphen{} Elimine uma das pendências (ex.: vá em \sphinxstyleemphasis{Análise de Juntada} e clique em \sphinxstyleemphasis{Dispensar})
\sphinxhyphen{} Retorne e repita o procedimento

\end{itemize}

\sphinxAtStartPar
\sphinxstylestrong{Passo 3: Segunda movimentação \sphinxhyphen{} Suspensão}

\sphinxAtStartPar
Após a intimação:
\begin{itemize}
\item {} 
\sphinxAtStartPar
Clique em \sphinxstylestrong{Movimentar}

\item {} 
\sphinxAtStartPar
Selecione: \sphinxstylestrong{Suspender}

\item {} 
\sphinxAtStartPar
Preencha:
\sphinxhyphen{} Data de início
\sphinxhyphen{} Tempo determinado? Se \sphinxstyleemphasis{não}, informe o prazo (ex.: conforme suspensão condicional do processo)
\sphinxhyphen{} Tipo de suspensão (ex.: aguardando julgamento do IRDR)

\item {} 
\sphinxAtStartPar
Clique em \sphinxstylestrong{Próximo passo}

\item {} 
\sphinxAtStartPar
Revise os processos

\item {} 
\sphinxAtStartPar
Clique em \sphinxstylestrong{Salvar}

\end{itemize}

\sphinxAtStartPar
Mensagem:

\sphinxAtStartPar
Processos movimentados com sucesso.


\subsubsection{Verificação}
\label{\detokenize{projud_08_multiplaeunitariamov:verificacao}}
\sphinxAtStartPar
Ao acessar um dos processos movimentados:
\begin{itemize}
\item {} 
\sphinxAtStartPar
\sphinxstylestrong{Status:} “Suspenso” ou “Sobrestado”

\item {} 
\sphinxAtStartPar
\sphinxstylestrong{Movimentações:}
\sphinxhyphen{} Análise da decisão
\sphinxhyphen{} Intimação das partes
\sphinxhyphen{} Suspensão do processo

\end{itemize}


\subsubsection{Resumo}
\label{\detokenize{projud_08_multiplaeunitariamov:resumo}}\begin{itemize}
\item {} 
\sphinxAtStartPar
Com a análise múltipla + movimentação contínua, é possível aplicar \sphinxstylestrong{duas ou mais ações em sequência} sobre um grupo de processos.

\item {} 
\sphinxAtStartPar
Ideal para decisões com \sphinxstylestrong{efeitos em cadeia}, como:
\sphinxhyphen{} Intimação + suspensão
\sphinxhyphen{} Citação + remessa
\sphinxhyphen{} Intimação + arquivamento

\item {} 
\sphinxAtStartPar
Gera \sphinxstylestrong{economia de tempo}: dezenas de movimentações em poucos minutos

\end{itemize}

\sphinxstepscope


\subsection{Aba Outros Cumprimentos}
\label{\detokenize{projud_09_outroscumprimentos:aba-outros-cumprimentos}}\label{\detokenize{projud_09_outroscumprimentos::doc}}
\sphinxAtStartPar
Nesta aula, vamos conhecer a aba \sphinxstylestrong{Outros Cumprimentos}, localizada no centro da mesa do sistema ProJUDI. Essa aba reúne \sphinxstylestrong{documentos assinados pela secretaria ou magistrado} que \sphinxstylestrong{não se enquadram} nas abas de Citações e Intimações ou Análise de Juntadas.


\subsubsection{Objetivo da Aba}
\label{\detokenize{projud_09_outroscumprimentos:objetivo-da-aba}}
\sphinxAtStartPar
Agrupar documentos processuais diversos que exigem assinatura e tramitação, tais como:
\begin{itemize}
\item {} 
\sphinxAtStartPar
Alvarás

\item {} 
\sphinxAtStartPar
Alvarás eletrônicos

\item {} 
\sphinxAtStartPar
Cartas eletrônicas (enviadas e recebidas)

\item {} 
\sphinxAtStartPar
Mandados

\item {} 
\sphinxAtStartPar
Guias de recolhimento definitiva (área criminal)

\item {} 
\sphinxAtStartPar
Outros documentos

\end{itemize}


\subsubsection{Colunas Disponíveis}
\label{\detokenize{projud_09_outroscumprimentos:colunas-disponiveis}}
\sphinxAtStartPar
A aba está dividida em colunas que indicam o status de cada documento:
\begin{enumerate}
\sphinxsetlistlabels{\arabic}{enumi}{enumii}{}{.}%
\item {} 
\sphinxAtStartPar
\sphinxstylestrong{Para conferir} \textendash{} Documentos expedidos e aguardando assinatura (magistrado ou servidor)

\item {} 
\sphinxAtStartPar
\sphinxstylestrong{Para expedir} \textendash{} Pendências criadas para geração de documentos ainda não expedidos

\item {} 
\sphinxAtStartPar
\sphinxstylestrong{Com urgência} \textendash{} Documentos classificados como prioritários

\item {} 
\sphinxAtStartPar
\sphinxstylestrong{Devolvido pelo juiz} \textendash{} Documentos que já foram assinados e devolvidos

\item {} 
\sphinxAtStartPar
\sphinxstylestrong{Decurso do prazo} \textendash{} Documentos cujo prazo para cumprimento já expirou

\end{enumerate}


\subsubsection{Funcionalidade por Coluna}
\label{\detokenize{projud_09_outroscumprimentos:funcionalidade-por-coluna}}
\sphinxAtStartPar
\#\#\# 1. Para Conferir
\begin{itemize}
\item {} 
\sphinxAtStartPar
Documentos já expedidos, aguardando assinatura.

\item {} 
\sphinxAtStartPar
Exemplo: Mandado aguardando assinatura do magistrado.

\item {} 
\sphinxAtStartPar
Acesso:
\sphinxhyphen{} Clique no número correspondente
\sphinxhyphen{} Na tela do documento (ex: Mandado), é possível:
\begin{itemize}
\item {} 
\sphinxAtStartPar
Verificar dados do documento

\item {} 
\sphinxAtStartPar
Editar documento (se necessário)

\item {} 
\sphinxAtStartPar
Visualizar PDF ou versão editável

\item {} 
\sphinxAtStartPar
Consultar o despacho referenciado na expedição

\end{itemize}

\end{itemize}

\sphinxAtStartPar
\#\#\# 2. Para Expedir
\begin{itemize}
\item {} 
\sphinxAtStartPar
Documentos com pendência criada, mas ainda \sphinxstylestrong{não expedidos}.

\item {} 
\sphinxAtStartPar
Exemplo: Mandado ainda sem arquivo gerado.

\item {} 
\sphinxAtStartPar
Acesso:
\sphinxhyphen{} Clique na quantidade listada
\sphinxhyphen{} Clique sobre o documento
\sphinxhyphen{} Use o botão \sphinxstylestrong{“Analisar”}
\sphinxhyphen{} Na nova tela, insira:
\begin{itemize}
\item {} 
\sphinxAtStartPar
Tipo de arquivo

\item {} 
\sphinxAtStartPar
Descrição

\item {} 
\sphinxAtStartPar
Modelo (mandado, alvará, etc.)

\end{itemize}
\begin{itemize}
\item {} 
\sphinxAtStartPar
\sphinxstyleemphasis{Obs.: A criação do documento será explicada na próxima aula.}

\end{itemize}

\end{itemize}

\sphinxAtStartPar
\#\#\# 3. Com Urgência
\begin{itemize}
\item {} 
\sphinxAtStartPar
Reúne os documentos prioritários

\item {} 
\sphinxAtStartPar
Utilizado como ferramenta de triagem e fluxo ágil

\end{itemize}

\sphinxAtStartPar
\#\#\# 4. Devolvido pelo Juiz
\begin{itemize}
\item {} 
\sphinxAtStartPar
Indica que o documento já foi \sphinxstylestrong{assinado pelo magistrado}

\item {} 
\sphinxAtStartPar
Pronto para cumprimento posterior

\end{itemize}

\sphinxAtStartPar
\#\#\# 5. Decurso do Prazo
\begin{itemize}
\item {} 
\sphinxAtStartPar
Indica que o documento (mandado, alvará, carta, etc.) já teve seu \sphinxstylestrong{prazo expirado}

\item {} 
\sphinxAtStartPar
Necessita providência por parte do cartório

\end{itemize}


\subsubsection{Diferenças na Área Criminal}
\label{\detokenize{projud_09_outroscumprimentos:diferencas-na-area-criminal}}
\sphinxAtStartPar
Na \sphinxstylestrong{mesa criminal}, a estrutura é semelhante, com as mesmas colunas.

\sphinxAtStartPar
Diferença principal:
\begin{itemize}
\item {} 
\sphinxAtStartPar
Inclusão da categoria \sphinxstylestrong{“Guia de Recolhimento Definitiva”}, que será gerada a partir da função \sphinxstylestrong{“Ordenar Cumprimento”}

\end{itemize}


\subsubsection{Resumo}
\label{\detokenize{projud_09_outroscumprimentos:resumo}}
\sphinxAtStartPar
A aba \sphinxstylestrong{Outros Cumprimentos} funciona como:
\begin{itemize}
\item {} 
\sphinxAtStartPar
Uma \sphinxstylestrong{fila de trabalho}

\item {} 
\sphinxAtStartPar
Uma \sphinxstylestrong{rede de energia processual} para organizar e cumprir decisões que envolvam documentos diversos

\end{itemize}

\sphinxAtStartPar
Essa aba é essencial para o bom andamento da secretaria e será aprofundada em aulas específicas sobre a expedição de documentos como mandados e alvarás.

\sphinxstepscope


\subsection{Filas Processuais e Mover de Fila}
\label{\detokenize{projud_10_filasprocessuais:filas-processuais-e-mover-de-fila}}\label{\detokenize{projud_10_filasprocessuais::doc}}
\sphinxAtStartPar
Nesta aula, vamos explorar a aba \sphinxstylestrong{Filas Processuais}, um recurso recente do sistema ProJUDI que visa proporcionar \sphinxstylestrong{maior organização e eficiência} na rotina cartorária.


\subsubsection{Objetivo da Aba}
\label{\detokenize{projud_10_filasprocessuais:objetivo-da-aba}}
\sphinxAtStartPar
A aba \sphinxstylestrong{Filas Processuais} permite:
\begin{itemize}
\item {} 
\sphinxAtStartPar
Agrupar processos conforme sua situação processual

\item {} 
\sphinxAtStartPar
Otimizar o fluxo de trabalho

\item {} 
\sphinxAtStartPar
Auxiliar no controle e distribuição de tarefas

\end{itemize}


\subsubsection{Acesso e Utilização}
\label{\detokenize{projud_10_filasprocessuais:acesso-e-utilizacao}}\begin{enumerate}
\sphinxsetlistlabels{\arabic}{enumi}{enumii}{}{.}%
\item {} 
\sphinxAtStartPar
Clique na aba \sphinxstylestrong{Filas Processuais}

\item {} 
\sphinxAtStartPar
Será exibida uma lista com diversas filas, como:
\sphinxhyphen{} Aguardando análise de AR
\sphinxhyphen{} Audiência
\sphinxhyphen{} Preliminar
\sphinxhyphen{} Avaliação
\sphinxhyphen{} Decisão
\sphinxhyphen{} Decurso de prazo
\sphinxhyphen{} Entre outras

\item {} 
\sphinxAtStartPar
Clique em uma das filas (ex: \sphinxstylestrong{Decisão}) para visualizar os processos ali contidos

\item {} 
\sphinxAtStartPar
Use o campo de busca para filtrar processos pelo número

\end{enumerate}


\subsubsection{Movimentações Disponíveis}
\label{\detokenize{projud_10_filasprocessuais:movimentacoes-disponiveis}}
\sphinxAtStartPar
\#\#\# 1. \sphinxstylestrong{Copiar para outra fila}
\begin{itemize}
\item {} 
\sphinxAtStartPar
O processo \sphinxstylestrong{permanece em ambas as filas}

\item {} 
\sphinxAtStartPar
Exemplo: da fila “Decisão” para “Aguardando cálculo de execução”

\end{itemize}

\sphinxAtStartPar
\#\#\# 2. \sphinxstylestrong{Mover para outra fila}
\begin{itemize}
\item {} 
\sphinxAtStartPar
O processo é \sphinxstylestrong{removido da fila atual} e transferido para a nova

\item {} 
\sphinxAtStartPar
Exemplo: da fila “Decisão” para “Aguardando audiência”

\end{itemize}

\sphinxAtStartPar
\#\#\# 3. \sphinxstylestrong{Remover processo da fila}
\begin{itemize}
\item {} 
\sphinxAtStartPar
Exclui o processo de determinada fila, sem movê\sphinxhyphen{}lo para outra

\end{itemize}

\sphinxAtStartPar
\#\#\# 4. \sphinxstylestrong{Mover Múltiplo}
\begin{itemize}
\item {} 
\sphinxAtStartPar
Selecione dois ou mais processos

\item {} 
\sphinxAtStartPar
Clique em \sphinxstylestrong{“Mover”} (canto inferior direito)

\item {} 
\sphinxAtStartPar
Escolha a nova fila e confirme

\end{itemize}

\sphinxAtStartPar
\#\#\# 5. \sphinxstylestrong{Copiar Múltiplo}
\begin{itemize}
\item {} 
\sphinxAtStartPar
Idêntico ao mover múltiplo, mas mantém os processos também na fila original

\end{itemize}


\subsubsection{Verificação}
\label{\detokenize{projud_10_filasprocessuais:verificacao}}\begin{itemize}
\item {} 
\sphinxAtStartPar
Após mover ou copiar, acesse novamente a fila para confirmar se os processos foram corretamente inseridos

\item {} 
\sphinxAtStartPar
Também é possível verificar a \sphinxstylestrong{fila atual} na tela principal do processo

\end{itemize}


\subsubsection{Funcionalidade Extra}
\label{\detokenize{projud_10_filasprocessuais:funcionalidade-extra}}
\sphinxAtStartPar
Na \sphinxstylestrong{tela principal do processo}, é possível:
\begin{itemize}
\item {} 
\sphinxAtStartPar
\sphinxstylestrong{Mover para outra fila}

\item {} 
\sphinxAtStartPar
\sphinxstylestrong{Copiar para outra fila}

\item {} 
\sphinxAtStartPar
\sphinxstylestrong{Excluir da fila}

\end{itemize}

\sphinxAtStartPar
\textgreater{} Essas ações estão disponíveis no painel lateral da página do processo.

\sphinxAtStartPar
⚠️ \sphinxstylestrong{Importante:}
Mover um processo para a fila “Decisão” \sphinxstylestrong{não significa} que ele será automaticamente direcionado para o gabinete ou será incluído em uma minuta.

\sphinxAtStartPar
Exemplo:
\begin{itemize}
\item {} 
\sphinxAtStartPar
Processo movido para a fila “Decisão”

\item {} 
\sphinxAtStartPar
Última movimentação: “Recebido os autos da instância superior”

\item {} 
\sphinxAtStartPar
Necessário ainda:
\sphinxhyphen{} Acessar aba \sphinxstylestrong{Movimentações}
\sphinxhyphen{} Clicar em \sphinxstyleemphasis{Movimentar a partir desta movimentação}
\sphinxhyphen{} Selecionar \sphinxstylestrong{“Enviar Concluso”}

\end{itemize}


\subsubsection{Resumo}
\label{\detokenize{projud_10_filasprocessuais:resumo}}
\sphinxAtStartPar
A aba \sphinxstylestrong{Filas Processuais} é uma ferramenta poderosa para:
\begin{itemize}
\item {} 
\sphinxAtStartPar
Organizar a carga de trabalho

\item {} 
\sphinxAtStartPar
Atribuir tarefas por status ou grupo

\item {} 
\sphinxAtStartPar
Planejar o andamento dos processos com mais clareza

\end{itemize}

\sphinxAtStartPar
Contudo, ela \sphinxstylestrong{não substitui as movimentações formais} exigidas pelo sistema para o encaminhamento ao magistrado ou para atos processuais específicos.

\sphinxstepscope


\subsection{Tela Inicial do Processo}
\label{\detokenize{projud_11_telainicialprocesso:tela-inicial-do-processo}}\label{\detokenize{projud_11_telainicialprocesso::doc}}
\sphinxAtStartPar
Nesta aula, iniciamos uma \sphinxstylestrong{nova sequência de aulas práticas}, explorando o que é possível realizar \sphinxstylestrong{dentro de um processo} no sistema ProJUDI, tanto na \sphinxstylestrong{mesa cível} quanto na \sphinxstylestrong{mesa criminal}.


\subsubsection{Objetivo}
\label{\detokenize{projud_11_telainicialprocesso:objetivo}}
\sphinxAtStartPar
Apresentar a estrutura da \sphinxstylestrong{tela inicial de um processo} no ProJUDI, destacando os campos, abas e funcionalidades disponíveis para alimentação e movimentação processual.


\subsubsection{Acesso ao Processo}
\label{\detokenize{projud_11_telainicialprocesso:acesso-ao-processo}}
\sphinxAtStartPar
Você pode acessar um processo por diversas rotinas, como:
\begin{itemize}
\item {} 
\sphinxAtStartPar
Citações e Intimações

\item {} 
\sphinxAtStartPar
Análise de Juntadas

\item {} 
\sphinxAtStartPar
Outros Cumprimentos

\item {} 
\sphinxAtStartPar
Filas Processuais

\end{itemize}

\sphinxAtStartPar
Para fins didáticos, o processo foi acessado a partir da aba \sphinxstylestrong{Análise de Juntadas}.


\subsubsection{Componentes da Tela Inicial}
\label{\detokenize{projud_11_telainicialprocesso:componentes-da-tela-inicial}}
\sphinxAtStartPar
Ao abrir um processo, a tela apresenta:
\begin{itemize}
\item {} 
\sphinxAtStartPar
\sphinxstylestrong{Número do processo}

\item {} 
\sphinxAtStartPar
\sphinxstylestrong{Advertência} (ex: necessidade de regularização de CPF/CNPJ)

\item {} 
\sphinxAtStartPar
\sphinxstylestrong{Dias de tramitação}

\item {} 
\sphinxAtStartPar
\sphinxstylestrong{Classe e assuntos do processo}

\item {} 
\sphinxAtStartPar
\sphinxstylestrong{Nível de prioridade}

\item {} 
\sphinxAtStartPar
\sphinxstylestrong{Agendamento de audiência}

\item {} 
\sphinxAtStartPar
\sphinxstylestrong{Anotações}

\item {} 
\sphinxAtStartPar
\sphinxstylestrong{Análise automática do processo}

\item {} 
\sphinxAtStartPar
\sphinxstylestrong{Pendências existentes}

\end{itemize}

\sphinxAtStartPar
Ações disponíveis (barra central superior):
\begin{itemize}
\item {} 
\sphinxAtStartPar
\sphinxstylestrong{Inserir pedido incidental}

\item {} 
\sphinxAtStartPar
\sphinxstylestrong{Juntar documento}

\item {} 
\sphinxAtStartPar
\sphinxstylestrong{Peticionar}

\item {} 
\sphinxAtStartPar
\sphinxstylestrong{Navegar} \textendash{} usada com frequência para consultar movimentações

\item {} 
\sphinxAtStartPar
\sphinxstylestrong{Exportar} \textendash{} para salvar documentos do processo

\item {} 
\sphinxAtStartPar
\sphinxstylestrong{Voltar} \textendash{} retorna à tela anterior

\end{itemize}

\sphinxAtStartPar
Abas de informações (parte inferior):
\begin{itemize}
\item {} 
\sphinxAtStartPar
\sphinxstylestrong{Informações gerais} \textendash{} dados básicos do processo

\item {} 
\sphinxAtStartPar
\sphinxstylestrong{Informações adicionais} \textendash{} dados complementares

\item {} 
\sphinxAtStartPar
\sphinxstylestrong{Partes} \textendash{} lista das partes envolvidas

\item {} 
\sphinxAtStartPar
\sphinxstylestrong{Movimentações} \textendash{} histórico de atos processuais

\item {} 
\sphinxAtStartPar
\sphinxstylestrong{Apensos} \textendash{} processos vinculados

\item {} 
\sphinxAtStartPar
\sphinxstylestrong{Vínculos} \textendash{} conexões com outros cadastros

\item {} 
\sphinxAtStartPar
\sphinxstylestrong{Guias de custas} \textendash{} informações sobre pagamento de custas

\end{itemize}


\subsubsection{Encaminhamento}
\label{\detokenize{projud_11_telainicialprocesso:encaminhamento}}
\sphinxAtStartPar
A partir das próximas aulas, cada um desses itens será abordado de forma \sphinxstylestrong{detalhada e prática}, mostrando como:
\begin{itemize}
\item {} 
\sphinxAtStartPar
Inserir documentos

\item {} 
\sphinxAtStartPar
Gerenciar partes e procuradores

\item {} 
\sphinxAtStartPar
Executar movimentações

\item {} 
\sphinxAtStartPar
Utilizar ferramentas de navegação e exportação

\end{itemize}

\sphinxAtStartPar
Essa abordagem será válida tanto para \sphinxstylestrong{varas cíveis} quanto para \sphinxstylestrong{varas criminais}, destacando aquilo que for específico de cada área, sempre que necessário.

\sphinxstepscope


\subsection{Favoritar, Anotações e Lembretes}
\label{\detokenize{projud_12_favoritar:favoritar-anotacoes-e-lembretes}}\label{\detokenize{projud_12_favoritar::doc}}
\sphinxAtStartPar
Nesta aula, vamos conhecer três ferramentas muito úteis para organização e comunicação dentro do processo digital no sistema ProJUDI:
\begin{itemize}
\item {} 
\sphinxAtStartPar
\sphinxstylestrong{Favoritar}

\item {} 
\sphinxAtStartPar
\sphinxstylestrong{Anotações nos Autos}

\item {} 
\sphinxAtStartPar
\sphinxstylestrong{Lembretes}

\end{itemize}

\sphinxAtStartPar
Esses recursos funcionam como os antigos post\sphinxhyphen{}its nos autos físicos, agora integrados digitalmente ao sistema.


\subsubsection{Favoritar Processos}
\label{\detokenize{projud_12_favoritar:favoritar-processos}}
\sphinxAtStartPar
O recurso de favoritos permite marcar processos com diferentes níveis de prioridade para facilitar a triagem e organização da rotina.

\sphinxAtStartPar
\sphinxstylestrong{Como adicionar um favorito:}
\begin{enumerate}
\sphinxsetlistlabels{\arabic}{enumi}{enumii}{}{.}%
\item {} 
\sphinxAtStartPar
Clique na \sphinxstylestrong{estrela} no topo da tela do processo.

\item {} 
\sphinxAtStartPar
Na tela \sphinxstyleemphasis{Adicionar Favorito}:
\sphinxhyphen{} Escolha a \sphinxstylestrong{relevância}: alta, média ou baixa
\sphinxhyphen{} Descreva o motivo do favoritamento (ex: \sphinxstyleemphasis{Prioridade pessoal}, \sphinxstyleemphasis{Réu preso com 90+ dias}, etc.)

\item {} 
\sphinxAtStartPar
Clique em \sphinxstylestrong{Salvar}

\end{enumerate}

\sphinxAtStartPar
\sphinxstylestrong{Visualização:}
\begin{itemize}
\item {} 
\sphinxAtStartPar
A estrela ficará \sphinxstylestrong{amarela} indicando que o processo está favoritado

\item {} 
\sphinxAtStartPar
Ao passar o cursor sobre a estrela, é possível visualizar:
\sphinxhyphen{} A descrição
\sphinxhyphen{} A relevância definida

\end{itemize}

\sphinxAtStartPar
\sphinxstylestrong{Como encontrar os processos favoritados:}
\begin{enumerate}
\sphinxsetlistlabels{\arabic}{enumi}{enumii}{}{.}%
\item {} 
\sphinxAtStartPar
Vá até a \sphinxstylestrong{aba “Processos”}

\item {} 
\sphinxAtStartPar
Clique em \sphinxstylestrong{“Favoritos”}

\item {} 
\sphinxAtStartPar
Uma lista será exibida com:
\sphinxhyphen{} Número do processo
\sphinxhyphen{} Relevância
\sphinxhyphen{} Descrição

\end{enumerate}


\subsubsection{Anotações nos Autos}
\label{\detokenize{projud_12_favoritar:anotacoes-nos-autos}}
\sphinxAtStartPar
As anotações permitem registrar informações padronizadas dentro do processo.

\sphinxAtStartPar
\sphinxstylestrong{Como inserir uma anotação:}
\begin{enumerate}
\sphinxsetlistlabels{\arabic}{enumi}{enumii}{}{.}%
\item {} 
\sphinxAtStartPar
Clique no ícone de \sphinxstylestrong{caneta}

\item {} 
\sphinxAtStartPar
Na tela de anotações:
\sphinxhyphen{} Clique na \sphinxstylestrong{seta} para selecionar uma anotação pré\sphinxhyphen{}definida pelo sistema (ex: \sphinxstyleemphasis{Liminar}, \sphinxstyleemphasis{Tutela antecipada}, \sphinxstyleemphasis{Embargos}, etc.)
\sphinxhyphen{} Clique em \sphinxstylestrong{Salvar}

\item {} 
\sphinxAtStartPar
As anotações aparecerão destacadas em \sphinxstylestrong{verde} na tela inicial do processo

\end{enumerate}

\sphinxAtStartPar
\textgreater{} É possível adicionar múltiplas anotações, cada uma representando uma situação processual relevante.

\sphinxAtStartPar
⚠️ \sphinxstylestrong{Observação:} Não há uma aba central para listar todas as anotações feitas em todos os processos.


\subsubsection{Lembretes}
\label{\detokenize{projud_12_favoritar:lembretes}}
\sphinxAtStartPar
Lembretes são usados para criar alertas específicos dentro do processo.

\sphinxAtStartPar
\sphinxstylestrong{Como adicionar um lembrete:}
\begin{enumerate}
\sphinxsetlistlabels{\arabic}{enumi}{enumii}{}{.}%
\item {} 
\sphinxAtStartPar
Vá até a aba \sphinxstylestrong{Informações Gerais}

\item {} 
\sphinxAtStartPar
Clique em \sphinxstylestrong{“Novo Lembrete”}

\item {} 
\sphinxAtStartPar
Preencha os campos:
\sphinxhyphen{} Assunto (ex: \sphinxstyleemphasis{Semana de Conciliação 2024})
\sphinxhyphen{} Descrição livre

\item {} 
\sphinxAtStartPar
Clique em \sphinxstylestrong{Salvar}

\end{enumerate}

\sphinxAtStartPar
\sphinxstylestrong{O lembrete aparecerá:}
\begin{itemize}
\item {} 
\sphinxAtStartPar
Na parte superior central da tela do processo, em \sphinxstylestrong{texto laranja destacado}

\item {} 
\sphinxAtStartPar
Na aba \sphinxstylestrong{“Início”}, opção \sphinxstylestrong{“Lembretes”}, onde você encontra:
\sphinxhyphen{} Todos os lembretes ativos
\sphinxhyphen{} Histórico de lembretes adicionados
\sphinxhyphen{} Possibilidade de:
\begin{itemize}
\item {} 
\sphinxAtStartPar
Encaminhar

\item {} 
\sphinxAtStartPar
Marcar como concluído

\item {} 
\sphinxAtStartPar
Excluir

\end{itemize}

\end{itemize}

\sphinxAtStartPar
\sphinxstylestrong{Exemplo de uso:}
\sphinxhyphen{} “Intimar por telefone”
\sphinxhyphen{} “Verificar cumprimento de prazo”
\sphinxhyphen{} “Agendar audiência”


\subsubsection{Resumo}
\label{\detokenize{projud_12_favoritar:resumo}}
\sphinxAtStartPar
Essas ferramentas são ideais para melhorar a \sphinxstylestrong{comunicação interna}, \sphinxstylestrong{organizar tarefas} e \sphinxstylestrong{personalizar a gestão dos processos}:
\begin{itemize}
\item {} 
\sphinxAtStartPar
\sphinxstylestrong{Favoritos:} triagem de prioridades

\item {} 
\sphinxAtStartPar
\sphinxstylestrong{Anotações:} marcações rápidas e padronizadas

\item {} 
\sphinxAtStartPar
\sphinxstylestrong{Lembretes:} alertas com data e conteúdo personalizado

\end{itemize}

\sphinxstepscope


\subsection{Alterar Sigilo de Processos, Documentos e Movimentações}
\label{\detokenize{projud_13_alterarsigilo:alterar-sigilo-de-processos-documentos-e-movimentacoes}}\label{\detokenize{projud_13_alterarsigilo::doc}}
\sphinxAtStartPar
Nesta aula, vamos aprender como alterar o \sphinxstylestrong{nível de sigilo} de processos, documentos e movimentações no sistema ProJUDI.


\subsubsection{Sigilo do Processo}
\label{\detokenize{projud_13_alterarsigilo:sigilo-do-processo}}
\sphinxAtStartPar
O sistema ProJUDI permite configurar diferentes níveis de sigilo para processos. Para acessar a função:
\begin{enumerate}
\sphinxsetlistlabels{\arabic}{enumi}{enumii}{}{.}%
\item {} 
\sphinxAtStartPar
Clique no ícone de \sphinxstylestrong{cadeado ao lado da palavra “Sigilo”}

\item {} 
\sphinxAtStartPar
Será exibida a tela com os seguintes níveis:
\begin{itemize}
\item {} 
\sphinxAtStartPar
\sphinxstylestrong{Público}: acessível a todos os servidores do Judiciário, órgãos públicos parceiros e advogados

\item {} 
\sphinxAtStartPar
\sphinxstylestrong{Médio}: acessível apenas aos servidores da unidade onde o processo tramita, às partes e envolvidos expressamente incluídos

\end{itemize}

\end{enumerate}

\sphinxAtStartPar
\sphinxstylestrong{Como alterar para sigilo médio:}
\begin{enumerate}
\sphinxsetlistlabels{\arabic}{enumi}{enumii}{}{.}%
\item {} 
\sphinxAtStartPar
Clique no campo de nível de sigilo

\item {} 
\sphinxAtStartPar
Selecione \sphinxstylestrong{“Sigilo Médio”}

\item {} 
\sphinxAtStartPar
Clique em \sphinxstylestrong{Salvar}

\end{enumerate}

\sphinxAtStartPar
\sphinxstylestrong{Adicionar permissões de acesso:}
\begin{itemize}
\item {} 
\sphinxAtStartPar
Caso o juiz determine que outro usuário tenha acesso:
1. Clique em \sphinxstylestrong{“Adicionar Permissão”}
2. Pesquise o usuário por \sphinxstylestrong{perfil} ou \sphinxstylestrong{login}
3. Selecione o usuário e clique em \sphinxstylestrong{Salvar}

\end{itemize}


\subsubsection{Sigilo de Documentos e Movimentações}
\label{\detokenize{projud_13_alterarsigilo:sigilo-de-documentos-e-movimentacoes}}
\sphinxAtStartPar
Além do processo como um todo, é possível tornar \sphinxstylestrong{documentos específicos} ou \sphinxstylestrong{movimentações} sigilosos.

\sphinxAtStartPar
\#\#\# Como tornar um documento sigiloso:
\begin{enumerate}
\sphinxsetlistlabels{\arabic}{enumi}{enumii}{}{.}%
\item {} 
\sphinxAtStartPar
Vá até a aba \sphinxstylestrong{Movimentações}

\item {} 
\sphinxAtStartPar
Clique sobre a movimentação desejada (ex: \sphinxstyleemphasis{Sentença})

\item {} 
\sphinxAtStartPar
Na tela aberta, altere o campo \sphinxstylestrong{“Nível de sigilo”}

\end{enumerate}

\sphinxAtStartPar
Níveis disponíveis:
\begin{itemize}
\item {} 
\sphinxAtStartPar
\sphinxstylestrong{Público} \textendash{} acessível a todos

\item {} 
\sphinxAtStartPar
\sphinxstylestrong{Segredo} \textendash{} acessível a partes e colaboradores da Justiça

\item {} 
\sphinxAtStartPar
\sphinxstylestrong{Mínimo} \textendash{} apenas servidores da mesma competência

\item {} 
\sphinxAtStartPar
\sphinxstylestrong{Médio} \textendash{} servidores da unidade e partes autorizadas

\item {} 
\sphinxAtStartPar
\sphinxstylestrong{Intenso} \textendash{} apenas magistrado, diretor, escrivão, oficial, assessor

\item {} 
\sphinxAtStartPar
\sphinxstylestrong{Absoluto} \textendash{} apenas o magistrado e usuários autorizados

\end{itemize}

\sphinxAtStartPar
\textgreater{} O sistema mostra uma \sphinxstylestrong{tabela comparativa} explicando cada nível.

\sphinxAtStartPar
\#\#\# Como ocultar a visibilidade externa de uma movimentação:
\begin{enumerate}
\sphinxsetlistlabels{\arabic}{enumi}{enumii}{}{.}%
\item {} 
\sphinxAtStartPar
Ainda na tela da movimentação, clique em \sphinxstylestrong{“Ocultar Visibilidade”}

\item {} 
\sphinxAtStartPar
Confirme a ação quando a caixa de diálogo for exibida

\item {} 
\sphinxAtStartPar
Uma tarja será exibida com a mensagem:
\sphinxhyphen{} \sphinxstyleemphasis{“Movimentação sem visibilidade externa”}

\end{enumerate}

\sphinxAtStartPar
\sphinxstylestrong{Importante:}
A movimentação continuará visível \sphinxstylestrong{somente para o cartório}. Para reverter, basta acessar novamente e permitir a visibilidade externa.


\subsubsection{Resumo}
\label{\detokenize{projud_13_alterarsigilo:resumo}}
\sphinxAtStartPar
No sistema ProJUDI, você pode configurar sigilo em três níveis:
\begin{enumerate}
\sphinxsetlistlabels{\arabic}{enumi}{enumii}{}{.}%
\item {} 
\sphinxAtStartPar
\sphinxstylestrong{Processo completo} (nível de sigilo geral)

\item {} 
\sphinxAtStartPar
\sphinxstylestrong{Documentos específicos}

\item {} 
\sphinxAtStartPar
\sphinxstylestrong{Movimentações específicas}

\end{enumerate}

\sphinxAtStartPar
Isso permite atender às determinações legais, proteger dados sensíveis e garantir o controle de acesso adequado aos processos.

\sphinxstepscope


\subsection{Análise Processual Automática (IA Arandu)}
\label{\detokenize{projud_14_iaarandu:analise-processual-automatica-ia-arandu}}\label{\detokenize{projud_14_iaarandu::doc}}
\sphinxAtStartPar
Nesta aula, vamos apresentar uma funcionalidade recente e inovadora do sistema ProJUDI: a \sphinxstylestrong{Análise Processual Automática}, realizada com o auxílio da \sphinxstylestrong{inteligência artificial Arandu}.


\subsubsection{Objetivo da Ferramenta}
\label{\detokenize{projud_14_iaarandu:objetivo-da-ferramenta}}
\sphinxAtStartPar
A ferramenta tem como finalidade:
\begin{itemize}
\item {} 
\sphinxAtStartPar
Identificar \sphinxstylestrong{semelhanças entre petições} distribuídas no sistema

\item {} 
\sphinxAtStartPar
Apontar possíveis \sphinxstylestrong{demandas predatórias} ou \sphinxstylestrong{ações repetitivas}

\item {} 
\sphinxAtStartPar
Oferecer suporte à triagem e despacho de processos por parte dos magistrados e servidores

\end{itemize}


\subsubsection{Funcionamento}
\label{\detokenize{projud_14_iaarandu:funcionamento}}
\sphinxAtStartPar
A IA Arandu utiliza modelos de aprendizado de máquina para:
\begin{itemize}
\item {} 
\sphinxAtStartPar
Comparar textos de petições

\item {} 
\sphinxAtStartPar
Calcular um \sphinxstylestrong{índice de similaridade}

\item {} 
\sphinxAtStartPar
Apresentar processos semelhantes com informações detalhadas

\end{itemize}

\sphinxAtStartPar
Como utilizar a funcionalidade:
\begin{enumerate}
\sphinxsetlistlabels{\arabic}{enumi}{enumii}{}{.}%
\item {} 
\sphinxAtStartPar
Acesse a \sphinxstylestrong{tela inicial de qualquer processo}

\item {} 
\sphinxAtStartPar
Clique no botão de \sphinxstylestrong{“Análise Processual Automática”}

\item {} 
\sphinxAtStartPar
O sistema exibirá o resultado da análise

\end{enumerate}

\sphinxAtStartPar
Exemplos práticos:
\begin{itemize}
\item {} 
\sphinxAtStartPar
\sphinxstylestrong{Processo 1}:
\sphinxhyphen{} Análise realizada
\sphinxhyphen{} Resultado: \sphinxstyleemphasis{“Nenhum processo semelhante encontrado”}

\item {} 
\sphinxAtStartPar
\sphinxstylestrong{Processo 2}:
\sphinxhyphen{} Análise realizada
\sphinxhyphen{} Resultado: \sphinxstyleemphasis{“Foi encontrado um processo com 93.58\% de similaridade”}

\end{itemize}

\sphinxAtStartPar
Informações fornecidas:
\begin{itemize}
\item {} 
\sphinxAtStartPar
Número do processo semelhante

\item {} 
\sphinxAtStartPar
Situação (ex: arquivado, julgado)

\item {} 
\sphinxAtStartPar
Polo ativo e passivo

\item {} 
\sphinxAtStartPar
Vara

\item {} 
\sphinxAtStartPar
Classe processual

\item {} 
\sphinxAtStartPar
Assunto

\item {} 
\sphinxAtStartPar
Data da distribuição

\item {} 
\sphinxAtStartPar
Estado de origem

\end{itemize}


\subsubsection{Análise do Resultado}
\label{\detokenize{projud_14_iaarandu:analise-do-resultado}}
\sphinxAtStartPar
Para examinar o processo semelhante:
\begin{enumerate}
\sphinxsetlistlabels{\arabic}{enumi}{enumii}{}{.}%
\item {} 
\sphinxAtStartPar
Clique sobre o processo listado

\item {} 
\sphinxAtStartPar
Avalie os dados apresentados

\item {} 
\sphinxAtStartPar
Se necessário, \sphinxstylestrong{encaminhe ao magistrado} para providências (ex: arquivamento de ação idêntica)

\end{enumerate}

\sphinxAtStartPar
⚠️ \sphinxstylestrong{Importante:} Essa ferramenta é \sphinxstylestrong{auxiliar}. A análise final continua sendo feita pelo servidor ou magistrado.


\subsubsection{Resumo}
\label{\detokenize{projud_14_iaarandu:resumo}}
\sphinxAtStartPar
A \sphinxstylestrong{IA Arandu} é uma ferramenta de apoio à gestão processual, que:
\begin{itemize}
\item {} 
\sphinxAtStartPar
Aumenta a eficiência

\item {} 
\sphinxAtStartPar
Ajuda na identificação de litígios massificados

\item {} 
\sphinxAtStartPar
Colabora com o enfrentamento de \sphinxstylestrong{demandas predatórias}

\end{itemize}

\sphinxAtStartPar
Esse recurso está disponível \sphinxstylestrong{em todos os processos}, diretamente na tela principal.

\sphinxstepscope


\subsection{Juntar Documento}
\label{\detokenize{projud_15_juntardocumento:juntar-documento}}\label{\detokenize{projud_15_juntardocumento::doc}}
\sphinxAtStartPar
Nesta aula, vamos aprender como utilizar a função \sphinxstylestrong{“Juntar Documento”} no sistema ProJUDI. Essa funcionalidade permite inserir documentos nos autos de forma direta, sem impactar os indicadores de produtividade.


\subsubsection{Objetivo}
\label{\detokenize{projud_15_juntardocumento:objetivo}}
\sphinxAtStartPar
A função \sphinxstylestrong{Juntar Documento} é usada principalmente para:
\begin{itemize}
\item {} 
\sphinxAtStartPar
\sphinxstylestrong{Recebimento de ofícios}

\item {} 
\sphinxAtStartPar
\sphinxstylestrong{Inserção de documentos administrativos}

\item {} 
\sphinxAtStartPar
\sphinxstylestrong{Anexos sem movimentação processual formal}

\end{itemize}

\sphinxAtStartPar
⚠️ \sphinxstylestrong{Importante:}
Movimentações realizadas por esse botão \sphinxstylestrong{não contabilizam para a produtividade}, por isso não são recomendadas para atos processuais formais.


\subsubsection{Como utilizar a funcionalidade}
\label{\detokenize{projud_15_juntardocumento:como-utilizar-a-funcionalidade}}\begin{enumerate}
\sphinxsetlistlabels{\arabic}{enumi}{enumii}{}{.}%
\item {} 
\sphinxAtStartPar
Acesse a \sphinxstylestrong{tela inicial de um processo}

\item {} 
\sphinxAtStartPar
Clique no botão \sphinxstylestrong{“Juntar Documento”} (localizado no centro da tela)

\item {} 
\sphinxAtStartPar
Será exibida a tela de inclusão do documento

\end{enumerate}

\sphinxAtStartPar
\#\#\# Etapas do procedimento:
\begin{enumerate}
\sphinxsetlistlabels{\arabic}{enumi}{enumii}{}{.}%
\item {} 
\sphinxAtStartPar
\sphinxstylestrong{Selecionar o tipo de documento}
\sphinxhyphen{} Clique na \sphinxstylestrong{lupa}
\sphinxhyphen{} Pesquise e selecione, por exemplo: \sphinxstyleemphasis{“Ofício de outros órgãos”}

\item {} 
\sphinxAtStartPar
\sphinxstylestrong{Adicionar o documento}
\sphinxhyphen{} Digite um texto (opcional)
\sphinxhyphen{} Clique em \sphinxstylestrong{“Selecionar Arquivo”} e escolha o arquivo desejado (ex: PDF)
\sphinxhyphen{} Defina a categoria (ex: \sphinxstyleemphasis{Outros})
\sphinxhyphen{} Insira o nome do documento (ex: \sphinxstyleemphasis{Ofício recebido})

\item {} 
\sphinxAtStartPar
\sphinxstylestrong{Assinar e confirmar}
\sphinxhyphen{} Clique em \sphinxstylestrong{“Assinar”}
\sphinxhyphen{} Em seguida, clique em \sphinxstylestrong{“Confirmar Inclusão”}
\sphinxhyphen{} Finalize com \sphinxstylestrong{“Concluir Movimento”}

\end{enumerate}

\sphinxAtStartPar
Mensagem de sucesso:

\sphinxAtStartPar
Dados registrados com sucesso.


\subsubsection{Verificação}
\label{\detokenize{projud_15_juntardocumento:verificacao}}
\sphinxAtStartPar
Após juntar o documento:
\begin{enumerate}
\sphinxsetlistlabels{\arabic}{enumi}{enumii}{}{.}%
\item {} 
\sphinxAtStartPar
Retorne à tela do processo

\item {} 
\sphinxAtStartPar
Clique em \sphinxstylestrong{“Navegar”}

\item {} 
\sphinxAtStartPar
O documento estará disponível com a descrição (ex: \sphinxstyleemphasis{Ofício Recebido})

\end{enumerate}


\subsubsection{Resumo}
\label{\detokenize{projud_15_juntardocumento:resumo}}
\sphinxAtStartPar
A funcionalidade \sphinxstylestrong{Juntar Documento} é indicada para anexos administrativos e documentos informativos que \sphinxstylestrong{não exigem movimentação processual registrada}.
\begin{itemize}
\item {} 
\sphinxAtStartPar
Não interfere na contagem de produtividade

\item {} 
\sphinxAtStartPar
Serve como complemento ao processo

\item {} 
\sphinxAtStartPar
Documentos inseridos ficam visíveis no histórico de movimentações

\end{itemize}

\sphinxstepscope


\subsection{Navegando nos Autos}
\label{\detokenize{projud_16_navegandoprocessos:navegando-nos-autos}}\label{\detokenize{projud_16_navegandoprocessos::doc}}
\sphinxAtStartPar
Nesta aula, vamos aprender como utilizar o botão \sphinxstylestrong{“Navegar”} nos autos do processo no sistema ProJUDI. Essa função permite uma \sphinxstylestrong{visualização ampla e detalhada} dos documentos e movimentações processuais.


\subsubsection{Acesso à Função}
\label{\detokenize{projud_16_navegandoprocessos:acesso-a-funcao}}
\sphinxAtStartPar
Existem duas formas de acessar o recurso:
\begin{itemize}
\item {} 
\sphinxAtStartPar
Clicando no botão \sphinxstylestrong{“Navegar”}

\item {} 
\sphinxAtStartPar
Utilizando o atalho de teclado \sphinxstylestrong{F12}

\end{itemize}

\sphinxAtStartPar
A tela é aberta em uma nova aba e pode ser movida para um \sphinxstylestrong{segundo monitor}, otimizando o trabalho simultâneo com outras telas do sistema.


\subsubsection{Funcionalidades da Tela Navegar}
\label{\detokenize{projud_16_navegandoprocessos:funcionalidades-da-tela-navegar}}
\sphinxAtStartPar
\#\#\# Juntar Documento
\begin{itemize}
\item {} 
\sphinxAtStartPar
Na tela de navegação, há um botão que \sphinxstylestrong{redireciona diretamente para a função “Juntar Documento”}

\end{itemize}

\sphinxAtStartPar
\#\#\# Capa do Processo
\begin{itemize}
\item {} 
\sphinxAtStartPar
Clique em \sphinxstylestrong{“Capa do Processo”}

\item {} 
\sphinxAtStartPar
O sistema fará o download automático

\item {} 
\sphinxAtStartPar
Contém as principais informações processuais

\end{itemize}

\sphinxAtStartPar
\#\#\# Exportar Processo
\begin{itemize}
\item {} 
\sphinxAtStartPar
Clique em \sphinxstylestrong{“Exportar”}

\item {} 
\sphinxAtStartPar
A tela permite:
\begin{itemize}
\item {} 
\sphinxAtStartPar
Escolher o \sphinxstylestrong{tipo de arquivo}

\item {} 
\sphinxAtStartPar
\sphinxstylestrong{Realçar movimentações específicas}, como:
\sphinxhyphen{} Magistrado
\sphinxhyphen{} Decisões

\item {} 
\sphinxAtStartPar
Incluir ou não:
\sphinxhyphen{} Movimentações
\sphinxhyphen{} Tarja de validação
\sphinxhyphen{} Capa do processo

\item {} 
\sphinxAtStartPar
Selecionar documentos individualmente (ex: apenas petição inicial)

\item {} 
\sphinxAtStartPar
Exportar \sphinxstylestrong{todos os documentos} do processo

\end{itemize}

\end{itemize}

\sphinxAtStartPar
\textgreater{} Ideal para leitura completa ou impressão seletiva do conteúdo dos autos.

\sphinxAtStartPar
\#\#\# Opções de Exibição
\begin{itemize}
\item {} 
\sphinxAtStartPar
Ao acessar a tela, é \sphinxstylestrong{recomendado ativar “Detalhes da Movimentação”} para visualizar:
\begin{itemize}
\item {} 
\sphinxAtStartPar
Data da movimentação

\item {} 
\sphinxAtStartPar
Quem realizou (servidor, sistema, advogado)

\item {} 
\sphinxAtStartPar
Número do protocolo

\item {} 
\sphinxAtStartPar
Transparência no histórico

\end{itemize}

\end{itemize}

\sphinxAtStartPar
\#\#\# Realçar Movimentos
\begin{itemize}
\item {} 
\sphinxAtStartPar
Permite \sphinxstylestrong{destaque visual} de agentes específicos (ex: magistrado)

\item {} 
\sphinxAtStartPar
As decisões interlocutórias, por exemplo, ficam realçadas em \sphinxstylestrong{laranja}

\end{itemize}

\sphinxAtStartPar
\#\#\# Resumo do Processo
\begin{itemize}
\item {} 
\sphinxAtStartPar
Aparece no lado direito da tela

\item {} 
\sphinxAtStartPar
Exibe:
\sphinxhyphen{} Partes
\sphinxhyphen{} Valor da causa
\sphinxhyphen{} Depósito judicial
\sphinxhyphen{} Trânsito em julgado
\sphinxhyphen{} CPF, entre outros

\end{itemize}

\sphinxAtStartPar
\#\#\# Visualização de Documentos
\begin{itemize}
\item {} 
\sphinxAtStartPar
Clique em uma movimentação para ver o conteúdo do documento correspondente

\item {} 
\sphinxAtStartPar
Clique nas \sphinxstylestrong{três barras laterais} para:
\sphinxhyphen{} Exibir somente o documento selecionado
\sphinxhyphen{} Navegar por múltiplas páginas
\sphinxhyphen{} Alternar rapidamente entre documentos

\end{itemize}


\subsubsection{Resumo}
\label{\detokenize{projud_16_navegandoprocessos:resumo}}
\sphinxAtStartPar
A funcionalidade \sphinxstylestrong{“Navegar”} oferece:
\begin{itemize}
\item {} 
\sphinxAtStartPar
Visualização ampla dos autos

\item {} 
\sphinxAtStartPar
Acesso direto a documentos, capa, e exportação

\item {} 
\sphinxAtStartPar
Realce de informações processuais importantes

\item {} 
\sphinxAtStartPar
Facilidade na organização e auditoria das ações no processo

\end{itemize}

\sphinxAtStartPar
Essa ferramenta é fundamental para uma análise processual detalhada e transparente.

\sphinxstepscope


\subsection{Visualizando as Movimentações}
\label{\detokenize{projud_17_visualizandomovimentos:visualizando-as-movimentacoes}}\label{\detokenize{projud_17_visualizandomovimentos::doc}}
\sphinxAtStartPar
Nesta aula, vamos aprender como \sphinxstylestrong{visualizar e interpretar as movimentações processuais} no sistema ProJUDI. Essa funcionalidade é essencial para acompanhar o histórico do processo, verificar documentos e entender o fluxo dos atos judiciais.


\subsubsection{Acessando as Movimentações}
\label{\detokenize{projud_17_visualizandomovimentos:acessando-as-movimentacoes}}
\sphinxAtStartPar
Existem duas formas principais para acessar as movimentações:
\begin{enumerate}
\sphinxsetlistlabels{\arabic}{enumi}{enumii}{}{.}%
\item {} 
\sphinxAtStartPar
\sphinxstylestrong{Pela aba “Movimentações”} na tela inicial do processo

\item {} 
\sphinxAtStartPar
\sphinxstylestrong{Pelo botão “Navegar”} (atalho: F12)

\end{enumerate}

\sphinxAtStartPar
\#\#\# Tela Inicial \textendash{} Aba Movimentações
\begin{itemize}
\item {} 
\sphinxAtStartPar
Exibe todas as movimentações do processo desde a petição inicial

\item {} 
\sphinxAtStartPar
É possível aplicar \sphinxstylestrong{filtros por agente} (ex: magistrado, advogado)

\item {} 
\sphinxAtStartPar
Ferramenta útil para processos com muitas movimentações

\end{itemize}

\sphinxAtStartPar
\sphinxstylestrong{Exemplo:}
\begin{itemize}
\item {} 
\sphinxAtStartPar
Clique em \sphinxstylestrong{“Realçar movimentações do magistrado”}
\sphinxhyphen{} As movimentações do juiz serão destacadas em \sphinxstylestrong{laranja}

\item {} 
\sphinxAtStartPar
Para visualizar apenas atos do advogado:
\sphinxhyphen{} Selecione o filtro e clique em \sphinxstylestrong{“Filtrar”}

\end{itemize}

\sphinxAtStartPar
\#\#\# Visualizando Documentos
\begin{itemize}
\item {} 
\sphinxAtStartPar
Ao lado de cada movimentação que possui documento, há um botão \sphinxstylestrong{“+”}

\item {} 
\sphinxAtStartPar
Clique nele e selecione \sphinxstylestrong{“Online”} para abrir o documento em PDF

\item {} 
\sphinxAtStartPar
Pode ser usado para:
\sphinxhyphen{} Decisões interlocutórias
\sphinxhyphen{} Manifestações das partes
\sphinxhyphen{} Petições, despachos, etc.

\end{itemize}


\subsubsection{Análise Detalhada via “Navegar”}
\label{\detokenize{projud_17_visualizandomovimentos:analise-detalhada-via-navegar}}
\sphinxAtStartPar
Na tela do botão \sphinxstylestrong{Navegar}, ative a opção \sphinxstylestrong{“Detalhes da Movimentação”}.

\sphinxAtStartPar
Ao clicar em \sphinxstylestrong{“Detalhes”} de uma movimentação, são exibidos:
\begin{itemize}
\item {} 
\sphinxAtStartPar
Data da movimentação

\item {} 
\sphinxAtStartPar
Nome de quem movimentou (advogado, magistrado, servidor)

\item {} 
\sphinxAtStartPar
Evento ou número sequencial da movimentação

\item {} 
\sphinxAtStartPar
Protocolo gerado

\end{itemize}

\sphinxAtStartPar
\#\#\# Exemplo de Contagem de Prazo:
\begin{itemize}
\item {} 
\sphinxAtStartPar
\sphinxstylestrong{Evento 8}: Decisão interlocutória

\item {} 
\sphinxAtStartPar
\sphinxstylestrong{Evento 9}: Expedição de intimação (referente ao evento 8)

\item {} 
\sphinxAtStartPar
\sphinxstylestrong{Evento 10}: Leitura da intimação

\end{itemize}

\sphinxAtStartPar
No detalhe da movimentação de leitura, será exibido:
\begin{itemize}
\item {} 
\sphinxAtStartPar
Nome do advogado que realizou a leitura

\item {} 
\sphinxAtStartPar
Data da leitura

\item {} 
\sphinxAtStartPar
Prazo contado

\item {} 
\sphinxAtStartPar
Referência ao evento que gerou a intimação

\end{itemize}

\sphinxAtStartPar
\textgreater{} Caso a leitura não ocorra dentro do prazo, o sistema gera uma \sphinxstylestrong{leitura automática}, indicando \sphinxstylestrong{“sistema”} como responsável.


\subsubsection{Resumo}
\label{\detokenize{projud_17_visualizandomovimentos:resumo}}
\sphinxAtStartPar
A aba de movimentações permite:
\begin{itemize}
\item {} 
\sphinxAtStartPar
Acompanhar o histórico processual completo

\item {} 
\sphinxAtStartPar
Visualizar documentos diretamente

\item {} 
\sphinxAtStartPar
Aplicar filtros por autor da movimentação

\item {} 
\sphinxAtStartPar
Ver detalhes como prazos, responsáveis e protocolos

\end{itemize}

\sphinxAtStartPar
O botão \sphinxstylestrong{“Navegar”} complementa essa função com uma visualização ampliada e organizada dos eventos.

\sphinxstepscope


\subsection{Exportar Processo}
\label{\detokenize{projud_18_exportarprocesso:exportar-processo}}\label{\detokenize{projud_18_exportarprocesso::doc}}
\sphinxAtStartPar
Nesta aula, vamos aprender a utilizar o botão \sphinxstylestrong{Exportar} no sistema ProJUDI, com foco especial em:
\begin{itemize}
\item {} 
\sphinxAtStartPar
Como gerar o processo completo em PDF

\item {} 
\sphinxAtStartPar
Como entender a \sphinxstylestrong{numeração de páginas e eventos}

\item {} 
\sphinxAtStartPar
Como exportar documentos específicos

\end{itemize}


\subsubsection{Acesso à Função}
\label{\detokenize{projud_18_exportarprocesso:acesso-a-funcao}}
\sphinxAtStartPar
Na tela inicial do processo, clique em:
\begin{itemize}
\item {} 
\sphinxAtStartPar
\sphinxstylestrong{“Exportar”} \(\rightarrow\) \sphinxstyleemphasis{“Exportar Processo”}

\end{itemize}

\sphinxAtStartPar
O sistema abrirá uma nova tela com opções de exportação personalizáveis.


\subsubsection{Opções de Exportação}
\label{\detokenize{projud_18_exportarprocesso:opcoes-de-exportacao}}
\sphinxAtStartPar
\#\#\# 1. \sphinxstylestrong{Gerar movimentações}: \sphinxstyleemphasis{Sim}
\begin{itemize}
\item {} 
\sphinxAtStartPar
O sistema incluirá:
\sphinxhyphen{} \sphinxstylestrong{Detalhes de cada movimentação}
\sphinxhyphen{} Número do evento (ou movimentação)
\sphinxhyphen{} Data da juntada
\sphinxhyphen{} Tipo de documento

\item {} 
\sphinxAtStartPar
Permite rastrear, no PDF gerado, qual documento corresponde a qual \sphinxstylestrong{evento no sistema}

\end{itemize}

\sphinxAtStartPar
\sphinxstylestrong{Exemplo:}

\begin{DUlineblock}{0em}
\item[] Página | Conteúdo                      | Evento       |
\end{DUlineblock}

\sphinxAtStartPar
{\color{red}\bfseries{}|\sphinxhyphen{}\sphinxhyphen{}\sphinxhyphen{}\sphinxhyphen{}\sphinxhyphen{}\sphinxhyphen{}\sphinxhyphen{}\sphinxhyphen{}|}——————————\sphinxhyphen{}{\color{red}\bfseries{}|\sphinxhyphen{}\sphinxhyphen{}\sphinxhyphen{}\sphinxhyphen{}\sphinxhyphen{}\sphinxhyphen{}\sphinxhyphen{}\sphinxhyphen{}\sphinxhyphen{}\sphinxhyphen{}\sphinxhyphen{}\sphinxhyphen{}\sphinxhyphen{}\sphinxhyphen{}|}
| 1      | Capa do processo              | \sphinxhyphen{}            |
| 2      | Certidão ou petição inicial   | 1.0          |
| 3      | Documento complementar        | 1.1          |

\sphinxAtStartPar
\#\#\# 2. \sphinxstylestrong{Gerar movimentações}: \sphinxstyleemphasis{Não}
\begin{itemize}
\item {} 
\sphinxAtStartPar
O sistema \sphinxstylestrong{omite os detalhes das movimentações}

\item {} 
\sphinxAtStartPar
O PDF resultante conterá apenas os documentos, sem a numeração de eventos ou data da juntada

\item {} 
\sphinxAtStartPar
Útil para versões mais enxutas, porém \sphinxstylestrong{sem referência cruzada com o sistema}

\end{itemize}

\sphinxAtStartPar
\#\#\# 3. \sphinxstylestrong{Exportar documentos específicos}
\begin{enumerate}
\sphinxsetlistlabels{\arabic}{enumi}{enumii}{}{.}%
\item {} 
\sphinxAtStartPar
Clique em \sphinxstylestrong{“Desmarcar todos”}

\item {} 
\sphinxAtStartPar
Marque apenas os documentos desejados (ex: \sphinxstyleemphasis{Petição Inicial})

\item {} 
\sphinxAtStartPar
Clique em \sphinxstylestrong{“Exportar”}

\item {} 
\sphinxAtStartPar
O sistema gerará um PDF apenas com os documentos selecionados

\end{enumerate}


\subsubsection{Resumo das Vantagens}
\label{\detokenize{projud_18_exportarprocesso:resumo-das-vantagens}}
\sphinxAtStartPar
✅ Exportar com movimentações:
\sphinxhyphen{} Permite identificar qual documento está ligado a qual evento (ex: evento 8.0)
\sphinxhyphen{} Útil para conferência, auditoria e prazos

\sphinxAtStartPar
✅ Exportar sem movimentações:
\sphinxhyphen{} Gera um PDF mais limpo e leve
\sphinxhyphen{} Ideal para leitura rápida ou impressão simplificada

\sphinxAtStartPar
✅ Exportar por seleção:
\sphinxhyphen{} Personaliza o conteúdo exportado
\sphinxhyphen{} Evita incluir documentos desnecessários


\subsubsection{Resumo}
\label{\detokenize{projud_18_exportarprocesso:resumo}}
\sphinxAtStartPar
A função \sphinxstylestrong{Exportar Processo} oferece total controle sobre o que será gerado em PDF, com ou sem detalhamento técnico, e é ideal para:
\begin{itemize}
\item {} 
\sphinxAtStartPar
Impressão

\item {} 
\sphinxAtStartPar
Conferência de prazos

\item {} 
\sphinxAtStartPar
Compartilhamento com partes

\item {} 
\sphinxAtStartPar
Consulta offline

\end{itemize}

\sphinxstepscope


\subsection{Evolução e Retificação de Classe Processual e Prioridade}
\label{\detokenize{projud_19_evolucao_retificacao:evolucao-e-retificacao-de-classe-processual-e-prioridade}}\label{\detokenize{projud_19_evolucao_retificacao::doc}}
\sphinxAtStartPar
Nesta aula, vamos explorar a aba \sphinxstylestrong{Informações Processuais}, onde é possível realizar \sphinxstylestrong{alterações importantes} no cadastro do processo no sistema ProJUDI. Essas alterações fazem parte da rotina da secretaria, especialmente nos casos de:
\begin{itemize}
\item {} 
\sphinxAtStartPar
\sphinxstylestrong{Retificação ou evolução de classe processual}

\item {} 
\sphinxAtStartPar
\sphinxstylestrong{Alteração de prioridade}

\item {} 
\sphinxAtStartPar
\sphinxstylestrong{Atualização de dados processuais e físicos}

\end{itemize}


\subsubsection{Acesso à Aba}
\label{\detokenize{projud_19_evolucao_retificacao:acesso-a-aba}}\begin{enumerate}
\sphinxsetlistlabels{\arabic}{enumi}{enumii}{}{.}%
\item {} 
\sphinxAtStartPar
Acesse a \sphinxstylestrong{tela inicial do processo}

\item {} 
\sphinxAtStartPar
Role até o final e clique em \sphinxstylestrong{“Informações Gerais”}

\end{enumerate}

\sphinxAtStartPar
Essa seção fornece dados como:
\begin{itemize}
\item {} 
\sphinxAtStartPar
Comarca e vara

\item {} 
\sphinxAtStartPar
Datas de autuação, distribuição e arquivamento

\item {} 
\sphinxAtStartPar
Situação e chave do processo

\item {} 
\sphinxAtStartPar
Local físico do processo (ex: arquivo, secretaria)

\item {} 
\sphinxAtStartPar
Número antigo, lembretes, localizadores, juiz responsável, etc.

\end{itemize}


\subsubsection{Alteração de Classe Processual}
\label{\detokenize{projud_19_evolucao_retificacao:alteracao-de-classe-processual}}
\sphinxAtStartPar
Para alterar a classe processual:
\begin{enumerate}
\sphinxsetlistlabels{\arabic}{enumi}{enumii}{}{.}%
\item {} 
\sphinxAtStartPar
Clique em \sphinxstylestrong{“Alterar”}

\item {} 
\sphinxAtStartPar
Na nova tela, clique na \sphinxstylestrong{lupa} ao lado de \sphinxstyleemphasis{Classe Processual}

\item {} 
\sphinxAtStartPar
Pesquise a nova classe (ex: \sphinxstyleemphasis{Carta Precatória Criminal})

\item {} 
\sphinxAtStartPar
Selecione o \sphinxstylestrong{Tipo de Movimento}

\item {} 
\sphinxAtStartPar
Escolha entre:
\sphinxhyphen{} \sphinxstylestrong{Retificação de Classe}: quando foi cadastrada incorretamente
\sphinxhyphen{} \sphinxstylestrong{Evolução de Classe}: quando o processo muda de fase (ex: conhecimento \(\rightarrow\) execução)

\item {} 
\sphinxAtStartPar
Clique em \sphinxstylestrong{Salvar}

\end{enumerate}

\sphinxAtStartPar
\textgreater{} Também é possível corrigir o \sphinxstylestrong{assunto principal} do processo nessa tela.

\sphinxAtStartPar
Exemplo:
\begin{itemize}
\item {} 
\sphinxAtStartPar
De: \sphinxstyleemphasis{Ação de Conhecimento}

\item {} 
\sphinxAtStartPar
Para: \sphinxstyleemphasis{Execução de Sentença}

\end{itemize}


\subsubsection{Alteração de Prioridade}
\label{\detokenize{projud_19_evolucao_retificacao:alteracao-de-prioridade}}
\sphinxAtStartPar
Para incluir ou modificar a prioridade:
\begin{enumerate}
\sphinxsetlistlabels{\arabic}{enumi}{enumii}{}{.}%
\item {} 
\sphinxAtStartPar
Clique novamente em \sphinxstylestrong{“Alterar”}

\item {} 
\sphinxAtStartPar
Clique na \sphinxstylestrong{lupa} ao lado de \sphinxstyleemphasis{Prioridade}

\item {} 
\sphinxAtStartPar
Exemplo: selecione \sphinxstyleemphasis{Violência Doméstica}

\item {} 
\sphinxAtStartPar
Clique em \sphinxstylestrong{Selecionar} e depois \sphinxstylestrong{Salvar}

\end{enumerate}

\sphinxAtStartPar
A prioridade aparecerá registrada em destaque no processo.


\subsubsection{Outras Alterações Possíveis}
\label{\detokenize{projud_19_evolucao_retificacao:outras-alteracoes-possiveis}}
\sphinxAtStartPar
Você pode ainda editar:
\begin{itemize}
\item {} 
\sphinxAtStartPar
\sphinxstylestrong{Magistrado responsável}

\item {} 
\sphinxAtStartPar
\sphinxstylestrong{Situação processual} (ex: sentenciado, arquivado)

\item {} 
\sphinxAtStartPar
\sphinxstylestrong{Local físico do processo} (ex: \sphinxstyleemphasis{Caixa Criminal nº 11 de 2016})

\item {} 
\sphinxAtStartPar
\sphinxstylestrong{Objeto e pedido}

\item {} 
\sphinxAtStartPar
\sphinxstylestrong{Necessidade de intervenção do MP}

\item {} 
\sphinxAtStartPar
\sphinxstylestrong{Dados complementares} para procedimentos investigatórios:
\sphinxhyphen{} Data da infração
\sphinxhyphen{} Delegacia, comarca
\sphinxhyphen{} Latitude, localidade
\sphinxhyphen{} Número do inquérito
\sphinxhyphen{} Magnitude ambiental, etc.

\end{itemize}


\subsubsection{Finalização}
\label{\detokenize{projud_19_evolucao_retificacao:finalizacao}}
\sphinxAtStartPar
Após preencher os campos desejados, clique em \sphinxstylestrong{Salvar}. O sistema registrará todas as alterações com sucesso.


\subsubsection{Resumo}
\label{\detokenize{projud_19_evolucao_retificacao:resumo}}
\sphinxAtStartPar
A aba \sphinxstylestrong{Informações Processuais} permite ajustes fundamentais no cadastro do processo, sendo especialmente útil para:
\begin{itemize}
\item {} 
\sphinxAtStartPar
Corrigir erros de classe ou assunto

\item {} 
\sphinxAtStartPar
Atualizar prioridade (ex: idosos, violência doméstica)

\item {} 
\sphinxAtStartPar
Evoluir processos conforme sua tramitação

\item {} 
\sphinxAtStartPar
Organizar o acervo físico e eletrônico

\end{itemize}

\sphinxAtStartPar
Essas ações garantem a \sphinxstylestrong{correção, transparência e rastreabilidade} da tramitação processual no ProJUDI.

\sphinxstepscope


\subsection{Cadastro de Depósito Judicial}
\label{\detokenize{projud_20_cadastrodeposito:cadastro-de-deposito-judicial}}\label{\detokenize{projud_20_cadastrodeposito::doc}}
\sphinxAtStartPar
Nesta aula, vamos aprender a utilizar a aba \sphinxstylestrong{Informações Adicionais} do sistema ProJUDI, com foco no \sphinxstylestrong{cadastro de depósitos judiciais} e seus respectivos detalhes financeiros.


\subsubsection{Acesso à Funcionalidade}
\label{\detokenize{projud_20_cadastrodeposito:acesso-a-funcionalidade}}\begin{enumerate}
\sphinxsetlistlabels{\arabic}{enumi}{enumii}{}{.}%
\item {} 
\sphinxAtStartPar
Acesse um processo qualquer

\item {} 
\sphinxAtStartPar
Clique na aba \sphinxstylestrong{“Informações Adicionais”}

\item {} 
\sphinxAtStartPar
Localize o campo \sphinxstylestrong{“Depósito Judicial”}

\end{enumerate}


\subsubsection{Cadastro de Conta Judicial}
\label{\detokenize{projud_20_cadastrodeposito:cadastro-de-conta-judicial}}\begin{enumerate}
\sphinxsetlistlabels{\arabic}{enumi}{enumii}{}{.}%
\item {} 
\sphinxAtStartPar
Clique em \sphinxstylestrong{“Novo”} para cadastrar uma conta

\item {} 
\sphinxAtStartPar
Preencha os seguintes dados:
\sphinxhyphen{} Banco (ex: Banco do Brasil, Caixa Econômica)
\sphinxhyphen{} Número da agência e dígito
\sphinxhyphen{} Número da conta e dígito

\item {} 
\sphinxAtStartPar
Clique em \sphinxstylestrong{Salvar}

\end{enumerate}

\sphinxAtStartPar
\textgreater{} Após salvar, é possível \sphinxstylestrong{alterar ou remover} a conta cadastrada.


\subsubsection{Cadastro de Depósito}
\label{\detokenize{projud_20_cadastrodeposito:cadastro-de-deposito}}\begin{enumerate}
\sphinxsetlistlabels{\arabic}{enumi}{enumii}{}{.}%
\item {} 
\sphinxAtStartPar
Clique em \sphinxstylestrong{“Novo”} no campo \sphinxstyleemphasis{Depósitos}

\item {} 
\sphinxAtStartPar
Selecione a conta judicial previamente cadastrada

\item {} 
\sphinxAtStartPar
Escolha a \sphinxstylestrong{natureza do depósito}:
\sphinxhyphen{} Alimentar
\sphinxhyphen{} Multa
\sphinxhyphen{} Fiança
\sphinxhyphen{} Outros (válido para áreas cível e criminal)

\item {} 
\sphinxAtStartPar
Insira a \sphinxstylestrong{data do depósito}

\end{enumerate}

\sphinxAtStartPar
\#\#\# Itens do Depósito:
\begin{enumerate}
\sphinxsetlistlabels{\arabic}{enumi}{enumii}{}{.}%
\item {} 
\sphinxAtStartPar
Clique em \sphinxstylestrong{“Adicionar”}

\item {} 
\sphinxAtStartPar
Preencha:
\sphinxhyphen{} Finalidade: pagamento, custas, taxas, honorários, garantia, bloqueio, etc.
\sphinxhyphen{} Valor
\sphinxhyphen{} Observação (opcional)

\end{enumerate}

\sphinxAtStartPar
\#\#\# Comprovante do Depósito:
\begin{enumerate}
\sphinxsetlistlabels{\arabic}{enumi}{enumii}{}{.}%
\item {} 
\sphinxAtStartPar
Clique em \sphinxstylestrong{“Adicionar Comprovante”}

\item {} 
\sphinxAtStartPar
Selecione o arquivo (preferencialmente em PDF)

\item {} 
\sphinxAtStartPar
Preencha o nome do documento

\item {} 
\sphinxAtStartPar
Clique em \sphinxstylestrong{Assinar}

\item {} 
\sphinxAtStartPar
Em seguida, \sphinxstylestrong{Confirmar Inclusão} e \sphinxstylestrong{Salvar}

\end{enumerate}

\sphinxAtStartPar
\textgreater{} Após o cadastro, o depósito aparecerá vinculado ao processo e poderá ser consultado, alterado ou removido.


\subsubsection{Levantamentos}
\label{\detokenize{projud_20_cadastrodeposito:levantamentos}}
\sphinxAtStartPar
Além dos depósitos, também é possível registrar \sphinxstylestrong{levantamentos} (resgates dos valores depositados), mesmo que hoje exista o \sphinxstylestrong{Alvará Eletrônico}.
\begin{enumerate}
\sphinxsetlistlabels{\arabic}{enumi}{enumii}{}{.}%
\item {} 
\sphinxAtStartPar
Clique na aba \sphinxstylestrong{Levantamentos}

\item {} 
\sphinxAtStartPar
Informe:
\sphinxhyphen{} Data
\sphinxhyphen{} Valor
\sphinxhyphen{} Beneficiário
\sphinxhyphen{} Finalidade (ex: pagamento, restituição, etc.)

\end{enumerate}


\subsubsection{Resumo}
\label{\detokenize{projud_20_cadastrodeposito:resumo}}
\sphinxAtStartPar
A aba \sphinxstylestrong{Informações Adicionais} permite o gerenciamento de dados financeiros ligados ao processo, sendo útil para:
\begin{itemize}
\item {} 
\sphinxAtStartPar
\sphinxstylestrong{Cadastrar contas judiciais}

\item {} 
\sphinxAtStartPar
\sphinxstylestrong{Registrar depósitos e comprovantes}

\item {} 
\sphinxAtStartPar
\sphinxstylestrong{Controlar levantamentos e beneficiários}

\end{itemize}

\sphinxAtStartPar
Essa funcionalidade é importante tanto na área \sphinxstylestrong{cível} quanto \sphinxstylestrong{criminal}, e ajuda a manter o controle e a transparência dos valores movimentados no processo.

\sphinxstepscope


\subsection{Cadastro de Bem de Penhora}
\label{\detokenize{projud_21_cadastrobempenhora:cadastro-de-bem-de-penhora}}\label{\detokenize{projud_21_cadastrobempenhora::doc}}
\sphinxAtStartPar
Nesta aula, você aprenderá a realizar o \sphinxstylestrong{cadastro de bens penhorados} no sistema ProJUDI, acessando a funcionalidade disponível em \sphinxstylestrong{Informações Adicionais} do processo.


\subsubsection{Acesso à Funcionalidade}
\label{\detokenize{projud_21_cadastrobempenhora:acesso-a-funcionalidade}}\begin{enumerate}
\sphinxsetlistlabels{\arabic}{enumi}{enumii}{}{.}%
\item {} 
\sphinxAtStartPar
Acesse a aba \sphinxstylestrong{Informações Adicionais}

\item {} 
\sphinxAtStartPar
Clique em \sphinxstylestrong{“Auto de Penhora”}

\item {} 
\sphinxAtStartPar
A tela \sphinxstyleemphasis{Auto de Penhora} será exibida

\end{enumerate}


\subsubsection{Cadastro do Auto de Penhora}
\label{\detokenize{projud_21_cadastrobempenhora:cadastro-do-auto-de-penhora}}\begin{enumerate}
\sphinxsetlistlabels{\arabic}{enumi}{enumii}{}{.}%
\item {} 
\sphinxAtStartPar
Clique em \sphinxstylestrong{“Novo”}

\item {} 
\sphinxAtStartPar
Preencha os campos obrigatórios:
\begin{itemize}
\item {} 
\sphinxAtStartPar
\sphinxstylestrong{Data da Penhora}

\item {} 
\sphinxAtStartPar
\sphinxstylestrong{Data do Bloqueio} (se aplicável)

\item {} 
\sphinxAtStartPar
\sphinxstylestrong{Tipo de penhora}: física ou online

\item {} 
\sphinxAtStartPar
\sphinxstylestrong{Tipo de depositário}:
\sphinxhyphen{} Parte
\sphinxhyphen{} Depositário Judicial
\sphinxhyphen{} Depositário Particular

\end{itemize}

\item {} 
\sphinxAtStartPar
Identifique o \sphinxstylestrong{depositário}:
\begin{itemize}
\item {} 
\sphinxAtStartPar
Clique em \sphinxstylestrong{“Depositário”}

\item {} 
\sphinxAtStartPar
Pesquise pelo nome

\item {} 
\sphinxAtStartPar
Selecione o nome cadastrado

\end{itemize}

\item {} 
\sphinxAtStartPar
Clique em \sphinxstylestrong{Salvar}

\end{enumerate}


\subsubsection{Cadastro do Bem Penhorado}
\label{\detokenize{projud_21_cadastrobempenhora:cadastro-do-bem-penhorado}}\begin{enumerate}
\sphinxsetlistlabels{\arabic}{enumi}{enumii}{}{.}%
\item {} 
\sphinxAtStartPar
Após cadastrar o auto de penhora, clique em \sphinxstylestrong{“Adicionar Bem”}

\item {} 
\sphinxAtStartPar
Informe os dados do bem:
\begin{itemize}
\item {} 
\sphinxAtStartPar
Nome do bem (ex: veículo, imóvel)

\item {} 
\sphinxAtStartPar
Tipo do bem

\item {} 
\sphinxAtStartPar
Valor estimado

\item {} 
\sphinxAtStartPar
Parte a quem pertence o bem

\end{itemize}

\item {} 
\sphinxAtStartPar
Clique em \sphinxstylestrong{Salvar}

\item {} 
\sphinxAtStartPar
Selecione o bem cadastrado e clique em \sphinxstylestrong{“Adicionar ao Auto de Penhora”}

\item {} 
\sphinxAtStartPar
Finalize clicando em \sphinxstylestrong{“Salvar”} novamente

\end{enumerate}


\subsubsection{Confirmação}
\label{\detokenize{projud_21_cadastrobempenhora:confirmacao}}
\sphinxAtStartPar
Após a conclusão do cadastro:
\begin{itemize}
\item {} 
\sphinxAtStartPar
Uma mensagem de sucesso será exibida:
\sphinxstyleemphasis{“Dados registrados com sucesso”}

\item {} 
\sphinxAtStartPar
O auto de penhora será listado na aba \sphinxstylestrong{Informações Adicionais}

\end{itemize}


\subsubsection{Resumo}
\label{\detokenize{projud_21_cadastrobempenhora:resumo}}
\sphinxAtStartPar
Essa funcionalidade permite:
\begin{itemize}
\item {} 
\sphinxAtStartPar
Controlar e registrar autos de penhora de forma padronizada

\item {} 
\sphinxAtStartPar
Vincular bens e depositários a cada penhora

\item {} 
\sphinxAtStartPar
Organizar as informações patrimoniais do processo

\end{itemize}

\sphinxAtStartPar
O recurso é especialmente importante nos processos de \sphinxstylestrong{execução}, \sphinxstylestrong{cumprimento de sentença}, ou em \sphinxstylestrong{processos criminais com apreensão de bens}.

\sphinxstepscope


\subsection{Cadastro de Perito, Acordo e Suspeição}
\label{\detokenize{projud_22_cadastroperiodo:cadastro-de-perito-acordo-e-suspeicao}}\label{\detokenize{projud_22_cadastroperiodo::doc}}
\sphinxAtStartPar
Nesta aula, vamos abordar três funcionalidades importantes da aba \sphinxstylestrong{Informações Adicionais} no sistema ProJUDI:
\begin{itemize}
\item {} 
\sphinxAtStartPar
Cadastro de \sphinxstylestrong{perito}

\item {} 
\sphinxAtStartPar
Registro de \sphinxstylestrong{acordo}

\item {} 
\sphinxAtStartPar
Registro de \sphinxstylestrong{impedimento ou suspeição}

\end{itemize}


\subsubsection{Cadastro de Perito}
\label{\detokenize{projud_22_cadastroperiodo:cadastro-de-perito}}\begin{enumerate}
\sphinxsetlistlabels{\arabic}{enumi}{enumii}{}{.}%
\item {} 
\sphinxAtStartPar
Acesse a aba \sphinxstylestrong{Informações Adicionais}

\item {} 
\sphinxAtStartPar
Clique em \sphinxstylestrong{“Habilitações Provisórias”}

\item {} 
\sphinxAtStartPar
Clique em \sphinxstylestrong{“Novo”}

\end{enumerate}

\sphinxAtStartPar
\textgreater{} Lembre\sphinxhyphen{}se: o perito deve estar previamente cadastrado no sistema.

\sphinxAtStartPar
\#\#\# Como cadastrar o perito:
\begin{enumerate}
\sphinxsetlistlabels{\arabic}{enumi}{enumii}{}{.}%
\item {} 
\sphinxAtStartPar
Clique no botão de \sphinxstylestrong{pesquisa}

\item {} 
\sphinxAtStartPar
Selecione o nome do perito (ou pesquise por login, CPF ou nome)

\item {} 
\sphinxAtStartPar
Defina a \sphinxstylestrong{validade da habilitação provisória} (conforme decisão judicial)

\item {} 
\sphinxAtStartPar
Clique em \sphinxstylestrong{Salvar}

\end{enumerate}

\sphinxAtStartPar
Mensagem de confirmação:
Dados registrados com sucesso.

\sphinxAtStartPar
O perito será habilitado provisoriamente, e o sistema irá gerar uma \sphinxstylestrong{movimentação automática} vinculada ao processo.


\subsubsection{Cadastro de Acordo}
\label{\detokenize{projud_22_cadastroperiodo:cadastro-de-acordo}}\begin{enumerate}
\sphinxsetlistlabels{\arabic}{enumi}{enumii}{}{.}%
\item {} 
\sphinxAtStartPar
Ainda em \sphinxstylestrong{Informações Adicionais}, clique em \sphinxstylestrong{“Acordo”}

\item {} 
\sphinxAtStartPar
Clique em \sphinxstylestrong{“Novo”}

\item {} 
\sphinxAtStartPar
Preencha os campos:
\begin{itemize}
\item {} 
\sphinxAtStartPar
\sphinxstylestrong{Data do acordo}

\item {} 
\sphinxAtStartPar
\sphinxstylestrong{Data de cumprimento}

\item {} 
\sphinxAtStartPar
Cumprimento foi voluntário? (\sphinxstylestrong{Sim} ou \sphinxstylestrong{Não})

\item {} 
\sphinxAtStartPar
Observações (opcional)

\end{itemize}

\item {} 
\sphinxAtStartPar
Clique em \sphinxstylestrong{Salvar}

\end{enumerate}

\sphinxAtStartPar
Mensagem de confirmação:
Acordo cadastrado com sucesso.

\sphinxAtStartPar
O acordo aparecerá listado em \sphinxstylestrong{Informações Adicionais} e será registrado nas movimentações.


\subsubsection{Cadastro de Impedimento ou Suspensão}
\label{\detokenize{projud_22_cadastroperiodo:cadastro-de-impedimento-ou-suspensao}}\begin{enumerate}
\sphinxsetlistlabels{\arabic}{enumi}{enumii}{}{.}%
\item {} 
\sphinxAtStartPar
Clique em \sphinxstylestrong{“Impedimento/Suspeição”}

\item {} 
\sphinxAtStartPar
Clique em \sphinxstylestrong{“Novo”}

\item {} 
\sphinxAtStartPar
Preencha os campos:
\begin{itemize}
\item {} 
\sphinxAtStartPar
\sphinxstylestrong{Tipo}: \sphinxstyleemphasis{Impedimento} ou \sphinxstyleemphasis{Suspeição}

\item {} 
\sphinxAtStartPar
\sphinxstylestrong{Motivo}: selecione na lista de motivos disponíveis

\item {} 
\sphinxAtStartPar
\sphinxstylestrong{Pessoa impedida}: pode ser o magistrado, analista, promotor, defensor público, etc.

\item {} 
\sphinxAtStartPar
Observações (opcional)

\end{itemize}

\item {} 
\sphinxAtStartPar
Clique em \sphinxstylestrong{Salvar}

\end{enumerate}

\sphinxAtStartPar
Essas informações também serão registradas nas \sphinxstylestrong{movimentações} do processo.


\subsubsection{Resumo}
\label{\detokenize{projud_22_cadastroperiodo:resumo}}
\sphinxAtStartPar
A aba \sphinxstylestrong{Informações Adicionais} permite registrar elementos fundamentais que impactam diretamente a tramitação do processo:
\begin{itemize}
\item {} 
\sphinxAtStartPar
Habilitação de \sphinxstylestrong{peritos judiciais}

\item {} 
\sphinxAtStartPar
Formalização de \sphinxstylestrong{acordos}

\item {} 
\sphinxAtStartPar
Registro de \sphinxstylestrong{impedimentos e suspeições}

\end{itemize}

\sphinxAtStartPar
Esses registros geram movimentações automáticas, garantindo \sphinxstylestrong{transparência e rastreabilidade}.

\sphinxstepscope


\subsection{Cadastro de Parte e CPF/CNPJ}
\label{\detokenize{projud_23_cadastroparte:cadastro-de-parte-e-cpf-cnpj}}\label{\detokenize{projud_23_cadastroparte::doc}}
\sphinxAtStartPar
Nesta aula, vamos aprender como \sphinxstylestrong{cadastrar, alterar, excluir e regularizar partes} em um processo no sistema ProJUDI, incluindo a inserção de CPF ou CNPJ conforme exigência do \sphinxstylestrong{Provimento nº 61/2017 do CNJ}.


\subsubsection{Acessando a Aba de Partes}
\label{\detokenize{projud_23_cadastroparte:acessando-a-aba-de-partes}}\begin{enumerate}
\sphinxsetlistlabels{\arabic}{enumi}{enumii}{}{.}%
\item {} 
\sphinxAtStartPar
Acesse a \sphinxstylestrong{tela inicial do processo}

\item {} 
\sphinxAtStartPar
Clique no botão \sphinxstylestrong{“Partes”}

\end{enumerate}

\sphinxAtStartPar
Aparecerá a tela com as partes já cadastradas (autor, réu, vítima, etc.)


\subsubsection{Adicionando uma Nova Parte}
\label{\detokenize{projud_23_cadastroparte:adicionando-uma-nova-parte}}\begin{enumerate}
\sphinxsetlistlabels{\arabic}{enumi}{enumii}{}{.}%
\item {} 
\sphinxAtStartPar
Clique em \sphinxstylestrong{“Adicionar Parte”}

\item {} 
\sphinxAtStartPar
Selecione o \sphinxstylestrong{polo processual} (ex: polo ativo, passivo, vítima, testemunha)

\item {} 
\sphinxAtStartPar
Clique na \sphinxstylestrong{lupa} ao lado de \sphinxstyleemphasis{Nome} para abrir a tela de seleção de partes

\end{enumerate}

\sphinxAtStartPar
\#\#\# Tela de Seleção de Parte
\begin{itemize}
\item {} 
\sphinxAtStartPar
Você pode buscar por:
\sphinxhyphen{} CPF
\sphinxhyphen{} Nome
\sphinxhyphen{} Nome da mãe
\sphinxhyphen{} Tipo (pessoa física, jurídica, órgão, entidade)

\end{itemize}

\sphinxAtStartPar
\#\#\# Criando uma nova parte:
\begin{itemize}
\item {} 
\sphinxAtStartPar
Se não encontrar o cadastro, clique em \sphinxstylestrong{“Criar Novo”}

\item {} 
\sphinxAtStartPar
Informe o \sphinxstylestrong{CPF} (para pessoa física) ou \sphinxstylestrong{CNPJ} (para pessoa jurídica)

\item {} 
\sphinxAtStartPar
Preencha os dados obrigatórios

\item {} 
\sphinxAtStartPar
Clique em \sphinxstylestrong{Salvar}

\end{itemize}

\sphinxAtStartPar
Após isso, volte à tela de seleção, selecione a parte criada e \sphinxstylestrong{adicione ao processo}.

\sphinxAtStartPar
\textgreater{} Você também pode adicionar um \sphinxstylestrong{representante legal}, se necessário.


\subsubsection{Alterando uma Parte}
\label{\detokenize{projud_23_cadastroparte:alterando-uma-parte}}\begin{enumerate}
\sphinxsetlistlabels{\arabic}{enumi}{enumii}{}{.}%
\item {} 
\sphinxAtStartPar
Na tela de partes, clique sobre o nome da parte

\item {} 
\sphinxAtStartPar
Serão exibidas as opções:
\begin{itemize}
\item {} 
\sphinxAtStartPar
\sphinxstylestrong{Alterar Parte}: Corrigir nome, inserir CPF/CNPJ, alterar dados pessoais

\item {} 
\sphinxAtStartPar
\sphinxstylestrong{Alterar Polo}: Mudar o polo (ex: de autor para vítima)

\item {} 
\sphinxAtStartPar
\sphinxstylestrong{Dar Baixa}: Excluir a parte do processo (sem apagá\sphinxhyphen{}la do histórico)

\end{itemize}

\item {} 
\sphinxAtStartPar
Após qualquer alteração, clique em \sphinxstylestrong{Salvar}

\end{enumerate}

\sphinxAtStartPar
\textgreater{} Partes com baixa continuam visíveis no histórico e podem ser \sphinxstylestrong{reativadas}.


\subsubsection{Regularizando CPF ou CNPJ}
\label{\detokenize{projud_23_cadastroparte:regularizando-cpf-ou-cnpj}}
\sphinxAtStartPar
Se houver uma \sphinxstylestrong{tarja de atenção} indicando ausência de CPF ou CNPJ, siga os passos:
\begin{enumerate}
\sphinxsetlistlabels{\arabic}{enumi}{enumii}{}{.}%
\item {} 
\sphinxAtStartPar
Clique sobre a parte que está com pendência

\item {} 
\sphinxAtStartPar
Clique em \sphinxstylestrong{“Cadastrar CPF/CNPJ”}

\item {} 
\sphinxAtStartPar
Insira o número correto e clique em \sphinxstylestrong{Salvar}

\item {} 
\sphinxAtStartPar
O sistema mostrará:
\begin{itemize}
\item {} 
\sphinxAtStartPar
Como estava antes

\item {} 
\sphinxAtStartPar
Como ficará com a atualização

\end{itemize}

\item {} 
\sphinxAtStartPar
Clique em \sphinxstylestrong{Confirmar}

\end{enumerate}

\sphinxAtStartPar
\textgreater{} Após isso, a tarja de alerta desaparecerá, indicando que o processo está \sphinxstylestrong{em conformidade com o Provimento nº 61/2017}.


\subsubsection{Histórico de Partes}
\label{\detokenize{projud_23_cadastroparte:historico-de-partes}}
\sphinxAtStartPar
A aba de Partes mantém o \sphinxstylestrong{histórico completo}:
\begin{itemize}
\item {} 
\sphinxAtStartPar
Inclusões e exclusões

\item {} 
\sphinxAtStartPar
Alterações de polo

\item {} 
\sphinxAtStartPar
Reativações

\end{itemize}


\subsubsection{Resumo}
\label{\detokenize{projud_23_cadastroparte:resumo}}
\sphinxAtStartPar
A funcionalidade de \sphinxstylestrong{cadastro de partes} permite:

\sphinxAtStartPar
✅ Adicionar partes ao processo (físicas ou jurídicas)
✅ Corrigir dados pessoais e regularizar CPF/CNPJ
✅ Alterar o polo processual
✅ Excluir (dar baixa) ou reativar partes
✅ Atender às exigências do \sphinxstylestrong{CNJ} com transparência e rastreabilidade

\sphinxAtStartPar
Essas ações garantem um cadastro processual \sphinxstylestrong{completo, válido e atualizado} no ProJUDI.

\sphinxstepscope


\subsection{Cadastro de Parte com Citação Online}
\label{\detokenize{projud_24_cadastropartecitacaoonline:cadastro-de-parte-com-citacao-online}}\label{\detokenize{projud_24_cadastropartecitacaoonline::doc}}
\sphinxAtStartPar
Nesta aula, você vai aprender como \sphinxstylestrong{cadastrar uma parte (pessoa jurídica ou órgão público)} no sistema ProJUDI para que ela receba \sphinxstylestrong{citação eletrônica (citação online)}.


\subsubsection{O que é a Citação Online?}
\label{\detokenize{projud_24_cadastropartecitacaoonline:o-que-e-a-citacao-online}}
\sphinxAtStartPar
A \sphinxstylestrong{citação online} é realizada \sphinxstylestrong{dentro do próprio sistema ProJUDI}, sem uso do Diário da Justiça Eletrônico.

\sphinxAtStartPar
Características:
\begin{itemize}
\item {} 
\sphinxAtStartPar
O sistema envia \sphinxstylestrong{notificação automática via e\sphinxhyphen{}mail} para a parte cadastrada

\item {} 
\sphinxAtStartPar
Há integração com o sistema \sphinxstylestrong{push}, que avisa o usuário sobre a movimentação

\item {} 
\sphinxAtStartPar
Utilizada para grandes demandantes:
\sphinxhyphen{} Bancos
\sphinxhyphen{} Companhias aéreas
\sphinxhyphen{} Estado, INSS, PGE, etc.

\end{itemize}


\subsubsection{Passo a Passo para Cadastro da Parte com Citação Online}
\label{\detokenize{projud_24_cadastropartecitacaoonline:passo-a-passo-para-cadastro-da-parte-com-citacao-online}}\begin{enumerate}
\sphinxsetlistlabels{\arabic}{enumi}{enumii}{}{.}%
\item {} 
\sphinxAtStartPar
Clique na aba \sphinxstylestrong{Partes}

\item {} 
\sphinxAtStartPar
Selecione \sphinxstylestrong{Adicionar Parte}

\item {} 
\sphinxAtStartPar
Escolha o \sphinxstylestrong{Polo Processual} (ex: Polo Passivo)

\item {} 
\sphinxAtStartPar
Clique na \sphinxstylestrong{lupa} ao lado do campo \sphinxstyleemphasis{Nome} para abrir a tela de seleção

\end{enumerate}

\sphinxAtStartPar
\#\#\# Filtros recomendados:
\begin{itemize}
\item {} 
\sphinxAtStartPar
\sphinxstylestrong{Pessoa Jurídica} (ou órgão público)

\item {} 
\sphinxAtStartPar
Insira o \sphinxstylestrong{CNPJ} ou o nome completo

\item {} 
\sphinxAtStartPar
Marque a opção: \sphinxstylestrong{“Somente com citação online”}

\item {} 
\sphinxAtStartPar
Clique em \sphinxstylestrong{Pesquisar}

\end{itemize}
\begin{enumerate}
\sphinxsetlistlabels{\arabic}{enumi}{enumii}{}{.}%
\setcounter{enumi}{4}
\item {} 
\sphinxAtStartPar
Na lista de resultados, selecione a parte que exibe \sphinxstylestrong{“Citação Online”} ao lado do nome

\item {} 
\sphinxAtStartPar
Confirme o endereço e demais dados

\item {} 
\sphinxAtStartPar
Clique em \sphinxstylestrong{Salvar}

\end{enumerate}

\sphinxAtStartPar
\textgreater{} ✅ A parte está agora apta a receber \sphinxstylestrong{citações e intimações eletrônicas} diretamente pelo sistema ProJUDI.


\subsubsection{Informações Adicionais (opcional)}
\label{\detokenize{projud_24_cadastropartecitacaoonline:informacoes-adicionais-opcional}}
\sphinxAtStartPar
Antes de salvar, você pode preencher:
\begin{itemize}
\item {} 
\sphinxAtStartPar
\sphinxstylestrong{Se a parte é revel}

\item {} 
\sphinxAtStartPar
\sphinxstylestrong{Prioridade}:
\sphinxhyphen{} Maior de 60 anos
\sphinxhyphen{} Pessoa com deficiência
\sphinxhyphen{} Portador de doença grave

\end{itemize}


\subsubsection{Verificando o Cadastro}
\label{\detokenize{projud_24_cadastropartecitacaoonline:verificando-o-cadastro}}
\sphinxAtStartPar
Após salvar:
\begin{itemize}
\item {} 
\sphinxAtStartPar
Volte à tela inicial do processo

\item {} 
\sphinxAtStartPar
A parte aparecerá com o \sphinxstylestrong{selo de “Citação Online”}

\item {} 
\sphinxAtStartPar
Ao realizar uma citação ou intimação, o sistema notificará automaticamente a parte cadastrada

\end{itemize}


\subsubsection{Resumo}
\label{\detokenize{projud_24_cadastropartecitacaoonline:resumo}}
\sphinxAtStartPar
O cadastro correto da parte com \sphinxstylestrong{citação online} é essencial para:

\sphinxAtStartPar
✅ Garantir que empresas e órgãos recebam comunicações judiciais eletrônicas
✅ Automatizar citações e intimações
✅ Reduzir custos e acelerar a tramitação processual

\sphinxAtStartPar
Essa funcionalidade é indispensável para processos envolvendo \sphinxstylestrong{grandes demandantes}.

\sphinxstepscope


\subsection{Cadastro de Réu Preso}
\label{\detokenize{projud_25_cadastro reupreso:cadastro-de-reu-preso}}\label{\detokenize{projud_25_cadastro reupreso::doc}}
\sphinxAtStartPar
Nesta aula, você aprenderá como \sphinxstylestrong{registrar prisões e solturas} no sistema ProJUDI, tanto na área \sphinxstylestrong{criminal} quanto \sphinxstylestrong{cível} (prisões civis, como as de alimentos).


\subsubsection{Cadastro de Prisão}
\label{\detokenize{projud_25_cadastro reupreso:cadastro-de-prisao}}\begin{enumerate}
\sphinxsetlistlabels{\arabic}{enumi}{enumii}{}{.}%
\item {} 
\sphinxAtStartPar
Acesse a aba \sphinxstylestrong{Partes}

\item {} 
\sphinxAtStartPar
Clique sobre o nome da parte (réu/iniciado)

\item {} 
\sphinxAtStartPar
Na tela de edição, clique em \sphinxstylestrong{Informações Adicionais}

\item {} 
\sphinxAtStartPar
Clique na aba \sphinxstylestrong{Prisões}

\item {} 
\sphinxAtStartPar
Clique em \sphinxstylestrong{“Adicionar”}

\end{enumerate}

\sphinxAtStartPar
\#\#\# Preencha os campos:
\begin{itemize}
\item {} 
\sphinxAtStartPar
\sphinxstylestrong{Data da prisão}

\item {} 
\sphinxAtStartPar
\sphinxstylestrong{Número da guia de prisão}

\item {} 
\sphinxAtStartPar
\sphinxstylestrong{Motivo da prisão} (ex: flagrante, prisão civil)

\item {} 
\sphinxAtStartPar
\sphinxstylestrong{Local da prisão}

\end{itemize}
\begin{enumerate}
\sphinxsetlistlabels{\arabic}{enumi}{enumii}{}{.}%
\setcounter{enumi}{5}
\item {} 
\sphinxAtStartPar
Clique em \sphinxstylestrong{Salvar}

\end{enumerate}

\sphinxAtStartPar
\textgreater{} ✅ A prisão será registrada e constará no sistema com \sphinxstylestrong{transparência e rastreabilidade}.


\subsubsection{Identificação do Réu Preso}
\label{\detokenize{projud_25_cadastro reupreso:identificacao-do-reu-preso}}
\sphinxAtStartPar
Após o cadastro da prisão:
\begin{itemize}
\item {} 
\sphinxAtStartPar
O número do processo aparece em \sphinxstylestrong{vermelho} na tela inicial

\item {} 
\sphinxAtStartPar
Um \sphinxstylestrong{ícone de cela com figura humana} é exibido

\item {} 
\sphinxAtStartPar
Ao passar o mouse sobre o ícone, aparecem:
\sphinxhyphen{} Status: \sphinxstylestrong{Réu Preso}
\sphinxhyphen{} \sphinxstylestrong{Quantidade de dias preso}

\end{itemize}

\sphinxAtStartPar
Na aba \sphinxstylestrong{Partes}, o nome do réu também aparece com a tag \sphinxstylestrong{“Réu Preso”}.


\subsubsection{Conversão da Prisão (Ex: Flagrante \(\rightarrow\) Preventiva)}
\label{\detokenize{projud_25_cadastro reupreso:conversao-da-prisao-ex-flagrante-preventiva}}\begin{enumerate}
\sphinxsetlistlabels{\arabic}{enumi}{enumii}{}{.}%
\item {} 
\sphinxAtStartPar
Na aba \sphinxstylestrong{Prisões}, clique em \sphinxstylestrong{“Converter tipo”}

\item {} 
\sphinxAtStartPar
Insira:
\sphinxhyphen{} \sphinxstylestrong{Data da nova prisão} (ex: conversão para preventiva)
\sphinxhyphen{} \sphinxstylestrong{Novo número da guia}
\sphinxhyphen{} \sphinxstylestrong{Novo motivo da prisão}
\sphinxhyphen{} \sphinxstylestrong{Local}
\sphinxhyphen{} \sphinxstylestrong{Mandado de prisão} (se aplicável)

\item {} 
\sphinxAtStartPar
Clique em \sphinxstylestrong{Salvar}

\end{enumerate}

\sphinxAtStartPar
O sistema manterá o \sphinxstylestrong{histórico das prisões} (ex: flagrante, preventiva), com contagem automática de dias.


\subsubsection{Cadastro de Soltura}
\label{\detokenize{projud_25_cadastro reupreso:cadastro-de-soltura}}\begin{enumerate}
\sphinxsetlistlabels{\arabic}{enumi}{enumii}{}{.}%
\item {} 
\sphinxAtStartPar
Acesse a \sphinxstylestrong{última prisão} registrada

\item {} 
\sphinxAtStartPar
Clique na \sphinxstylestrong{data da prisão}

\item {} 
\sphinxAtStartPar
Na tela de alteração, selecione:
\sphinxhyphen{} \sphinxstylestrong{“Continua preso?” \(\rightarrow\) Não}

\item {} 
\sphinxAtStartPar
Preencha:
\sphinxhyphen{} \sphinxstylestrong{Data da soltura}
\sphinxhyphen{} \sphinxstylestrong{Número da guia de alvará de soltura}
\sphinxhyphen{} \sphinxstylestrong{Motivo da soltura}:
\begin{itemize}
\item {} 
\sphinxAtStartPar
Absolvição

\item {} 
\sphinxAtStartPar
Habeas corpus

\item {} 
\sphinxAtStartPar
Liberdade provisória

\item {} 
\sphinxAtStartPar
Fiança ou sem fiança

\item {} 
\sphinxAtStartPar
Outros

\end{itemize}
\begin{itemize}
\item {} 
\sphinxAtStartPar
\sphinxstylestrong{Anexe o alvará} (em PDF)

\end{itemize}

\item {} 
\sphinxAtStartPar
Clique em \sphinxstylestrong{Salvar}

\end{enumerate}

\sphinxAtStartPar
Resultado:
\begin{itemize}
\item {} 
\sphinxAtStartPar
O status de \sphinxstylestrong{réu preso} é \sphinxstylestrong{removido}

\item {} 
\sphinxAtStartPar
O processo deixa de ter \sphinxstylestrong{prioridade visual}

\item {} 
\sphinxAtStartPar
O ícone da cela desaparece da tela inicial e da aba de partes

\end{itemize}


\subsubsection{Resumo}
\label{\detokenize{projud_25_cadastro reupreso:resumo}}
\sphinxAtStartPar
A funcionalidade de \sphinxstylestrong{Cadastro de Prisão e Soltura} no ProJUDI permite:

\sphinxAtStartPar
✅ Transparência na \sphinxstylestrong{situação carcerária} da parte
✅ Controle automático de \sphinxstylestrong{dias de prisão}
✅ Registro de \sphinxstylestrong{histórico de prisões e solturas}
✅ Conformidade com decisões judiciais e mandados

\sphinxAtStartPar
É essencial para o andamento adequado de processos \sphinxstylestrong{criminais e civis com prisão}.

\sphinxstepscope


\subsection{Habilitar Advogado}
\label{\detokenize{projud_26_habilitaradvogado:habilitar-advogado}}\label{\detokenize{projud_26_habilitaradvogado::doc}}
\sphinxAtStartPar
Nesta aula, você vai aprender como \sphinxstylestrong{habilitar um advogado} no sistema ProJUDI, vinculando\sphinxhyphen{}o corretamente à parte que ele irá representar no processo.


\subsubsection{Passo a Passo para Habilitar Advogado}
\label{\detokenize{projud_26_habilitaradvogado:passo-a-passo-para-habilitar-advogado}}\begin{enumerate}
\sphinxsetlistlabels{\arabic}{enumi}{enumii}{}{.}%
\item {} 
\sphinxAtStartPar
Acesse o processo desejado

\item {} 
\sphinxAtStartPar
Clique na aba \sphinxstylestrong{“Advogados”}

\item {} 
\sphinxAtStartPar
Clique em \sphinxstylestrong{“Adicionar”} (localizado na parte inferior da tela)

\end{enumerate}

\sphinxAtStartPar
\#\#\# Tela de Seleção de Advogado
\begin{itemize}
\item {} 
\sphinxAtStartPar
Você pode filtrar por:
\sphinxhyphen{} \sphinxstylestrong{Número da OAB}
\sphinxhyphen{} \sphinxstylestrong{Nome do advogado}

\end{itemize}

\sphinxAtStartPar
\textgreater{} Também é possível selecionar apenas advogados \sphinxstylestrong{particulares} ou da \sphinxstylestrong{Defensoria Pública}.
\begin{enumerate}
\sphinxsetlistlabels{\arabic}{enumi}{enumii}{}{.}%
\setcounter{enumi}{3}
\item {} 
\sphinxAtStartPar
Após localizar o advogado desejado, clique em \sphinxstylestrong{Selecionar}

\item {} 
\sphinxAtStartPar
Escolha a \sphinxstylestrong{parte que será representada}:
\sphinxhyphen{} Requerente
\sphinxhyphen{} Requerido
\sphinxhyphen{} Iniciado
\sphinxhyphen{} Vítima
\sphinxhyphen{} Outro (conforme o caso)

\item {} 
\sphinxAtStartPar
Clique em \sphinxstylestrong{Salvar}

\end{enumerate}


\subsubsection{Confirmação}
\label{\detokenize{projud_26_habilitaradvogado:confirmacao}}
\sphinxAtStartPar
Após a habilitação:
\begin{itemize}
\item {} 
\sphinxAtStartPar
O advogado será \sphinxstylestrong{vinculado à parte selecionada}

\item {} 
\sphinxAtStartPar
A informação ficará visível na aba \sphinxstylestrong{Partes} e \sphinxstylestrong{Advogados} do processo

\end{itemize}


\subsubsection{Resumo}
\label{\detokenize{projud_26_habilitaradvogado:resumo}}
\sphinxAtStartPar
A funcionalidade de habilitação de advogado permite:

\sphinxAtStartPar
✅ Associar corretamente representantes legais às partes processuais
✅ Registrar defensores públicos ou advogados particulares
✅ Garantir a atuação jurídica no processo conforme a legislação

\sphinxAtStartPar
Essa etapa é essencial para assegurar a \sphinxstylestrong{representação adequada} das partes nos autos.

\sphinxstepscope


\subsection{Desmembramento de Processos}
\label{\detokenize{projud_27_desmembramentoprocesso:desmembramento-de-processos}}\label{\detokenize{projud_27_desmembramentoprocesso::doc}}
\sphinxAtStartPar
Nesta aula, você vai aprender como realizar o \sphinxstylestrong{desmembramento de processos} no sistema ProJUDI — procedimento utilizado quando é necessário \sphinxstylestrong{separar partes ou documentos} de um processo para gerar um \sphinxstylestrong{novo processo autônomo}.


\subsubsection{Acesso à Funcionalidade}
\label{\detokenize{projud_27_desmembramentoprocesso:acesso-a-funcionalidade}}\begin{enumerate}
\sphinxsetlistlabels{\arabic}{enumi}{enumii}{}{.}%
\item {} 
\sphinxAtStartPar
Acesse a \sphinxstylestrong{aba Partes}

\item {} 
\sphinxAtStartPar
No canto inferior direito, clique no botão \sphinxstylestrong{“Desmembrar”}

\end{enumerate}


\subsubsection{Etapas do Desmembramento}
\label{\detokenize{projud_27_desmembramentoprocesso:etapas-do-desmembramento}}
\sphinxAtStartPar
\#\#\# 1. Seleção de Documentos
\begin{itemize}
\item {} 
\sphinxAtStartPar
A tela \sphinxstylestrong{Cadastro de Desmembramento} será exibida

\item {} 
\sphinxAtStartPar
Selecione os \sphinxstylestrong{documentos} que devem compor o novo processo

\item {} 
\sphinxAtStartPar
Você pode:
\sphinxhyphen{} Marcar documento por documento
\sphinxhyphen{} Marcar todos de uma vez (opção recomendada)

\end{itemize}

\sphinxAtStartPar
Clique em \sphinxstylestrong{OK} para continuar.

\sphinxAtStartPar
\#\#\# 2. Seleção de Partes

\sphinxAtStartPar
Na próxima tela:
\begin{itemize}
\item {} 
\sphinxAtStartPar
Selecione as \sphinxstylestrong{partes que serão copiadas} para o novo processo
(ex: Ministério Público, testemunhas, alguns indiciados)

\item {} 
\sphinxAtStartPar
Selecione as \sphinxstylestrong{partes que serão movidas} definitivamente
(ex: o promovido, réu específico, etc.)

\end{itemize}

\sphinxAtStartPar
Clique em \sphinxstylestrong{Próximo Passo}.

\sphinxAtStartPar
\#\#\# 3. Verificação e Confirmação
\begin{itemize}
\item {} 
\sphinxAtStartPar
Verifique se as seleções estão corretas

\item {} 
\sphinxAtStartPar
Confirme:
\sphinxhyphen{} Quais partes serão \sphinxstylestrong{copiadas}
\sphinxhyphen{} Quais partes serão \sphinxstylestrong{movidas}

\end{itemize}

\sphinxAtStartPar
Clique em \sphinxstylestrong{Salvar}.


\subsubsection{Resultado}
\label{\detokenize{projud_27_desmembramentoprocesso:resultado}}\begin{itemize}
\item {} 
\sphinxAtStartPar
O desmembramento será realizado com sucesso

\item {} 
\sphinxAtStartPar
Um \sphinxstylestrong{novo número de processo} será gerado, com \sphinxstylestrong{0 dias de tramitação}

\item {} 
\sphinxAtStartPar
O processo original exibirá uma \sphinxstylestrong{referência ao processo desmembrado}

\item {} 
\sphinxAtStartPar
O novo processo herdará os documentos e partes conforme selecionado

\end{itemize}

\sphinxAtStartPar
\textgreater{} ✅ É possível clicar no número do novo processo para acessá\sphinxhyphen{}lo diretamente
\textgreater{} ✅ A tramitação dos processos seguirá de forma \sphinxstylestrong{independente}


\subsubsection{Resumo}
\label{\detokenize{projud_27_desmembramentoprocesso:resumo}}
\sphinxAtStartPar
O \sphinxstylestrong{desmembramento de processos} é útil em situações como:
\begin{itemize}
\item {} 
\sphinxAtStartPar
Separação de réus em processo criminal

\item {} 
\sphinxAtStartPar
Destacamento de parte da ação para outra vara ou juízo

\item {} 
\sphinxAtStartPar
Agilização da tramitação de temas autônomos

\end{itemize}

\sphinxAtStartPar
No ProJUDI, esse procedimento é feito de forma \sphinxstylestrong{guiada, segura e rastreável}.

\sphinxstepscope


\subsection{Apensamentos}
\label{\detokenize{projud_28_apensamentos:apensamentos}}\label{\detokenize{projud_28_apensamentos::doc}}
\sphinxAtStartPar
Nesta aula, você vai aprender a realizar o \sphinxstylestrong{apensamento de processos} no sistema ProJUDI. O apensamento é utilizado quando dois ou mais processos precisam tramitar \sphinxstylestrong{de forma conjunta}, por possuírem conexão ou dependência temática.


\subsubsection{Acessando a Aba de Apensamentos}
\label{\detokenize{projud_28_apensamentos:acessando-a-aba-de-apensamentos}}\begin{enumerate}
\sphinxsetlistlabels{\arabic}{enumi}{enumii}{}{.}%
\item {} 
\sphinxAtStartPar
Na \sphinxstylestrong{tela inicial do processo}, clique na aba \sphinxstylestrong{“Apensamentos”}

\item {} 
\sphinxAtStartPar
No canto inferior direito, clique em \sphinxstylestrong{“Gerenciar”}

\item {} 
\sphinxAtStartPar
Na tela que abrir, clique em \sphinxstylestrong{“Adicionar”}

\end{enumerate}


\subsubsection{Cadastro de Apensamento}
\label{\detokenize{projud_28_apensamentos:cadastro-de-apensamento}}
\sphinxAtStartPar
\#\#\# Tipo de Apensamento

\sphinxAtStartPar
Você tem duas opções:
\begin{itemize}
\item {} 
\sphinxAtStartPar
\sphinxstylestrong{Apensar este processo a um principal} (mais comum)

\item {} 
\sphinxAtStartPar
\sphinxstylestrong{Tornar este processo o principal e apensar outros a ele}

\end{itemize}

\sphinxAtStartPar
Selecione a segunda opção:
\sphinxstylestrong{“Apensar este processo a um principal”}

\sphinxAtStartPar
\#\#\# Informações a Preencher
\begin{enumerate}
\sphinxsetlistlabels{\arabic}{enumi}{enumii}{}{.}%
\item {} 
\sphinxAtStartPar
\sphinxstylestrong{Processo principal}:
\sphinxhyphen{} Informe o número do processo ao qual este será apensado
\sphinxhyphen{} Escolha o tipo (Projudi ou físico)

\item {} 
\sphinxAtStartPar
\sphinxstylestrong{Motivo do apensamento}:
\sphinxhyphen{} Exemplo: Determinação Judicial, conexão, continência, etc.

\item {} 
\sphinxAtStartPar
\sphinxstylestrong{Forma de bloqueio}:
\sphinxhyphen{} \sphinxstylestrong{Bloquear o principal}
\sphinxhyphen{} \sphinxstylestrong{Bloquear o apenso}
\sphinxhyphen{} \sphinxstylestrong{Não bloquear}

\item {} 
\sphinxAtStartPar
\sphinxstylestrong{Observações} (opcional):
\sphinxhyphen{} Ex: \sphinxstyleemphasis{“Conforme decisão de fls. 24 do processo 0001234\sphinxhyphen{}56.2024.8.04.0001”}

\item {} 
\sphinxAtStartPar
Clique em \sphinxstylestrong{Salvar}

\end{enumerate}


\subsubsection{Confirmação do Apensamento}
\label{\detokenize{projud_28_apensamentos:confirmacao-do-apensamento}}
\sphinxAtStartPar
Após salvar:
\begin{itemize}
\item {} 
\sphinxAtStartPar
O apensamento será registrado com:
\sphinxhyphen{} \sphinxstylestrong{Data}
\sphinxhyphen{} \sphinxstylestrong{Número do processo principal}
\sphinxhyphen{} \sphinxstylestrong{Número do apenso (atual)}

\end{itemize}


\subsubsection{Visualização}
\label{\detokenize{projud_28_apensamentos:visualizacao}}
\sphinxAtStartPar
Na \sphinxstylestrong{tela inicial do processo}:
\begin{itemize}
\item {} 
\sphinxAtStartPar
Será exibida a informação de \sphinxstylestrong{processo apensado}

\item {} 
\sphinxAtStartPar
Você verá:
\sphinxhyphen{} Número do \sphinxstylestrong{processo principal}
\sphinxhyphen{} Relação dos \sphinxstylestrong{apensos}, se houver

\end{itemize}


\subsubsection{Resumo}
\label{\detokenize{projud_28_apensamentos:resumo}}
\sphinxAtStartPar
A funcionalidade de \sphinxstylestrong{apensamento} é útil para:

\sphinxAtStartPar
✅ Processos com objetos similares ou interdependentes
✅ Situações de conexão ou continência
✅ Execuções relacionadas à mesma sentença

\sphinxAtStartPar
\textgreater{} O ProJUDI permite fazer o controle de apensamento com rastreabilidade e bloqueio, se necessário.

\sphinxstepscope


\subsection{Intimação Pessoal por AR Digital}
\label{\detokenize{projud_29_intimacaoARdigital:intimacao-pessoal-por-ar-digital}}\label{\detokenize{projud_29_intimacaoARdigital::doc}}
\sphinxAtStartPar
Nesta aula, você vai aprender como realizar uma \sphinxstylestrong{intimação pessoal por AR digital} no sistema ProJUDI. Esse tipo de intimação é utilizada quando a parte precisa ser notificada fisicamente, mas o processo tramita digitalmente.


\subsubsection{Início do Procedimento}
\label{\detokenize{projud_29_intimacaoARdigital:inicio-do-procedimento}}\begin{enumerate}
\sphinxsetlistlabels{\arabic}{enumi}{enumii}{}{.}%
\item {} 
\sphinxAtStartPar
Acesse o processo desejado

\item {} 
\sphinxAtStartPar
Vá até a aba \sphinxstylestrong{“Movimentações”}

\item {} 
\sphinxAtStartPar
Clique sobre a movimentação que determina a expedição da intimação (ex: decisão, sentença ou ato ordinatório)

\item {} 
\sphinxAtStartPar
Clique em \sphinxstylestrong{“Movimentar a partir desta”}

\item {} 
\sphinxAtStartPar
Selecione \sphinxstylestrong{“Intimação”}

\end{enumerate}


\subsubsection{Configuração da Intimação}
\label{\detokenize{projud_29_intimacaoARdigital:configuracao-da-intimacao}}
\sphinxAtStartPar
Na tela de intimação:
\begin{itemize}
\item {} 
\sphinxAtStartPar
Selecione: \sphinxstylestrong{“Intimação pessoal”}

\item {} 
\sphinxAtStartPar
Informe o \sphinxstylestrong{prazo} em dias

\item {} 
\sphinxAtStartPar
Se desejar, marque como \sphinxstylestrong{urgente}

\item {} 
\sphinxAtStartPar
Clique em \sphinxstylestrong{“Intimar”}

\end{itemize}

\sphinxAtStartPar
\textgreater{} ✅ A intimação será registrada como \sphinxstylestrong{pendente de expedição}


\subsubsection{Expedição da Intimação}
\label{\detokenize{projud_29_intimacaoARdigital:expedicao-da-intimacao}}\begin{enumerate}
\sphinxsetlistlabels{\arabic}{enumi}{enumii}{}{.}%
\item {} 
\sphinxAtStartPar
Clique sobre a pendência gerada (indicada na tela)

\item {} 
\sphinxAtStartPar
Na tela \sphinxstylestrong{“Expedir Intimação”}, insira:
\sphinxhyphen{} \sphinxstylestrong{Prazo}
\sphinxhyphen{} Marque a caixa \sphinxstylestrong{“Usar anexo”} (se necessário)
\sphinxhyphen{} Selecione os documentos adicionais (ex: petição inicial, denúncia)

\item {} 
\sphinxAtStartPar
Clique em \sphinxstylestrong{“Usar AR Digital”}

\item {} 
\sphinxAtStartPar
Selecione:
\sphinxhyphen{} O \sphinxstylestrong{endereço}
\sphinxhyphen{} O \sphinxstylestrong{destinatário}

\end{enumerate}


\subsubsection{Inclusão da Carta de Intimação}
\label{\detokenize{projud_29_intimacaoARdigital:inclusao-da-carta-de-intimacao}}
\sphinxAtStartPar
Você pode usar duas formas para incluir o conteúdo da carta:

\sphinxAtStartPar
\#\#\# Opção 1: Anexar PDF
\begin{itemize}
\item {} 
\sphinxAtStartPar
Clique em \sphinxstylestrong{“Anexar”}

\item {} 
\sphinxAtStartPar
Insira a descrição

\item {} 
\sphinxAtStartPar
Escolha o arquivo PDF

\item {} 
\sphinxAtStartPar
Clique em \sphinxstylestrong{“Enviar”}

\end{itemize}

\sphinxAtStartPar
\#\#\# Opção 2: Digitar texto
\begin{itemize}
\item {} 
\sphinxAtStartPar
Clique em \sphinxstylestrong{“Digitar texto”}

\item {} 
\sphinxAtStartPar
Selecione um modelo já salvo no sistema (ex: carta de intimação)

\item {} 
\sphinxAtStartPar
O texto será preenchido automaticamente com as variáveis configuradas

\end{itemize}


\subsubsection{Finalização}
\label{\detokenize{projud_29_intimacaoARdigital:finalizacao}}\begin{enumerate}
\sphinxsetlistlabels{\arabic}{enumi}{enumii}{}{.}%
\item {} 
\sphinxAtStartPar
Clique em \sphinxstylestrong{“Continuar”}

\item {} 
\sphinxAtStartPar
Clique em \sphinxstylestrong{“Salvar e Concluir”}

\item {} 
\sphinxAtStartPar
Informe a sua \sphinxstylestrong{senha}

\item {} 
\sphinxAtStartPar
Clique em \sphinxstylestrong{“Assinar e Expedir”}

\end{enumerate}

\sphinxAtStartPar
\textgreater{} ✅ Expedição realizada com sucesso


\subsubsection{Verificação no Processo}
\label{\detokenize{projud_29_intimacaoARdigital:verificacao-no-processo}}
\sphinxAtStartPar
Para verificar a expedição:
\begin{enumerate}
\sphinxsetlistlabels{\arabic}{enumi}{enumii}{}{.}%
\item {} 
\sphinxAtStartPar
Acesse a aba \sphinxstylestrong{“Navegar”}

\item {} 
\sphinxAtStartPar
Clique em \sphinxstylestrong{“Detalhes da Movimentação”}

\item {} 
\sphinxAtStartPar
Localize a movimentação:
\sphinxhyphen{} \sphinxstyleemphasis{Carta expedida \textendash{} AR digital}
\sphinxhyphen{} Documento da \sphinxstylestrong{carta}
\sphinxhyphen{} \sphinxstyleemphasis{Aviso de Recebimento dos Correios}, se houver

\end{enumerate}


\subsubsection{Resumo}
\label{\detokenize{projud_29_intimacaoARdigital:resumo}}
\sphinxAtStartPar
A \sphinxstylestrong{intimação pessoal via AR digital} permite:

\sphinxAtStartPar
✅ Maior agilidade e rastreabilidade na comunicação processual
✅ Geração de AR digital e controle pela própria secretaria
✅ Registro automático na movimentação do processo

\sphinxAtStartPar
Essa funcionalidade é essencial para processos que exigem intimações físicas, mesmo em meio eletrônico.

\sphinxstepscope


\subsection{Citação Eletrônica}
\label{\detokenize{projud_30_cita_xe7_xe3oeletronica:citacao-eletronica}}\label{\detokenize{projud_30_cita_xe7_xe3oeletronica::doc}}
\sphinxAtStartPar
Nesta aula, você vai aprender como realizar a \sphinxstylestrong{citação eletrônica} no sistema ProJUDI, utilizando \sphinxstylestrong{mandado digital com contrafé}, especialmente para partes que possuem \sphinxstylestrong{cadastro com citação online} (como municípios, entes públicos ou grandes demandantes).


\subsubsection{Iniciando a Citação}
\label{\detokenize{projud_30_cita_xe7_xe3oeletronica:iniciando-a-citacao}}\begin{enumerate}
\sphinxsetlistlabels{\arabic}{enumi}{enumii}{}{.}%
\item {} 
\sphinxAtStartPar
Acesse o processo desejado

\item {} 
\sphinxAtStartPar
Vá até a aba \sphinxstylestrong{“Movimentações”}

\item {} 
\sphinxAtStartPar
Localize a \sphinxstylestrong{decisão} que determinou a expedição de citação

\item {} 
\sphinxAtStartPar
Clique em \sphinxstylestrong{“Movimentar a partir desta”}

\item {} 
\sphinxAtStartPar
No painel lateral esquerdo, clique em \sphinxstylestrong{“Citar Partes”}

\end{enumerate}


\subsubsection{Configuração da Citação}
\label{\detokenize{projud_30_cita_xe7_xe3oeletronica:configuracao-da-citacao}}\begin{itemize}
\item {} 
\sphinxAtStartPar
Selecione as partes a serem citadas

\item {} 
\sphinxAtStartPar
Informe o \sphinxstylestrong{prazo legal} (ex: \sphinxstyleemphasis{prazo em dobro} se for ente público)

\item {} 
\sphinxAtStartPar
Clique em \sphinxstylestrong{“Citar”}

\end{itemize}

\sphinxAtStartPar
\textgreater{} ✅ Será gerada uma \sphinxstylestrong{pendência de expedição de citação}, com possibilidade de incluir mandado e contrafé


\subsubsection{Expedição da Citação}
\label{\detokenize{projud_30_cita_xe7_xe3oeletronica:expedicao-da-citacao}}\begin{enumerate}
\sphinxsetlistlabels{\arabic}{enumi}{enumii}{}{.}%
\item {} 
\sphinxAtStartPar
Clique em \sphinxstylestrong{“Citações”} (alerta de pendência)

\item {} 
\sphinxAtStartPar
Clique em \sphinxstylestrong{“Visualizar”} para acessar a tela de expedição

\end{enumerate}

\sphinxAtStartPar
Nesta tela, preencha:
\begin{itemize}
\item {} 
\sphinxAtStartPar
\sphinxstylestrong{Município} de destino

\item {} 
\sphinxAtStartPar
Marque a opção \sphinxstylestrong{“Enviar contrafé/petição inicial”}

\item {} 
\sphinxAtStartPar
Clique em \sphinxstylestrong{“Selecionar Arquivo”} e anexe a \sphinxstylestrong{petição inicial}

\end{itemize}


\subsubsection{Criação do Mandado de Citação}
\label{\detokenize{projud_30_cita_xe7_xe3oeletronica:criacao-do-mandado-de-citacao}}
\sphinxAtStartPar
Você tem duas opções:

\sphinxAtStartPar
\#\#\# 1. Anexar Arquivo PDF
\begin{itemize}
\item {} 
\sphinxAtStartPar
Clique em \sphinxstylestrong{“Anexar Arquivo”}

\item {} 
\sphinxAtStartPar
Descreva o conteúdo

\item {} 
\sphinxAtStartPar
Escolha o arquivo PDF

\item {} 
\sphinxAtStartPar
Envie o documento

\end{itemize}

\sphinxAtStartPar
\#\#\# 2. Utilizar Texto Digitável (modelo interno)
\begin{itemize}
\item {} 
\sphinxAtStartPar
Clique em \sphinxstylestrong{“Digitar Texto”}

\item {} 
\sphinxAtStartPar
Escolha um modelo já cadastrado (ex: \sphinxstyleemphasis{Mandado de Citação})

\item {} 
\sphinxAtStartPar
Preencha o texto (com variáveis automáticas)

\item {} 
\sphinxAtStartPar
Clique em \sphinxstylestrong{“Continuar”}

\item {} 
\sphinxAtStartPar
Revise e clique em \sphinxstylestrong{“Salvar”}

\end{itemize}


\subsubsection{Finalização}
\label{\detokenize{projud_30_cita_xe7_xe3oeletronica:finalizacao}}\begin{itemize}
\item {} 
\sphinxAtStartPar
Clique em \sphinxstylestrong{“Salvar e Concluir”}

\item {} 
\sphinxAtStartPar
Digite sua \sphinxstylestrong{senha de certificação}

\item {} 
\sphinxAtStartPar
Clique em \sphinxstylestrong{“Assinar e Expedir”}

\end{itemize}

\sphinxAtStartPar
\textgreater{} 💡 Se você \sphinxstylestrong{não tiver permissão para assinar}, clique apenas em \sphinxstylestrong{Salvar}, e o documento ficará pendente de assinatura por outro usuário autorizado.


\subsubsection{Verificação da Citação}
\label{\detokenize{projud_30_cita_xe7_xe3oeletronica:verificacao-da-citacao}}
\sphinxAtStartPar
\#\#\# Tela Inicial
\begin{itemize}
\item {} 
\sphinxAtStartPar
A citação aparecerá listada com o status \sphinxstylestrong{“online”}

\item {} 
\sphinxAtStartPar
O mandado de citação gerado poderá ser visualizado em PDF

\end{itemize}

\sphinxAtStartPar
\#\#\# Tela Navegar
\begin{enumerate}
\sphinxsetlistlabels{\arabic}{enumi}{enumii}{}{.}%
\item {} 
\sphinxAtStartPar
Acesse a aba \sphinxstylestrong{“Navegar”}

\item {} 
\sphinxAtStartPar
Desça até a movimentação de citação

\item {} 
\sphinxAtStartPar
Clique em \sphinxstylestrong{“Detalhes”} para verificar:
\sphinxhyphen{} Data da expedição
\sphinxhyphen{} Documento expedido (mandado)
\sphinxhyphen{} Confirmação de envio da contrafé

\end{enumerate}


\subsubsection{Resumo}
\label{\detokenize{projud_30_cita_xe7_xe3oeletronica:resumo}}
\sphinxAtStartPar
A \sphinxstylestrong{citação eletrônica} via ProJUDI possibilita:

\sphinxAtStartPar
✅ Comunicação ágil com entes cadastrados no sistema
✅ Geração automatizada de mandados com contrafé
✅ Registro e controle da entrega pelo sistema

\sphinxAtStartPar
\textgreater{} Essa funcionalidade é amplamente utilizada para citar \sphinxstylestrong{entes públicos}, como \sphinxstylestrong{municípios, estados, INSS, bancos} e \sphinxstylestrong{empresas conveniadas} com citação online.

\sphinxstepscope


\subsection{Contagem do Prazo nas Intimações e Citações}
\label{\detokenize{projud_31_contagemprazo:contagem-do-prazo-nas-intimacoes-e-citacoes}}\label{\detokenize{projud_31_contagemprazo::doc}}
\sphinxAtStartPar
Nesta aula, vamos aprender como o sistema \sphinxstylestrong{ProJUDI} realiza a \sphinxstylestrong{contagem dos prazos} em citações e intimações eletrônicas.


\subsubsection{Visão Geral}
\label{\detokenize{projud_31_contagemprazo:visao-geral}}
\sphinxAtStartPar
O sistema ProJUDI permite que você visualize de forma \sphinxstylestrong{automática e detalhada} a contagem de prazos a partir da \sphinxstylestrong{data da intimação ou citação} e da \sphinxstylestrong{leitura realizada}.


\subsubsection{Acesso à Contagem}
\label{\detokenize{projud_31_contagemprazo:acesso-a-contagem}}\begin{enumerate}
\sphinxsetlistlabels{\arabic}{enumi}{enumii}{}{.}%
\item {} 
\sphinxAtStartPar
Acesse o processo desejado

\item {} 
\sphinxAtStartPar
Vá até a aba \sphinxstylestrong{“Movimentações”}

\item {} 
\sphinxAtStartPar
Clique na movimentação relacionada à citação ou intimação

\item {} 
\sphinxAtStartPar
Ou clique diretamente em \sphinxstylestrong{“Pendências”} (Ex: \sphinxstyleemphasis{decorrência de prazo})

\end{enumerate}


\subsubsection{Análise por Decurso de Prazo}
\label{\detokenize{projud_31_contagemprazo:analise-por-decurso-de-prazo}}\begin{itemize}
\item {} 
\sphinxAtStartPar
A aba de \sphinxstylestrong{citações e intimações} possui uma coluna chamada \sphinxstylestrong{“Decurso do Prazo”}

\item {} 
\sphinxAtStartPar
Nela, você encontra os atos cujo prazo \sphinxstylestrong{já transcorreu}

\item {} 
\sphinxAtStartPar
Clique sobre a pendência e acesse a tela da intimação

\end{itemize}


\subsubsection{Tela da Intimação}
\label{\detokenize{projud_31_contagemprazo:tela-da-intimacao}}
\sphinxAtStartPar
Na tela de intimação, o sistema exibe:
\begin{itemize}
\item {} 
\sphinxAtStartPar
\sphinxstylestrong{Data da postagem}

\item {} 
\sphinxAtStartPar
\sphinxstylestrong{Data da leitura} (quando realizada)

\item {} 
\sphinxAtStartPar
\sphinxstylestrong{Data do decurso do prazo}

\item {} 
\sphinxAtStartPar
\sphinxstylestrong{Status do prazo} (aguardando análise, em curso, decorrido)

\end{itemize}

\sphinxAtStartPar
\textgreater{} ✅ Clique em \sphinxstylestrong{“Data do decurso”} para ver mais detalhes


\subsubsection{Detalhamento do Cálculo do Prazo}
\label{\detokenize{projud_31_contagemprazo:detalhamento-do-calculo-do-prazo}}
\sphinxAtStartPar
O botão \sphinxstylestrong{“Detalhamento do Cálculo do Prazo”} exibe:
\begin{itemize}
\item {} 
\sphinxAtStartPar
Data da \sphinxstylestrong{leitura eletrônica}

\item {} 
\sphinxAtStartPar
\sphinxstylestrong{Início do prazo}

\item {} 
\sphinxAtStartPar
\sphinxstylestrong{Dias úteis contabilizados}

\item {} 
\sphinxAtStartPar
\sphinxstylestrong{Feriados e fins de semana desconsiderados}

\item {} 
\sphinxAtStartPar
\sphinxstylestrong{Data de término do prazo}

\item {} 
\sphinxAtStartPar
Situação atual (ex: \sphinxstyleemphasis{Prazo Decorrido}, \sphinxstyleemphasis{Aguardando Leitura}, etc.)

\end{itemize}

\sphinxAtStartPar
Exemplo de visualização:

\begin{sphinxVerbatim}[commandchars=\\\{\}]
\PYG{n}{Leitura}\PYG{p}{:} \PYG{l+m+mi}{10}\PYG{o}{/}\PYG{l+m+mi}{09}
\PYG{n}{Início} \PYG{n}{do} \PYG{n}{Prazo}\PYG{p}{:} \PYG{l+m+mi}{11}\PYG{o}{/}\PYG{l+m+mi}{09}
\PYG{n}{Dias}\PYG{p}{:} \PYG{l+m+mi}{5} \PYG{n}{úteis}
\PYG{n}{Feriado}\PYG{p}{:} \PYG{l+m+mi}{13}\PYG{o}{/}\PYG{l+m+mi}{09} \PYG{p}{(}\PYG{n}{excluído}\PYG{p}{)}
\PYG{n}{Término}\PYG{p}{:} \PYG{l+m+mi}{18}\PYG{o}{/}\PYG{l+m+mi}{09}
\PYG{n}{Status}\PYG{p}{:} \PYG{n}{Prazo} \PYG{n}{Decorrido}
\end{sphinxVerbatim}


\subsubsection{Ações Possíveis}
\label{\detokenize{projud_31_contagemprazo:acoes-possiveis}}
\sphinxAtStartPar
Ao visualizar o decurso do prazo, você pode:
\begin{itemize}
\item {} \begin{description}
\sphinxlineitem{\sphinxstylestrong{Analisar o Decurso}:}\begin{itemize}
\item {} 
\sphinxAtStartPar
Inserir \sphinxstylestrong{certidão de decurso}

\item {} 
\sphinxAtStartPar
Inserir \sphinxstylestrong{ato ordinatório}

\item {} 
\sphinxAtStartPar
Movimentar diretamente (ex: \sphinxstyleemphasis{concluir}, \sphinxstyleemphasis{intimar}, \sphinxstyleemphasis{suspender})

\end{itemize}

\end{description}

\item {} 
\sphinxAtStartPar
\sphinxstylestrong{Dispensar} a pendência (caso não necessite ação)

\item {} 
\sphinxAtStartPar
\sphinxstylestrong{Acompanhar prazos futuros} para novas intimações

\end{itemize}


\subsubsection{Intimações Recentes}
\label{\detokenize{projud_31_contagemprazo:intimacoes-recentes}}
\sphinxAtStartPar
Ao clicar sobre uma \sphinxstylestrong{intimação recente}, o sistema mostrará apenas a \sphinxstylestrong{data da postagem}. A \sphinxstylestrong{leitura e contagem do prazo} ainda não estarão disponíveis.

\sphinxAtStartPar
\textgreater{} ⚠️ Isso ocorre porque o sistema respeita o \sphinxstylestrong{prazo de até 10 dias} para que o destinatário leia a intimação (art. 5º da Lei 11.419/2006)


\subsubsection{Resumo}
\label{\detokenize{projud_31_contagemprazo:resumo}}
\sphinxAtStartPar
A contagem automática de prazos no ProJUDI oferece:

\sphinxAtStartPar
✅ Transparência nos atos processuais
✅ Precisão nos prazos, com exclusão de feriados e fins de semana
✅ Facilidade para análise e movimentação a partir das pendências

\sphinxAtStartPar
\textgreater{} 📌 Utilize sempre a aba \sphinxstylestrong{“Detalhamento do Cálculo do Prazo”} para uma visualização segura e juridicamente respaldada dos prazos em curso.

\sphinxstepscope


\subsection{Expedição de Mandado para Oficial de Justiça}
\label{\detokenize{projud_32_expedicaomandado:expedicao-de-mandado-para-oficial-de-justica}}\label{\detokenize{projud_32_expedicaomandado::doc}}
\sphinxAtStartPar
Nesta aula, aprenderemos como realizar a \sphinxstylestrong{expedição de mandado} de citação, intimação ou outro tipo de diligência para cumprimento por \sphinxstylestrong{oficial de justiça}, utilizando o sistema \sphinxstylestrong{Projudi}.


\subsubsection{Acesso à Expedição de Mandado}
\label{\detokenize{projud_32_expedicaomandado:acesso-a-expedicao-de-mandado}}\begin{enumerate}
\sphinxsetlistlabels{\arabic}{enumi}{enumii}{}{.}%
\item {} 
\sphinxAtStartPar
Acesse o processo desejado

\item {} 
\sphinxAtStartPar
Vá para a aba \sphinxstylestrong{“Movimentações”}

\item {} 
\sphinxAtStartPar
Clique sobre a \sphinxstylestrong{decisão ou ato que determinou o cumprimento}

\item {} 
\sphinxAtStartPar
Clique em \sphinxstylestrong{“Movimentar a partir desta”}

\item {} 
\sphinxAtStartPar
No menu lateral, selecione a opção \sphinxstylestrong{“Ordenar Cumprimentos”}

\end{enumerate}


\subsubsection{Tipos de Cumprimentos Disponíveis}
\label{\detokenize{projud_32_expedicaomandado:tipos-de-cumprimentos-disponiveis}}
\sphinxAtStartPar
Ao clicar em “Ordenar Cumprimentos”, o sistema apresenta uma lista com os tipos de documentos que podem ser expedidos:
\begin{itemize}
\item {} 
\sphinxAtStartPar
Alvará

\item {} 
\sphinxAtStartPar
Carta de adjudicação

\item {} 
\sphinxAtStartPar
Carta precatória

\item {} 
\sphinxAtStartPar
Edital de citação

\item {} 
\sphinxAtStartPar
\sphinxstylestrong{Mandado}

\item {} 
\sphinxAtStartPar
Nota de foro

\item {} 
\sphinxAtStartPar
Ofício

\item {} 
\sphinxAtStartPar
Termo

\end{itemize}


\subsubsection{Configuração do Mandado}
\label{\detokenize{projud_32_expedicaomandado:configuracao-do-mandado}}
\sphinxAtStartPar
Ao escolher o tipo \sphinxstylestrong{Mandado}, siga os passos:
\begin{enumerate}
\sphinxsetlistlabels{\arabic}{enumi}{enumii}{}{.}%
\item {} 
\sphinxAtStartPar
\sphinxstylestrong{Natureza do Mandado}: ex: \sphinxstyleemphasis{Citação, Intimação de Penhora, Depoimento Pessoal, etc.}

\item {} 
\sphinxAtStartPar
\sphinxstylestrong{Assinatura do Mandado}:
\sphinxhyphen{} Sim: será assinado por magistrado
\sphinxhyphen{} Não: será assinado pela secretaria

\item {} 
\sphinxAtStartPar
\sphinxstylestrong{Seleção das partes} que receberão o cumprimento

\item {} 
\sphinxAtStartPar
\sphinxstylestrong{Urgência}: selecione o tipo, se aplicável

\item {} 
\sphinxAtStartPar
\sphinxstylestrong{Prazos}:
\sphinxhyphen{} Prazo da parte (ex: 15 dias)
\sphinxhyphen{} Prazo para o oficial de justiça cumprir

\end{enumerate}


\paragraph{Distribuição}
\label{\detokenize{projud_32_expedicaomandado:distribuicao}}\begin{itemize}
\item {} 
\sphinxAtStartPar
Se \sphinxstylestrong{nenhum oficial} for selecionado, o mandado será encaminhado automaticamente à \sphinxstylestrong{Central de Mandados}

\item {} 
\sphinxAtStartPar
É o padrão na capital e em outras comarcas de maior porte

\end{itemize}


\subsubsection{Expedição do Mandado}
\label{\detokenize{projud_32_expedicaomandado:expedicao-do-mandado}}
\sphinxAtStartPar
Após ordenar o cumprimento, será criada uma pendência em \sphinxstylestrong{“Cumprimentos para expedir”}.
\begin{enumerate}
\sphinxsetlistlabels{\arabic}{enumi}{enumii}{}{.}%
\item {} 
\sphinxAtStartPar
Clique sobre o número da pendência

\item {} 
\sphinxAtStartPar
Na tela que se abrirá, clique em \sphinxstylestrong{“Ordenação”}

\item {} 
\sphinxAtStartPar
O sistema mostrará a \sphinxstylestrong{tela do mandado}, com:
\sphinxhyphen{} Endereço
\sphinxhyphen{} Detalhamento do prazo
\sphinxhyphen{} Dados da parte destinatária
\sphinxhyphen{} Seção para inserir o \sphinxstylestrong{modelo do mandado}

\end{enumerate}


\paragraph{Inserção do Documento}
\label{\detokenize{projud_32_expedicaomandado:insercao-do-documento}}
\sphinxAtStartPar
Você pode:
\begin{itemize}
\item {} 
\sphinxAtStartPar
\sphinxstylestrong{Anexar arquivo PDF}

\item {} 
\sphinxAtStartPar
\sphinxstylestrong{Digitar texto} a partir de modelo

\end{itemize}

\sphinxAtStartPar
Siga os passos:
\begin{enumerate}
\sphinxsetlistlabels{\arabic}{enumi}{enumii}{}{.}%
\item {} 
\sphinxAtStartPar
Clique em \sphinxstylestrong{“Analisar”}

\item {} 
\sphinxAtStartPar
Selecione o modelo: ex: \sphinxstyleemphasis{Mandado de Citação}

\item {} 
\sphinxAtStartPar
Preencha a descrição

\item {} 
\sphinxAtStartPar
Clique em \sphinxstylestrong{“Digitar texto”}

\item {} 
\sphinxAtStartPar
Edite se necessário e clique em \sphinxstylestrong{“Salvar e Concluir”}

\end{enumerate}


\subsubsection{Inclusão de Documentos Anexos}
\label{\detokenize{projud_32_expedicaomandado:inclusao-de-documentos-anexos}}
\sphinxAtStartPar
É \sphinxstylestrong{obrigatório} anexar documentos essenciais para o cumprimento:
\begin{itemize}
\item {} 
\sphinxAtStartPar
Petição Inicial

\item {} 
\sphinxAtStartPar
Denúncia

\item {} 
\sphinxAtStartPar
TCO (Termo Circunstanciado de Ocorrência)

\item {} 
\sphinxAtStartPar
Despachos

\end{itemize}

\sphinxAtStartPar
\textgreater{} ⚠️ A ausência destes documentos pode fazer com que o mandado seja devolvido sem cumprimento pela Central de Mandados.


\subsubsection{Assinatura e Expedição}
\label{\detokenize{projud_32_expedicaomandado:assinatura-e-expedicao}}\begin{enumerate}
\sphinxsetlistlabels{\arabic}{enumi}{enumii}{}{.}%
\item {} 
\sphinxAtStartPar
Clique em \sphinxstylestrong{“Assinar”}

\item {} 
\sphinxAtStartPar
Insira sua \sphinxstylestrong{senha do certificado digital}

\item {} 
\sphinxAtStartPar
Clique em \sphinxstylestrong{“Assinar e Expedir”}

\end{enumerate}


\subsubsection{Confirmação}
\label{\detokenize{projud_32_expedicaomandado:confirmacao}}\begin{itemize}
\item {} 
\sphinxAtStartPar
O mandado será registrado com sucesso

\item {} 
\sphinxAtStartPar
Ele constará como \sphinxstylestrong{“Expedido e não lido”}, aguardando a distribuição pela Central de Mandados

\end{itemize}


\subsubsection{Resumo}
\label{\detokenize{projud_32_expedicaomandado:resumo}}
\sphinxAtStartPar
✅ Expedição de mandados via sistema Projudi
✅ Envio automático à Central de Mandados
✅ Opção de digitação ou inserção de modelo
✅ Inclusão de documentos obrigatórios
✅ Assinatura e controle de leitura do mandado

\sphinxstepscope


\subsection{Ordenar Cumprimentos}
\label{\detokenize{projud_33_ordenarcumprimento:ordenar-cumprimentos}}\label{\detokenize{projud_33_ordenarcumprimento::doc}}
\sphinxAtStartPar
Nesta aula, aprofundaremos o uso da funcionalidade \sphinxstylestrong{Ordenar Cumprimentos} no sistema \sphinxstylestrong{Projudi}, especialmente com foco em varas criminais, ampliando os exemplos vistos anteriormente.


\subsubsection{O que é Ordenar Cumprimentos?}
\label{\detokenize{projud_33_ordenarcumprimento:o-que-e-ordenar-cumprimentos}}
\sphinxAtStartPar
A função \sphinxstylestrong{“Ordenar Cumprimentos”} permite ao usuário expedir diversos tipos de documentos com base em determinações processuais. Ela está localizada na aba \sphinxstylestrong{Movimentações}, acessível ao clicar sobre a decisão ou ato que fundamenta o cumprimento.


\subsubsection{Tipos de Documentos Disponíveis}
\label{\detokenize{projud_33_ordenarcumprimento:tipos-de-documentos-disponiveis}}
\sphinxAtStartPar
Na vara \sphinxstylestrong{criminal}, a lista de documentos disponíveis se amplia, incluindo:
\begin{itemize}
\item {} 
\sphinxAtStartPar
Alvará

\item {} 
\sphinxAtStartPar
Carta de arrematação

\item {} 
\sphinxAtStartPar
Carta precatória

\item {} 
\sphinxAtStartPar
Edital de citação

\item {} 
\sphinxAtStartPar
Mandado de prisão

\item {} 
\sphinxAtStartPar
Guias de:
\sphinxhyphen{} Internação
\sphinxhyphen{} Execução
\sphinxhyphen{} Recolhimento

\end{itemize}

\sphinxAtStartPar
\textgreater{} 🔜 A carta precatória eletrônica será abordada em uma aula específica.


\subsubsection{Exemplo: Expedição de Edital de Citação}
\label{\detokenize{projud_33_ordenarcumprimento:exemplo-expedicao-de-edital-de-citacao}}\begin{enumerate}
\sphinxsetlistlabels{\arabic}{enumi}{enumii}{}{.}%
\item {} 
\sphinxAtStartPar
Acesse o processo desejado

\item {} 
\sphinxAtStartPar
Vá até a aba \sphinxstylestrong{“Movimentações”}

\item {} 
\sphinxAtStartPar
Clique na movimentação\sphinxhyphen{}base (ex: decisão)

\item {} 
\sphinxAtStartPar
Clique em \sphinxstylestrong{“Movimentar a partir desta”}

\item {} 
\sphinxAtStartPar
No menu lateral, clique em \sphinxstylestrong{“Ordenar Cumprimentos”}

\item {} 
\sphinxAtStartPar
No campo \sphinxstylestrong{Tipo de documento}, selecione \sphinxstylestrong{Edital de Citação}

\item {} 
\sphinxAtStartPar
Indique:
\sphinxhyphen{} Se será assinado pelo magistrado: \sphinxstylestrong{Sim}
\sphinxhyphen{} Nome do magistrado
\sphinxhyphen{} Parte destinatária
\sphinxhyphen{} Urgência (Sim/Não)
\sphinxhyphen{} Se necessita retorno
\sphinxhyphen{} Prazo

\end{enumerate}

\sphinxAtStartPar
Clique em \sphinxstylestrong{“Ordenar”} para gerar a pendência.


\paragraph{Inserção do Documento}
\label{\detokenize{projud_33_ordenarcumprimento:insercao-do-documento}}\begin{enumerate}
\sphinxsetlistlabels{\arabic}{enumi}{enumii}{}{.}%
\item {} 
\sphinxAtStartPar
Clique sobre a pendência gerada (ex: edital de citação)

\item {} 
\sphinxAtStartPar
Clique em \sphinxstylestrong{“Visualizar”}

\item {} 
\sphinxAtStartPar
Clique em \sphinxstylestrong{“Analisar”}

\item {} 
\sphinxAtStartPar
Na tela de pré\sphinxhyphen{}análise, escolha:
\sphinxhyphen{} \sphinxstylestrong{Digitar texto} (usando modelo previamente salvo)
\sphinxhyphen{} Ou \sphinxstylestrong{Anexar arquivo PDF}

\item {} 
\sphinxAtStartPar
Preencha os campos necessários

\item {} 
\sphinxAtStartPar
Clique em \sphinxstylestrong{“Continuar”}, depois \sphinxstylestrong{“Salvar e Concluir”}

\end{enumerate}


\subsubsection{Encaminhamento para Assinatura}
\label{\detokenize{projud_33_ordenarcumprimento:encaminhamento-para-assinatura}}
\sphinxAtStartPar
Diferente dos documentos assinados pela secretaria, \sphinxstylestrong{documentos que exigem assinatura do magistrado}, como editais, devem ser \sphinxstylestrong{encaminhados ao juiz}.
\begin{enumerate}
\sphinxsetlistlabels{\arabic}{enumi}{enumii}{}{.}%
\item {} 
\sphinxAtStartPar
Após salvar e concluir a pré\sphinxhyphen{}análise, clique em \sphinxstylestrong{“Encaminhar ao juiz”}

\item {} 
\sphinxAtStartPar
Confirme o encaminhamento

\item {} 
\sphinxAtStartPar
O documento será enviado para a \sphinxstylestrong{fila do magistrado}, que poderá assiná\sphinxhyphen{}lo digitalmente

\end{enumerate}


\subsubsection{Resumo}
\label{\detokenize{projud_33_ordenarcumprimento:resumo}}
\sphinxAtStartPar
✅ Ordenar Cumprimentos é um recurso versátil do Projudi
✅ Permite expedir documentos como mandados, alvarás, cartas e guias
✅ Em varas criminais, há documentos específicos como mandado de prisão e guia de recolhimento
✅ Documentos como \sphinxstylestrong{editais} devem ser \sphinxstylestrong{encaminhados para assinatura judicial}

\sphinxstepscope


\subsection{Intimar Perito e Oficial de Justiça}
\label{\detokenize{projud_34_intimarperitooj:intimar-perito-e-oficial-de-justica}}\label{\detokenize{projud_34_intimarperitooj::doc}}
\sphinxAtStartPar
Nesta aula, vamos aprender como realizar a \sphinxstylestrong{intimação de peritos} e \sphinxstylestrong{oficiais de justiça} no sistema Projudi.


\subsubsection{Pré\sphinxhyphen{}requisitos}
\label{\detokenize{projud_34_intimarperitooj:pre-requisitos}}
\sphinxAtStartPar
Antes de realizar a intimação de um perito, é necessário que ele esteja \sphinxstylestrong{previamente cadastrado no processo}. Este procedimento foi abordado na aula sobre \sphinxstylestrong{Cadastro de Perito}.


\subsubsection{Passo a Passo: Intimar Perito}
\label{\detokenize{projud_34_intimarperitooj:passo-a-passo-intimar-perito}}\begin{enumerate}
\sphinxsetlistlabels{\arabic}{enumi}{enumii}{}{.}%
\item {} 
\sphinxAtStartPar
Acesse o processo desejado

\item {} 
\sphinxAtStartPar
Vá até a aba \sphinxstylestrong{Movimentações}

\item {} 
\sphinxAtStartPar
Clique sobre o \sphinxstylestrong{ato que determinou a intimação} (ato ordinatório, despacho, decisão etc.)

\item {} 
\sphinxAtStartPar
Clique em \sphinxstylestrong{Movimentar a partir desta movimentação}

\item {} 
\sphinxAtStartPar
No menu lateral esquerdo, clique em \sphinxstylestrong{Intimar peritos e auxiliares da justiça}

\item {} 
\sphinxAtStartPar
O sistema irá listar os peritos já cadastrados no processo

\item {} 
\sphinxAtStartPar
Preencha:
\sphinxhyphen{} \sphinxstylestrong{Número de dias} (corridos ou úteis)
\sphinxhyphen{} \sphinxstylestrong{Urgência} (Sim ou Não)

\item {} 
\sphinxAtStartPar
Clique em \sphinxstylestrong{Intimar}

\end{enumerate}

\sphinxAtStartPar
\textgreater{} 📌 Uma nova movimentação será gerada no processo, registrando a intimação do perito.


\subsubsection{Passo a Passo: Intimar Oficial de Justiça}
\label{\detokenize{projud_34_intimarperitooj:passo-a-passo-intimar-oficial-de-justica}}
\sphinxAtStartPar
Este procedimento é utilizado, por exemplo, \sphinxstylestrong{quando o cumprimento de um mandado está demorando} e é necessário solicitar o retorno do oficial de justiça.
\begin{enumerate}
\sphinxsetlistlabels{\arabic}{enumi}{enumii}{}{.}%
\item {} 
\sphinxAtStartPar
Acesse o mesmo processo

\item {} 
\sphinxAtStartPar
Vá até a aba \sphinxstylestrong{Movimentações}

\item {} 
\sphinxAtStartPar
Clique na movimentação de referência

\item {} 
\sphinxAtStartPar
Clique em \sphinxstylestrong{Movimentar a partir desta}

\item {} 
\sphinxAtStartPar
No menu lateral, clique em \sphinxstylestrong{Oficial de Justiça}

\item {} 
\sphinxAtStartPar
O sistema exibirá o nome do oficial de justiça \sphinxstylestrong{vinculado ao processo}

\item {} 
\sphinxAtStartPar
Preencha o prazo, se necessário

\item {} 
\sphinxAtStartPar
Clique em \sphinxstylestrong{Intimar}

\end{enumerate}

\sphinxAtStartPar
\textgreater{} 📌 A intimação será registrada na aba \sphinxstylestrong{Movimentações}, informando que o oficial de justiça foi intimado a devolver ou manifestar\sphinxhyphen{}se sobre o mandado.


\subsubsection{Resumo}
\label{\detokenize{projud_34_intimarperitooj:resumo}}
\sphinxAtStartPar
✅ Intimações de \sphinxstylestrong{peritos} e \sphinxstylestrong{oficiais de justiça} são realizadas a partir da movimentação base
✅ O perito precisa estar \sphinxstylestrong{previamente habilitado} no processo
✅ O oficial de justiça deve estar \sphinxstylestrong{vinculado ao mandado}
✅ A intimação gera uma \sphinxstylestrong{movimentação automática no processo}

\sphinxstepscope


\subsection{Enviar Concluso}
\label{\detokenize{projud_35_enviarconcluso:enviar-concluso}}\label{\detokenize{projud_35_enviarconcluso::doc}}
\sphinxAtStartPar
Nesta aula, vamos aprender como \sphinxstylestrong{enviar um processo concluso ao magistrado} no sistema Projudi. Este procedimento é utilizado para que o juiz possa proferir uma decisão, despacho ou sentença a partir de uma movimentação específica no processo.


\subsubsection{Passo a Passo: Enviar Concluso}
\label{\detokenize{projud_35_enviarconcluso:passo-a-passo-enviar-concluso}}\begin{enumerate}
\sphinxsetlistlabels{\arabic}{enumi}{enumii}{}{.}%
\item {} 
\sphinxAtStartPar
Acesse o processo desejado na tela inicial do sistema Projudi.

\item {} 
\sphinxAtStartPar
Vá até a aba \sphinxstylestrong{Movimentações}.

\item {} 
\sphinxAtStartPar
Clique sobre a movimentação que direciona a necessidade de conclusão:
\sphinxhyphen{} Pode ser um \sphinxstylestrong{ato ordinatório}
\sphinxhyphen{} Uma \sphinxstylestrong{certidão}
\sphinxhyphen{} Um \sphinxstylestrong{recurso}
\sphinxhyphen{} Ou, por exemplo, uma movimentação de \sphinxstylestrong{decurso de prazo}

\item {} 
\sphinxAtStartPar
Clique em \sphinxstylestrong{Movimentar a partir desta movimentação}.

\item {} 
\sphinxAtStartPar
No menu lateral esquerdo, selecione a ação \sphinxstylestrong{Enviar Concluso}.

\end{enumerate}


\subsubsection{Configurações da Conclusão}
\label{\detokenize{projud_35_enviarconcluso:configuracoes-da-conclusao}}
\sphinxAtStartPar
Na tela de envio, preencha as seguintes informações:
\begin{itemize}
\item {} 
\sphinxAtStartPar
\sphinxstylestrong{Tipo de Conclusão}:
\sphinxhyphen{} Decisão
\sphinxhyphen{} Despacho
\sphinxhyphen{} Sentença

\item {} 
\sphinxAtStartPar
\sphinxstylestrong{Magistrado responsável} pelo julgamento

\item {} 
\sphinxAtStartPar
\sphinxstylestrong{Urgência}:
\sphinxhyphen{} Marque \sphinxstylestrong{Sim} se for matéria urgente
\begin{itemize}
\item {} 
\sphinxAtStartPar
Exemplos: Liberdade provisória, Habeas Corpus, Mandado de Segurança

\end{itemize}
\begin{itemize}
\item {} 
\sphinxAtStartPar
Marque \sphinxstylestrong{Não} para casos comuns

\end{itemize}

\item {} 
\sphinxAtStartPar
\sphinxstylestrong{Agrupador}:
\sphinxhyphen{} Ferramenta importante que organiza e \sphinxstylestrong{categoriza o tipo de conclusão}
\sphinxhyphen{} Permite que \sphinxstylestrong{assessores e magistrados identifiquem rapidamente} a natureza do pedido
\sphinxhyphen{} Exemplos:
\begin{itemize}
\item {} 
\sphinxAtStartPar
Cumprimento de decisão do TRF

\item {} 
\sphinxAtStartPar
Emendar inicial

\item {} 
\sphinxAtStartPar
Pedido de reconsideração

\end{itemize}

\item {} 
\sphinxAtStartPar
\sphinxstylestrong{Assessor responsável}:
\sphinxhyphen{} (Opcional) Defina o servidor que irá realizar a minuta da decisão

\end{itemize}
\begin{enumerate}
\sphinxsetlistlabels{\arabic}{enumi}{enumii}{}{.}%
\setcounter{enumi}{5}
\item {} 
\sphinxAtStartPar
Após preencher as informações, clique em \sphinxstylestrong{Enviar Concluso}.

\end{enumerate}


\subsubsection{Resultado}
\label{\detokenize{projud_35_enviarconcluso:resultado}}
\sphinxAtStartPar
✅ O processo é movimentado com sucesso
✅ A conclusão aparece na \sphinxstylestrong{fila do gabinete do magistrado}, aguardando análise
✅ O \sphinxstylestrong{agrupador} facilita a triagem das conclusões
✅ O juiz poderá despachar, decidir ou sentenciar o processo

\sphinxstepscope


\subsection{Realizar Remessa}
\label{\detokenize{projud_36_realizarremessa:realizar-remessa}}\label{\detokenize{projud_36_realizarremessa::doc}}
\sphinxAtStartPar
Nesta aula, vamos aprender como realizar \sphinxstylestrong{remessas de processos} no sistema Projudi para diferentes órgãos como \sphinxstylestrong{Delegacia, Ministério Público, Defensoria, Distribuidor e Contadoria}.


\subsubsection{Passo a Passo: Como Realizar uma Remessa}
\label{\detokenize{projud_36_realizarremessa:passo-a-passo-como-realizar-uma-remessa}}\begin{enumerate}
\sphinxsetlistlabels{\arabic}{enumi}{enumii}{}{.}%
\item {} 
\sphinxAtStartPar
Na tela inicial do processo, vá até a aba \sphinxstylestrong{Movimentações}.

\item {} 
\sphinxAtStartPar
Localize a movimentação que determinou a remessa (pode ser uma decisão, despacho etc.).

\item {} 
\sphinxAtStartPar
Clique em \sphinxstylestrong{Movimentar a partir desta movimentação}.

\item {} 
\sphinxAtStartPar
No menu lateral esquerdo, clique em \sphinxstylestrong{Realizar Remessa}.

\end{enumerate}


\subsubsection{Tipos de Remessa e Destinatários}
\label{\detokenize{projud_36_realizarremessa:tipos-de-remessa-e-destinatarios}}
\sphinxAtStartPar
Ao acessar a tela de remessa, o sistema permite que você selecione o destinatário:


\paragraph{\sphinxstylestrong{Delegacia}}
\label{\detokenize{projud_36_realizarremessa:delegacia}}\begin{itemize}
\item {} 
\sphinxAtStartPar
Selecione a \sphinxstylestrong{Comarca}.

\item {} 
\sphinxAtStartPar
Escolha a \sphinxstylestrong{Delegacia cadastrada}.

\item {} 
\sphinxAtStartPar
Informe o \sphinxstylestrong{prazo} para cumprimento da remessa.

\end{itemize}


\paragraph{\sphinxstylestrong{Distribuidor}}
\label{\detokenize{projud_36_realizarremessa:distribuidor}}\begin{itemize}
\item {} 
\sphinxAtStartPar
Finalidades possíveis:
\sphinxhyphen{} Registro de distribuição
\sphinxhyphen{} Baixa
\sphinxhyphen{} Assistida de antecedentes
\sphinxhyphen{} Redistribuição
\sphinxhyphen{} Cancelamento

\end{itemize}


\paragraph{\sphinxstylestrong{Ministério Público}}
\label{\detokenize{projud_36_realizarremessa:ministerio-publico}}\begin{itemize}
\item {} 
\sphinxAtStartPar
Selecione a \sphinxstylestrong{Promotoria}.

\item {} 
\sphinxAtStartPar
Informe a \sphinxstylestrong{Finalidade da remessa}:
\sphinxhyphen{} Parecer
\sphinxhyphen{} Ciência
\sphinxhyphen{} Manifestação
\sphinxhyphen{} Remessa física
\sphinxhyphen{} Razões e contra\sphinxhyphen{}razões
\sphinxhyphen{} Promoção
\sphinxhyphen{} Recurso
\sphinxhyphen{} Denúncia

\item {} 
\sphinxAtStartPar
Defina o \sphinxstylestrong{prazo em dias}.

\item {} 
\sphinxAtStartPar
Indique se a remessa é \sphinxstylestrong{urgente}.

\end{itemize}


\paragraph{\sphinxstylestrong{Defensoria Pública}}
\label{\detokenize{projud_36_realizarremessa:defensoria-publica}}\begin{itemize}
\item {} 
\sphinxAtStartPar
Finalidades:
\sphinxhyphen{} Ciência
\sphinxhyphen{} Manifestação
\sphinxhyphen{} Contra\sphinxhyphen{}razões

\item {} 
\sphinxAtStartPar
Informe o \sphinxstylestrong{prazo}.

\end{itemize}


\paragraph{\sphinxstylestrong{Contadoria (Área Cível)}}
\label{\detokenize{projud_36_realizarremessa:contadoria-area-civel}}\begin{itemize}
\item {} 
\sphinxAtStartPar
Exemplo: Terceira Contadoria Judicial

\item {} 
\sphinxAtStartPar
Também segue os mesmos passos.

\end{itemize}


\subsubsection{Exemplo Prático: Remessa ao Ministério Público}
\label{\detokenize{projud_36_realizarremessa:exemplo-pratico-remessa-ao-ministerio-publico}}\begin{enumerate}
\sphinxsetlistlabels{\arabic}{enumi}{enumii}{}{.}%
\item {} 
\sphinxAtStartPar
Selecione \sphinxstylestrong{Ministério Público} como destinatário.

\item {} 
\sphinxAtStartPar
Escolha a \sphinxstylestrong{finalidade}: por exemplo, “Ciência”.

\item {} 
\sphinxAtStartPar
Informe o \sphinxstylestrong{prazo em dias}.

\item {} 
\sphinxAtStartPar
Indique \sphinxstylestrong{urgência}, se aplicável.

\item {} 
\sphinxAtStartPar
Clique em \sphinxstylestrong{Realizar Remessa}.

\end{enumerate}


\paragraph{Resultado}
\label{\detokenize{projud_36_realizarremessa:resultado}}
\sphinxAtStartPar
✅ O processo é remetido ao MP
✅ Ainda é possível realizar outras ações, como:
\begin{itemize}
\item {} 
\sphinxAtStartPar
Enviar Concluso

\item {} 
\sphinxAtStartPar
Fazer nova remessa à Defensoria

\end{itemize}

\sphinxAtStartPar
⚠️ Observação:
\sphinxhyphen{} \sphinxstylestrong{Remessas não bloqueiam} o processo para outras ações.
\sphinxhyphen{} \sphinxstylestrong{Conclusões sim bloqueiam}: não é possível fazer intimações, citações, ou ordenar cumprimento enquanto o processo estiver concluso.

\sphinxstepscope


\subsection{Remessa ao 2º Grau e Turma Recursal}
\label{\detokenize{projud_37_remessasegundograu:remessa-ao-2o-grau-e-turma-recursal}}\label{\detokenize{projud_37_remessasegundograu::doc}}
\sphinxAtStartPar
Nesta aula, veremos como realizar a \sphinxstylestrong{remessa de um processo} para a \sphinxstylestrong{Turma Recursal} (Juizados Especiais) ou para o \sphinxstylestrong{Segundo Grau} (Tribunal).


\subsubsection{Requisitos Antes da Remessa}
\label{\detokenize{projud_37_remessasegundograu:requisitos-antes-da-remessa}}
\sphinxAtStartPar
Antes de realizar a remessa, é necessário verificar:
\begin{description}
\sphinxlineitem{✅ \sphinxstylestrong{Sem pendências}:}\begin{itemize}
\item {} 
\sphinxAtStartPar
O processo não pode conter pendências ativas, como:
\sphinxhyphen{} Citações pendentes
\sphinxhyphen{} Intimações pendentes
\sphinxhyphen{} Conclusões ainda não resolvidas
\sphinxhyphen{} Remessas em andamento (MP, Defensoria, etc.)

\end{itemize}

\sphinxlineitem{✅ \sphinxstylestrong{Cadastro das partes}:}\begin{itemize}
\item {} 
\sphinxAtStartPar
Todas as partes devem estar devidamente cadastradas com:
\sphinxhyphen{} \sphinxstylestrong{Nome completo}
\sphinxhyphen{} \sphinxstylestrong{CPF/CNPJ}
\sphinxhyphen{} \sphinxstylestrong{Advogado habilitado}

\end{itemize}

\end{description}

\sphinxAtStartPar
Caso essas condições não sejam atendidas, o sistema \sphinxstylestrong{bloqueará} a remessa.


\subsubsection{Procedimento de Remessa}
\label{\detokenize{projud_37_remessasegundograu:procedimento-de-remessa}}\begin{enumerate}
\sphinxsetlistlabels{\arabic}{enumi}{enumii}{}{.}%
\item {} 
\sphinxAtStartPar
Acesse a aba \sphinxstylestrong{Movimentações} no processo.

\item {} 
\sphinxAtStartPar
Localize a decisão que determinou o envio à instância superior.

\item {} 
\sphinxAtStartPar
Clique em \sphinxstylestrong{Movimentar a partir desta movimentação}.

\item {} 
\sphinxAtStartPar
No menu lateral, clique em \sphinxstylestrong{Autos à Turma Recursal} (para Juizados) ou \sphinxstylestrong{Autos ao 2º Grau} (para varas comuns).

\sphinxAtStartPar
Exemplo:

\begin{sphinxVerbatim}[commandchars=\\\{\}]
\PYG{p}{[} \PYG{n}{Ações} \PYG{p}{]}
\PYG{o}{\PYGZhy{}}\PYG{o}{\PYGZgt{}} \PYG{n}{Autos} \PYG{n}{à} \PYG{n}{Turma} \PYG{n}{Recursal}
\end{sphinxVerbatim}

\item {} 
\sphinxAtStartPar
Na tela seguinte, clique em \sphinxstylestrong{Confirmar}.

\end{enumerate}


\subsubsection{Resultado da Remessa}
\label{\detokenize{projud_37_remessasegundograu:resultado-da-remessa}}
\sphinxAtStartPar
Após a confirmação:

\sphinxAtStartPar
✅ O processo será remetido com sucesso.
✅ Aparecerá no topo da tela a seguinte \sphinxstylestrong{mensagem}:
\begin{quote}

\begin{sphinxVerbatim}[commandchars=\\\{\}]
\PYG{n}{Processo} \PYG{n}{em} \PYG{n}{instância} \PYG{n}{superior}
\end{sphinxVerbatim}
\end{quote}

\sphinxAtStartPar
⚠️ \sphinxstylestrong{Restrições}:
\sphinxhyphen{} O processo \sphinxstylestrong{fica bloqueado para movimentações}.
\sphinxhyphen{} \sphinxstylestrong{Não é possível} inserir documentos, realizar intimações, conclusões ou outras movimentações enquanto ele estiver em trâmite na instância superior.

\sphinxstepscope


\subsection{Interrupção do Prazo}
\label{\detokenize{projud_38_interrupcaoprazo:interrupcao-do-prazo}}\label{\detokenize{projud_38_interrupcaoprazo::doc}}
\sphinxAtStartPar
Nesta aula, veremos como \sphinxstylestrong{interromper a contagem de um prazo} no sistema Projudi.


\subsubsection{Quando utilizar}
\label{\detokenize{projud_38_interrupcaoprazo:quando-utilizar}}
\sphinxAtStartPar
O recurso de \sphinxstylestrong{interrupção de prazo} pode ser utilizado nos casos em que há necessidade de pausar temporariamente o prazo de uma \sphinxstylestrong{intimação} ou \sphinxstylestrong{citação}.

\sphinxAtStartPar
⚠️ Atualmente, o sistema contempla \sphinxstylestrong{apenas prazos de citações e intimações} para esse tipo de operação.


\subsubsection{Passo a passo para interromper um prazo}
\label{\detokenize{projud_38_interrupcaoprazo:passo-a-passo-para-interromper-um-prazo}}\begin{enumerate}
\sphinxsetlistlabels{\arabic}{enumi}{enumii}{}{.}%
\item {} 
\sphinxAtStartPar
Acesse a aba \sphinxstylestrong{Movimentações} dentro do processo.

\item {} 
\sphinxAtStartPar
Localize a decisão ou movimentação que gerou o prazo a ser interrompido.

\item {} 
\sphinxAtStartPar
Clique em \sphinxstylestrong{Movimentar a partir desta movimentação}.

\item {} 
\sphinxAtStartPar
No menu lateral, clique em \sphinxstylestrong{Outras Ações} e depois em \sphinxstylestrong{Interromper o Prazo}.

\begin{sphinxVerbatim}[commandchars=\\\{\}]
\PYG{p}{[} \PYG{n}{Ações} \PYG{p}{]}
\PYG{o}{\PYGZhy{}}\PYG{o}{\PYGZgt{}} \PYG{n}{Outras} \PYG{n}{Ações}
   \PYG{o}{\PYGZhy{}}\PYG{o}{\PYGZgt{}} \PYG{n}{Interromper} \PYG{n}{o} \PYG{n}{Prazo}
\end{sphinxVerbatim}

\item {} 
\sphinxAtStartPar
Na tela \sphinxstylestrong{Interromper o Prazo}:
\sphinxhyphen{} O sistema irá listar automaticamente os prazos ativos elegíveis para interrupção (citações/intimações).
\sphinxhyphen{} Selecione o prazo desejado.
\sphinxhyphen{} Informe a \sphinxstylestrong{data de interrupção}.
\sphinxhyphen{} Clique em \sphinxstylestrong{Interromper o Prazo}.

\end{enumerate}


\subsubsection{Resultado da Interrupção}
\label{\detokenize{projud_38_interrupcaoprazo:resultado-da-interrupcao}}
\sphinxAtStartPar
✅ A movimentação será registrada no processo.

\sphinxAtStartPar
✅ Ao voltar à aba \sphinxstylestrong{Movimentações}, aparecerá um novo evento informando:
\begin{itemize}
\item {} 
\sphinxAtStartPar
Qual foi o prazo interrompido.

\item {} 
\sphinxAtStartPar
A quem se referia (autor, réu, etc.).

\item {} 
\sphinxAtStartPar
Desde quando ele foi interrompido.

\item {} 
\sphinxAtStartPar
Qual foi o movimento gerador do prazo (ex: citação, intimação).

\end{itemize}


\subsubsection{O que acontece após a interrupção?}
\label{\detokenize{projud_38_interrupcaoprazo:o-que-acontece-apos-a-interrupcao}}
\sphinxAtStartPar
🕒 O \sphinxstylestrong{sistema Projudi recalcula o prazo} de forma automática, preservando a parte restante da contagem original.

\sphinxstepscope


\subsection{Suspensão e Sobrestamento do Processo}
\label{\detokenize{projud_39_suspensaosobrestamento:suspensao-e-sobrestamento-do-processo}}\label{\detokenize{projud_39_suspensaosobrestamento::doc}}
\sphinxAtStartPar
Nesta aula, você aprenderá como realizar e finalizar a \sphinxstylestrong{suspensão} ou o \sphinxstylestrong{sobrestamento} de um processo no sistema Projudi.


\subsubsection{Quando utilizar}
\label{\detokenize{projud_39_suspensaosobrestamento:quando-utilizar}}
\sphinxAtStartPar
A suspensão ou o sobrestamento pode ser utilizado quando há necessidade de \sphinxstylestrong{interromper a tramitação do processo temporariamente}, por exemplo, em razão de:
\begin{itemize}
\item {} 
\sphinxAtStartPar
Determinação judicial

\item {} 
\sphinxAtStartPar
IRDRs (Incidentes de Resolução de Demandas Repetitivas)

\item {} 
\sphinxAtStartPar
Processos aguardando julgamento de matéria em instância superior

\item {} 
\sphinxAtStartPar
Entre outros motivos legais ou estratégicos

\end{itemize}


\subsubsection{Como suspender ou sobrestar um processo}
\label{\detokenize{projud_39_suspensaosobrestamento:como-suspender-ou-sobrestar-um-processo}}\begin{enumerate}
\sphinxsetlistlabels{\arabic}{enumi}{enumii}{}{.}%
\item {} 
\sphinxAtStartPar
Na \sphinxstylestrong{tela inicial do processo}, acesse a aba \sphinxstylestrong{Movimentações}.

\item {} 
\sphinxAtStartPar
Localize a decisão que determinou a suspensão ou sobrestamento.

\item {} 
\sphinxAtStartPar
Clique em \sphinxstylestrong{Movimentar a partir desta movimentação}.

\item {} 
\sphinxAtStartPar
No menu lateral esquerdo, selecione \sphinxstylestrong{Outras Ações} \(\rightarrow\) \sphinxstylestrong{Suspender/Sobrestar Processo}.

\begin{sphinxVerbatim}[commandchars=\\\{\}]
\PYG{p}{[} \PYG{n}{Ações} \PYG{p}{]}
\PYG{o}{\PYGZhy{}}\PYG{o}{\PYGZgt{}} \PYG{n}{Outras} \PYG{n}{Ações}
   \PYG{o}{\PYGZhy{}}\PYG{o}{\PYGZgt{}} \PYG{n}{Suspender} \PYG{n}{ou} \PYG{n}{Sobrestar} \PYG{n}{Processo}
\end{sphinxVerbatim}

\item {} 
\sphinxAtStartPar
Na tela de suspensão:
\sphinxhyphen{} Informe a \sphinxstylestrong{data de início da suspensão}.
\sphinxhyphen{} Escolha o tipo de prazo:
\begin{itemize}
\item {} 
\sphinxAtStartPar
\sphinxstylestrong{Prazo fixo} (com data final determinada)

\item {} 
\sphinxAtStartPar
\sphinxstylestrong{Prazo por número de dias}

\item {} 
\sphinxAtStartPar
\sphinxstylestrong{Prazo indeterminado}

\end{itemize}

\item {} 
\sphinxAtStartPar
Clique em \sphinxstylestrong{Suspender/Sobrestar}.

\end{enumerate}

\sphinxAtStartPar
✅ O \sphinxstylestrong{status do processo será atualizado} para “\sphinxstylestrong{Suspenso ou Sobrestado}”.


\subsubsection{Como finalizar a suspensão ou sobrestamento}
\label{\detokenize{projud_39_suspensaosobrestamento:como-finalizar-a-suspensao-ou-sobrestamento}}
\sphinxAtStartPar
Quando a causa da suspensão cessar, será necessário \sphinxstylestrong{encerrar o status suspenso} para poder continuar movimentando o processo.
\begin{enumerate}
\sphinxsetlistlabels{\arabic}{enumi}{enumii}{}{.}%
\item {} 
\sphinxAtStartPar
Acesse a aba \sphinxstylestrong{Movimentações} do processo.

\item {} 
\sphinxAtStartPar
Localize a movimentação que registrou a suspensão.

\item {} 
\sphinxAtStartPar
Clique em \sphinxstylestrong{Movimentar a partir desta movimentação}.

\item {} 
\sphinxAtStartPar
No menu lateral esquerdo, selecione \sphinxstylestrong{Finalizar Suspensão/Sobrestamento}.

\begin{sphinxVerbatim}[commandchars=\\\{\}]
\PYG{p}{[} \PYG{n}{Ações} \PYG{p}{]}
\PYG{o}{\PYGZhy{}}\PYG{o}{\PYGZgt{}} \PYG{n}{Finalizar} \PYG{n}{Suspensão}\PYG{o}{/}\PYG{n}{Sobrestamento}
\end{sphinxVerbatim}

\item {} 
\sphinxAtStartPar
Clique em \sphinxstylestrong{Finalizar}.

\end{enumerate}

\sphinxAtStartPar
✅ A movimentação será registrada e o \sphinxstylestrong{processo voltará ao seu estado normal} de tramitação.

\sphinxstepscope


\subsection{Transitar em Julgado}
\label{\detokenize{projud_40_transitaremjulgado:transitar-em-julgado}}\label{\detokenize{projud_40_transitaremjulgado::doc}}
\sphinxAtStartPar
Nesta aula, você aprenderá como realizar o \sphinxstylestrong{registro de trânsito em julgado}, tanto \sphinxstylestrong{por parte} quanto \sphinxstylestrong{para o processo inteiro}, no sistema Projudi.


\subsubsection{Quando utilizar}
\label{\detokenize{projud_40_transitaremjulgado:quando-utilizar}}
\sphinxAtStartPar
O trânsito em julgado é registrado quando não há mais possibilidade de \sphinxstylestrong{recurso} contra uma decisão judicial, seja ela \sphinxstylestrong{sentença} ou \sphinxstylestrong{acórdão}.


\subsubsection{Como registrar trânsito em julgado por parte}
\label{\detokenize{projud_40_transitaremjulgado:como-registrar-transito-em-julgado-por-parte}}\begin{enumerate}
\sphinxsetlistlabels{\arabic}{enumi}{enumii}{}{.}%
\item {} 
\sphinxAtStartPar
Na tela inicial do processo, clique em \sphinxstylestrong{Movimentações}.

\item {} 
\sphinxAtStartPar
Localize a movimentação de referência (ex: sentença, acórdão, certidão de decurso).

\item {} 
\sphinxAtStartPar
Clique em \sphinxstylestrong{Movimentar a partir desta movimentação}.

\item {} 
\sphinxAtStartPar
No menu lateral esquerdo, clique em \sphinxstylestrong{Transitar em Julgado}.

\item {} 
\sphinxAtStartPar
Na tela aberta, marque a parte (ex: \sphinxstylestrong{Ministério Público}, \sphinxstylestrong{Réu}, \sphinxstylestrong{Defensor}) a quem se refere o trânsito.

\item {} 
\sphinxAtStartPar
Informe a \sphinxstylestrong{data do trânsito em julgado}.

\item {} 
\sphinxAtStartPar
Clique em \sphinxstylestrong{Salvar}.

\end{enumerate}

\sphinxAtStartPar
✅ A movimentação de trânsito em julgado será registrada para a parte selecionada.

\sphinxAtStartPar
Exemplo de uso:
\begin{itemize}
\item {} 
\sphinxAtStartPar
Trânsito em julgado apenas para o Ministério Público.

\item {} 
\sphinxAtStartPar
Posteriormente, trânsito em julgado para o acusado e seu defensor.

\end{itemize}


\subsubsection{Como registrar trânsito em julgado para o processo inteiro}
\label{\detokenize{projud_40_transitaremjulgado:como-registrar-transito-em-julgado-para-o-processo-inteiro}}\begin{enumerate}
\sphinxsetlistlabels{\arabic}{enumi}{enumii}{}{.}%
\item {} 
\sphinxAtStartPar
Acesse um processo (cível ou criminal).

\item {} 
\sphinxAtStartPar
Vá até a aba \sphinxstylestrong{Movimentações}.

\item {} 
\sphinxAtStartPar
Localize e clique na movimentação de referência.

\item {} 
\sphinxAtStartPar
Clique em \sphinxstylestrong{Movimentar a partir desta movimentação}.

\item {} 
\sphinxAtStartPar
Selecione a opção \sphinxstylestrong{Transitar em Julgado}.

\item {} 
\sphinxAtStartPar
Marque a opção \sphinxstylestrong{Informar trânsito para o processo (todas as partes)}.

\item {} 
\sphinxAtStartPar
Informe a \sphinxstylestrong{data do trânsito em julgado}.

\item {} 
\sphinxAtStartPar
Clique em \sphinxstylestrong{Salvar}.

\end{enumerate}

\sphinxAtStartPar
✅ O sistema registrará o trânsito em julgado para \sphinxstylestrong{todas as partes do processo}.


\subsubsection{Visualizando a movimentação}
\label{\detokenize{projud_40_transitaremjulgado:visualizando-a-movimentacao}}
\sphinxAtStartPar
Você pode visualizar o registro do trânsito em julgado de duas formas:
\begin{itemize}
\item {} 
\sphinxAtStartPar
\sphinxstylestrong{Na aba Movimentações}: aparecerá a movimentação “Trânsito em julgado da parte” ou “Trânsito em julgado do processo”.

\item {} 
\sphinxAtStartPar
\sphinxstylestrong{Na opção Navegar} \(\rightarrow\) clique em \sphinxstylestrong{Detalhes da movimentação} para ver:
\sphinxhyphen{} Data
\sphinxhyphen{} Parte (se aplicável)
\sphinxhyphen{} Usuário responsável pela movimentação

\end{itemize}


\subsubsection{Observações}
\label{\detokenize{projud_40_transitaremjulgado:observacoes}}\begin{itemize}
\item {} 
\sphinxAtStartPar
É possível que a gestão da secretaria exija também a juntada de uma \sphinxstylestrong{certidão} confirmando o trânsito.

\item {} 
\sphinxAtStartPar
O uso correto da funcionalidade garante que o sistema Projudi reconheça o fim da fase recursal e bloqueie novos recursos.

\end{itemize}

\sphinxstepscope


\subsection{Arquivamento dos Autos}
\label{\detokenize{projud_41_arquivamento:arquivamento-dos-autos}}\label{\detokenize{projud_41_arquivamento::doc}}
\sphinxAtStartPar
Nesta aula, vamos aprender a \sphinxstylestrong{arquivar um processo} no sistema Projudi, tanto de forma \sphinxstylestrong{definitiva} quanto \sphinxstylestrong{provisória}.


\subsubsection{Quando arquivar}
\label{\detokenize{projud_41_arquivamento:quando-arquivar}}
\sphinxAtStartPar
O arquivamento é realizado após o \sphinxstylestrong{trânsito em julgado} do processo, quando \sphinxstylestrong{não há mais providências} a serem tomadas no momento.


\subsubsection{Como arquivar um processo}
\label{\detokenize{projud_41_arquivamento:como-arquivar-um-processo}}\begin{enumerate}
\sphinxsetlistlabels{\arabic}{enumi}{enumii}{}{.}%
\item {} 
\sphinxAtStartPar
Acesse a \sphinxstylestrong{tela inicial do processo}.

\item {} 
\sphinxAtStartPar
Clique na aba \sphinxstylestrong{Movimentações}.

\item {} 
\sphinxAtStartPar
Localize a movimentação que deu origem ao arquivamento (ex: trânsito em julgado).

\item {} 
\sphinxAtStartPar
Clique em \sphinxstylestrong{Movimentar a partir desta movimentação}.

\item {} 
\sphinxAtStartPar
No menu lateral esquerdo, clique em \sphinxstylestrong{Arquivar o processo}.

\item {} 
\sphinxAtStartPar
Escolha o tipo de arquivamento:
\sphinxhyphen{} \sphinxstylestrong{Definitivo}: quando não há mais necessidade de reativação.
\sphinxhyphen{} \sphinxstylestrong{Provisório}: quando o processo pode ser desarquivado futuramente para novas providências.

\item {} 
\sphinxAtStartPar
Clique em \sphinxstylestrong{Arquivar}.

\end{enumerate}

\sphinxAtStartPar
✅ O sistema confirmará: \sphinxstyleemphasis{“Processo arquivado com sucesso.”}


\subsubsection{O que muda após o arquivamento}
\label{\detokenize{projud_41_arquivamento:o-que-muda-apos-o-arquivamento}}\begin{itemize}
\item {} 
\sphinxAtStartPar
O status do processo passa a ser exibido como \sphinxstylestrong{Arquivado}, em destaque laranja.

\item {} 
\sphinxAtStartPar
Todos os botões de ação na tela do processo aparecem \sphinxstylestrong{riscados} ou \sphinxstylestrong{desabilitados}:
\sphinxhyphen{} Não é mais possível \sphinxstylestrong{intimar}, \sphinxstylestrong{notificar}, \sphinxstylestrong{enviar concluso}, \sphinxstylestrong{ordenar cumprimento}, etc.

\item {} 
\sphinxAtStartPar
A aba \sphinxstylestrong{Movimentações} incluirá a informação: \sphinxstylestrong{Arquivado definitivamente} ou \sphinxstylestrong{Arquivado provisoriamente}.

\end{itemize}


\subsubsection{Como desarquivar um processo}
\label{\detokenize{projud_41_arquivamento:como-desarquivar-um-processo}}
\sphinxAtStartPar
Caso seja necessário movimentar novamente o processo, será preciso realizar o \sphinxstylestrong{desarquivamento}. Essa funcionalidade é abordada em uma aula posterior.


\subsubsection{Observações}
\label{\detokenize{projud_41_arquivamento:observacoes}}\begin{itemize}
\item {} 
\sphinxAtStartPar
O arquivamento definitivo é recomendado apenas quando \sphinxstylestrong{todas as etapas processuais estiverem encerradas}.

\item {} 
\sphinxAtStartPar
O arquivamento provisório é útil em situações que dependem de prazo ou manifestação futura.

\end{itemize}

\sphinxstepscope


\subsection{Aba Superior: Processos}
\label{\detokenize{projud_42_abasuperiorprocesso:aba-superior-processos}}\label{\detokenize{projud_42_abasuperiorprocesso::doc}}
\sphinxAtStartPar
Nesta aula, vamos conhecer as funcionalidades disponíveis na aba \sphinxstylestrong{Processos}, localizada no menu superior do sistema Projudi.


\subsubsection{Funcionalidades da Aba Processos}
\label{\detokenize{projud_42_abasuperiorprocesso:funcionalidades-da-aba-processos}}
\sphinxAtStartPar
Na aba \sphinxstylestrong{Processos}, o usuário encontra diversas opções que facilitam o acompanhamento da tramitação processual:
\begin{description}
\sphinxlineitem{\sphinxstylestrong{1. Favoritos}}\begin{itemize}
\item {} 
\sphinxAtStartPar
Exibe todos os processos marcados como favoritos.

\item {} 
\sphinxAtStartPar
Já demonstrado em aula anterior.

\end{itemize}

\sphinxlineitem{\sphinxstylestrong{2. Ativos e Arquivados}}\begin{itemize}
\item {} 
\sphinxAtStartPar
Permite acessar separadamente os processos em andamento (\sphinxstylestrong{ativos}) e os que já foram \sphinxstylestrong{arquivados}.

\end{itemize}

\sphinxlineitem{\sphinxstylestrong{3. Paralisados}}\begin{itemize}
\item {} 
\sphinxAtStartPar
Já apresentado na aba \sphinxstylestrong{Início}.

\item {} 
\sphinxAtStartPar
Exibe processos que estão sem movimentação há determinado tempo.

\end{itemize}

\sphinxlineitem{\sphinxstylestrong{4. Suspensos}}\begin{itemize}
\item {} 
\sphinxAtStartPar
Exibe os processos que estão \sphinxstylestrong{suspensos}.

\item {} 
\sphinxAtStartPar
É possível filtrar por tempo de suspensão:
\sphinxhyphen{} Suspensos há \sphinxstylestrong{30 dias}, \sphinxstylestrong{90 dias} ou \sphinxstylestrong{tempo indeterminado}.

\item {} 
\sphinxAtStartPar
Basta selecionar a opção desejada e clicar em \sphinxstylestrong{Pesquisar}.

\end{itemize}

\sphinxlineitem{\sphinxstylestrong{5. Instância Superior}}\begin{itemize}
\item {} 
\sphinxAtStartPar
Permite visualizar os processos \sphinxstylestrong{remetidos ao segundo grau} (Tribunais ou Turmas Recursais).

\item {} 
\sphinxAtStartPar
Filtros disponíveis:
\sphinxhyphen{} \sphinxstylestrong{Remetidos}, \sphinxstylestrong{Retornados}, ou \sphinxstylestrong{Ambos}.
\sphinxhyphen{} Filtragem por \sphinxstylestrong{período de tempo}.

\end{itemize}

\sphinxlineitem{\sphinxstylestrong{6. Remetidos}}\begin{itemize}
\item {} 
\sphinxAtStartPar
Lista todos os processos remetidos para:
\sphinxhyphen{} \sphinxstylestrong{Distribuidor}
\sphinxhyphen{} \sphinxstylestrong{Procuradoria}
\sphinxhyphen{} \sphinxstylestrong{Ministério Público}
\sphinxhyphen{} \sphinxstylestrong{Defensoria}

\item {} 
\sphinxAtStartPar
Permite selecionar o \sphinxstylestrong{destino da remessa} e a \sphinxstylestrong{situação}:
\sphinxhyphen{} Aguardando retorno
\sphinxhyphen{} Enviado
\sphinxhyphen{} Retornado

\item {} 
\sphinxAtStartPar
Também permite filtrar por período.

\end{itemize}

\sphinxlineitem{\sphinxstylestrong{7. Buscas}}\begin{itemize}
\item {} 
\sphinxAtStartPar
Aqui será possível realizar buscas avançadas por diversos critérios.

\item {} 
\sphinxAtStartPar
Será abordado com mais detalhes em aula posterior.

\end{itemize}

\sphinxlineitem{\sphinxstylestrong{8. Novação}}\begin{itemize}
\item {} 
\sphinxAtStartPar
Ferramenta utilizada pelo distribuidor.

\item {} 
\sphinxAtStartPar
Permite realizar tarefas relacionadas à nova distribuição.

\end{itemize}

\sphinxlineitem{\sphinxstylestrong{9. Analisar Suspeita de Prevenção}}\begin{itemize}
\item {} 
\sphinxAtStartPar
Apresenta uma \sphinxstylestrong{relação de processos} que podem estar relacionados a casos de prevenção.

\item {} 
\sphinxAtStartPar
Permite selecionar individualmente e verificar se há conexão ou prevenção entre processos.

\end{itemize}

\sphinxlineitem{\sphinxstylestrong{10. Suspeita de Repetição}}\begin{itemize}
\item {} 
\sphinxAtStartPar
Possibilita identificar processos repetitivos.

\item {} 
\sphinxAtStartPar
Filtros disponíveis:
\sphinxhyphen{} Por nome do processo
\sphinxhyphen{} \sphinxstylestrong{Aguardando análise}
\sphinxhyphen{} \sphinxstylestrong{Prevenção confirmada}
\sphinxhyphen{} \sphinxstylestrong{Prevenção descartada}

\end{itemize}

\sphinxlineitem{\sphinxstylestrong{11. Cartas Precatórias Eletrônicas}}\begin{itemize}
\item {} 
\sphinxAtStartPar
Será tema de uma \sphinxstylestrong{aula específica futura}.

\end{itemize}

\end{description}

\sphinxstepscope


\subsection{Buscas e Relação de Réu Preso}
\label{\detokenize{projud_43_buscarelacaoreupreso:buscas-e-relacao-de-reu-preso}}\label{\detokenize{projud_43_buscarelacaoreupreso::doc}}
\sphinxAtStartPar
Nesta aula, vamos explorar as funcionalidades da aba \sphinxstylestrong{Processos}, especificamente no menu de \sphinxstylestrong{Buscas} do sistema Projudi.


\subsubsection{Funcionalidades de Busca}
\label{\detokenize{projud_43_buscarelacaoreupreso:funcionalidades-de-busca}}
\sphinxAtStartPar
A aba \sphinxstylestrong{Processos \textgreater{} Busca} permite ao usuário localizar processos de forma simples ou avançada, além de realizar buscas específicas como a de réus presos. Abaixo estão as principais ferramentas:


\paragraph{Busca Simples}
\label{\detokenize{projud_43_buscarelacaoreupreso:busca-simples}}\begin{itemize}
\item {} 
\sphinxAtStartPar
Permite localizar um processo diretamente pelo \sphinxstylestrong{número do processo}.

\item {} 
\sphinxAtStartPar
O histórico de processos visualizados recentemente é acessível via o \sphinxstylestrong{ícone da lupa}.

\end{itemize}


\paragraph{Busca Avançada}
\label{\detokenize{projud_43_buscarelacaoreupreso:busca-avancada}}
\sphinxAtStartPar
Permite pesquisar processos com filtros detalhados:
\begin{itemize}
\item {} 
\sphinxAtStartPar
\sphinxstylestrong{Nome da parte}

\item {} 
\sphinxAtStartPar
\sphinxstylestrong{Nome da mãe ou pai}

\item {} 
\sphinxAtStartPar
\sphinxstylestrong{Juízo:} manter “todos” para buscar em toda a jurisdição (capital e interior).

\item {} 
\sphinxAtStartPar
\sphinxstylestrong{Classificação e Classe Processual}

\item {} 
\sphinxAtStartPar
\sphinxstylestrong{Status do processo:} ativos, arquivados, suspensos, etc.

\item {} 
\sphinxAtStartPar
\sphinxstylestrong{Segredo de Justiça}

\item {} 
\sphinxAtStartPar
\sphinxstylestrong{Localizador}

\item {} 
\sphinxAtStartPar
\sphinxstylestrong{Data da distribuição}

\item {} 
\sphinxAtStartPar
\sphinxstylestrong{Nome do advogado e número da OAB}

\end{itemize}

\sphinxAtStartPar
Após configurar os filtros desejados, clique em \sphinxstylestrong{Pesquisar}.


\paragraph{Busca de Prisões, Acolhimentos e Internações}
\label{\detokenize{projud_43_buscarelacaoreupreso:busca-de-prisoes-acolhimentos-e-internacoes}}
\sphinxAtStartPar
Acesse a opção \sphinxstylestrong{Prisões} para visualizar e filtrar os processos com réus presos:
\begin{itemize}
\item {} 
\sphinxAtStartPar
\sphinxstylestrong{Nome do réu}

\item {} 
\sphinxAtStartPar
\sphinxstylestrong{Nome da mãe}

\item {} 
\sphinxAtStartPar
\sphinxstylestrong{Agrupar por processo ou por réu}

\item {} 
\sphinxAtStartPar
\sphinxstylestrong{Situação do réu:} preso, não preso ou ambos

\item {} 
\sphinxAtStartPar
\sphinxstylestrong{Data da prisão}

\item {} 
\sphinxAtStartPar
\sphinxstylestrong{Número da guia}

\item {} 
\sphinxAtStartPar
\sphinxstylestrong{Motivo da prisão:} flagrante, preventiva, domiciliar, etc.

\item {} 
\sphinxAtStartPar
\sphinxstylestrong{Motivo da soltura:} absolvição, arquivamento, liberdade provisória, etc.

\item {} 
\sphinxAtStartPar
\sphinxstylestrong{Classe processual e status do processo}

\item {} 
\sphinxAtStartPar
\sphinxstylestrong{Artigo da lei:} permite buscar por artigos específicos (ex.: art. 171 \sphinxhyphen{} estelionato)

\end{itemize}

\sphinxAtStartPar
Resultado da Busca de Réus Presos:
\begin{itemize}
\item {} 
\sphinxAtStartPar
Número do processo

\item {} 
\sphinxAtStartPar
Nome do réu

\item {} 
\sphinxAtStartPar
Data de nascimento

\item {} 
\sphinxAtStartPar
Data da prisão

\item {} 
\sphinxAtStartPar
Unidade prisional

\item {} 
\sphinxAtStartPar
\sphinxstylestrong{Quantidade de dias preso}

\end{itemize}


\paragraph{Outras Formas de Busca}
\label{\detokenize{projud_43_buscarelacaoreupreso:outras-formas-de-busca}}
\sphinxAtStartPar
Além de prisões, é possível realizar buscas por:
\begin{itemize}
\item {} 
\sphinxAtStartPar
\sphinxstylestrong{Denúncia}

\item {} 
\sphinxAtStartPar
\sphinxstylestrong{Queixa}

\item {} 
\sphinxAtStartPar
\sphinxstylestrong{Cumprimento de medidas}

\item {} 
\sphinxAtStartPar
\sphinxstylestrong{Benefícios ou suspensão de medidas}

\end{itemize}

\sphinxAtStartPar
Essas opções também estarão disponíveis na \sphinxstylestrong{Mesa do Escrivão Criminal}, que será abordada em aula futura.

\sphinxstepscope


\subsection{Carta Precatória Eletrônica}
\label{\detokenize{projud_44_cartaprecatoriaeletronica:carta-precatoria-eletronica}}\label{\detokenize{projud_44_cartaprecatoriaeletronica::doc}}
\sphinxAtStartPar
Nesta aula, aprendemos como confeccionar, enviar e acompanhar \sphinxstylestrong{cartas precatórias eletrônicas} no sistema Projudi, diferenciando\sphinxhyphen{}as das cartas precatórias físicas que são utilizadas quando o destinatário \sphinxstylestrong{não usa o sistema Projudi}.


\subsubsection{Tipos de Carta Precatória}
\label{\detokenize{projud_44_cartaprecatoriaeletronica:tipos-de-carta-precatoria}}\begin{enumerate}
\sphinxsetlistlabels{\arabic}{enumi}{enumii}{}{.}%
\item {} 
\sphinxAtStartPar
\sphinxstylestrong{Carta Precatória Física (via Ordenar Cumprimentos)}:
\sphinxhyphen{} Utilizada para enviar comunicações a unidades judiciárias \sphinxstylestrong{fora do Projudi} (ex: PJe, SAJ).
\sphinxhyphen{} Emitida via: \sphinxtitleref{Movimentações \textgreater{} Ordenar Cumprimentos \textgreater{} Carta Precatória}
\sphinxhyphen{} Após assinatura do magistrado, é enviada por \sphinxstylestrong{malote digital} ou cadastrada em outro sistema externo.

\item {} 
\sphinxAtStartPar
\sphinxstylestrong{Carta Precatória Eletrônica (internamente via Projudi)}:
\sphinxhyphen{} Usada quando o destinatário também \sphinxstylestrong{utiliza o sistema Projudi}, geralmente dentro do mesmo estado.
\sphinxhyphen{} Tramitada integralmente pelo próprio sistema.

\end{enumerate}


\subsubsection{Acesso às Cartas Precatórias}
\label{\detokenize{projud_44_cartaprecatoriaeletronica:acesso-as-cartas-precatorias}}
\sphinxAtStartPar
Acesse pelo menu superior:

\sphinxAtStartPar
\sphinxcode{\sphinxupquote{Processos \textgreater{} Cartas \textgreater{} Carta Precatória}}
\begin{itemize}
\item {} 
\sphinxAtStartPar
\sphinxstylestrong{Recebidas}: Cartas recebidas de outras unidades Projudi.

\item {} 
\sphinxAtStartPar
\sphinxstylestrong{Enviadas}: Cartas criadas pela sua unidade.

\item {} 
\sphinxAtStartPar
Filtros disponíveis: Tipo, status, comarca de origem/destino, prazo urgente, finalidade, etc.

\end{itemize}


\subsubsection{Como Criar uma Carta Precatória Eletrônica}
\label{\detokenize{projud_44_cartaprecatoriaeletronica:como-criar-uma-carta-precatoria-eletronica}}
\sphinxAtStartPar
Há duas formas de iniciar:
\begin{enumerate}
\sphinxsetlistlabels{\arabic}{enumi}{enumii}{}{.}%
\item {} 
\sphinxAtStartPar
\sphinxstylestrong{Via Menu Processos}
\sphinxhyphen{} \sphinxtitleref{Processos \textgreater{} Cartas \textgreater{} Carta Precatória \textgreater{} Criar Carta Precatória}

\item {} 
\sphinxAtStartPar
\sphinxstylestrong{Dentro do Processo (mais comum)}
\sphinxhyphen{} \sphinxtitleref{Movimentações \textgreater{} {[}Selecione a decisão{]} \textgreater{} Movimentar a partir desta \textgreater{} Ordenar Carta Precatória Eletrônica}

\end{enumerate}


\paragraph{Passos para Cadastro da Carta Precatória Eletrônica}
\label{\detokenize{projud_44_cartaprecatoriaeletronica:passos-para-cadastro-da-carta-precatoria-eletronica}}\begin{enumerate}
\sphinxsetlistlabels{\arabic}{enumi}{enumii}{}{.}%
\item {} 
\sphinxAtStartPar
\sphinxstylestrong{Informações Iniciais}
\sphinxhyphen{} Número do processo originário
\sphinxhyphen{} Nome do magistrado
\sphinxhyphen{} Urgência (sim ou não)

\item {} 
\sphinxAtStartPar
\sphinxstylestrong{Finalidade e Destino}
\sphinxhyphen{} Finalidade: citação, intimação etc.
\sphinxhyphen{} Comarca e vara de destino
\sphinxhyphen{} Competência e prazo para cumprimento

\item {} 
\sphinxAtStartPar
\sphinxstylestrong{Seleção de Partes}
\sphinxhyphen{} Selecionar as partes relevantes que serão copiadas automaticamente para o novo processo.

\item {} 
\sphinxAtStartPar
\sphinxstylestrong{Documento da Capa}
\sphinxhyphen{} Inserir o documento principal (modelo da carta).

\item {} 
\sphinxAtStartPar
\sphinxstylestrong{Documentos Anexos}
\sphinxhyphen{} Anexar petição inicial, decisão, denúncia, etc.

\item {} 
\sphinxAtStartPar
\sphinxstylestrong{Verificação e Cadastro}
\sphinxhyphen{} Conferência final de todos os dados inseridos.
\sphinxhyphen{} Clique em \sphinxstylestrong{Salvar}.

\end{enumerate}


\subsubsection{Gestão da Carta Precatória}
\label{\detokenize{projud_44_cartaprecatoriaeletronica:gestao-da-carta-precatoria}}\begin{itemize}
\item {} 
\sphinxAtStartPar
Após cadastrada, a carta irá para \sphinxstylestrong{assinatura do juiz}.

\item {} 
\sphinxAtStartPar
Após assinada, ficará visível em:
\sphinxhyphen{} \sphinxtitleref{Análise de Juntadas \textgreater{} Tipo de movimento: expedição de carta precatória}

\item {} 
\sphinxAtStartPar
Para acompanhar o status:
\sphinxhyphen{} \sphinxtitleref{Processos \textgreater{} Cartas \textgreater{} Carta Precatória \textgreater{} Enviadas ou Recebidas}
\sphinxhyphen{} Status: aguardando análise, enviada ao deprecado, recebida, etc.

\end{itemize}


\subsubsection{Resumo}
\label{\detokenize{projud_44_cartaprecatoriaeletronica:resumo}}\begin{itemize}
\item {} 
\sphinxAtStartPar
Cartas precatórias eletrônicas são tramitadas internamente entre comarcas que usam o Projudi.

\item {} 
\sphinxAtStartPar
O sistema permite \sphinxstylestrong{confeccionar, gerenciar, acompanhar e arquivar} todas as fases da carta.

\item {} 
\sphinxAtStartPar
A ferramenta é fundamental para a comunicação interjurisdicional eficiente e ágil dentro do mesmo sistema.

\end{itemize}

\sphinxstepscope


\subsection{Listas de Audiências}
\label{\detokenize{projud_45_listasaudiencias:listas-de-audiencias}}\label{\detokenize{projud_45_listasaudiencias::doc}}
\sphinxAtStartPar
Nesta aula, aprendemos a utilizar a aba \sphinxstylestrong{Audiências} do sistema Projudi para \sphinxstylestrong{listar}, \sphinxstylestrong{pesquisar} e \sphinxstylestrong{movimentar} audiências de forma eficiente e organizada.


\subsubsection{Listagem de Audiências}
\label{\detokenize{projud_45_listasaudiencias:listagem-de-audiencias}}
\sphinxAtStartPar
Acesse a listagem de audiências via:

\sphinxAtStartPar
\sphinxcode{\sphinxupquote{Audiências \textgreater{} Listagem}}

\sphinxAtStartPar
Você pode visualizar:
\begin{itemize}
\item {} 
\sphinxAtStartPar
\sphinxstylestrong{Audiências para hoje}: mostra as audiências agendadas para o dia atual.

\item {} 
\sphinxAtStartPar
\sphinxstylestrong{Pendentes}: audiências agendadas que ainda \sphinxstylestrong{não foram movimentadas}.

\item {} 
\sphinxAtStartPar
\sphinxstylestrong{Movimentadas hoje}: audiências realizadas ou movimentadas no dia atual.

\item {} 
\sphinxAtStartPar
\sphinxstylestrong{Data futura}: todas as audiências agendadas para dias posteriores.

\end{itemize}

\sphinxAtStartPar
Para pesquisar:
\begin{enumerate}
\sphinxsetlistlabels{\arabic}{enumi}{enumii}{}{.}%
\item {} 
\sphinxAtStartPar
Escolha o \sphinxstylestrong{tipo de audiência} (ex: conciliação, instrução).

\item {} 
\sphinxAtStartPar
Clique em \sphinxstylestrong{Pesquisar}.

\item {} 
\sphinxAtStartPar
O sistema retornará uma \sphinxstylestrong{lista com os processos}, \sphinxstylestrong{partes envolvidas}, \sphinxstylestrong{data}, \sphinxstylestrong{local}, \sphinxstylestrong{tipo} e \sphinxstylestrong{situação da audiência} (realizada, cancelada, redesignada, etc).

\end{enumerate}


\subsubsection{Movimentar Audiência Pendente}
\label{\detokenize{projud_45_listasaudiencias:movimentar-audiencia-pendente}}
\sphinxAtStartPar
Caso uma audiência tenha sido designada, mas \sphinxstylestrong{não ocorreu} (mesmo que o processo tenha sido arquivado), é necessário \sphinxstylestrong{movimentar} a audiência para retirar a pendência:
\begin{enumerate}
\sphinxsetlistlabels{\arabic}{enumi}{enumii}{}{.}%
\item {} 
\sphinxAtStartPar
Acesse o processo na lista de pendentes.

\item {} 
\sphinxAtStartPar
Clique em \sphinxstylestrong{Movimentar Audiência}.

\item {} 
\sphinxAtStartPar
Na tela de movimentação:
\sphinxhyphen{} Insira um \sphinxstylestrong{ato ordinatório} ou uma \sphinxstylestrong{certidão} informando o motivo da não realização.
\sphinxhyphen{} Pode ser digitável ou em \sphinxstylestrong{PDF}.
\sphinxhyphen{} Defina o \sphinxstylestrong{status} da audiência: cancelada, negativa ou redesignada.
\sphinxhyphen{} Preencha campos como: número de pessoas ouvidas, valor de acordo, nome de quem movimentou.

\item {} 
\sphinxAtStartPar
Clique em \sphinxstylestrong{Salvar}.

\end{enumerate}


\subsubsection{Buscar a Pauta}
\label{\detokenize{projud_45_listasaudiencias:buscar-a-pauta}}
\sphinxAtStartPar
A ferramenta \sphinxstylestrong{Buscar a Pauta} permite localizar audiências com filtros avançados:

\sphinxAtStartPar
\sphinxcode{\sphinxupquote{Audiências \textgreater{} Buscar a Pauta}}

\sphinxAtStartPar
Filtros disponíveis:
\begin{itemize}
\item {} 
\sphinxAtStartPar
Número do processo

\item {} 
\sphinxAtStartPar
Tipo de audiência

\item {} 
\sphinxAtStartPar
Classe ou classificação processual

\item {} 
\sphinxAtStartPar
Assunto ou objeto do pedido

\item {} 
\sphinxAtStartPar
Situação da audiência: movimentada, não movimentada ou ambos

\item {} 
\sphinxAtStartPar
Status da audiência: designada, realizada, cancelada, negativa

\item {} 
\sphinxAtStartPar
Prioridade (ex: idoso, réu preso)

\item {} 
\sphinxAtStartPar
Período por data

\end{itemize}

\sphinxAtStartPar
Após definir os critérios, clique em \sphinxstylestrong{Pesquisar} para visualizar a lista de audiências correspondente.


\subsubsection{Resumo}
\label{\detokenize{projud_45_listasaudiencias:resumo}}
\sphinxAtStartPar
A aba de \sphinxstylestrong{Audiências} permite:
\sphinxhyphen{} Acompanhar e organizar as audiências por status.
\sphinxhyphen{} Controlar as pendências de audiências não movimentadas.
\sphinxhyphen{} Pesquisar rapidamente qualquer audiência por diversos critérios.

\sphinxAtStartPar
É essencial movimentar corretamente as audiências para evitar pendências e manter o fluxo processual em ordem.

\sphinxstepscope


\subsection{Como Pautar Audiência}
\label{\detokenize{projud_46_comopautaraudiencia:como-pautar-audiencia}}\label{\detokenize{projud_46_comopautaraudiencia::doc}}
\sphinxAtStartPar
Nesta aula, aprendemos a \sphinxstylestrong{abrir pautas de audiência} no sistema Projudi e realizar o \sphinxstylestrong{agendamento de processos} nessas pautas.


\subsubsection{Acessar a Pauta de Horários}
\label{\detokenize{projud_46_comopautaraudiencia:acessar-a-pauta-de-horarios}}
\sphinxAtStartPar
Acesse pela aba superior:

\sphinxAtStartPar
\sphinxcode{\sphinxupquote{Audiências \textgreater{} Ver Pauta de Horários}}

\sphinxAtStartPar
Nesta tela, é possível:
\begin{itemize}
\item {} 
\sphinxAtStartPar
Visualizar pautas existentes

\item {} 
\sphinxAtStartPar
Pesquisar por tipo de audiência:
\sphinxhyphen{} Conciliação
\sphinxhyphen{} Preliminar
\sphinxhyphen{} Transação penal
\sphinxhyphen{} Oitiva
\sphinxhyphen{} Audiência de não persecução penal

\end{itemize}


\subsubsection{Criar Nova Pauta}
\label{\detokenize{projud_46_comopautaraudiencia:criar-nova-pauta}}\begin{enumerate}
\sphinxsetlistlabels{\arabic}{enumi}{enumii}{}{.}%
\item {} 
\sphinxAtStartPar
Clique no botão \sphinxstylestrong{Novo}.

\item {} 
\sphinxAtStartPar
Selecione o \sphinxstylestrong{tipo de audiência} (ex: preliminar).

\item {} 
\sphinxAtStartPar
Escolha a \sphinxstylestrong{data da audiência}.

\item {} 
\sphinxAtStartPar
Se necessário, marque a opção \sphinxstylestrong{“Criar em dias não úteis”}.

\item {} 
\sphinxAtStartPar
Para criar pauta para \sphinxstylestrong{um único horário}, clique em \sphinxstylestrong{Adicionar}.

\item {} 
\sphinxAtStartPar
Para criar pauta em \sphinxstylestrong{lote (vários horários)}, clique em \sphinxstylestrong{Adicionar em lote}.

\sphinxAtStartPar
Na tela de lote:
\begin{itemize}
\item {} 
\sphinxAtStartPar
Informe o \sphinxstylestrong{horário inicial} e \sphinxstylestrong{horário final}.

\item {} 
\sphinxAtStartPar
Defina a \sphinxstylestrong{duração da audiência} (em minutos).

\item {} 
\sphinxAtStartPar
Informe o \sphinxstylestrong{número de audiências por horário}.

\item {} 
\sphinxAtStartPar
Se desejar que o sistema agende automaticamente os horários, marque \sphinxstylestrong{“Agendar automaticamente: Sim”}.

\end{itemize}

\item {} 
\sphinxAtStartPar
Clique em \sphinxstylestrong{Adicionar}.

\item {} 
\sphinxAtStartPar
Após configurar a pauta, clique em \sphinxstylestrong{Salvar} para finalizar o cadastro.

\end{enumerate}


\subsubsection{Agendar Processo na Pauta}
\label{\detokenize{projud_46_comopautaraudiencia:agendar-processo-na-pauta}}\begin{enumerate}
\sphinxsetlistlabels{\arabic}{enumi}{enumii}{}{.}%
\item {} 
\sphinxAtStartPar
Acesse o processo desejado.

\item {} 
\sphinxAtStartPar
Clique no botão \sphinxstylestrong{Agendar}, localizado abaixo do campo de \sphinxstylestrong{Nível de Sigilo}.

\item {} 
\sphinxAtStartPar
Na nova tela:
\sphinxhyphen{} Selecione o \sphinxstylestrong{tipo de audiência} (ex: preliminar).
\sphinxhyphen{} Clique em \sphinxstylestrong{Manual}.
\sphinxhyphen{} Escolha o \sphinxstylestrong{horário disponível na pauta} que deseja utilizar.
\sphinxhyphen{} Clique em \sphinxstylestrong{Agendar}.

\sphinxAtStartPar
Confirme o agendamento.

\end{enumerate}


\subsubsection{Resultado do Agendamento}
\label{\detokenize{projud_46_comopautaraudiencia:resultado-do-agendamento}}
\sphinxAtStartPar
Após o agendamento:
\begin{itemize}
\item {} 
\sphinxAtStartPar
Na \sphinxstylestrong{tela inicial do processo}, será gerada uma \sphinxstylestrong{pendência} de audiência.

\item {} 
\sphinxAtStartPar
A informação aparecerá como:
\sphinxcode{\sphinxupquote{Audiência Preliminar em aberto \sphinxhyphen{} {[}Data e horário{]}}}

\item {} 
\sphinxAtStartPar
Em \sphinxstylestrong{Movimentações}, será registrada a movimentação de designação de audiência.

\end{itemize}


\subsubsection{Resumo}
\label{\detokenize{projud_46_comopautaraudiencia:resumo}}
\sphinxAtStartPar
Com os recursos da aba \sphinxstylestrong{Audiências}, é possível:
\begin{itemize}
\item {} 
\sphinxAtStartPar
Criar pautas manuais ou em lote para diferentes tipos de audiência.

\item {} 
\sphinxAtStartPar
Agendar facilmente um processo para um horário disponível.

\item {} 
\sphinxAtStartPar
Acompanhar o agendamento diretamente na tela do processo.

\end{itemize}

\sphinxstepscope


\subsection{Movimentação de Audiência}
\label{\detokenize{projud_47_movimentacaoaudiencia:movimentacao-de-audiencia}}\label{\detokenize{projud_47_movimentacaoaudiencia::doc}}
\sphinxAtStartPar
Nesta aula, você aprenderá como \sphinxstylestrong{movimentar uma audiência} no sistema Projudi, incluindo a inserção do \sphinxstylestrong{termo de audiência}, definição de \sphinxstylestrong{status}, \sphinxstylestrong{resultado}, e até a \sphinxstylestrong{prolação de sentença}, quando cabível.


\subsubsection{Importante}
\label{\detokenize{projud_47_movimentacaoaudiencia:importante}}
\sphinxAtStartPar
Após agendar uma audiência, \sphinxstylestrong{não é possível excluir a pendência} gerada. Caso necessário, deve\sphinxhyphen{}se movimentá\sphinxhyphen{}la informando, por exemplo, que foi:
\begin{itemize}
\item {} 
\sphinxAtStartPar
Cancelada

\item {} 
\sphinxAtStartPar
Redesignada

\item {} 
\sphinxAtStartPar
Inserida por engano

\end{itemize}


\subsubsection{Movimentar Audiência}
\label{\detokenize{projud_47_movimentacaoaudiencia:movimentar-audiencia}}\begin{enumerate}
\sphinxsetlistlabels{\arabic}{enumi}{enumii}{}{.}%
\item {} 
\sphinxAtStartPar
Acesse o processo e clique em \sphinxstylestrong{Movimentar Audiência}.

\item {} 
\sphinxAtStartPar
Em \sphinxstylestrong{Arquivos}, adicione o \sphinxstylestrong{termo de audiência}:
\begin{itemize}
\item {} 
\sphinxAtStartPar
Você pode \sphinxstylestrong{digitar o texto} no sistema (ex: termo de retratação, ausência de vítima, etc.)

\item {} 
\sphinxAtStartPar
Ou \sphinxstylestrong{anexar um arquivo PDF} (ex: termo assinado e digitalizado)

\end{itemize}

\item {} 
\sphinxAtStartPar
Informe:
\begin{itemize}
\item {} 
\sphinxAtStartPar
\sphinxstylestrong{Status da audiência}: realizada, cancelada, redesignada, etc.

\item {} 
\sphinxAtStartPar
\sphinxstylestrong{Resultado}: acordo, transação, declínio de competência, revelia, etc.

\item {} 
\sphinxAtStartPar
\sphinxstylestrong{Número de pessoas ouvidas}

\item {} 
\sphinxAtStartPar
\sphinxstylestrong{Valor do acordo} (se aplicável)

\item {} 
\sphinxAtStartPar
\sphinxstylestrong{Nome do responsável pela movimentação}

\end{itemize}

\end{enumerate}


\subsubsection{Sentença ou Decisão}
\label{\detokenize{projud_47_movimentacaoaudiencia:sentenca-ou-decisao}}
\sphinxAtStartPar
Se o termo de audiência \sphinxstylestrong{contiver uma sentença}, é possível já lançar a movimentação como \sphinxstylestrong{sentença homologatória}:
\begin{itemize}
\item {} 
\sphinxAtStartPar
Em \sphinxstylestrong{Tipo de movimentação}:
\sphinxhyphen{} Escolha “Sentença homologatória” (ex: movimentação 466 \textendash{} homologação de acordo)

\item {} 
\sphinxAtStartPar
Em \sphinxstylestrong{Juiz}:
\sphinxhyphen{} Informe o nome do magistrado

\item {} 
\sphinxAtStartPar
Em \sphinxstylestrong{Realizado por}:
\sphinxhyphen{} Também insira o nome do juiz, para que gere \sphinxstylestrong{produtividade ao magistrado}

\item {} 
\sphinxAtStartPar
Marque a opção: \sphinxstylestrong{Movimentar processo automaticamente}

\item {} 
\sphinxAtStartPar
Clique em \sphinxstylestrong{Salvar}

\end{itemize}

\sphinxAtStartPar
Se não houver sentença, mas apenas uma \sphinxstylestrong{decisão interlocutória}, repita o procedimento escolhendo:
\begin{itemize}
\item {} 
\sphinxAtStartPar
\sphinxstylestrong{Tipo de movimentação}: Decisão

\item {} 
\sphinxAtStartPar
\sphinxstylestrong{Juiz}: Informe o nome do magistrado

\item {} 
\sphinxAtStartPar
\sphinxstylestrong{Realizado por}: Servidor responsável

\item {} 
\sphinxAtStartPar
O sistema \sphinxstylestrong{não gerará produtividade} ao juiz nesse caso.

\end{itemize}


\subsubsection{Resultado da Movimentação}
\label{\detokenize{projud_47_movimentacaoaudiencia:resultado-da-movimentacao}}\begin{itemize}
\item {} 
\sphinxAtStartPar
A movimentação será exibida na aba \sphinxstylestrong{Movimentações} do processo

\item {} 
\sphinxAtStartPar
A depender da escolha entre \sphinxstylestrong{decisão ou sentença}, ela pode:
\sphinxhyphen{} Ser apenas uma anotação
\sphinxhyphen{} Ou \sphinxstylestrong{gerar produtividade diretamente ao juiz}, sem necessidade de enviar concluso

\end{itemize}


\subsubsection{Resumo}
\label{\detokenize{projud_47_movimentacaoaudiencia:resumo}}
\sphinxAtStartPar
O sistema Projudi permite:
\begin{itemize}
\item {} 
\sphinxAtStartPar
Inserção do termo de audiência de forma digital ou anexada

\item {} 
\sphinxAtStartPar
Registro do status e resultado da audiência

\item {} 
\sphinxAtStartPar
Prolação direta de sentença homologatória com impacto na produtividade

\item {} 
\sphinxAtStartPar
Movimentações ágeis, sem necessidade de envio concluso ao magistrado em alguns casos

\end{itemize}

\sphinxstepscope


\subsection{Criando Modelos de Documento}
\label{\detokenize{projud_48_criandomodelo:criando-modelos-de-documento}}\label{\detokenize{projud_48_criandomodelo::doc}}
\sphinxAtStartPar
Nesta aula, você aprenderá a criar modelos de documentos dentro do sistema Projudi, utilizando a funcionalidade disponível na aba \sphinxstylestrong{Outros \textgreater{} Meus modelos de documentos}.


\subsubsection{Acessando a funcionalidade}
\label{\detokenize{projud_48_criandomodelo:acessando-a-funcionalidade}}\begin{enumerate}
\sphinxsetlistlabels{\arabic}{enumi}{enumii}{}{.}%
\item {} 
\sphinxAtStartPar
Clique na aba superior \sphinxstylestrong{Outros}.

\item {} 
\sphinxAtStartPar
Selecione \sphinxstylestrong{Meus modelos de documentos}.

\item {} 
\sphinxAtStartPar
Para visualizar modelos existentes, clique em \sphinxstylestrong{Pesquisar}.

\item {} 
\sphinxAtStartPar
Para criar um novo modelo, clique em \sphinxstylestrong{Novo}.

\end{enumerate}


\subsubsection{Criando um novo modelo}
\label{\detokenize{projud_48_criandomodelo:criando-um-novo-modelo}}
\sphinxAtStartPar
Na tela de criação de modelo:
\begin{itemize}
\item {} 
\sphinxAtStartPar
\sphinxstylestrong{Descrição}: Informe o nome do modelo (ex: \sphinxstyleemphasis{Mandado de Citação Criminal}).

\item {} 
\sphinxAtStartPar
\sphinxstylestrong{Tipo de Documento}: Escolha entre opções como \sphinxstyleemphasis{Citação}, \sphinxstyleemphasis{Intimação}, \sphinxstyleemphasis{Alvará}, etc.

\item {} 
\sphinxAtStartPar
\sphinxstylestrong{Competência}: Indique se é da área Cível, Criminal, Juizado, etc.

\end{itemize}

\begin{sphinxadmonition}{note}{Nota:}
\sphinxAtStartPar
O modelo criado ficará \sphinxstylestrong{disponível para todos os usuários da vara}.
\end{sphinxadmonition}


\subsubsection{Inserindo e editando o conteúdo}
\label{\detokenize{projud_48_criandomodelo:inserindo-e-editando-o-conteudo}}\begin{itemize}
\item {} 
\sphinxAtStartPar
Copie e cole o texto no campo “Inserir texto”.

\item {} 
\sphinxAtStartPar
Utilize o botão \sphinxstylestrong{Maximizar} para melhor visualização.

\item {} 
\sphinxAtStartPar
Cuidado com \sphinxstylestrong{âncoras ocultas} (ícones de bandeira azul) que podem aparecer ao colar de documentos externos. Para removê\sphinxhyphen{}las:
\sphinxhyphen{} Clique com o botão direito sobre a âncora
\sphinxhyphen{} Selecione \sphinxstylestrong{Remover âncora}

\end{itemize}


\subsubsection{Formatando o texto}
\label{\detokenize{projud_48_criandomodelo:formatando-o-texto}}
\sphinxAtStartPar
Use o botão \sphinxstylestrong{Estilo} para aplicar formatações padronizadas:
\begin{itemize}
\item {} 
\sphinxAtStartPar
\sphinxstylestrong{Parágrafo 1}: Recuo de 2cm

\item {} 
\sphinxAtStartPar
\sphinxstylestrong{Parágrafo 2}: Recuo de 3cm (padrão mais utilizado)

\item {} 
\sphinxAtStartPar
\sphinxstylestrong{Parágrafo 3}: Recuo de 4cm

\item {} 
\sphinxAtStartPar
\sphinxstylestrong{Emenda}: Para títulos ou frases destacadas

\end{itemize}

\begin{sphinxadmonition}{tip}{Dica:}
\sphinxAtStartPar
O espaçamento entre linhas (ex: 1,15 / 1,5) será mantido conforme o texto colado.
\end{sphinxadmonition}


\subsubsection{Inserindo variáveis}
\label{\detokenize{projud_48_criandomodelo:inserindo-variaveis}}
\sphinxAtStartPar
O sistema permite o uso de \sphinxstylestrong{variáveis automáticas}, que serão preenchidas dinamicamente de acordo com o processo.

\sphinxAtStartPar
Exemplos:
\begin{itemize}
\item {} 
\sphinxAtStartPar
\sphinxcode{\sphinxupquote{\$\{nomeParte\}}} \textendash{} Nome do réu ou parte

\item {} 
\sphinxAtStartPar
\sphinxcode{\sphinxupquote{\$\{numeroProcesso\}}} \textendash{} Número do processo

\item {} 
\sphinxAtStartPar
\sphinxcode{\sphinxupquote{\$\{assinatura\}}} \textendash{} Local da assinatura

\end{itemize}

\begin{sphinxadmonition}{note}{Nota:}
\sphinxAtStartPar
Variáveis serão explicadas com mais detalhes na \sphinxstylestrong{próxima aula}.
\end{sphinxadmonition}


\subsubsection{Finalizando o modelo}
\label{\detokenize{projud_48_criandomodelo:finalizando-o-modelo}}\begin{itemize}
\item {} 
\sphinxAtStartPar
Clique em \sphinxstylestrong{Salvar} para registrar seu modelo.

\item {} 
\sphinxAtStartPar
Para visualizar como ficará o documento final, clique em \sphinxstylestrong{Pré\sphinxhyphen{}visualizar}.

\item {} 
\sphinxAtStartPar
Ajuste quebras de linha e espaços excessivos para manter o documento em \sphinxstylestrong{uma única lauda}.

\end{itemize}


\subsubsection{Edição e remoção}
\label{\detokenize{projud_48_criandomodelo:edicao-e-remocao}}\begin{itemize}
\item {} 
\sphinxAtStartPar
Para editar um modelo, clique em \sphinxstylestrong{Alterar}.

\item {} 
\sphinxAtStartPar
Para remover, clique em \sphinxstylestrong{Remover}.

\item {} 
\sphinxAtStartPar
Ao editar, você pode:
\sphinxhyphen{} Excluir espaços acima do cabeçalho (ex: “Autos nº…”)
\sphinxhyphen{} Centralizar títulos (ex: \sphinxstyleemphasis{Mandado de Citação})
\sphinxhyphen{} Corrigir recuos e parágrafos com o botão \sphinxstylestrong{Estilo}

\end{itemize}


\subsubsection{Conclusão}
\label{\detokenize{projud_48_criandomodelo:conclusao}}
\sphinxAtStartPar
Após o salvamento, seu modelo estará disponível na listagem de \sphinxstylestrong{Meus modelos de documentos}, pronto para ser utilizado em futuras expedições, com variáveis preenchidas automaticamente.

\sphinxstepscope


\subsection{Variáveis}
\label{\detokenize{projud_49_variaveis:variaveis}}\label{\detokenize{projud_49_variaveis::doc}}
\sphinxAtStartPar
Nesta aula, você aprenderá a inserir \sphinxstylestrong{variáveis} nos seus modelos de documentos dentro do sistema Projudi. O uso de variáveis é \sphinxstylestrong{opcional}, mas muito útil para automação e padronização dos documentos.


\subsubsection{O que são variáveis?}
\label{\detokenize{projud_49_variaveis:o-que-sao-variaveis}}
\sphinxAtStartPar
Variáveis são comandos que são substituídos automaticamente por dados do processo quando o documento é gerado.

\sphinxAtStartPar
Por exemplo:
\begin{itemize}
\item {} 
\sphinxAtStartPar
\sphinxcode{\sphinxupquote{\$\{assinaturaJuiz\}}} \(\rightarrow\) Exibe o nome e cargo do juiz.

\item {} 
\sphinxAtStartPar
\sphinxcode{\sphinxupquote{\$\{dataAtual\}}} \(\rightarrow\) Exibe a data corrente.

\item {} 
\sphinxAtStartPar
\sphinxcode{\sphinxupquote{\$\{partePassiva\}}} \(\rightarrow\) Exibe o nome da parte ré no processo.

\end{itemize}


\subsubsection{Como inserir uma variável}
\label{\detokenize{projud_49_variaveis:como-inserir-uma-variavel}}\begin{enumerate}
\sphinxsetlistlabels{\arabic}{enumi}{enumii}{}{.}%
\item {} 
\sphinxAtStartPar
Ao editar um modelo de documento, clique no botão \sphinxstylestrong{Variável}.

\item {} 
\sphinxAtStartPar
Uma janela será aberta com uma lista de variáveis disponíveis.

\item {} 
\sphinxAtStartPar
Localize a variável desejada e clique nela para copiá\sphinxhyphen{}la.

\item {} 
\sphinxAtStartPar
Cole no local do documento onde deseja que a informação apareça.

\end{enumerate}


\subsubsection{Exemplos práticos}
\label{\detokenize{projud_49_variaveis:exemplos-praticos}}
\sphinxAtStartPar
\sphinxstylestrong{Assinatura do Juiz}
\begin{itemize}
\item {} 
\sphinxAtStartPar
Variável: \sphinxcode{\sphinxupquote{\$\{assinaturaJuiz\}}}

\item {} 
\sphinxAtStartPar
Finalidade: Exibe nome e cargo do juiz de direito

\item {} 
\sphinxAtStartPar
Onde usar: No rodapé do documento, na linha de assinatura

\end{itemize}

\sphinxAtStartPar
\sphinxstylestrong{Parte passiva (réu)}
\begin{itemize}
\item {} 
\sphinxAtStartPar
Variável: \sphinxcode{\sphinxupquote{\$\{partePassiva\}}}

\item {} 
\sphinxAtStartPar
Finalidade: Inserir automaticamente o nome da parte ré

\item {} 
\sphinxAtStartPar
Onde usar: Ao longo da sentença ou mandado, para evitar digitação manual

\end{itemize}

\sphinxAtStartPar
\sphinxstylestrong{Data atual}
\begin{itemize}
\item {} 
\sphinxAtStartPar
Variável: \sphinxcode{\sphinxupquote{\$\{dataAtual\}}}

\item {} 
\sphinxAtStartPar
Finalidade: Preencher a data corrente no documento

\item {} 
\sphinxAtStartPar
Onde usar: Cabeçalhos, rodapés, corpo do documento

\end{itemize}

\begin{sphinxadmonition}{note}{Nota:}
\sphinxAtStartPar
Algumas variáveis possuem nomes similares e resultados idênticos. Você pode testar qual se adequa melhor ao seu modelo.
\end{sphinxadmonition}


\subsubsection{Pré\sphinxhyphen{}visualização do documento}
\label{\detokenize{projud_49_variaveis:pre-visualizacao-do-documento}}
\sphinxAtStartPar
Após inserir as variáveis, clique em \sphinxstylestrong{Pré\sphinxhyphen{}visualizar} para ver como o documento ficará quando for utilizado. As variáveis serão substituídas automaticamente pelos dados reais do processo:
\begin{itemize}
\item {} 
\sphinxAtStartPar
O nome do réu (\sphinxcode{\sphinxupquote{\$\{partePassiva\}}}) aparecerá conforme o cadastro.

\item {} 
\sphinxAtStartPar
A assinatura (\sphinxcode{\sphinxupquote{\$\{assinaturaJuiz\}}}) mostrará o nome do juiz logado.

\item {} 
\sphinxAtStartPar
A data (\sphinxcode{\sphinxupquote{\$\{dataAtual\}}}) refletirá o dia da expedição do documento.

\end{itemize}


\subsubsection{Dicas importantes}
\label{\detokenize{projud_49_variaveis:dicas-importantes}}\begin{itemize}
\item {} 
\sphinxAtStartPar
Variáveis não precisam de asterisco (*) para funcionar.

\item {} 
\sphinxAtStartPar
Utilize variáveis para evitar erros de digitação e agilizar o trabalho.

\item {} 
\sphinxAtStartPar
Você pode utilizar várias variáveis no mesmo documento.

\end{itemize}


\subsubsection{Conclusão}
\label{\detokenize{projud_49_variaveis:conclusao}}
\sphinxAtStartPar
Agora que você já sabe utilizar variáveis, seus modelos ficarão mais dinâmicos, personalizados e prontos para uso automático no sistema Projudi.

\sphinxAtStartPar
Na próxima aula, veremos como \sphinxstylestrong{utilizar modelos criados com variáveis} em documentos reais.

\sphinxstepscope


\subsection{Localizador}
\label{\detokenize{projud_50_localizador:localizador}}\label{\detokenize{projud_50_localizador::doc}}
\sphinxAtStartPar
O \sphinxstylestrong{Localizador} no sistema Projudi é uma ferramenta que permite organizar e agrupar processos manualmente, de forma personalizada.

\sphinxAtStartPar
Diferente das \sphinxstylestrong{filas processuais}, que são criadas e gerenciadas pelo próprio sistema, o \sphinxstylestrong{Localizador} pode ser criado, nomeado e utilizado livremente por cada usuário, para melhor organização do seu fluxo de trabalho.


\subsubsection{O que é um Localizador?}
\label{\detokenize{projud_50_localizador:o-que-e-um-localizador}}
\sphinxAtStartPar
Um localizador funciona como uma “etiqueta” ou “caixa” virtual onde você pode alocar processos relacionados a uma determinada ação ou fase do trabalho. Exemplos:
\begin{itemize}
\item {} 
\sphinxAtStartPar
Processos aguardando resposta da Renajude

\item {} 
\sphinxAtStartPar
Processos com pendências documentais

\item {} 
\sphinxAtStartPar
Processos aguardando providência externa

\end{itemize}


\subsubsection{Criando um Localizador}
\label{\detokenize{projud_50_localizador:criando-um-localizador}}\begin{enumerate}
\sphinxsetlistlabels{\arabic}{enumi}{enumii}{}{.}%
\item {} 
\sphinxAtStartPar
Vá até a aba \sphinxstylestrong{Outros} \(\rightarrow\) \sphinxstylestrong{Cadastrar Localizador}

\item {} 
\sphinxAtStartPar
Clique em \sphinxstylestrong{Novo}

\item {} 
\sphinxAtStartPar
Informe a descrição do localizador (ex: \sphinxcode{\sphinxupquote{Registrar na Renajude}})

\item {} 
\sphinxAtStartPar
Clique em \sphinxstylestrong{Salvar}

\end{enumerate}


\subsubsection{Adicionando um processo a um Localizador}
\label{\detokenize{projud_50_localizador:adicionando-um-processo-a-um-localizador}}\begin{enumerate}
\sphinxsetlistlabels{\arabic}{enumi}{enumii}{}{.}%
\item {} 
\sphinxAtStartPar
Acesse a tela inicial de um processo

\item {} 
\sphinxAtStartPar
Vá até a aba \sphinxstylestrong{Informações Gerais}

\item {} 
\sphinxAtStartPar
Na seção \sphinxstylestrong{Localizadores}, clique em \sphinxstylestrong{Adicionar}

\item {} 
\sphinxAtStartPar
Selecione o localizador desejado

\item {} 
\sphinxAtStartPar
Clique em \sphinxstylestrong{Adicionar}

\end{enumerate}

\begin{sphinxadmonition}{tip}{Dica:}
\sphinxAtStartPar
Você também pode criar um novo localizador diretamente por essa tela.
\end{sphinxadmonition}


\subsubsection{Buscando processos por Localizador}
\label{\detokenize{projud_50_localizador:buscando-processos-por-localizador}}\begin{enumerate}
\sphinxsetlistlabels{\arabic}{enumi}{enumii}{}{.}%
\item {} 
\sphinxAtStartPar
Vá até a aba \sphinxstylestrong{Processos} \(\rightarrow\) \sphinxstylestrong{Busca Avançada}

\item {} 
\sphinxAtStartPar
Preencha os campos necessários (ex: juízo)

\item {} 
\sphinxAtStartPar
No campo \sphinxstylestrong{Localizador}, selecione o desejado

\item {} 
\sphinxAtStartPar
Clique em \sphinxstylestrong{Pesquisar}

\end{enumerate}


\subsubsection{Utilizando o Localizador em Análise Múltipla}
\label{\detokenize{projud_50_localizador:utilizando-o-localizador-em-analise-multipla}}
\sphinxAtStartPar
Você pode aproveitar o recurso de análise múltipla (como em \sphinxstylestrong{Juntadas} ou \sphinxstylestrong{Retorno de Conclusão}) para aplicar um mesmo localizador a vários processos de uma só vez.

\sphinxAtStartPar
Exemplo prático:
\begin{enumerate}
\sphinxsetlistlabels{\arabic}{enumi}{enumii}{}{.}%
\item {} 
\sphinxAtStartPar
Acesse a aba \sphinxstylestrong{Análise de Juntadas}

\item {} 
\sphinxAtStartPar
Selecione os processos desejados

\item {} 
\sphinxAtStartPar
Clique em \sphinxstylestrong{Análise Múltipla}

\item {} 
\sphinxAtStartPar
Realize a movimentação desejada (ex: intimação de partes)

\item {} 
\sphinxAtStartPar
No passo final, selecione o \sphinxstylestrong{Localizador} (ex: \sphinxcode{\sphinxupquote{Registrar no Renajud}})

\item {} 
\sphinxAtStartPar
Clique em \sphinxstylestrong{Salvar}

\end{enumerate}

\sphinxAtStartPar
Todos os processos movimentados serão também alocados no localizador selecionado.


\subsubsection{Resumo}
\label{\detokenize{projud_50_localizador:resumo}}\begin{itemize}
\item {} 
\sphinxAtStartPar
O Localizador é uma ferramenta de organização \sphinxstylestrong{manual e flexível}.

\item {} 
\sphinxAtStartPar
Pode ser usado para \sphinxstylestrong{monitoramento de grupos de processos}.

\item {} 
\sphinxAtStartPar
É útil em \sphinxstylestrong{intimações em lote}, \sphinxstylestrong{retorno de conclusões} e para \sphinxstylestrong{ações administrativas} específicas.

\end{itemize}

\sphinxAtStartPar
Utilize o recurso para tornar sua rotina mais produtiva e os processos mais fáceis de localizar no sistema.

\sphinxstepscope


\subsection{Agrupador}
\label{\detokenize{projud_51_agrupador:agrupador}}\label{\detokenize{projud_51_agrupador::doc}}
\sphinxAtStartPar
Os \sphinxstylestrong{agrupadores} no sistema Projudi são etiquetas ou marcadores que ajudam a organizar os processos conforme seu conteúdo, natureza ou necessidade de providência, facilitando a triagem, a análise e o cumprimento de diligências, tanto pela secretaria quanto pela assessoria.


\subsubsection{O que é um Agrupador?}
\label{\detokenize{projud_51_agrupador:o-que-e-um-agrupador}}
\sphinxAtStartPar
O agrupador serve como uma \sphinxstylestrong{categoria} personalizada associada a uma movimentação, especialmente no momento da \sphinxstylestrong{conclusão para o magistrado}.

\sphinxAtStartPar
Ele permite que a equipe da assessoria identifique rapidamente o tipo de minuta a ser elaborada, e que a secretaria faça filtros em massa para facilitar ações como análise múltipla, remessas, intimações, etc.


\subsubsection{Como cadastrar um Agrupador?}
\label{\detokenize{projud_51_agrupador:como-cadastrar-um-agrupador}}
\sphinxAtStartPar
Existem \sphinxstylestrong{duas formas} de cadastrar agrupadores:
\begin{enumerate}
\sphinxsetlistlabels{\arabic}{enumi}{enumii}{}{.}%
\item {} 
\sphinxAtStartPar
\sphinxstylestrong{Via menu Cadastro}:
\sphinxhyphen{} Acesse a aba superior \sphinxstylestrong{Cadastro} \(\rightarrow\) \sphinxstylestrong{Agrupadores}
\sphinxhyphen{} Clique em \sphinxstylestrong{Novo}
\sphinxhyphen{} Digite a descrição do agrupador (ex: \sphinxcode{\sphinxupquote{Decisão Inicial com Cautelar}})
\sphinxhyphen{} Clique em \sphinxstylestrong{Salvar}

\item {} 
\sphinxAtStartPar
\sphinxstylestrong{No momento da conclusão}:
\sphinxhyphen{} Acesse a aba \sphinxstylestrong{Movimentações} do processo
\sphinxhyphen{} Clique em \sphinxstylestrong{Movimentar a partir desta}
\sphinxhyphen{} Selecione a opção \sphinxstylestrong{Enviar Concluso}
\sphinxhyphen{} No campo \sphinxstylestrong{Agrupador}, selecione um já existente ou clique no símbolo ➕ para \sphinxstylestrong{criar um novo}
\sphinxhyphen{} Exemplo de agrupadores: \sphinxcode{\sphinxupquote{Recebimento da denúncia}}, \sphinxcode{\sphinxupquote{Vista ao MP}}, \sphinxcode{\sphinxupquote{Suspensão e prazo prescricional}}

\end{enumerate}


\subsubsection{Utilidade para a Assessoria}
\label{\detokenize{projud_51_agrupador:utilidade-para-a-assessoria}}
\sphinxAtStartPar
Na aba \sphinxstylestrong{Minutas}, ao acessar por tipo de documento (ex: Decisão), os processos conclusos estarão listados com os agrupadores visíveis na coluna lateral.

\sphinxAtStartPar
Isso \sphinxstylestrong{orienta o assessor} sobre:
\begin{itemize}
\item {} 
\sphinxAtStartPar
Qual tipo de minuta produzir (ex: decisão de recebimento de denúncia)

\item {} 
\sphinxAtStartPar
Qual diligência ou encaminhamento será necessário

\end{itemize}


\subsubsection{Utilidade para a Secretaria}
\label{\detokenize{projud_51_agrupador:utilidade-para-a-secretaria}}
\sphinxAtStartPar
Na aba \sphinxstylestrong{Retorno de Conclusão} ou \sphinxstylestrong{Análise de Juntadas}, é possível:
\begin{itemize}
\item {} 
\sphinxAtStartPar
\sphinxstylestrong{Ordenar os processos por agrupador} (clicando no cabeçalho da coluna)

\item {} 
\sphinxAtStartPar
\sphinxstylestrong{Selecionar múltiplos processos com o mesmo agrupador}

\item {} 
\sphinxAtStartPar
Executar ações em lote, como:
\sphinxhyphen{} Remessa ao Ministério Público
\sphinxhyphen{} Intimações
\sphinxhyphen{} Conclusões
\sphinxhyphen{} Cumprimentos de diligência

\end{itemize}


\paragraph{Exemplo:}
\label{\detokenize{projud_51_agrupador:exemplo}}
\sphinxAtStartPar
Selecionar todos os processos com o agrupador \sphinxcode{\sphinxupquote{Vista ao MP}}:
\begin{enumerate}
\sphinxsetlistlabels{\arabic}{enumi}{enumii}{}{.}%
\item {} 
\sphinxAtStartPar
Acesse \sphinxstylestrong{Retorno de Conclusão}

\item {} 
\sphinxAtStartPar
Ordene por agrupador

\item {} 
\sphinxAtStartPar
Marque todos os processos \sphinxcode{\sphinxupquote{Vista ao MP}}

\item {} 
\sphinxAtStartPar
Clique em \sphinxstylestrong{Análise Múltipla}

\item {} 
\sphinxAtStartPar
Realize a remessa ao MP para manifestação

\item {} 
\sphinxAtStartPar
Defina prazo, urgência e finalize

\end{enumerate}


\subsubsection{Resumo}
\label{\detokenize{projud_51_agrupador:resumo}}\begin{itemize}
\item {} 
\sphinxAtStartPar
Agrupadores facilitam a organização, produtividade e comunicação entre secretaria e assessoria.

\item {} 
\sphinxAtStartPar
Podem ser criados de forma independente ou no momento da conclusão.

\item {} 
\sphinxAtStartPar
São filtros valiosos para ações em massa.

\end{itemize}

\sphinxAtStartPar
Use agrupadores para transformar seu fluxo de trabalho em algo mais ágil, padronizado e eficiente.

\sphinxstepscope


\subsection{Cadastro de Infrações e Penas para fins de Prescrição}
\label{\detokenize{projud_52_cadastroinfracoes:cadastro-de-infracoes-e-penas-para-fins-de-prescricao}}\label{\detokenize{projud_52_cadastroinfracoes::doc}}
\sphinxAtStartPar
O cadastro de \sphinxstylestrong{infrações e penas} no sistema Projudi é fundamental para que o próprio sistema possa calcular automaticamente o \sphinxstylestrong{prazo prescricional} de forma precisa, com base na legislação penal.

\sphinxAtStartPar
Esse recurso é essencial na gestão de processos criminais, tanto para a \sphinxstylestrong{secretaria judicial} quanto para o \sphinxstylestrong{gabinete do magistrado}.


\subsubsection{Onde cadastrar?}
\label{\detokenize{projud_52_cadastroinfracoes:onde-cadastrar}}
\sphinxAtStartPar
No processo criminal aberto:
\begin{enumerate}
\sphinxsetlistlabels{\arabic}{enumi}{enumii}{}{.}%
\item {} 
\sphinxAtStartPar
Desça até a seção \sphinxstylestrong{Informações Adicionais}

\item {} 
\sphinxAtStartPar
Clique em \sphinxstylestrong{Infrações e Penas}

\item {} 
\sphinxAtStartPar
Se aparecer “Sem infração e pena”, clique para adicionar

\end{enumerate}


\subsubsection{Preenchimento dos Campos}
\label{\detokenize{projud_52_cadastroinfracoes:preenchimento-dos-campos}}
\sphinxAtStartPar
Na tela de cadastro:
\begin{itemize}
\item {} 
\sphinxAtStartPar
\sphinxstylestrong{Parte}: Será selecionada automaticamente

\item {} 
\sphinxAtStartPar
\sphinxstylestrong{Lei}: Clique nos atalhos para selecionar rapidamente (ex: \sphinxstyleemphasis{Código Penal})

\item {} 
\sphinxAtStartPar
\sphinxstylestrong{Artigo}: Informe o número do artigo (ex: \sphinxcode{\sphinxupquote{121}} \textendash{} Homicídio)

\item {} 
\sphinxAtStartPar
\sphinxstylestrong{Descrição}: Especifique se for qualificado, tentado, consumado etc.

\item {} 
\sphinxAtStartPar
\sphinxstylestrong{Complemento}: Use para detalhamentos (ex: motivo torpe)

\item {} 
\sphinxAtStartPar
\sphinxstylestrong{Hediondo?}: Marque se o crime é hediondo

\item {} 
\sphinxAtStartPar
\sphinxstylestrong{Incidente}: Se o réu é reincidente comum ou específico

\item {} 
\sphinxAtStartPar
\sphinxstylestrong{Concurso de Crimes?}: Marque se houver

\end{itemize}

\sphinxAtStartPar
Após preencher, clique em \sphinxstylestrong{Salvar}


\subsubsection{Resultado e Cálculo da Prescrição}
\label{\detokenize{projud_52_cadastroinfracoes:resultado-e-calculo-da-prescricao}}
\sphinxAtStartPar
Após o cadastro:
\begin{itemize}
\item {} 
\sphinxAtStartPar
O sistema exibirá a \sphinxstylestrong{data prevista para a prescrição} (ex: \sphinxcode{\sphinxupquote{05/09/2025}})

\item {} 
\sphinxAtStartPar
Clique em \sphinxstylestrong{Detalhes} para visualizar o cálculo completo:
\sphinxhyphen{} \sphinxstylestrong{Data do fato}
\sphinxhyphen{} \sphinxstylestrong{Data de nascimento do réu}
\sphinxhyphen{} \sphinxstylestrong{Tempo considerado}
\sphinxhyphen{} \sphinxstylestrong{Reduções aplicadas} (ex: menor de 21 anos na data do fato \(\rightarrow\) redução de 50\%)
\sphinxhyphen{} \sphinxstylestrong{Regra jurídica utilizada} (ex: Art. 109 do CP)
\sphinxhyphen{} \sphinxstylestrong{Resumo} da prescrição da pretensão punitiva ou executória

\end{itemize}


\paragraph{Importante}
\label{\detokenize{projud_52_cadastroinfracoes:importante}}
\sphinxAtStartPar
Se o sistema não encontrar informações como \sphinxstylestrong{data de recebimento da denúncia}, ele utilizará \sphinxstylestrong{a data do fato} como termo inicial do cálculo. Essas informações precisam estar corretamente lançadas nas movimentações para precisão do resultado.


\subsubsection{Utilidade}
\label{\detokenize{projud_52_cadastroinfracoes:utilidade}}
\sphinxAtStartPar
Este recurso é útil para:
\begin{itemize}
\item {} 
\sphinxAtStartPar
\sphinxstylestrong{Acompanhar o prazo de prescrição de forma automatizada}

\item {} 
\sphinxAtStartPar
\sphinxstylestrong{Evitar extinção da punibilidade por decurso de prazo}

\item {} 
\sphinxAtStartPar
\sphinxstylestrong{Planejar as diligências e movimentações com antecedência}

\item {} 
\sphinxAtStartPar
\sphinxstylestrong{Gerar relatórios de controle de prescrição}

\end{itemize}

\sphinxstepscope


\subsection{Cadastro da Denúncia ou Queixa para fins da Prescrição}
\label{\detokenize{projud_53_cadastrodenuncia:cadastro-da-denuncia-ou-queixa-para-fins-da-prescricao}}\label{\detokenize{projud_53_cadastrodenuncia::doc}}
\sphinxAtStartPar
A contagem precisa do \sphinxstylestrong{prazo prescricional} no sistema Projudi depende também do cadastro formal da \sphinxstylestrong{denúncia ou queixa} criminal. Mesmo que o processo já tenha registrado o oferecimento e recebimento da denúncia em suas \sphinxstylestrong{movimentações}, é imprescindível inserir essas informações na aba \sphinxstylestrong{Informações Adicionais} para que o sistema possa calcular corretamente os prazos legais.


\subsubsection{Local do Cadastro}
\label{\detokenize{projud_53_cadastrodenuncia:local-do-cadastro}}
\sphinxAtStartPar
Na tela do processo:
\begin{enumerate}
\sphinxsetlistlabels{\arabic}{enumi}{enumii}{}{.}%
\item {} 
\sphinxAtStartPar
Clique na aba \sphinxstylestrong{Informações Adicionais}

\item {} 
\sphinxAtStartPar
Vá até \sphinxstylestrong{Denunciado e Querelado}

\item {} 
\sphinxAtStartPar
Clique em \sphinxstylestrong{Novo} para iniciar o cadastro

\end{enumerate}


\subsubsection{Preenchimento dos Campos}
\label{\detokenize{projud_53_cadastrodenuncia:preenchimento-dos-campos}}\begin{itemize}
\item {} 
\sphinxAtStartPar
\sphinxstylestrong{Tipo}: Selecione se é \sphinxstylestrong{Denúncia} ou \sphinxstylestrong{Queixa}

\item {} 
\sphinxAtStartPar
\sphinxstylestrong{Parte}: O denunciado ou querelado será listado

\item {} 
\sphinxAtStartPar
\sphinxstylestrong{Assunto Principal}: Exemplo: Homicídio Qualificado

\item {} 
\sphinxAtStartPar
\sphinxstylestrong{Imputações}:
\sphinxhyphen{} Código Penal (ex: \sphinxstylestrong{Art. 121, \S{}2º})
\sphinxhyphen{} Se é \sphinxstylestrong{hediondo}
\sphinxhyphen{} Se é \sphinxstylestrong{consumado} ou \sphinxstylestrong{tentado}
\sphinxhyphen{} Se há \sphinxstylestrong{concurso de crimes}

\item {} 
\sphinxAtStartPar
\sphinxstylestrong{Data do Oferecimento da Denúncia}

\item {} 
\sphinxAtStartPar
\sphinxstylestrong{Data do Recebimento da Denúncia}

\item {} 
\sphinxAtStartPar
\sphinxstylestrong{Documentos} (se houver, podem ser vinculados)

\end{itemize}

\sphinxAtStartPar
Clique em \sphinxstylestrong{Salvar}


\subsubsection{Importância para o Cálculo da Prescrição}
\label{\detokenize{projud_53_cadastrodenuncia:importancia-para-o-calculo-da-prescricao}}
\sphinxAtStartPar
Com essas informações, o sistema deixa de utilizar a \sphinxstylestrong{data do fato} como termo inicial e passa a utilizar a \sphinxstylestrong{data de recebimento da denúncia}, como prevê a legislação penal.


\paragraph{Análise Detalhada}
\label{\detokenize{projud_53_cadastrodenuncia:analise-detalhada}}
\sphinxAtStartPar
Clicando em \sphinxstylestrong{Detalhes}, será exibido:
\begin{itemize}
\item {} 
\sphinxAtStartPar
Nome das partes (acusado e MP)

\item {} 
\sphinxAtStartPar
Crime imputado

\item {} 
\sphinxAtStartPar
Termo inicial: \sphinxstylestrong{recebimento da denúncia}

\item {} 
\sphinxAtStartPar
Tempo máximo considerado

\item {} 
\sphinxAtStartPar
Idade do réu na data do fato

\item {} 
\sphinxAtStartPar
Redução aplicada (ex: \sphinxhyphen{}50\% se \textless{} 21 anos)

\item {} 
\sphinxAtStartPar
\sphinxstylestrong{Prescrição calculada automaticamente}

\end{itemize}
\begin{description}
\sphinxlineitem{Exemplo:}\begin{itemize}
\item {} 
\sphinxAtStartPar
Termo inicial: 03/03/2017 (data do recebimento)

\item {} 
\sphinxAtStartPar
Prescrição da pretensão punitiva: 02/03/2025

\end{itemize}

\end{description}


\subsubsection{Conclusão}
\label{\detokenize{projud_53_cadastrodenuncia:conclusao}}
\sphinxAtStartPar
O cadastro de denúncia ou queixa é essencial para:
\begin{itemize}
\item {} 
\sphinxAtStartPar
Calcular corretamente o prazo da \sphinxstylestrong{prescrição penal}

\item {} 
\sphinxAtStartPar
Garantir segurança jurídica e agilidade processual

\item {} 
\sphinxAtStartPar
Oferecer dados claros e acessíveis para \sphinxstylestrong{magistrados, assessorias e secretarias}

\end{itemize}

\sphinxAtStartPar
Na próxima aula, abordaremos a aba \sphinxstylestrong{Mesa do Escrivão}, com foco nos relatórios de controle.

\sphinxstepscope


\subsection{Cadastro de Medidas Alternativas: Transação Penal}
\label{\detokenize{projud_54_cadastromedidasalternativas:cadastro-de-medidas-alternativas-transacao-penal}}\label{\detokenize{projud_54_cadastromedidasalternativas::doc}}
\sphinxAtStartPar
O cadastro de \sphinxstylestrong{transações penais} no sistema Projudi é realizado na aba
\sphinxstylestrong{Informações Adicionais}, especificamente em:

\sphinxAtStartPar
\sphinxcode{\sphinxupquote{Benefícios, Medidas e Suspensões}} \(\rightarrow\) \sphinxcode{\sphinxupquote{Transação Penal}}


\subsubsection{Cadastro de Medida Alternativa}
\label{\detokenize{projud_54_cadastromedidasalternativas:cadastro-de-medida-alternativa}}
\sphinxAtStartPar
Ao clicar em \sphinxstylestrong{Transação Penal}, abrirá a tela de cadastro da medida homologada em juízo.
\begin{enumerate}
\sphinxsetlistlabels{\arabic}{enumi}{enumii}{}{.}%
\item {} 
\sphinxAtStartPar
\sphinxstylestrong{Data de Início}

\item {} 
\sphinxAtStartPar
Clique em \sphinxstylestrong{Novo}

\item {} 
\sphinxAtStartPar
Selecione a \sphinxstylestrong{modalidade} da medida:
\sphinxhyphen{} Exemplo: \sphinxstylestrong{Prestação de serviço à comunidade}

\item {} 
\sphinxAtStartPar
Informe:
\sphinxhyphen{} \sphinxstylestrong{Quantidade de horas totais}
\sphinxhyphen{} \sphinxstylestrong{Horas mensais}
\sphinxhyphen{} \sphinxstylestrong{Nome da instituição beneficiária}

\sphinxAtStartPar
A instituição deve estar previamente cadastrada no sistema:

\sphinxAtStartPar
\sphinxstylestrong{Cadastro \textgreater{} Entidades Beneficiárias}

\sphinxAtStartPar
Preencha:
\sphinxhyphen{} Nome da entidade
\sphinxhyphen{} Endereço
\sphinxhyphen{} Dados bancários (se necessário)
\sphinxhyphen{} Outros dados requeridos

\item {} 
\sphinxAtStartPar
Informe:
\sphinxhyphen{} \sphinxstylestrong{Dias trabalhados por semana}
\sphinxhyphen{} \sphinxstylestrong{Período de cumprimento}
\sphinxhyphen{} \sphinxstylestrong{Observações (se houver)}

\end{enumerate}


\paragraph{Planejamento de Cumprimento}
\label{\detokenize{projud_54_cadastromedidasalternativas:planejamento-de-cumprimento}}
\sphinxAtStartPar
Você poderá criar um \sphinxstylestrong{planejamento de datas} para recebimento dos comprovantes mensais da prestação:
\begin{itemize}
\item {} 
\sphinxAtStartPar
Informe:
\sphinxhyphen{} Data de início
\sphinxhyphen{} Número de períodos (ex: 6 meses)
\sphinxhyphen{} Periodicidade (ex: 30 dias)

\item {} 
\sphinxAtStartPar
Clique em \sphinxstylestrong{Gerar Datas}

\item {} 
\sphinxAtStartPar
Clique em \sphinxstylestrong{Salvar}

\end{itemize}


\subsubsection{Cadastro de Prestação Pecuniária}
\label{\detokenize{projud_54_cadastromedidasalternativas:cadastro-de-prestacao-pecuniaria}}
\sphinxAtStartPar
Além do serviço comunitário, pode haver \sphinxstylestrong{prestação pecuniária} acumulada. Clique em \sphinxstylestrong{Adicionar}.
\begin{enumerate}
\sphinxsetlistlabels{\arabic}{enumi}{enumii}{}{.}%
\item {} 
\sphinxAtStartPar
\sphinxstylestrong{Valor total da prestação}

\item {} 
\sphinxAtStartPar
\sphinxstylestrong{Número de parcelas}

\item {} 
\sphinxAtStartPar
\sphinxstylestrong{Destinatário da prestação}:
\sphinxhyphen{} Entidade beneficiária
\sphinxhyphen{} Pessoa física (ex: vítima)

\item {} 
\sphinxAtStartPar
\sphinxstylestrong{Período de pagamento}:
\sphinxhyphen{} Data de início
\sphinxhyphen{} Periodicidade (ex: 30 dias)

\item {} 
\sphinxAtStartPar
Clique em \sphinxstylestrong{Gerar Datas}

\item {} 
\sphinxAtStartPar
Clique em \sphinxstylestrong{Salvar}

\end{enumerate}


\subsubsection{Gerenciamento das Medidas}
\label{\detokenize{projud_54_cadastromedidasalternativas:gerenciamento-das-medidas}}
\sphinxAtStartPar
Na mesma tela, o sistema permite:
\begin{itemize}
\item {} 
\sphinxAtStartPar
Adicionar novas medidas alternativas

\item {} 
\sphinxAtStartPar
\sphinxstylestrong{Informar descumprimento}

\item {} 
\sphinxAtStartPar
\sphinxstylestrong{Alterar ou remover} medidas já inseridas

\item {} 
\sphinxAtStartPar
\sphinxstylestrong{Gerar relatórios de acompanhamento}

\end{itemize}


\paragraph{Relatório de Acompanhamento}
\label{\detokenize{projud_54_cadastromedidasalternativas:relatorio-de-acompanhamento}}
\sphinxAtStartPar
Clique em \sphinxstylestrong{Relatório} para visualizar:
\begin{itemize}
\item {} 
\sphinxAtStartPar
Dados do beneficiário

\item {} 
\sphinxAtStartPar
Medidas aplicadas

\item {} 
\sphinxAtStartPar
Porcentagem de cumprimento

\item {} 
\sphinxAtStartPar
Datas planejadas de comprovação

\end{itemize}

\sphinxAtStartPar
Exemplo:
\begin{itemize}
\item {} 
\sphinxAtStartPar
\sphinxstylestrong{Prestação pecuniária}:
\sphinxhyphen{} 100\% ainda a pagar
\sphinxhyphen{} Parcelas em: 15/03, 15/04, 15/05…

\item {} 
\sphinxAtStartPar
\sphinxstylestrong{Prestação de serviço}:
\sphinxhyphen{} 10 horas por mês
\sphinxhyphen{} Dias específicos registrados

\end{itemize}


\subsubsection{Conclusão}
\label{\detokenize{projud_54_cadastromedidasalternativas:conclusao}}
\sphinxAtStartPar
Esse módulo permite à secretaria e ao juízo \sphinxstylestrong{acompanhar e controlar} com precisão a execução das medidas alternativas fixadas em transação penal, favorecendo a transparência e o cumprimento das determinações judiciais.

\sphinxstepscope


\subsection{Cadastro de Suspensões (Parte)}
\label{\detokenize{projud_55_cadastrosuspensao:cadastro-de-suspensoes-parte}}\label{\detokenize{projud_55_cadastrosuspensao::doc}}
\sphinxAtStartPar
Nesta aula, é demonstrado como realizar \sphinxstylestrong{a suspensão processual referente a uma parte específica} no sistema Projudi. Este procedimento difere da \sphinxstylestrong{suspensão do processo inteiro}, tratada em aula anterior.


\subsubsection{Acesso ao Cadastro}
\label{\detokenize{projud_55_cadastrosuspensao:acesso-ao-cadastro}}\begin{enumerate}
\sphinxsetlistlabels{\arabic}{enumi}{enumii}{}{.}%
\item {} 
\sphinxAtStartPar
Na \sphinxstylestrong{tela principal} do processo, vá até:

\sphinxAtStartPar
\sphinxcode{\sphinxupquote{Informações Adicionais}} \(\rightarrow\) \sphinxcode{\sphinxupquote{Benefícios, Medidas e Suspensões}} \(\rightarrow\) \sphinxcode{\sphinxupquote{Suspensão}}

\item {} 
\sphinxAtStartPar
Clique em \sphinxstylestrong{“Cadastrar”}

\end{enumerate}


\subsubsection{Suspensão de Parte (Art. 366 do CPP)}
\label{\detokenize{projud_55_cadastrosuspensao:suspensao-de-parte-art-366-do-cpp}}\begin{enumerate}
\sphinxsetlistlabels{\arabic}{enumi}{enumii}{}{.}%
\item {} 
\sphinxAtStartPar
\sphinxstylestrong{Data da suspensão}

\item {} 
\sphinxAtStartPar
\sphinxstylestrong{Motivo da suspensão}:
\sphinxhyphen{} Artigo \sphinxstylestrong{366 do CPP} (ausência do réu citado por edital)

\item {} 
\sphinxAtStartPar
Clique em \sphinxstylestrong{Salvar}

\sphinxAtStartPar
Resultado: A suspensão é aplicada apenas à parte selecionada.

\end{enumerate}


\subsubsection{Suspensão com Base no Art. 89 da Lei 9.099/95}
\label{\detokenize{projud_55_cadastrosuspensao:suspensao-com-base-no-art-89-da-lei-9-099-95}}\begin{enumerate}
\sphinxsetlistlabels{\arabic}{enumi}{enumii}{}{.}%
\item {} 
\sphinxAtStartPar
Acesse novamente:

\sphinxAtStartPar
\sphinxcode{\sphinxupquote{Informações Adicionais}} \(\rightarrow\) \sphinxcode{\sphinxupquote{Benefícios, Medidas e Suspensões}} \(\rightarrow\) \sphinxcode{\sphinxupquote{Suspensão}}

\item {} 
\sphinxAtStartPar
Insira:
\sphinxhyphen{} \sphinxstylestrong{Data de início da suspensão}
\sphinxhyphen{} \sphinxstylestrong{Motivo da suspensão}: Artigo 89 da Lei 9.099/95

\item {} 
\sphinxAtStartPar
O sistema abrirá a aba de \sphinxstylestrong{Condições impostas}. Clique em \sphinxstylestrong{Novo}.

\item {} 
\sphinxAtStartPar
Preencha:
\sphinxhyphen{} \sphinxstylestrong{Tipo de medida alternativa} (ex: Comparecimento em juízo)
\sphinxhyphen{} \sphinxstylestrong{Prazo da medida} (ex: 2 anos)
\sphinxhyphen{} \sphinxstylestrong{Local de comparecimento}:
\begin{itemize}
\item {} 
\sphinxAtStartPar
Pode ser \sphinxstyleemphasis{em juízo} ou \sphinxstyleemphasis{entidade beneficiada}

\item {} 
\sphinxAtStartPar
Caso entidade, deve estar previamente cadastrada no sistema (\sphinxcode{\sphinxupquote{Cadastro \textgreater{} Entidades Beneficiadas}})

\end{itemize}

\item {} 
\sphinxAtStartPar
Insira \sphinxstylestrong{observações}, se necessário.

\item {} 
\sphinxAtStartPar
Em \sphinxstylestrong{Cumprimentos}, configure o planejamento:
\sphinxhyphen{} \sphinxstylestrong{Data de início}
\sphinxhyphen{} \sphinxstylestrong{Quantidade de períodos} (ex: 24 para 2 anos)
\sphinxhyphen{} \sphinxstylestrong{Periodicidade} (ex: 30 dias)
\sphinxhyphen{} Clique em \sphinxstylestrong{Gerar datas}
\sphinxhyphen{} Clique em \sphinxstylestrong{Salvar}

\item {} 
\sphinxAtStartPar
Se necessário, \sphinxstylestrong{adicione novas condições} clicando em \sphinxstylestrong{Adicionar}

\end{enumerate}


\subsubsection{Recursos Disponíveis na Tela}
\label{\detokenize{projud_55_cadastrosuspensao:recursos-disponiveis-na-tela}}\begin{itemize}
\item {} 
\sphinxAtStartPar
\sphinxstylestrong{Relatório}: Geração de relatório com resumo das medidas impostas

\item {} 
\sphinxAtStartPar
\sphinxstylestrong{Suspender processo}: Caso necessário suspender o processo inteiro (encaminha para a tela geral de suspensão)

\item {} 
\sphinxAtStartPar
\sphinxstylestrong{Alterar ou remover}: Possibilidade de editar ou excluir as suspensões aplicadas

\end{itemize}


\subsubsection{Conclusão}
\label{\detokenize{projud_55_cadastrosuspensao:conclusao}}
\sphinxAtStartPar
O cadastro correto da suspensão \sphinxstylestrong{por parte} é essencial para o controle processual, especialmente em casos de réus ausentes ou que tenham aceitado propostas de suspensão condicional do processo.

\sphinxstepscope


\subsection{Cadastro de Medida Cautelar de Monitoração Eletrônica}
\label{\detokenize{projud_56_cadastromedidacautelar:cadastro-de-medida-cautelar-de-monitoracao-eletronica}}\label{\detokenize{projud_56_cadastromedidacautelar::doc}}
\sphinxAtStartPar
Nesta aula é demonstrado o procedimento para \sphinxstylestrong{cadastrar uma medida cautelar de monitoração eletrônica} no sistema Projudi, bem como a forma de localizar todos os processos com essa medida vigente.


\subsubsection{Cadastro da Medida}
\label{\detokenize{projud_56_cadastromedidacautelar:cadastro-da-medida}}\begin{enumerate}
\sphinxsetlistlabels{\arabic}{enumi}{enumii}{}{.}%
\item {} 
\sphinxAtStartPar
Acesse o processo desejado.

\item {} 
\sphinxAtStartPar
Vá em:

\sphinxAtStartPar
\sphinxcode{\sphinxupquote{Informações Adicionais}} \(\rightarrow\) \sphinxcode{\sphinxupquote{Medida Cautelar \sphinxhyphen{} Monitoração Eletrônica}} \(\rightarrow\) \sphinxcode{\sphinxupquote{Cadastrar}}

\item {} 
\sphinxAtStartPar
Na tela de cadastro:
\sphinxhyphen{} Informe a \sphinxstylestrong{data de início}
\sphinxhyphen{} Clique em \sphinxstylestrong{Novo} para inserir a medida
\sphinxhyphen{} Em \sphinxstylestrong{Condição}, selecione \sphinxstyleemphasis{Monitoração Eletrônica}
\sphinxhyphen{} Preencha os seguintes campos:
\begin{itemize}
\item {} 
\sphinxAtStartPar
\sphinxstylestrong{Prazo da monitoração}

\item {} 
\sphinxAtStartPar
\sphinxstylestrong{Considerar período para detração?} (Sim ou Não)

\item {} 
\sphinxAtStartPar
\sphinxstylestrong{Processo que concedeu a medida} (pode ser este ou outro)

\item {} 
\sphinxAtStartPar
\sphinxstylestrong{Área de inclusão} (ex: residência, bairro, cidade, conforme decisão judicial)

\item {} 
\sphinxAtStartPar
\sphinxstylestrong{Área de exclusão}

\item {} 
\sphinxAtStartPar
\sphinxstylestrong{Observações}

\end{itemize}

\item {} 
\sphinxAtStartPar
Clique em \sphinxstylestrong{Salvar} duas vezes (na medida e na tela principal)

\sphinxAtStartPar
Resultado: Medida cadastrada com sucesso.

\end{enumerate}


\subsubsection{Gerenciamento da Medida}
\label{\detokenize{projud_56_cadastromedidacautelar:gerenciamento-da-medida}}
\sphinxAtStartPar
Na tela de medidas cautelares, você poderá:
\begin{itemize}
\item {} 
\sphinxAtStartPar
\sphinxstylestrong{Gerar relatório}

\item {} 
\sphinxAtStartPar
\sphinxstylestrong{Informar descumprimento}

\item {} 
\sphinxAtStartPar
\sphinxstylestrong{Alterar dados}

\item {} 
\sphinxAtStartPar
\sphinxstylestrong{Remover medida}

\item {} 
\sphinxAtStartPar
\sphinxstylestrong{Adicionar outra medida cumulativa} (ex: proibição de contato)

\end{itemize}


\subsubsection{Pendência de Monitoração}
\label{\detokenize{projud_56_cadastromedidacautelar:pendencia-de-monitoracao}}
\sphinxAtStartPar
Após o cadastro, será gerada uma \sphinxstylestrong{pendência} do tipo \sphinxstyleemphasis{monitoração eletrônica sem prisão vinculada}. Para tratar:
\begin{enumerate}
\sphinxsetlistlabels{\arabic}{enumi}{enumii}{}{.}%
\item {} 
\sphinxAtStartPar
Clique na pendência

\item {} 
\sphinxAtStartPar
Será aberta a tela de cadastro da monitoração com possibilidade de ajustes adicionais

\end{enumerate}


\subsubsection{Gestão de Processos com Medida Vigente}
\label{\detokenize{projud_56_cadastromedidacautelar:gestao-de-processos-com-medida-vigente}}\begin{enumerate}
\sphinxsetlistlabels{\arabic}{enumi}{enumii}{}{.}%
\item {} 
\sphinxAtStartPar
Vá para a aba superior: \sphinxcode{\sphinxupquote{Início}}

\item {} 
\sphinxAtStartPar
Acesse: \sphinxcode{\sphinxupquote{Mesa do Escrivão Criminal}}

\sphinxAtStartPar
Dentro dessa tela, você verá \sphinxstylestrong{quadros} (filas de trabalho) com títulos, como:
\begin{itemize}
\item {} 
\sphinxAtStartPar
\sphinxcode{\sphinxupquote{Monitoração Eletrônica}}
\sphinxhyphen{} \sphinxcode{\sphinxupquote{Ativas}}
\sphinxhyphen{} \sphinxcode{\sphinxupquote{Inspiradas}}
\sphinxhyphen{} \sphinxcode{\sphinxupquote{Expiráveis}}

\end{itemize}

\item {} 
\sphinxAtStartPar
Clique em \sphinxstylestrong{Ativas}

\sphinxAtStartPar
Resultado: lista de todos os processos com \sphinxstylestrong{medida cautelar de monitoração eletrônica vigente}.

\item {} 
\sphinxAtStartPar
Clique sobre o nome do processo para:
\sphinxhyphen{} Ver detalhes
\sphinxhyphen{} Gerar relatórios
\sphinxhyphen{} Informar descumprimento
\sphinxhyphen{} Alterar ou remover a medida

\end{enumerate}


\subsubsection{Conclusão}
\label{\detokenize{projud_56_cadastromedidacautelar:conclusao}}
\sphinxAtStartPar
Este recurso permite \sphinxstylestrong{controle centralizado e eficaz} das medidas cautelares eletrônicas, com previsão de vencimento, relatórios e vinculação a outras condições cautelares.

\sphinxstepscope


\subsection{Cadastro de Medida Protetiva ao Agressor}
\label{\detokenize{projud_57_medidaprotetivaagressor:cadastro-de-medida-protetiva-ao-agressor}}\label{\detokenize{projud_57_medidaprotetivaagressor::doc}}
\sphinxAtStartPar
Nesta aula é apresentada a funcionalidade (ainda em fase de implantação) de \sphinxstylestrong{cadastro de medidas protetivas direcionadas ao agressor} no sistema Projudi. Esta funcionalidade será especialmente útil para as unidades especializadas em violência doméstica.


\subsubsection{Cadastro de Medida Protetiva ao Agressor}
\label{\detokenize{projud_57_medidaprotetivaagressor:id1}}\begin{enumerate}
\sphinxsetlistlabels{\arabic}{enumi}{enumii}{}{.}%
\item {} 
\sphinxAtStartPar
Na \sphinxstylestrong{tela principal do processo}, acesse:

\sphinxAtStartPar
\sphinxcode{\sphinxupquote{Informações Adicionais}} \(\rightarrow\) \sphinxcode{\sphinxupquote{Medida Protetiva ao Agressor}} \(\rightarrow\) \sphinxcode{\sphinxupquote{Cadastrar}}

\item {} 
\sphinxAtStartPar
Na tela de cadastro:
\sphinxhyphen{} Informe a \sphinxstylestrong{data de início da medida}
\sphinxhyphen{} Clique em \sphinxstylestrong{Novo} para adicionar uma nova medida protetiva

\item {} 
\sphinxAtStartPar
Preencha os seguintes campos:
\sphinxhyphen{} \sphinxstylestrong{Tipo de medida}: \sphinxstyleemphasis{Medida Protetiva}
\sphinxhyphen{} \sphinxstylestrong{Processo que concedeu a medida}: informe se é o processo atual ou outro
\sphinxhyphen{} \sphinxstylestrong{Espécie da medida} (ex: \sphinxstyleemphasis{afastamento do lar}, \sphinxstyleemphasis{proibição de contato} etc.)
\sphinxhyphen{} \sphinxstylestrong{Prazo de duração} (ex: 1 ano, 2 anos)
\sphinxhyphen{} \sphinxstylestrong{Data prevista para término}
\sphinxhyphen{} \sphinxstylestrong{Observações adicionais} (opcional)

\item {} 
\sphinxAtStartPar
Clique em \sphinxstylestrong{Salvar}

\end{enumerate}


\subsubsection{Status da Funcionalidade}
\label{\detokenize{projud_57_medidaprotetivaagressor:status-da-funcionalidade}}\begin{itemize}
\item {} 
\sphinxAtStartPar
A funcionalidade está \sphinxstylestrong{em fase de implantação}

\item {} 
\sphinxAtStartPar
Atualmente, o campo \sphinxstylestrong{Espécie da medida} ainda não está disponível

\item {} 
\sphinxAtStartPar
Em versões futuras, será possível escolher a espécie da medida a partir de uma lista padronizada (ex: \sphinxstyleemphasis{afastamento do lar}, \sphinxstyleemphasis{monitoramento eletrônico}, \sphinxstyleemphasis{proibição de aproximação}, entre outras)

\end{itemize}


\subsubsection{Aplicabilidade}
\label{\detokenize{projud_57_medidaprotetivaagressor:aplicabilidade}}
\sphinxAtStartPar
Essa ferramenta será especialmente útil para \sphinxstylestrong{Juizados de Violência Doméstica e Familiar contra a Mulher}, permitindo melhor gestão e controle das medidas protetivas concedidas aos agressores.


\subsubsection{Considerações Finais}
\label{\detokenize{projud_57_medidaprotetivaagressor:consideracoes-finais}}
\sphinxAtStartPar
Mesmo estando em fase de implantação, a estrutura do cadastro já segue o padrão do sistema Projudi e poderá, em breve, ser utilizada de forma completa.

\sphinxstepscope


\subsection{Cumprimentos de Medidas Alternativas}
\label{\detokenize{projud_58_cadastrocumprimentomedida:cumprimentos-de-medidas-alternativas}}\label{\detokenize{projud_58_cadastrocumprimentomedida::doc}}
\sphinxAtStartPar
Esta aula apresenta a aba \sphinxstylestrong{Cumprimentos de Medidas} do sistema Projudi, utilizada para registrar e controlar o cumprimento de \sphinxstylestrong{medidas alternativas}, como \sphinxstylestrong{prestação de serviço à comunidade} ou \sphinxstylestrong{prestação pecuniária}, aplicadas em casos como \sphinxstylestrong{transação penal}.


\subsubsection{Acesso à Aba}
\label{\detokenize{projud_58_cadastrocumprimentomedida:acesso-a-aba}}
\sphinxAtStartPar
Acesse a aba \sphinxstylestrong{Cumprimentos de Medidas} no menu superior. Nela, é possível visualizar:
\begin{itemize}
\item {} 
\sphinxAtStartPar
Medidas \sphinxstylestrong{atrasadas}

\item {} 
\sphinxAtStartPar
Medidas \sphinxstylestrong{sem cumprimentos}

\item {} 
\sphinxAtStartPar
Medidas \sphinxstylestrong{a vencer}

\end{itemize}


\subsubsection{Busca de Cumprimentos}
\label{\detokenize{projud_58_cadastrocumprimentomedida:busca-de-cumprimentos}}
\sphinxAtStartPar
Você pode buscar pelos seguintes filtros:
\begin{itemize}
\item {} 
\sphinxAtStartPar
Número do processo

\item {} 
\sphinxAtStartPar
Nome da parte

\item {} 
\sphinxAtStartPar
Entidade beneficiada

\item {} 
\sphinxAtStartPar
Tipo da medida (transação penal, suspensão condicional do processo, etc.)

\item {} 
\sphinxAtStartPar
Situação: a cumprir, cumprida, atrasada, não cumprida, a vencer

\item {} 
\sphinxAtStartPar
Data de cumprimento

\end{itemize}


\subsubsection{Inserção de Comprovante de Cumprimento}
\label{\detokenize{projud_58_cadastrocumprimentomedida:insercao-de-comprovante-de-cumprimento}}\begin{enumerate}
\sphinxsetlistlabels{\arabic}{enumi}{enumii}{}{.}%
\item {} 
\sphinxAtStartPar
Clique sobre a \sphinxstylestrong{data prevista de cumprimento}

\item {} 
\sphinxAtStartPar
Clique em \sphinxstylestrong{Comprovante}

\item {} 
\sphinxAtStartPar
Selecione o documento:
\sphinxhyphen{} De dentro do processo (já juntado)
\sphinxhyphen{} Ou \sphinxstylestrong{junte um novo arquivo} (PDF)

\item {} 
\sphinxAtStartPar
Preencha os dados:
\sphinxhyphen{} Se cumpriu: \sphinxstylestrong{Sim}
\sphinxhyphen{} Data do cumprimento
\sphinxhyphen{} Horas previstas e efetivamente cumpridas
\sphinxhyphen{} Informações adicionais obrigatórias:
\begin{itemize}
\item {} 
\sphinxAtStartPar
Faltas

\item {} 
\sphinxAtStartPar
Bom comportamento

\item {} 
\sphinxAtStartPar
Atos de indisciplina

\item {} 
\sphinxAtStartPar
Qualidade do serviço prestado

\end{itemize}

\item {} 
\sphinxAtStartPar
Clique em \sphinxstylestrong{Salvar}

\end{enumerate}


\subsubsection{Prestação Pecuniária}
\label{\detokenize{projud_58_cadastrocumprimentomedida:prestacao-pecuniaria}}\begin{enumerate}
\sphinxsetlistlabels{\arabic}{enumi}{enumii}{}{.}%
\item {} 
\sphinxAtStartPar
Clique sobre a \sphinxstylestrong{parcela} da prestação pecuniária

\item {} 
\sphinxAtStartPar
Insira o \sphinxstylestrong{comprovante de pagamento}

\item {} 
\sphinxAtStartPar
Clique em \sphinxstylestrong{Alterar}

\item {} 
\sphinxAtStartPar
Informe:
\sphinxhyphen{} Se cumpriu: \sphinxstylestrong{Sim}
\sphinxhyphen{} Data do pagamento
\sphinxhyphen{} Valor pago

\item {} 
\sphinxAtStartPar
Clique em \sphinxstylestrong{Salvar}

\end{enumerate}

\sphinxAtStartPar
O sistema atualizará automaticamente o total pago e o saldo restante, exibindo graficamente o andamento.


\subsubsection{Informar Descumprimento}
\label{\detokenize{projud_58_cadastrocumprimentomedida:informar-descumprimento}}\begin{enumerate}
\sphinxsetlistlabels{\arabic}{enumi}{enumii}{}{.}%
\item {} 
\sphinxAtStartPar
Clique no \sphinxstylestrong{tipo de medida}

\item {} 
\sphinxAtStartPar
Clique em \sphinxstylestrong{Descumprir}

\item {} 
\sphinxAtStartPar
Confirme a ação

\end{enumerate}

\sphinxAtStartPar
O sistema marcará como \sphinxstylestrong{inválidos} todos os cumprimentos abertos daquela medida.

\sphinxAtStartPar
Você também pode reativar medidas posteriormente.


\subsubsection{Relatórios}
\label{\detokenize{projud_58_cadastrocumprimentomedida:relatorios}}\begin{itemize}
\item {} 
\sphinxAtStartPar
O sistema fornece \sphinxstylestrong{relatórios detalhados} com:
\sphinxhyphen{} Total de parcelas previstas
\sphinxhyphen{} Parcelas cumpridas
\sphinxhyphen{} Situação individual de cada medida

\item {} 
\sphinxAtStartPar
Através da aba \sphinxstylestrong{Cumprimentos de Medidas}, a secretaria consegue manter um \sphinxstylestrong{controle eficaz} das obrigações impostas ao beneficiário, \sphinxstylestrong{sem necessidade de planilhas externas}.

\end{itemize}


\subsubsection{Considerações Finais}
\label{\detokenize{projud_58_cadastrocumprimentomedida:consideracoes-finais}}\begin{itemize}
\item {} 
\sphinxAtStartPar
Os registros de cumprimento, descumprimento e comprovações são automaticamente inseridos nas movimentações do processo.

\item {} 
\sphinxAtStartPar
O sistema oferece uma \sphinxstylestrong{gestão clara, detalhada e atualizada} das medidas alternativas aplicadas.

\end{itemize}

\sphinxstepscope


\subsection{Juntada de Habeas Corpus}
\label{\detokenize{projud_59_juntadahabeascorpus:juntada-de-habeas-corpus}}\label{\detokenize{projud_59_juntadahabeascorpus::doc}}

\subsubsection{Objetivo}
\label{\detokenize{projud_59_juntadahabeascorpus:objetivo}}
\sphinxAtStartPar
Demonstrar o procedimento para cadastrar e realizar a juntada de um \sphinxstylestrong{Habeas Corpus (HC)} físico recebido por \sphinxstylestrong{malote digital} ou \sphinxstylestrong{outro meio externo}, especialmente quando proveniente de \sphinxstylestrong{Tribunais Superiores} como o STJ ou STF.


\subsubsection{Atenção}
\label{\detokenize{projud_59_juntadahabeascorpus:atencao}}
\sphinxAtStartPar
A aba de Habeas Corpus deve ser utilizada \sphinxstylestrong{somente} para:
\begin{itemize}
\item {} 
\sphinxAtStartPar
Habeas Corpus \sphinxstylestrong{físico};

\item {} 
\sphinxAtStartPar
Impetrado \sphinxstylestrong{no TJ ou em Tribunais Superiores} (STJ, STF).

\end{itemize}

\sphinxAtStartPar
\sphinxstylestrong{Não deve ser usada} para:
\begin{itemize}
\item {} 
\sphinxAtStartPar
Habeas Corpus impetrado na vara;

\item {} 
\sphinxAtStartPar
Habeas Corpus eletrônico autuado no Projudi de 2º Grau.

\end{itemize}


\subsubsection{Passo a Passo}
\label{\detokenize{projud_59_juntadahabeascorpus:passo-a-passo}}\begin{enumerate}
\sphinxsetlistlabels{\arabic}{enumi}{enumii}{}{.}%
\item {} 
\sphinxAtStartPar
\sphinxstylestrong{Acesso}
\sphinxhyphen{} Vá até a aba superior: \sphinxstylestrong{HC/STJ}.
\sphinxhyphen{} Clique em \sphinxstylestrong{Adicionar}.

\item {} 
\sphinxAtStartPar
\sphinxstylestrong{Preenchimento do Cadastro}
\sphinxhyphen{} Informe o \sphinxstylestrong{número dos autos} do Habeas Corpus.
\sphinxhyphen{} Selecione o \sphinxstylestrong{órgão julgador} (STJ, STF, etc.).
\sphinxhyphen{} Escolha a \sphinxstylestrong{parte do processo} referente ao HC.
\sphinxhyphen{} Selecione o \sphinxstylestrong{resultado}:
\begin{itemize}
\item {} 
\sphinxAtStartPar
Liberdade concedida

\item {} 
\sphinxAtStartPar
Salvo\sphinxhyphen{}conduto concedido

\item {} 
\sphinxAtStartPar
HC denegado

\end{itemize}
\begin{itemize}
\item {} 
\sphinxAtStartPar
Insira \sphinxstylestrong{observações}, se necessário.

\end{itemize}

\item {} 
\sphinxAtStartPar
\sphinxstylestrong{Juntada da Peça}
\sphinxhyphen{} Clique em \sphinxstylestrong{Adicionar Peça}.
\sphinxhyphen{} Duas opções:
\begin{itemize}
\item {} 
\sphinxAtStartPar
\sphinxstylestrong{Selecionar} um documento já juntado no processo.

\item {} 
\sphinxAtStartPar
\sphinxstylestrong{Juntar um novo documento} do computador:
\sphinxhyphen{} Clique em \sphinxstylestrong{Juntar Documento}.
\sphinxhyphen{} Escolha o arquivo PDF.
\sphinxhyphen{} Nomeie o arquivo (ex: \sphinxtitleref{Habeas\_Corpus\_STF.pdf}).
\sphinxhyphen{} Insira sua \sphinxstylestrong{senha do sistema} e \sphinxstylestrong{assine digitalmente}.

\end{itemize}
\begin{itemize}
\item {} 
\sphinxAtStartPar
Clique em \sphinxstylestrong{Salvar}.

\end{itemize}

\item {} 
\sphinxAtStartPar
\sphinxstylestrong{Verificação}
\sphinxhyphen{} Vá até a aba \sphinxstylestrong{Movimentações}:
\begin{itemize}
\item {} 
\sphinxAtStartPar
Verifique a \sphinxstylestrong{juntada do HC}.

\end{itemize}
\begin{itemize}
\item {} 
\sphinxAtStartPar
Vá até a aba \sphinxstylestrong{Navegar}:
\sphinxhyphen{} Verifique o documento e o vínculo ao processo.

\end{itemize}

\end{enumerate}


\subsubsection{Resultado}
\label{\detokenize{projud_59_juntadahabeascorpus:resultado}}
\sphinxAtStartPar
A juntada do Habeas Corpus foi realizada com sucesso. A movimentação correspondente é registrada e vinculada ao processo principal, com visibilidade em todas as abas relevantes.

\sphinxstepscope


\subsection{Contagem dos Prazos Processuais}
\label{\detokenize{projud_60_contagemprazosprocessuais:contagem-dos-prazos-processuais}}\label{\detokenize{projud_60_contagemprazosprocessuais::doc}}

\subsubsection{Objetivo}
\label{\detokenize{projud_60_contagemprazosprocessuais:objetivo}}
\sphinxAtStartPar
Demonstrar como utilizar a aba \sphinxstylestrong{Prazos} no sistema Projudi para acompanhar e gerenciar a contagem de prazos processuais de forma organizada e centralizada.


\subsubsection{Localização}
\label{\detokenize{projud_60_contagemprazosprocessuais:localizacao}}
\sphinxAtStartPar
A aba \sphinxstylestrong{Prazos} encontra\sphinxhyphen{}se na parte final do menu superior do sistema Projudi, ao lado das demais abas como \sphinxstylestrong{Processo}, \sphinxstylestrong{Movimentações}, \sphinxstylestrong{Audiências}, entre outras.


\subsubsection{Funcionalidade da Aba}
\label{\detokenize{projud_60_contagemprazosprocessuais:funcionalidade-da-aba}}
\sphinxAtStartPar
Ao clicar na aba \sphinxstylestrong{Prazos}, o sistema exibirá uma tabela com os seguintes dados:
\begin{itemize}
\item {} 
\sphinxAtStartPar
\sphinxstylestrong{Data da Leitura}

\item {} 
\sphinxAtStartPar
\sphinxstylestrong{Prazo}

\item {} 
\sphinxAtStartPar
\sphinxstylestrong{Data de Cumprimento}

\item {} 
\sphinxAtStartPar
\sphinxstylestrong{Data do Decurso}

\item {} 
\sphinxAtStartPar
\sphinxstylestrong{Data da Interrupção (se houver)}

\item {} 
\sphinxAtStartPar
\sphinxstylestrong{Status do Prazo} (em curso, decorrido, interrompido, etc.)

\item {} 
\sphinxAtStartPar
\sphinxstylestrong{Tipo de Ato} (ex: Vista ao Ministério Público, Intimação, Citação)

\end{itemize}


\subsubsection{Visualizando o Detalhamento}
\label{\detokenize{projud_60_contagemprazosprocessuais:visualizando-o-detalhamento}}\begin{itemize}
\item {} 
\sphinxAtStartPar
Clique sobre o campo \sphinxstylestrong{Data da Leitura} de uma das linhas da tabela.

\item {} 
\sphinxAtStartPar
Será aberta a tela de \sphinxstylestrong{Detalhamento do Cálculo do Prazo}, contendo:
\begin{itemize}
\item {} 
\sphinxAtStartPar
Data da leitura (início da contagem)

\item {} 
\sphinxAtStartPar
Dias úteis computados

\item {} 
\sphinxAtStartPar
Exclusão de finais de semana e feriados (nacional, estadual ou municipal)

\item {} 
\sphinxAtStartPar
Data final do prazo

\item {} 
\sphinxAtStartPar
Indicação de decurso ou pendência

\end{itemize}

\end{itemize}


\subsubsection{Benefícios da Aba}
\label{\detokenize{projud_60_contagemprazosprocessuais:beneficios-da-aba}}\begin{itemize}
\item {} 
\sphinxAtStartPar
Centraliza \sphinxstylestrong{todas as contagens de prazo} em um único local por processo.

\item {} 
\sphinxAtStartPar
Evita a necessidade de verificar prazos apenas por pendências ou movimentações.

\item {} 
\sphinxAtStartPar
Acelera o trabalho de servidores ao permitir visualização e controle direto.

\end{itemize}


\subsubsection{Observações}
\label{\detokenize{projud_60_contagemprazosprocessuais:observacoes}}
\sphinxAtStartPar
Esta funcionalidade é complementar às análises já vistas nas aulas anteriores sobre:
\begin{itemize}
\item {} 
\sphinxAtStartPar
\sphinxstylestrong{Intimações e Citações}

\item {} 
\sphinxAtStartPar
\sphinxstylestrong{Análise de Pendências}

\item {} 
\sphinxAtStartPar
\sphinxstylestrong{Interrupção de Prazo}

\end{itemize}

\sphinxAtStartPar
A aba \sphinxstylestrong{Prazos} agrega todas essas movimentações em uma única visualização, facilitando o gerenciamento do processo.

\sphinxstepscope


\subsection{Mesa do Escrivão Criminal: Análise de Prescrições}
\label{\detokenize{projud_61_acompanhamentoprescricao:mesa-do-escrivao-criminal-analise-de-prescricoes}}\label{\detokenize{projud_61_acompanhamentoprescricao::doc}}

\subsubsection{Objetivo}
\label{\detokenize{projud_61_acompanhamentoprescricao:objetivo}}
\sphinxAtStartPar
Apresentar as funcionalidades da \sphinxstylestrong{aba exclusiva da área criminal} chamada \sphinxstylestrong{Mesa do Escrivão Criminal}, com destaque para a gestão de \sphinxstylestrong{prescrições} e outras pendências relevantes à tramitação de processos penais.


\subsubsection{Acesso}
\label{\detokenize{projud_61_acompanhamentoprescricao:acesso}}
\sphinxAtStartPar
Esta funcionalidade está disponível apenas para unidades da área criminal, como:
\begin{itemize}
\item {} 
\sphinxAtStartPar
Juizado Especial Criminal

\item {} 
\sphinxAtStartPar
Varas Criminais Comuns

\item {} 
\sphinxAtStartPar
Varas de Meio Ambiente Criminal

\item {} 
\sphinxAtStartPar
Varas Especializadas

\end{itemize}

\sphinxAtStartPar
Localização: Aba superior \sphinxstylestrong{“Mesa do Escrivão Criminal”}.


\subsubsection{Organização Visual}
\label{\detokenize{projud_61_acompanhamentoprescricao:organizacao-visual}}
\sphinxAtStartPar
Ao acessar a aba, o sistema exibe \sphinxstylestrong{painéis em formato de cartões}, semelhantes a filas organizadas por tipo de pendência ou dado relevante. Exemplos:
\begin{itemize}
\item {} 
\sphinxAtStartPar
\sphinxstylestrong{Feitos com réu preso (sem sentença anotada)}

\item {} 
\sphinxAtStartPar
\sphinxstylestrong{Mandados com prazo vencido}

\item {} 
\sphinxAtStartPar
\sphinxstylestrong{Preções vencidas e a vencer}

\item {} 
\sphinxAtStartPar
\sphinxstylestrong{Prisões provisórias com mais de 90 dias}

\item {} 
\sphinxAtStartPar
\sphinxstylestrong{Processos paralisados}

\item {} 
\sphinxAtStartPar
\sphinxstylestrong{Inquéritos pendentes de arquivamento}

\item {} 
\sphinxAtStartPar
\sphinxstylestrong{Ações penais sem denúncia}

\item {} 
\sphinxAtStartPar
\sphinxstylestrong{Feitos com réu sem CPF ou RG}

\item {} 
\sphinxAtStartPar
\sphinxstylestrong{Medidas alternativas atrasadas}

\end{itemize}


\subsubsection{Prescrições: vencidas e a vencer}
\label{\detokenize{projud_61_acompanhamentoprescricao:prescricoes-vencidas-e-a-vencer}}
\sphinxAtStartPar
Destaque para o painel \sphinxstylestrong{Prescrições}, que se divide em:
\begin{itemize}
\item {} 
\sphinxAtStartPar
\sphinxstylestrong{Prescrições Vencidas}: Exibe processos com prazos prescricionais já ultrapassados.

\item {} 
\sphinxAtStartPar
\sphinxstylestrong{Prescrições a Vencer}: Mostra processos com prescrição próxima da data\sphinxhyphen{}limite.

\end{itemize}

\sphinxAtStartPar
Cada item exibe:
\begin{itemize}
\item {} 
\sphinxAtStartPar
Número do processo

\item {} 
\sphinxAtStartPar
Classe processual

\item {} 
\sphinxAtStartPar
Data da infração e da distribuição

\item {} 
\sphinxAtStartPar
Data prevista para a prescrição

\item {} 
\sphinxAtStartPar
Dias paralisados

\end{itemize}

\begin{sphinxadmonition}{note}{Nota:}
\sphinxAtStartPar
Para que os processos apareçam corretamente nesses painéis, \sphinxstylestrong{é necessário que os dados estejam corretamente alimentados}, especialmente:
\begin{itemize}
\item {} 
\sphinxAtStartPar
\sphinxstylestrong{Data do fato}

\item {} 
\sphinxAtStartPar
\sphinxstylestrong{Data de nascimento do réu}

\item {} 
\sphinxAtStartPar
\sphinxstylestrong{Cadastro de infrações e penas}

\item {} 
\sphinxAtStartPar
\sphinxstylestrong{Cadastro de denúncia/queixa com datas de oferecimento e recebimento}

\end{itemize}
\end{sphinxadmonition}


\subsubsection{Funcionalidades Adicionais da Mesa}
\label{\detokenize{projud_61_acompanhamentoprescricao:funcionalidades-adicionais-da-mesa}}
\sphinxAtStartPar
Outros painéis relevantes:
\begin{itemize}
\item {} 
\sphinxAtStartPar
\sphinxstylestrong{Prisões provisórias sem sentença com mais de 90 dias}

\item {} 
\sphinxAtStartPar
\sphinxstylestrong{Feitos sem anotação de sentença}

\item {} 
\sphinxAtStartPar
\sphinxstylestrong{Processos com apreensão sem depósito}

\item {} 
\sphinxAtStartPar
\sphinxstylestrong{Habeas corpus pendentes (TJ/STJ/STF)}

\item {} 
\sphinxAtStartPar
\sphinxstylestrong{Monitoramento eletrônico (ativas/expiradas)}

\item {} 
\sphinxAtStartPar
\sphinxstylestrong{Ações penais sem audiência agendada}

\end{itemize}


\subsubsection{Relatórios e Organização}
\label{\detokenize{projud_61_acompanhamentoprescricao:relatorios-e-organizacao}}
\sphinxAtStartPar
Os painéis permitem:
\begin{itemize}
\item {} 
\sphinxAtStartPar
Ordenação por data (ascendente/descendente)

\item {} 
\sphinxAtStartPar
Visualização rápida de gargalos

\item {} 
\sphinxAtStartPar
Auditoria de processos paralisados ou sem movimentações essenciais

\item {} 
\sphinxAtStartPar
Tomada de decisões preventivas para evitar prescrição

\end{itemize}


\subsubsection{Importância}
\label{\detokenize{projud_61_acompanhamentoprescricao:importancia}}
\sphinxAtStartPar
A aba \sphinxstylestrong{Mesa do Escrivão Criminal} é uma poderosa ferramenta de \sphinxstylestrong{gestão da vara penal}, pois permite:
\begin{itemize}
\item {} 
\sphinxAtStartPar
Visualização centralizada de processos críticos

\item {} 
\sphinxAtStartPar
Detecção proativa de riscos de prescrição

\item {} 
\sphinxAtStartPar
Priorização no andamento de feitos sensíveis

\item {} 
\sphinxAtStartPar
Apoio direto à chefia de secretaria e gabinete

\end{itemize}



\renewcommand{\indexname}{Índice}
\printindex
\end{document}